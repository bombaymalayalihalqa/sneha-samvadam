\chapter{ഇസ്‌ലാമും ഭീകരവാദവും}
\section{  ലോകമെങ്ങുമുള്ള മുസ്‌ലിംകള്‍ ഭീകരവാദികളും തീവ്രവാദികളുമാകാന്‍ കാരണം ഇസ്‌ലാമല്ലേ?}
    
 അല്പം വിശദീകരണമര്‍ഹിക്കുന്ന ചോദ്യമാണിത്. 1492 മനുഷ്യചരിത്രത്തിലെ ഏറ്റവും വലിയ ദുരന്തത്തിന്റെ വര്‍ഷമായിരുന്നു. നീണ്ട നിരവധി നൂറ്റാണ്ടുകാലം ശാസ്ത്രസാങ്കേതിക രംഗങ്ങളിലും കലാ,സാഹിത്യ, സാംസ്‌കാരിക, നാഗരിക, വൈജ്ഞാനിക മേഖലകളിലും ലോകത്തിന് നേതൃത്വം നല്‍കിപ്പോന്ന മുസ്‌ലിം സ്‌പെയിനിലെ അവസാനത്തെ ഭരണാധികാരി അബൂ അബ്ദുല്ലയായിരുന്നു. ഗ്രാനഡെ നഗരം മാത്രമേ അദ്ദേഹത്തിന്റെ അധീനതയിലുണ്ടായിരുന്നുള്ളൂ. 1492 ജനുവരി അവസാനത്തോടെയാണ് അയാളെ പുറംതള്ളി സ്പാനിഷുകാര്‍ അവിടെ ആധിപത്യമുറപ്പിച്ചത്. സ്‌പെയിനിന്റെ പതനം പൂര്‍ത്തിയായ അതേ വര്‍ഷമാണ് സാമ്രാജ്യത്വാധിനിവേശം ആരംഭിച്ചതെന്ന വസ്തുത ഏറെ ശ്രദ്ധേയമത്രെ. 1492ലാണല്ലോ കൊളംബസ് തന്റെ 'കണ്ടെത്തല്‍' യാത്രക്ക് തുടക്കം കുറിച്ചത്.
അതേവര്‍ഷം ഒക്ടോബര്‍ 12ന് അയാള്‍ ഗ്വാനാഹാനി ദ്വീപിലെത്തി. ആയുധങ്ങളുമായി കപ്പലിറങ്ങിയ കൊളംബസും കൂട്ടുകാരും, അതുമുതല്‍ ആ നാട് സ്പാനിഷ് രാജാവിന്റെതായിരിക്കുമെന്ന് പ്രഖ്യാപിച്ചു. തദ്ദേശീയരെ അവര്‍ക്കറിയാത്ത സ്പാനിഷ് ഭാഷയിലുള്ള ഉത്തരവ് വായിച്ചുകേള്‍പ്പിച്ചു. അത് അനുസരിച്ചില്ലെങ്കില്‍ എന്താണ് സംഭവിക്കുകയെന്ന് ഇങ്ങനെ വിശദീകരിച്ചു: 'ഞാന്‍ ഉറപ്പിച്ചു പറയുന്നു, ദൈവസഹായത്താല്‍ ഞങ്ങള്‍ നിങ്ങളുടെ രാജ്യത്ത് ബലമായി പ്രവേശിക്കും. നിങ്ങളോട് ആവുംവിധം യുദ്ധം ചെയ്യും. നിങ്ങളെ ക്രിസ്ത്യന്‍ പള്ളിക്കും തമ്പ്രാക്കന്മാര്‍ക്കും കീഴ്‌പെടുത്തും. നിങ്ങളെയും ഭാര്യമാരെയും കുട്ടികളെയുമെല്ലാം പിടികൂടി അടിമകളാക്കും. നിങ്ങളുടെ സാമാനങ്ങള്‍ പിടിച്ചടക്കും. ഞങ്ങളാലാവുന്ന എല്ലാ ദ്രോഹവും നാശവും നിങ്ങള്‍ക്ക് വരുത്തും'' ( ഉദ്ധരണം: പരാന്നഭോജികള്‍: പാശ്ചാത്യവല്‍ക്കരണത്തിന്റെ അഞ്ഞൂറു വര്‍ഷം, ഐ.പി.എച്ച്, പുറം 17)
ഇതോടെയാണ് യൂറോപ്പിന്റെ അധിനിവേശം ആരംഭിച്ചത്. മുസ്‌ലിം സ്‌പെയിന്‍ തകര്‍ന്ന് കൃത്യം ആറുവര്‍ഷം കഴിഞ്ഞ് 1498ലാണല്ലോ വാസ്‌കോഡഗാമ കോഴിക്കോട്ട് കപ്പലിറങ്ങിയത്.
1492ല്‍ അമേരിക്കയില്‍ ഏഴരകോടിക്കും പത്തുകോടിക്കുമിടയില്‍ ആദിവാസികളുണ്ടായിരുന്നു. യൂറോപ്യന്‍ അധിനിവേശത്തെത്തുടര്‍ന്ന് ഒന്നര നൂറ്റാണ്ടുകൊണ്ട് അവരില്‍ 90 ശതമാനവും സ്വന്തം മണ്ണില്‍നിന്ന് തുടച്ചുനീക്കപ്പെട്ടു. ക്രൂരമായ കൂട്ടക്കൊലകളിലൂടെ തദ്ദേശീയരെ നശിപ്പിച്ചശേഷം 1776ലെ 'സ്വാതന്ത്യ്രപ്രഖ്യാപനത്തോടെ' യൂറോപ്യര്‍ അമേരിക്ക അധീനപ്പെടുത്തുകയായിരുന്നു. തീര്‍ത്തും അനീതിയിലും അതിക്രമത്തിലും അധിഷ്ഠിതമായ ഈ പ്രഖ്യാപനത്തിലൂടെയാണ് ഇന്നത്തെ അമേരിക്ക സ്ഥാപിതമായത്.
1527ല്‍ പോര്‍ച്ചുഗീസുകാര്‍ ബഹ്‌റൈന്‍ പിടിച്ചടക്കി. തൊട്ടുടനെ ഒമാനും കീഴ്‌പ്പെടുത്തി. എങ്കിലും ഉസ്മാനികള്‍ ആ നാടുകള്‍ തിരിച്ചുപിടിച്ചു. പിന്നീട് 17981801 കാലത്ത് നെപ്പോളിയന്റെ ഫ്രഞ്ച് സേന ഈജിപ്തിലെ അലക്‌സാണ്ട്രിയ, അക്കാ നഗരങ്ങള്‍ അധീനപ്പെടുത്തി.
പിന്നിട്ട രണ്ട് നൂറ്റാണ്ടുകള്‍ അറബ് മുസ്‌ലിം നാടുകള്‍ ഇതര ഏഷ്യനാഫ്രിക്കന്‍ രാജ്യങ്ങളെപ്പോലെത്തന്നെ പാശ്ചാത്യാധിനിവേശത്തിന്റേതായിരുന്നു. ഫ്രാന്‍സ് 1830ല്‍ അള്‍ജീരിയയും 1859ല്‍ ജിബൂട്ടിയും 1881ല്‍ തുനീഷ്യയും 1919ല്‍ മൗറിത്താനിയയും അധീനപ്പെടുത്തി. ഇറ്റലി 1859ല്‍ സോമാലിയയും 1911ല്‍ ലിബിയയും 1880ല്‍ ഐരിത്രിയയും കീഴ്‌പ്പെടുത്തി. ബ്രിട്ടന്‍ 1800ല്‍ മസ്‌കത്തും 1820ല്‍ ഒമാന്റെ ബാക്കി ഭാഗവും 1839ല്‍ ഏതനും 1863ല്‍ ബഹ്‌റൈനും 1878ല്‍ സൈപ്രസും 1882ല്‍ ഈജിപ്തും 1898ല്‍ സുഡാനും 1899ല്‍ കുവൈതും പിടിച്ചടക്കി. 1916ല്‍ ബ്രിട്ടനും ഫ്രാന്‍സും ചേര്‍ന്നുണ്ടാക്കിയ സൈക്‌സ്പിക്കോട്ട് രഹസ്യ കരാറനുസരിച്ച് ഉസ്മാനിയാ ഖലീഫയുടെ കീഴിലുണ്ടായിരുന്ന അറബ് പ്രവിശ്യകള്‍ ബ്രിട്ടനും ഫ്രാന്‍സും പങ്കിട്ടെടുത്തു. അങ്ങനെ ഇറാഖും ജോര്‍ദാനും ഫലസ്തീനും ഖത്തറും ബ്രിട്ടന്റെയും സിറിയയും ലബനാനും ഫ്രാന്‍സിന്റെയും പിടിയിലമര്‍ന്നു. മൊറോക്കോ സ്‌പെയിനിന്റെയും ഇന്തോനേഷ്യ ഡച്ചുകാരുടെയും കോളനികളായി. ഇന്ത്യയും അഫ്ഗാനിസ്ഥാനുമെല്ലാം സാമ്രാജ്യശക്തികളുടെ പിടിയിലമര്‍ന്നപോലെത്തന്നെ.
കഴിഞ്ഞ നൂറ്റാണ്ടിന്റെ ആദ്യപാദം പിന്നിട്ടതോടെ പടിഞ്ഞാറിന്റെ കോളനികളിലെല്ലാം സ്വാതന്ത്യ്രസമരം ശക്തിപ്പെട്ടു. തദ്ഫലമായി 1932ല്‍ ഇറാഖും '46ല്‍ സിറിയയും ലബനാനും '51ല്‍ ലിബിയയും ഒമാനും '52ല്‍ ഈജിപ്തും '56ല്‍ മൊറോക്കോയും സുഡാനും തുനീഷ്യയും '58ല്‍ ജോര്‍ദാനും '59ല്‍ മൗറിത്താനിയയും '60ല്‍ സോമാലിയയും '61ല്‍ കുവൈത്തും '62ല്‍ അള്‍ജീരിയയും '68ല്‍ യമനും '71ല്‍ ഖത്തറും ബഹ്‌റൈനും അറബ് എമിറേറ്റ്‌സും '77ല്‍ ജിബൂട്ടിയും സ്വാതന്ത്യ്രം നേടി.
എന്നാല്‍, സാമ്രാജ്യശക്തികള്‍ ഈ നാടുകളോട് വിടപറഞ്ഞത് അവിടങ്ങളിലടക്കം അവരുടെ താല്‍പര്യങ്ങള്‍ സംരക്ഷിക്കുന്ന ഏകാധിപതികളും സ്വേച്ഛാധികാരികളുമായ രാജാക്കന്മാരെയും ചക്രവര്‍ത്തിമാരെയും സുല്‍ത്താന്മാരെയും പ്രതിഷ്ഠിച്ച ശേഷമായിരുന്നു. അതോടൊപ്പം ഈ നാടുകള്‍ക്കിടയിലെല്ലാം അപരിഹാര്യങ്ങളായ അതിര്‍ത്തിത്തര്‍ക്കങ്ങളും ഉണ്ടാക്കിവെച്ചു. യമനും സൗദി അറേബ്യയും തമ്മിലും ഇറാനും ഇറാഖും തമ്മിലും ഇറാഖും കുവൈത്തും തമ്മിലും ഇറാനും യു.എ.ഇയും തമ്മിലും ഇനിയും പരിഹരിക്കപ്പെടാത്ത തര്‍ക്കങ്ങള്‍ നിലനില്‍ക്കാനുള്ള കാരണം സാമ്രാജ്യത്വശക്തികള്‍ ചെയ്തുവെച്ച കുതന്ത്രങ്ങളത്രെ. അറബ്മുസ്‌ലിം നാടുകള്‍ ഒന്നിക്കുന്നതിന് ഈ അതിര്‍ത്തി തര്‍ക്കങ്ങള്‍ സ്ഥിരമായ തടസ്സം സൃഷ്ടിക്കുന്നതോടൊപ്പം പലപ്പോഴും കിടമല്‍സരത്തിന് കാരണമായിത്തീരുകയും ചെയ്യുന്നു. അമേരിക്കക്കും ഇതര മുതലാളിത്ത നാടുകള്‍ക്കും അവിടങ്ങളിലെ പെട്രോളും വാതകവും ഇതര അസംസ്‌കൃത പദാര്‍ഥങ്ങളും തട്ടിയെടുക്കാനും ആ നാടുകളെ തങ്ങളുടെ ആയുധക്കമ്പോളമാക്കി മാറ്റാനും ഇത് അവസരമൊരുക്കുന്നു.
സാമ്രാജ്യശക്തികള്‍ തന്നെ സൃഷ്ടിച്ച തര്‍ക്കത്തിന്റെ പേരിലാണല്ലോ ഇറാഖും കുവൈത്തും തമ്മിലേറ്റുമുട്ടിയത്. ഇത് അമേരിക്കക്ക് മേഖലയില്‍ ഇടപെടാന്‍ അവസരമൊരുക്കി. അതുതന്നെയായിരുന്നുവല്ലോ അവരുടെ ലക്ഷ്യം. അറബ്മുസ്‌ലിം നാടുകള്‍ സ്വാതന്ത്യ്രം നേടിയശേഷവും അവിടങ്ങളിലെല്ലാം പടിഞ്ഞാറിന്റെ അദൃശ്യസാമ്രാജ്യത്വവും ചൂഷണവും നിയന്ത്രണവും ഇന്നോളം നിലനിന്നുപോന്നിട്ടുണ്ട്. ആ രാജ്യങ്ങള്‍ക്കൊന്നും തന്നെ സ്വന്തം നാടുകളിലെ വിഭവങ്ങള്‍ ഇഷ്ടാനുസൃതം വിനിയോഗിക്കാനോ നിയന്ത്രിക്കാനോ സാധിച്ചിട്ടില്ല. അവിടങ്ങളിലെ ഭരണാധികാരികള്‍ അറിഞ്ഞോ അറിയാതെയോ നിര്‍ബന്ധിതമായോ അല്ലാതെയോ പടിഞ്ഞാറിന്റെ താല്‍പര്യങ്ങള്‍ സംരക്ഷിച്ചുവരികയായിരുന്നു.
ഇറാഖില്‍ ഇടപെടാന്‍ അവസരം ലഭിച്ചതോടെ അമേരിക്ക അറബ് നാടുകളുടെ മേലുള്ള പിടിമുറുക്കുകയും തങ്ങളുടെ പട്ടാളത്തെ തീറ്റിപ്പോറ്റുന്ന ചുമതലയും സാമ്പത്തിക പ്രതിസന്ധി പരിഹരിക്കുന്ന ബാധ്യതയും ആ നാടുകളുടെ മേല്‍ വെച്ചുകെട്ടുകയും ചെയ്തു. ഇന്ന് എല്ലാ പ്രധാന അറബ് നാടുകളിലും അമേരിക്കക്ക് ആയുധശാലകളും സൈനികത്താവളങ്ങളുമുണ്ട്. അത് അവസാനിപ്പിക്കണമെന്ന് ആവശ്യപ്പെടുന്നതും ഈ സാമ്രാജ്യത്വ ചൂഷണത്തിന് അറുതിവരുത്തണമെന്ന് വാദിക്കുന്നതും കൊടിയ പാതകമായാണ് പാശ്ചാത്യസാമ്രാജ്യ ശക്തികള്‍ കണക്കാക്കുന്നത്.
കഴിഞ്ഞ നൂറ്റാണ്ടിലെ അവസാനദശകം വരെ ലോകത്ത് ശാക്തികമായ സന്തുലിതത്വം നിലനിന്നിരുന്നു. എന്നാല്‍ സോഷ്യലിസ്‌റ് ചേരി ദുര്‍ബലമായി ചരിത്രത്തിന്റെ ഭാഗമായതോടെ ശീതസമരം അവസാനിച്ചു. ലോകം അമേരിക്കയുടെ നേതൃത്വത്തില്‍ ഏകധ്രുവമായി മാറി. ഗള്‍ഫ് യുദ്ധത്തില്‍ അമേരിക്കന്‍ ചേരി വിജയിച്ചതോടെ ആ രാജ്യം ലോകപോലീസ് ചമയാന്‍ തുടങ്ങി. തങ്ങളുടെ താല്‍പര്യങ്ങള്‍ക്ക് എതിരു നില്‍ക്കുന്നവരെയെല്ലാം തീവ്രവാദികളും ഭീകരവാദികളുമായി മുദ്രകുത്തുകയും അവരെയൊക്കെ തകര്‍ക്കാന്‍ ഇറങ്ങിപ്പുറപ്പെടുകയും ചെയ്തു.
പാശ്ചാത്യ ചേരി കമ്യൂണിസ്‌റ് നാടുകളുടെ തകര്‍ച്ചയോടെ തങ്ങളുടെ മുഖ്യശത്രുവായി പ്രതിഷ്ഠിച്ചത് ഇസ്‌ലാമിനെയാണ്. അമേരിക്കയും അതിന്റെ നേതൃത്വത്തിലുള്ള നാറ്റോയും ഇക്കാര്യം സംശയലേശമന്യേ വ്യക്തമാക്കിയിട്ടുണ്ട്. അതിനാല്‍ ലോകമെങ്ങുമുള്ള ഇസ്‌ലാമിക നവോത്ഥാന ചലനങ്ങളെയും മുന്നേറ്റങ്ങളെയും അടിച്ചമര്‍ത്താനും നശിപ്പിക്കാനും അമേരിക്കയും കൂട്ടാളികളും ആവുന്നതൊക്കെ ചെയ്യുന്നു. അതിനായി ആടിനെ പട്ടിയാക്കും വിധമുള്ള പ്രചാരവേലകള്‍ സംഘടിപ്പിക്കുന്നു. ഇസ്‌ലാമിനും മുസ്‌ലിംകള്‍ക്കുമെതിരെ മതമൗലികവാദം, മതഭ്രാന്ത്, ഭീകരത, തീവ്രവാദം തുടങ്ങിയ പദങ്ങള്‍ നിരന്തരം നിര്‍ലോഭം പ്രയോഗിച്ചുകൊണ്ടിരിക്കുന്നു. 1993ല്‍ അമേരിക്കന്‍ കോണ്‍ഗ്രസ്സ് അംഗീകരിച്ചു പുറത്തിറക്കിയ കേവലം 93 പുറങ്ങളുള്ള 'പുതിയ ലോക ഇസ്‌ലാമിസ്‌റുകള്‍' എന്ന ഔദ്യോഗിക രേഖയില്‍ 288 പ്രാവശ്യം ഭീകരത, ഭീകരര്‍ എന്നീ പദങ്ങള്‍ പ്രയോഗിച്ചിട്ടുണ്ട്. സയണിസ്‌റ് പ്രസ്ഥാനത്തോട് കൂറുപുലര്‍ത്തുന്ന ഫാന്‍ഫോറെയ്‌സ്‌റ്, യോസഫ് സോദാന്‍സ്‌കി എന്നിവരാണ് പ്രസ്തുത രേഖ തയ്യാറാക്കിയത്.
യഥാര്‍ഥത്തില്‍ ആരാണ് ലോകത്ത് കൂട്ടക്കൊലകളും യുദ്ധങ്ങളും ഭീകരപ്രവര്‍ത്തനങ്ങളും നടത്തുന്നത്? ഒന്നാം ലോകയുദ്ധത്തില്‍ 80 ലക്ഷവും രണ്ടാം ലോകയുദ്ധത്തില്‍ അഞ്ചുകോടിയും വിയറ്റ്‌നാമില്‍ മുപ്പതു ലക്ഷവും കൊല്ലപ്പെട്ടു. പനാമയിലും ഗോട്ടിമലയിലും നിക്കരാഗ്വയിലും കമ്പൂച്ചിയയിലും കൊറിയയിലും ദക്ഷിണാഫ്രിക്കയിലുമെല്ലാം അനേകലക്ഷങ്ങള്‍ അറുകൊല ചെയ്യപ്പെടുകയുണ്ടായി. ഇതിലൊന്നും ഇസ്‌ലാമിന്നോ മുസ്‌ലിംകള്‍ക്കോ ഒരു പങ്കുമില്ല.
ജപ്പാന്‍ യുദ്ധത്തില്‍നിന്ന് പിന്മാറാന്‍ തയ്യാറായിട്ടും ഹിരോഷിമയിലും നാഗസാക്കിയിലും ബോംബ് വര്‍ഷിച്ച കൊടും ഭീകരനായ അമേരിക്കതന്നെയാണ് ഇന്നും ആ പേരിന്നര്‍ഹന്‍. തങ്ങളുടെ കുടില താല്‍പര്യങ്ങള്‍ക്കെതിരു നില്‍ക്കുന്ന എല്ലാ നാടുകളെയും സമൂഹങ്ങളെയും ആ രാജ്യം എതിര്‍ക്കുന്നു. സ്വന്തം വരുതിയില്‍ വരാത്ത നാടുകളിലെല്ലാം ഭീകരപ്രവര്‍ത്തനങ്ങള്‍ സംഘടിപ്പിക്കുന്നു. ചാരസംഘടനയായ സി.ഐ.എയെയും ഭീകരപ്രവര്‍ത്തകരായ സയണിസ്‌റുകളെയും അതിനായി ഉപയോഗിക്കുന്നു. ആഭ്യന്തര പ്രശ്‌നങ്ങള്‍ ഒഴിവാക്കാന്‍ പോലും ലോകത്ത് യുദ്ധങ്ങളുണ്ടാക്കുന്നു. അമേരിക്കന്‍ രാഷ്ട്രീയവൃത്തങ്ങളില്‍ നിറഞ്ഞുനില്‍ക്കുന്ന മിസ്‌റര്‍ റോസ്‌പെറോ പറയുന്നു: 'ആഭ്യന്തരസ്ഥിതി മോശമാകുമ്പോള്‍ ശ്രദ്ധ തിരിച്ചുവിടാനായി ഞങ്ങള്‍ ലോകത്ത് കൊച്ചുകൊച്ചു യുദ്ധങ്ങള്‍ സംഘടിപ്പിക്കുന്നു.''
ഇവ്വിധം കഴിഞ്ഞ ഒരു നൂറ്റാണ്ടിനകം അമേരിക്ക ലോകത്തിലെ അമ്പതിലേറെ രാജ്യങ്ങളില്‍ പ്രത്യക്ഷമായോ പരോക്ഷമായോ ഇടപെട്ട് കുഴപ്പങ്ങള്‍ കുത്തിപ്പൊക്കുകയും കൂട്ടക്കൊലകള്‍ സംഘടിപ്പിക്കുകയുമുണ്ടായി. ഇത്തരമൊരു ഭീകരരാഷ്ട്രത്തിന്റെ പ്രചാരണമാണ് ഇസ്‌ലാമിനെയും മുസ്‌ലിംകളെയും സംബന്ധിച്ച തെറ്റിദ്ധാരണകള്‍ വളര്‍ന്നുവരാനിടവരുത്തിയത്. എല്ലാ വിധ പ്രചാരണോപാധികളും കയ്യടക്കിവെക്കുന്ന അമേരിക്കയുടെ കുടിലതന്ത്രങ്ങള്‍ ലോകജനതയെ തെറ്റിദ്ധരിപ്പിക്കുന്നതില്‍ അനല്പമായ പങ്കുവഹിക്കുകയാണുണ്ടായത്.
യഥാര്‍ഥത്തില്‍, ഇന്ന് ലോകത്തിലെ ഏറ്റവും കൊടിയ ഭീകരകൃത്യം നടത്തുന്നത് സമാധാനത്തിന്റെ സംരക്ഷകരായി അറിയപ്പെടുന്ന ഐക്യരാഷ്ട്രസഭയുടെ സെക്യൂരിറ്റി കൗണ്‍സിലാണ്. ഏതെങ്കിലും ഒരു രാജ്യത്തെ ഭരണാധികാരിയോടുള്ള ശത്രുതയുടെ പേരില്‍ ആ രാജ്യത്തിന്റെ യാത്രാവിമാനം തട്ടിക്കൊണ്ടുപോയി അതിലെ നാനൂറോ അഞ്ഞൂറോ ആളുകളെ ബന്ദികളാക്കിയാല്‍ നാമവരെ തീവ്രവാദികളെന്നും ഭീകരവാദികളെന്നും വിളിക്കും. തീര്‍ച്ചയായും അത് ശരിയുമാണ്. നിരപരാധരായ യാത്രക്കാരെ ശാരീരികമായും മാനസികമായും പീഡിപ്പിക്കുന്നത് ക്രൂരതയാണ്; മനുഷ്യവിരുദ്ധവും. അതുകൊണ്ടുതന്നെ മതവിരുദ്ധവുമാണ്. എന്നാല്‍ സദ്ദാം ഹുസൈന്‍ എന്ന ഒരു ഭരണാധികാരിയോടുള്ള വിരോധത്തിന്റെ പേരില്‍ ഇറാഖിലെ ഒന്നേകാല്‍ കോടി മനുഷ്യരെ കഴിഞ്ഞ പതിനൊന്നുവര്‍ഷം ആഹാരവും മരുന്നും കൊടുക്കാതെ ഐക്യരാഷ്ട്രസഭ ബന്ദികളാക്കുകയും ആറുലക്ഷം കുട്ടികളുള്‍പ്പെടെ പതിനൊന്നു ലക്ഷത്തെ കൊന്നൊടുക്കുകയും ചെയ്തു. ഇവ്വിധം ലോകത്തിലെ ഏറ്റം ക്രൂരനായ കൊലയാളിയും കൊടുംഭീകരനുമായി മാറിയ സെക്യൂരിറ്റി കൗണ്‍സില്‍ ഇപ്പോഴും സമാധാനത്തിന്റെ കാവല്‍ക്കാരായാണ് അറിയപ്പെടുന്നതെന്നത് എത്രമാത്രം വിചിത്രവും വിരോധാഭാസവുമാണ്. അമേരിക്കയുടേതല്ലാത്ത ഒരു മാനദണ്ഡവും അളവുകോലും ലോകത്തിന് ഇന്നില്ല എന്നതാണ് ഇതിനു കാരണം.
മുസ്‌ലിംകളില്‍ തീവ്രവാദികളോ ഭീകരപ്രവര്‍ത്തകരോ ആയി ആരുമില്ലെന്ന് അവകാശപ്പെടാനാവില്ല. ഇസ്‌ലാമിനെ ഏറെ കളങ്കപ്പെടുത്തുകയും അതിന്റെ പ്രതിഛായ തകര്‍ക്കുകയും ചെയ്യുന്ന അപക്വവും വിവേകരഹിതവുമായ അത്തരം പ്രവര്‍ത്തനങ്ങളിലേര്‍പ്പെടുന്ന ഒറ്റപ്പെട്ട ചിലരെ അങ്ങിങ്ങായി കാണാന്‍ കഴിഞ്ഞേക്കും. മുസ്‌ലിം നാടുകളില്‍ അത്തരം ഭീകരപ്രവര്‍ത്തനങ്ങളും തീവ്രവാദചിന്തകളും വളര്‍ന്നുവരാന്‍ കാരണം, അവിടങ്ങളിലെ ഏകാധിപത്യസ്വേഛാധിപത്യ ഭരണകൂടങ്ങളും അവയുടെ കൊടിയ തിന്മകളുമാണ്. ജനാധിപത്യപരവും സമാധാനപരവുമായ മാര്‍ഗങ്ങളിലൂടെ ജനാഭിലാഷങ്ങളെ പ്രതിനിധീകരിക്കുന്ന ഭരണകൂടങ്ങള്‍ സ്ഥാപിക്കാനും വ്യവസ്ഥാമാറ്റത്തിനുമായി കഴിഞ്ഞ നിരവധി പതിറ്റാണ്ടുകളിലായി നടത്തിപ്പോന്ന ശ്രമങ്ങള്‍ സാമ്രാജ്യ ശക്തികളുടെ ഇടപെടല്‍കാരണം പരാജയമടഞ്ഞതിനാല്‍ ക്ഷമകെട്ട ഒരുപറ്റം ചെറുപ്പക്കാര്‍ തീവ്രവാദസമീപനം സ്വീകരിക്കുകയാണുണ്ടായത്. മുസ്‌ലിം ന്യൂനപക്ഷ പ്രദേശങ്ങളില്‍ ഭൂരിപക്ഷത്തോടൊപ്പം ഭരണകൂടവും ചേര്‍ന്ന് നടത്തുന്ന കൊടുംഭീകരവൃത്തികളാണ് ചില ചെറുപ്പക്കാരെ തുല്യനിലയില്‍ പ്രതികരിക്കാന്‍ പ്രേരിപ്പിക്കുന്നത്. തീര്‍ത്തും ഒറ്റപ്പെട്ട ഇത്തരം സംഭവങ്ങളെ പെരുപ്പിച്ചുകാണിച്ചാണ് തല്‍പര കക്ഷികള്‍ ഇസ്‌ലാമിനെതിരെ ഭീകരതയും തീവ്രവാദവും ആരോപിക്കുന്നത്.
ഇസ്‌ലാം എല്ലാവിധ ഭീകരതക്കും എതിരാണ് വ്യക്തികളുടെയും സംഘടിത പ്രസ്ഥാനങ്ങളുടെയും ഭരണകൂടങ്ങളുടെയും ഐക്യരാഷ്ട്രസഭയുടെയും. ഭീകരതയെയും തീവ്രവാദത്തെയും അത് തീര്‍ത്തും നിരാകരിക്കുകയും ശക്തിയായി എതിര്‍ക്കുകയും ചെയ്യുന്നു. നിരപരാധികളുടെ മരണത്തിനും സ്വത്തുനാശത്തിനും ഇടവരുത്തുന്ന ഭീകരപ്രവര്‍ത്തനങ്ങള്‍ മതവിരുദ്ധമാണ്. അകാരണമായി ഒരാളെ വധിക്കുന്നത് മുഴുവന്‍ മനുഷ്യരെയും വധിക്കുന്നതുപോലെയും, ഒരാള്‍ക്കു ജീവനേകുന്നത് മുഴുവന്‍ മനുഷ്യര്‍ക്കും ജീവിതമേകുന്നതുപോലെയുമാണെന്ന് ഖുര്‍ആന്‍ അസന്ദിഗ്ധമായി വ്യക്തമാക്കുന്നു.(5: 32)
അതിനാല്‍ യഥാര്‍ഥ വിശ്വാസികള്‍ക്ക് ഒരിക്കലും തീവ്രവാദികളോ ഭീകരപ്രവര്‍ത്തകരോ ആവുക സാധ്യമല്ല. അതോടൊപ്പം ലോകമെങ്ങും മുസ്‌ലിംകള്‍ ഭീകരപ്രവര്‍ത്തകരാണെന്നത് അമേരിക്കയുടെയും കൂട്ടാളികളുടെയും അവയുടെ മെഗഫോണുകളാകാന്‍ വിധിക്കപ്പെട്ട പൗരസ്ത്യനാടുകളുടെ അടിയാളസമൂഹങ്ങളുടെയും പ്രചാരണം മാത്രമാണെന്ന വസ്തുത വിസ്മരിക്കാവതല്ല.(വിശദമായ പഠനത്തിന് ഐ.പി.എച്ച് പ്രസിദ്ധീകരിച്ച 'ഭീകരവാദവും ഇസ്‌ലാമും', 'ഖുര്‍ആന്റെ യുദ്ധസമീപനം' എന്നീ കൃതികള്‍ കാണുക.)
\chapter{ബലിയുടെ ആത്മാവ് }
  \section{എന്നാലും ഹജ്ജിലും പെരുന്നാളിലും നടത്തുന്ന ബലികര്‍മം ശരിയാണോ? എന്തിനാണിത്ര കൂടുതല്‍ ജീവികളെ കൊല്ലുന്നത്?}
    
ചരിത്രത്തിന്റെ ഇരുളടഞ്ഞ ഇടനാഴികകളില്‍ ഇടക്കിടെ നടന്നുകൊണ്ടിരുന്ന നരബലിയെന്ന അത്യാചാരത്തിന് അറുതിവരുത്തിയ മഹല്‍സംഭവത്തിന്റെ അനുസ്മരണമാണ് ഇസ്‌ലാമിലെ ബലി. ഒപ്പം അസദൃശമായ ആത്മത്യാഗത്തിന്റെ ഓര്‍മപുതുക്കലിലൂടെയുള്ള സമര്‍പ്പണപ്രതിജ്ഞയും.
പ്രായമേറെയായിട്ടും ഇബ്‌റാഹീം പ്രവാചകന്നു സന്താനങ്ങളുണ്ടായില്ല. അതീവ ദുഃഖിതനായ അദ്ദേഹം ദൈവത്തോടു സന്താനലബ്ധിക്കായി മനംനൊന്തു കേണു. നിരന്തര പ്രാര്‍ഥനക്കൊടുവില്‍ പ്രപഞ്ചനാഥന്‍ അദ്ദേഹത്തിന് ഒരു മകനെ പ്രദാനം ചെയ്തു. ഇസ്മാഈല്‍ എന്നു വിളിക്കപ്പെട്ട ആ ഇഷ്ടപുത്രന്‍ കൂടെ നടക്കാറായപ്പോള്‍ അവനെ ബലി നല്‍കണമെന്ന ദൈവശാസനയുണ്ടായി. പിതാവും പുത്രനും ദൈവകല്‍പന പാലിച്ച് ബലിക്കൊരുങ്ങി. അപ്പോള്‍, മകനെ അറുക്കേണ്ടതില്ലെന്നും പകരം മൃഗത്തെ ബലിനല്‍കിയാല്‍ മതിയെന്നും ദൈവശാസനയുണ്ടായി.
'എന്തുകിട്ടു'മെന്ന ചോദ്യമുണര്‍ത്താനാണല്ലോ ഭൗതികജീവിതവീക്ഷണം മനുഷ്യനെ എപ്പോഴും പ്രേരിപ്പിക്കുക. എന്നാല്‍, 'എന്തു നല്‍കാനാവും' എന്ന ചിന്തയും ചോദ്യവുമാണ് മതം എപ്പോഴും വിശ്വാസികളിലുണര്‍ത്തുക. അതിനു 'പ്രയാസപ്പെട്ടതെന്തും' എന്ന് ജീവിതത്തിലൂടെ മറുപടി നല്‍കാനുള്ള പ്രചോദനമാണ് ബലി സൃഷ്ടിക്കുന്നത്.
തനിക്കേറ്റം പ്രിയപ്പെട്ടതുള്‍പ്പെടെ എന്തും ദൈവത്തിനു സമര്‍പ്പിക്കാന്‍ സന്നദ്ധനായ ഇബ്‌റാഹീം പ്രവാചകന്റെ ത്യാഗപൂര്‍ണമായ ഈ പ്രവൃത്തിയുടെ പ്രതീകാത്മകമായ ആവര്‍ത്തനമാണ് ഹജ്ജിലെയും അതിനോടനുബന്ധിച്ച പെരുന്നാളിലെയും ബലി. തനിക്കേറ്റം ഇഷ്ടപ്പെട്ടതുള്‍പ്പെടെ ആവശ്യമായതൊക്കെ നല്‍കാന്‍ ഒരുക്കമാണെന്നതിന്റെ പ്രതിജ്ഞയും പ്രഖ്യാപനവുമാണത്. പണമോ പദവിയോ പ്രതാപമോ പ്രശസ്തിയോ പെണ്ണോ പൊന്നോ കുലമോ കുടുംബമോ അന്തസ്സോ അധികാരമോ ഒന്നും തന്നെ ദൈവഹിതത്തിനെതിരായ ജീവിതത്തിനു കാരണമാവില്ലെന്ന ദൃഢവിശ്വാസത്തിന്റെ പ്രകാശനം കൂടി അതിലുണ്ട്.
സര്‍വപ്രധാനമെന്ന് കരുതുന്നവയുടെ സമര്‍പ്പണമാണല്ലോ ഏവര്‍ക്കും ഏറെ പ്രയാസകരം. താനതിനൊരുക്കമാണെന്ന് വിശ്വാസി ബലിയിലൂടെ വിളംബരം ചെയ്യുന്നു. പ്രത്യക്ഷത്തില്‍ അതൊരു ജീവന്‍ ഹനിക്കലാണ്. എന്നാല്‍, അതിന്റെ ആന്തരാര്‍ഥം അതിമഹത്തരമത്രെ. പ്രപഞ്ചനാഥന്റെ പ്രീതിക്കായി ഏറെ പ്രിയംകരമായതൊക്കെ കൊടുക്കാനും പ്രയാസകരമായത് ചെയ്യാനും തയ്യാറാണെന്ന പ്രതിജ്ഞയും അതുള്‍ക്കൊള്ളുന്നു. അതിനാലാണ് ഖുര്‍ആന്‍ അതിനെ സംബന്ധിച്ച് ഇങ്ങനെ പറഞ്ഞത്: 'അവയുടെ മാംസമോ രക്തമോ ദൈവത്തെ പ്രാപിക്കുന്നില്ല. മറിച്ച്, അവനെ പ്രാപിക്കുന്നത് നിങ്ങളുടെ അര്‍പ്പണബോധമാകുന്നു.''(22: 37)
ഇസ്‌ലാമിലെ ദൈവാരാധനകളേറെയും സമൂഹത്തിനു പൊതുവിലും അവരിലെ അഗതികള്‍ക്കും അശരണര്‍ക്കും വിശേഷിച്ചും ഗുണം ചെയ്യുന്നവയാണ്. ബലിയും അവ്വിധം തന്നെ. അല്ലാഹു ആജ്ഞാപിക്കുന്നു: 'അതില്‍നിന്ന് നിങ്ങള്‍ സ്വയം ഭക്ഷിക്കുക. പ്രയാസപ്പെടുന്ന ആവശ്യക്കാരെ ആഹരിപ്പിക്കുകയും ചെയ്യുക.''(ഖുര്‍ആന്‍ 22: 28)
സന്തോഷത്തിന്റെ സന്ദര്‍ഭങ്ങളിലെല്ലാം ദൈവത്തോടുള്ള നന്ദിപ്രകടനത്തിന്റെ ഭാഗമായി അവന്റെ സൃഷ്ടികളായ മനുഷ്യര്‍ക്ക്; പ്രത്യേകിച്ചും അവരിലെ പാവങ്ങള്‍ക്ക് അന്നദാനം നടത്തുന്നത് നല്ലതാണെന്ന് ഇസ്‌ലാം നിര്‍ദേശിക്കുന്നു. ആഹാരപദാര്‍ഥങ്ങളില്‍ ഏറ്റം പോഷകാംശമുള്ളതും ഉത്തമവും മാംസഭക്ഷണമായതിനാല്‍ അതു നല്‍കുന്നതില്‍ ഉദാരമതികളായ വിശ്വാസികള്‍ നിഷ്‌കര്‍ഷത പുലര്‍ത്തുകയും ചെയ്യുന്നു. ലോകത്തിന്റെ ഏതു ഭാഗത്തെ, ഏതു ജനവിഭാഗത്തിന്റേതായാലും മാംസവിഭവങ്ങളില്ലാത്ത സല്‍ക്കാരത്തളികകള്‍ വളരെ വളരെ വിരളമത്രെ.
\chapter{ബലികര്‍മവും ജീവകാരുണ്യവും }
  \section{കാരുണ്യത്തെക്കുറിച്ച് ഏറെ പറയുന്ന ഇസ്‌ലാം മൃഗങ്ങളോടും മറ്റു ജീവികളോടും കാണിക്കാറുള്ളത് ക്രൂരതയല്ലേ? അവയെ അറുക്കുന്നത് ശരിയാണോ?}
    
ഭൂമിയിലെ എല്ലാറ്റിനോടും കാരുണ്യം കാണിക്കണമെന്ന് ഇസ്‌ലാം കല്പിക്കുന്നു. പ്രവാചകന്‍ പറഞ്ഞു: 'ഭൂമിയിലുള്ളവരോട് കരുണ കാണിക്കുക. ഉപരിലോകത്തുള്ളവന്‍ നിങ്ങളോടും കരുണ കാണിക്കും.''(ത്വബ്‌റാനി)
'കരുണയില്ലാത്തവന് കാരുണ്യം കിട്ടുകയില്ല.''(ബുഖാരി, മുസ്‌ലിം)
'നിര്‍ഭാഗ്യവാനല്ലാതെ കരുണയില്ലാത്തവനാവതേയില്ല.''(അബൂദാവൂദ്)
ഭൂമിയിലെ ജീവികളെയെല്ലാം ഇസ്‌ലാം മനുഷ്യരെപ്പോലുള്ള സമുദായമായാണ് കാണുന്നത്. അല്ലാഹു പറയുന്നു: 'ഭൂമിയിലെ ഏതൊരു ജന്തുവും രണ്ടു ചിറകുകളാല്‍ പറക്കുന്ന ഏതൊരു പക്ഷിയും നിങ്ങളെപ്പോലുള്ള ചില സമുദായങ്ങള്‍ മാത്രമാകുന്നു.''(ഖുര്‍ആന്‍ 6: 38)
മനുഷ്യര്‍ ധിക്കാരികളാകുമ്പോഴും മഴവര്‍ഷിക്കുന്നത് ഇതര ജീവികളെ പരിഗണിച്ചാണെന്ന് പ്രവാചകന്‍ പഠിപ്പിക്കുന്നു. 'ജനം സകാത്ത് നല്കാതിരുന്നാല്‍ മഴ നിലക്കുമായിരുന്നു. ജന്തുക്കള്‍ കാരണമായാണ് എന്നിട്ടും മഴ വര്‍ഷിക്കുന്നത്.''(ഇബ്‌നുമാജ)
ജീവനുള്ള ഏതിനെ സഹായിക്കുന്നതും സേവിക്കുന്നതും പുണ്യകര്‍മമത്രെ. പ്രവാചകന്‍ പറയുന്നു: 'പച്ചക്കരളുള്ള ഏതൊരു ജീവിയുടെ കാര്യത്തിലും നിങ്ങള്‍ക്കു പുണ്യമുണ്ട്.''(ബുഖാരി)
നബിതിരുമേനി അരുള്‍ ചെയ്യുന്നു: 'ഒരാള്‍ ഒരു വഴിയിലൂടെ നടന്നുപോകവേ ദാഹിച്ചുവലഞ്ഞു. അയാള്‍ അവിടെ ഒരു കിണര്‍ കണ്ടു. അതിലിറങ്ങി വെള്ളം കുടിച്ചു. പുറത്തുവന്നപ്പോള്‍ ഒരു നായ ദാഹാധിക്യത്താല്‍ മണ്ണ് കപ്പുന്നതു കണ്ടു. 'ഈ നായക്ക് കഠിനമായ ദാഹമുണ്ട്; എനിക്കുണ്ടായിരുന്നപോലെ!' എന്ന് ആത്മഗതം ചെയ്ത് അയാള്‍ കിണറ്റിലിറങ്ങി. ഷൂവില്‍ വെള്ളം നിറച്ച് വായകൊണ്ട് കടിച്ചുപിടിച്ച് കരക്കുകയറി നായയെ കുടിപ്പിച്ചു. ഇതിന്റെ പേരില്‍ അല്ലാഹു അയാളോട് നന്ദികാണിച്ചു. അയാള്‍ക്കു പൊറുത്തു കൊടുത്തു.'' ഇതുകേട്ട് അവിടത്തെ അനുചരന്മാര്‍ ചോദിച്ചു: മൃഗങ്ങളുടെ കാര്യത്തിലും ഞങ്ങള്‍ക്കു പ്രതിഫലമുണ്ടോ? പ്രവാചകന്‍ പ്രതിവചിച്ചു: പച്ചക്കരളുള്ള എല്ലാറ്റിന്റെ കാര്യത്തിലും നിങ്ങള്‍ക്കു പ്രതിഫലമുണ്ട്.''(ബുഖാരി, മുസ്‌ലിം)
മറ്റൊരു സംഭവം പ്രവാചകന്‍ ഇങ്ങനെ വിവരിക്കുന്നു: 'ഒരു നായ കിണറ്റിനുചുറ്റും ഓടിനടക്കുകയായിരുന്നു. കഠിനമായ ദാഹം കാരണം അതു ചാവാറായിരുന്നു. അതുകണ്ട ഇസ്‌റാഈല്യരില്‍പെട്ട ഒരു വ്യഭിചാരി തന്റെ ഷൂ അഴിച്ച് അതില്‍ വെള്ളമെടുത്ത് അതിനെ കുടിപ്പിക്കുകയും സ്വയം കുടിക്കുകയും ചെയ്തു. അതിന്റെ പേരില്‍ അല്ലാഹു അവര്‍ക്ക് പൊറുത്തുകൊടുത്തു.''(ബുഖാരി)
ഏതു ജീവിയേയും ദ്രോഹിക്കുന്നത് പാപമാകുന്നു. പ്രവാചകനത് ശക്തമായി വിലക്കുന്നു. നബിതിരുമേനി അരുള്‍ചെയ്യുന്നു: 'പൂച്ച കാരണം ഒരു സ്ത്രീ ശിക്ഷിക്കപ്പെട്ടു. അവളതിനെ വിശന്നുചാകുവോളം കെട്ടിയിട്ടു. അങ്ങനെ അവര്‍ നരകാവകാശിയായി.'' (ബുഖാരി, മുസ്‌ലിം)
'ഒരു കുരുവിയെയോ അതിനെക്കാള്‍ ചെറിയ ജീവിയെയോ അന്യായമായി വധിക്കുന്നത് അല്ലാഹുവോട് ഉത്തരം പറയേണ്ട കാര്യമാണ്. 'ന്യായമായ ആവശ്യമെന്തെന്ന്' ചോദിച്ചപ്പോള്‍ അവിടന്ന് പറഞ്ഞു: ഭക്ഷണമുണ്ടാക്കലും ബലിയറുക്കലും കൊന്നശേഷം വെറുതെ ഉപേക്ഷിക്കാതിരിക്കലും.''(അഹ്മദ്)
വിനോദത്തിന് ജീവികളെ കൊല്ലുന്നതും പരസ്പരം പോരടിപ്പിച്ച് മത്സരിപ്പിക്കുന്നതും പ്രവാചകന്‍ നിരോധിച്ചിരിക്കുന്നു. (മുസ്‌ലിം, തിര്‍മുദി)
ഇപ്രകാരം തന്നെ മൃഗങ്ങളെ കല്ലെറിയുന്നതും തേനീച്ച, ഉറുമ്പ്, കുരുവി പോലുള്ളവയെ കൊല്ലുന്നതും അവിടന്ന് വിലക്കിയിരിക്കുന്നു.(മുസ്‌ലിം, അബൂദാവൂദ്)
നബിതിരുമേനി അരുള്‍ ചെയ്യുന്നു: 'ആരെങ്കിലും ഒരു പക്ഷിയെ അനാവശ്യമായി വധിച്ചാല്‍ അന്ത്യദിനത്തില്‍ അത് അലമുറയിട്ടുകൊണ്ട് അല്ലാഹുവോട് പറയും: എന്റെ നാഥാ! ഇന്നയാള്‍ എന്നെ അനാവശ്യമായി കൊന്നിരിക്കുന്നു. ഉപയോഗത്തിനു വേണ്ടിയല്ല അയാളെന്നെ വധിച്ചത്.''(നസാഈ, ഇബ്‌നുഹിബ്ബാന്‍)
തണുപ്പകറ്റാന്‍ തീയിട്ട അനുചരന്മാരോട്, ഉറുമ്പ് കരിയാന്‍ കാരണമാകുമോ എന്ന ആശങ്കയാല്‍ അത് കെടുത്താന്‍ കല്പിക്കുകയും ഒട്ടകത്തെ കെട്ടിയിട്ട് അതിന് ആഹാരം നല്‍കാതെ പട്ടിണിക്കിട്ടവനെ ശക്തമായി ശാസിക്കുകയും മൃഗങ്ങളുടെ മുഖത്ത് മുദ്രവെക്കുന്നതും പുറത്ത് ചൂടുവെക്കുന്നതും വിലക്കുകയും ജന്തുക്കളുടെ പുറംഭാഗം 'ഇരിപ്പിട'മാക്കരുതെന്ന് ആജ്ഞാപിക്കുകയും ചെയ്ത പ്രവാചകന്‍ വൃക്ഷങ്ങളോടുപോലും കരുണ കാണിക്കണമെന്ന് ഉപദേശിക്കുകയുണ്ടായി. മരത്തിനുനേരെ കല്ലെറിഞ്ഞ കുട്ടിയോട് അവിടന്ന് പറഞ്ഞു: 'ഇനിമേല്‍ നീ ഒരു മരത്തേയും കല്ലെറിയരുത്. കല്ലുകൊണ്ടാല്‍ അതിനു വേദനിക്കും.''
ഇവ്വിധം ഈ പ്രപഞ്ചത്തിലെ എല്ലാറ്റിനോടും നിറഞ്ഞ കാരുണ്യത്തോടെ വര്‍ത്തിക്കണമെന്നാണ് ഇസ്‌ലാം ആവശ്യപ്പെടുന്നത്.
ആഹാരമില്ലാതെ ഇവിടെ ഒന്നിനും ജീവിക്കാനാവില്ല. സസ്യങ്ങളും പ്രാണികളും ഇഴജീവികളും ജലജീവികളും കന്നുകാലികളും പറവകളുമെല്ലാം ജീവിക്കുന്നത് ഭക്ഷണം ഉപയോഗിച്ചാണ്. അത് സാധ്യമാവണമെങ്കില്‍ ഓരോന്നിനും മറ്റുള്ളവയെ ഉപയോഗപ്പെടുത്തുകതന്നെ വേണം. സസ്യങ്ങള്‍ നിലനില്‍പിനായി മറ്റു സസ്യങ്ങളെ ഉപയോഗിക്കുന്നു. അപൂര്‍വം ചിലത് ജീവികളെയും ആഹാരമായുപയോഗിക്കാറുണ്ട്. പ്രാണികള്‍ സസ്യങ്ങളെയും മറ്റു ജീവികളെയും ഭക്ഷിക്കുന്നു. വായുവിലും വെള്ളത്തിലും കരയിലും കടലിലുമുള്ള ജീവികളെല്ലാം നിലനില്‍ക്കുന്നത് സസ്യങ്ങളെയും മറ്റു ജീവികളെയും ആഹാരമായുപയോഗിച്ചാണ്. ഇവയില്‍ ഓരോ ജീവിക്കും അതിന്റെ ശരീരഘടനക്കനുസൃതമായ ജീവിതരീതിയാണുള്ളത്. മുയല്‍ സസ്യഭുക്കായതിനാല്‍ അതിനനുസൃതമായ പല്ലും വയറുമാണ് അതിനുള്ളത്. സിംഹം മാംസഭുക്കായതിനാല്‍ അതിന്റെ വായയുടെയും വയറിന്റെയും ഘടന അതിനു ചേരുംവിധമാണ്.
മനുഷ്യന്‍ സസ്യാഹാരവും മാംസാഹാരവും ഉപയോഗിക്കാന്‍ സാധിക്കുംവിധമാണ് സൃഷ്ടിക്കപ്പെട്ടിരിക്കുന്നത്. തീര്‍ത്തും സസ്യഭുക്കുകളായ ആട്, പശു, ചെമ്മരിയാട് തുടങ്ങിയവയുടെ പല്ലുകള്‍ സസ്യാഹാരം മാത്രം കഴിക്കാന്‍ കഴിയുംവിധം പരന്നതും നിരപ്പായതുമാണല്ലോ. പൂര്‍ണമായും മാംസഭുക്കായ കടുവ പോലുള്ളവയുടേത് കൂര്‍ത്തതും മൂര്‍ച്ചയുള്ളതുമത്രേ. എന്നാല്‍, മനുഷ്യനു രണ്ടിനും പറ്റുന്ന പല്ലുകളുണ്ട്. പരന്നതും നിരപ്പായതുമായ പല്ലുകളോടൊപ്പം മൂര്‍ച്ചയുള്ളവയും കൂര്‍ത്തവയുമുണ്ട്. അഥവാ, മനുഷ്യന്‍ സൃഷ്ടിക്കപ്പെട്ടതുതന്നെ മിശ്രഭുക്കായാണ്.
ദഹനേന്ദ്രിയങ്ങളുടെ സ്ഥിതിയും ഇതുതന്നെ. സസ്യഭുക്കുകള്‍ക്ക് സസ്യാഹാരം മാത്രം ദഹിപ്പിക്കാവുന്നതും മാംസഭുക്കുകള്‍ക്ക് അതിനനുസൃതമായതുമാണ് ദഹനേന്ദ്രിയങ്ങളെങ്കില്‍ മനുഷ്യന്റേത് രണ്ടിനെയും ദഹിപ്പിക്കാവുന്ന വിധമുള്ളവയാണ്. പല്ലുകളുടെയും ദഹനവ്യവസ്ഥയുടെയും അവസ്ഥതന്നെ മനുഷ്യന്‍ മിശ്രഭുക്കാണെന്ന് അസന്ദിഗ്ധമായി തെളിയിക്കുന്നു.
ഈ ഭൂമിയും അതിലുള്ളവയുമൊക്കെ മനുഷ്യസമൂഹത്തിനുവേണ്ടി സൃഷ്ടിക്കപ്പെട്ടവയാണ്. അഥവാ, മനുഷ്യനാണ് ഭൂമിയുടെ കേന്ദ്രബിന്ദു. അല്ലാഹു പറയുന്നു: 'ആകാശഭൂമികളിലുള്ളതൊക്കെയും നിങ്ങള്‍ക്ക് അധീനപ്പെടുത്തിത്തന്നത് നിങ്ങള്‍ കാണുന്നില്ലേ? പ്രത്യക്ഷവും പരോക്ഷവുമായ അനുഗ്രഹങ്ങള്‍ അവന്‍ നിങ്ങള്‍ക്ക് നിറവേറ്റിത്തരികയും ചെയ്തിരിക്കുന്നു.''(ഖുര്‍ആന്‍ 31: 20)
'കാലികളില്‍നിന്ന് ഭാരം ചുമക്കുന്നവയും അറുത്തു ഭക്ഷിക്കാനുള്ളവയും അവന്‍ സൃഷ്ടിച്ചിരിക്കുന്നു. നിങ്ങള്‍ക്ക് അല്ലാഹു നല്കിയതില്‍നിന്ന് ഭക്ഷിച്ചുകൊള്ളുക. പിശാചിന്റെ കാല്‍പാടുകളെ പിന്തുടരരുത്. അവന്‍ നിങ്ങളുടെ പ്രത്യക്ഷശത്രുവാകുന്നു.''(ഖുര്‍ആന്‍ 6: 142)
'ഉറപ്പായും നിങ്ങള്‍ക്ക് കന്നുകാലികളില്‍ ഗുണപാഠമുണ്ട്. അവയുടെ ഉദരത്തില്‍നിന്നുള്ളതില്‍നിന്ന് നിങ്ങള്‍ക്കു നാം കുടിക്കാന്‍ തരുന്നു. നിങ്ങള്‍ക്ക് അവയില്‍ ധാരാളം പ്രയോജനങ്ങളുണ്ട്. അവയില്‍നിന്ന് നിങ്ങള്‍ ഭക്ഷിക്കുകയും ചെയ്യുന്നു.''(ഖുര്‍ആന്‍ 23: 21)
'നിങ്ങള്‍ക്കു പുതുമാംസം എടുത്തു ഭക്ഷിക്കാനും നിങ്ങള്‍ക്കണിയാനുള്ള ആഭരണം പുറത്തെടുക്കാനും പാകത്തില്‍ കടലിനെ നിങ്ങള്‍ക്ക് അധീനപ്പെടുത്തിത്തന്നതും അല്ലാഹുവാകുന്നു.''(ഖുര്‍ആന്‍ 16: 14)
ഭൂമിയിലുള്ളതൊക്കെയും മനുഷ്യനുവേണ്ടി സംവിധാനിക്കപ്പെട്ടതാണെന്ന തത്വത്തെ നിരാകരിക്കുന്നവരും പ്രയോഗത്തില്‍ അതിനനുസൃതമായ സമീപനമാണ് സ്വീകരിക്കാറുള്ളത്. മനുഷ്യന്‍ തങ്ങളുടെ താല്‍പര്യങ്ങള്‍ സംരക്ഷിക്കാന്‍ ഭൂമിയെ ഉഴുതുമറിക്കുന്നു. അതില്‍ കിണറുകളും കുളങ്ങളും കുഴിക്കുന്നു. റോഡുകളും പാലങ്ങളും നിര്‍മിക്കുന്നു. വീടുകള്‍ ഉണ്ടാക്കുന്നു. ഇതൊക്കെ ചെയ്യുമ്പോള്‍ അവിടെയുള്ള പ്രാണികള്‍ക്കും ഇതര ജീവികള്‍ക്കും എന്തു സംഭവിക്കുന്നുവെന്ന് പരിഗണിക്കാറില്ല. എല്ലാ ജീവികള്‍ക്കും ഒന്നുപോലെ അവകാശപ്പെട്ട ഭൂമിയിലാണ് താന്‍ റോഡും കിണറുമൊക്കെ ഉണ്ടാക്കുന്നതെന്നോര്‍ക്കാറില്ല. ഇപ്രകാരം തന്നെ സസ്യങ്ങളെയും ഫലവൃക്ഷങ്ങളെയും വിളകളെയുമെല്ലാം മനുഷ്യന്‍ തന്റെ താല്‍പര്യത്തിനായുപയോഗിക്കുന്നു. അതിനാല്‍ ഭൂമിയും അതിലുള്ളവയും മനുഷ്യനുവേണ്ടി സജ്ജമാക്കപ്പെട്ടതാണെന്ന സത്യത്തെ പ്രയോഗതലത്തില്‍ അംഗീകരിക്കാത്ത ആരുമില്ല.
ഒരു ജീവിയെയും കൊല്ലുകയില്ല എന്നതാണ് അഹിംസകൊണ്ട് ഉദ്ദേശിക്കപ്പെടുന്നതെങ്കിലും അതിനനുസൃതമായി ജീവിതം നയിക്കുന്ന ആരും ഈ ഭൂമിയിലില്ല. മാംസാഹാരം കഴിക്കാത്തവര്‍ സസ്യാഹാരം ഭക്ഷിക്കുന്നവരാണല്ലോ. സസ്യങ്ങള്‍ക്ക് ജീവനും വികാരവുമുണ്ടെന്നത് സുസമ്മതസത്യമത്രെ. അതിനാല്‍ മാംസഭുക്കുകളെപ്പോലെ, സസ്യഭുക്കുകളും ജീവഹാനിവരുത്തുന്നവരും സസ്യങ്ങളെ വേദനിപ്പിക്കുന്നവരുമാണ്.
ഏതു മനുഷ്യനും ശരീരത്തില്‍ മുറിവുപറ്റിയാല്‍ അതിലെ വിഷാണുക്കളെ മരുന്നുപയോഗിച്ചു കൊല്ലുന്നു. ഉദരത്തിലെ കൃമികളെ നശിപ്പിക്കുന്നു. കൊതുകുകള്‍ മുട്ടയിട്ടുവിരിയുന്ന കെട്ടിനില്‍ക്കുന്ന മലിനജലത്തില്‍ വിഷം തളിച്ച് അവയെ കൊല്ലുന്നു. മൂട്ടയെയും കൊതുകിനെയും നശിപ്പിക്കുന്നു. ചുരുക്കത്തില്‍, ഏതെങ്കിലും ജീവിയെ ഹനിക്കാതെ ആരും എവിടെയുമില്ല; ഉണ്ടാവുക സാധ്യവുമല്ല.
മനുഷ്യസമൂഹത്തിന്റെ സുഗമമായ നിലനില്‍പിനായി വിഷാണുക്കളെ കൊല്ലാമെങ്കില്‍ അതേ കാര്യത്തില്‍ ഏറ്റം നല്ല പോഷകാഹാരമെന്ന നിലയില്‍ മാംസം ഉപയോഗിക്കാവുന്നതാണ്. പ്രോട്ടീന്‍, ഇരുമ്പ്, വിറ്റാമിന്‍ ബി. 1, നിയാസിന്‍ തുടങ്ങിയവ മാംസാഹാരത്തില്‍ ധാരാളമായി ഉണ്ടെന്നത് അനിഷേധ്യമാണ്. സസ്യങ്ങളെയും പ്രാണികളെയും അണുക്കളെയും മറ്റും തങ്ങളുടെ നിലനില്‍പിന്നായി കൊല്ലാമെന്ന് തീരുമാനിച്ചവര്‍ ജനകോടികള്‍ക്ക് ആഹാരമായി മാറുന്ന മാംസം ഉപേക്ഷിക്കണമെന്ന് പറയുന്നത് തീര്‍ത്തും നിരര്‍ഥകമത്രെ.
അതുകൊണ്ടുതന്നെ, ജീവജാലങ്ങളോട് പരമാവധി കാരുണ്യം കാണിക്കാന്‍ കല്‍പിക്കുന്ന ഇസ്‌ലാം അവയുടെ മാംസം ഭക്ഷിക്കുന്നത് അനുവദനീയമാക്കി. ലോകത്തിലെ കോടിക്കണക്കിന് ജനങ്ങള്‍ ജീവിതം നയിക്കുന്നത് മാംസാഹാരം ഉപയോഗിച്ചാണ്. അത് വിലക്കുന്നത് സാമൂഹ്യദ്രോഹവും ജനവിരുദ്ധവുമാണ്.
മാംസം അനുവദനീയമാകാന്‍ ജീവികളെ ദൈവനാമമുച്ചരിച്ച് അറുക്കണമെന്ന് ഇസ്‌ലാം നിഷ്‌കര്‍ഷിക്കുന്നു. ദൈവത്തിന്റെ അനുഗ്രഹവും കാരുണ്യവും അനുസരിച്ചുകൊണ്ടായിരിക്കണം മറ്റെല്ലാ കര്‍മങ്ങളുമെന്നപോലെ അവന്റെ സൃഷ്ടിയായ ജീവിയെ അനിവാര്യമായ ആവശ്യത്തിന് ഉപയോഗപ്പെടുത്തേണ്ടതെന്ന് ഇതു പഠിപ്പിക്കുന്നു. അതോടൊപ്പം ഉരുവിന് പരമാവധി സൌകര്യം നല്‍കിയും പ്രയാസം ലഘൂകരിച്ചുമായിരിക്കണം അറവെന്ന് കണിശമായി കല്പിക്കുകയും ചെയ്യുന്നു. പ്രവാചകന്‍ പറയുന്നു: 'അല്ലാഹു എല്ലാ കാര്യങ്ങളിലും നന്മ നിശ്ചയിച്ചിരിക്കുന്നു. അതിനാല്‍ നിങ്ങള്‍ കൊല്ലുന്നുവെങ്കില്‍ നല്ല നിലയിലത് നിര്‍വഹിക്കുക. അറുക്കുന്നുവെങ്കില്‍ അതും നല്ലനിലയിലാക്കുക. കത്തിയുടെ വായ്ത്തല മൂര്‍ച്ചകൂട്ടി ഉരുവിന് സൌകര്യം ചെയ്യുക.''(മുസ്‌ലിം)
ഒരിക്കല്‍ ഒരാള്‍ നബിതിരുമേനിയോടു പറഞ്ഞു: 'ഞാന്‍ ആടിനെ അറുക്കുമ്പോള്‍ ദയ കാണിക്കാറുണ്ട്.'' ഇതുകേട്ട് പ്രവാചകന്‍ പ്രതിവചിച്ചു: 'നീ അതിനോടു കരുണ കാണിച്ചാല്‍ അല്ലാഹു നിന്നോടും കരുണ കാണിക്കും.''(ഹാകിം)
ഒരാള്‍ അറുക്കാനുള്ള ആടിനെ അതിന്റെ കാലു പിടിച്ചുവലിച്ചു കൊണ്ടുപോവുന്നതു കാണാനിടയായ ഉമറുല്‍ ഫാറൂഖ് പറഞ്ഞു: 'നിനക്കു നാശം! അതിനെ നല്ലനിലയില്‍ മരണത്തിലേക്കു നയിക്കുക.''
ഇസ്‌ലാമിനെപ്പോലെ ഹൈന്ദവ ധര്‍മത്തിലും മാംസാഹാരം അനുവദനീയമാണ്. അത് മാംസഭക്ഷണം വിലക്കുന്നുവെന്ന ധാരണ അബദ്ധമാണ്. പല മഹര്‍ഷിമാരും പുണ്യവാളന്മാരും മാംസാഹാരം കഴിക്കുന്നവരായിരുന്നുവെന്ന് പുരാണങ്ങള്‍ വ്യക്തമാക്കുന്നു. ശ്രീരാമന്‍ വനവാസത്തിന് അയക്കപ്പെട്ടപ്പോള്‍, എനിക്കു രുചികരമായ മാംസത്തളിക ഉപേക്ഷിക്കേണ്ടിവരുമെന്ന് മാതാവിനോട് പറഞ്ഞിരുന്നതായി അയോധ്യാകാണ്ഡത്തിലെ 20, 26, 94 ശ്‌ളോകങ്ങളില്‍ പരാമര്‍ശിച്ചിട്ടുണ്ട്. ഇത് ശ്രീരാമന് മാംസഭക്ഷണത്തോട് പ്രിയമുണ്ടായിരുന്നതായി വ്യക്തമാക്കുന്നു.
ബൃഹദാരണ്യകോപനിഷത്ത് പറയുന്നു: 'എനിക്കു പണ്ഡിതനും പ്രസിദ്ധനും സഭകളില്‍ പോകുന്നവനും മറ്റുള്ളവര്‍ കേള്‍ക്കാന്‍ ആഗ്രഹിക്കുന്ന വാക്കുകള്‍ പറയുന്നവനുമായ പുത്രനുണ്ടാവണം, അവന്‍ എല്ലാ വേദങ്ങളും പഠിക്കണം, ആറ് വര്‍ഷം ജീവിക്കണം എന്ന് ആഗ്രഹിക്കുന്നുണ്ടെങ്കില്‍ മാംസത്തോടുകൂടിയ ഭക്ഷണം പാകം ചെയ്ത് നെയ്യോടുകൂടി രണ്ടുപേരും കഴിക്കണം. അങ്ങനെയുള്ള പുത്രനെ ജനിപ്പിക്കാന്‍ അവര്‍ ശക്തരാവും. മാംസം ഉക്ഷത്തിന്റെയോ ഋഷഭത്തിന്റെയോ ആകാം.''(6418) 
\chapter{സ്ത്രീക്ക് ഭരണരാഷ്ട്രീയ പങ്കാളിത്തം പാടില്ലേ? }
\section{ സ്ത്രീകള്‍ ഭരണരാഷ്ട്രീയ രംഗങ്ങളില്‍ പങ്കുവഹിക്കുന്നതിനെ ഇസ്‌ലാം വിലക്കുന്നുണ്ടോ? മുസ്‌ലിം സ്ത്രീകള്‍ ഭരണാധികാരികളാകാന്‍ പാടില്ലെന്ന് പറഞ്ഞുകേള്‍ക്കുന്നത് ശരിയാണോ?}
    
 ഈ ചോദ്യത്തിന് ഇരുപതാം നൂറ്റാണ്ട് കണ്ട പ്രമുഖ പണ്ഡിതന്മാരിലൊരാളായ ശൈഖ് മുഹമ്മദുല്‍ ഗസ്സാലി നല്‍കിയ വിശദീകരണം ഇവിടെ ഉദ്ധരിക്കുന്നത് ഉചിതമായിരിക്കുമെന്ന് കരുതുന്നു. അദ്ദേഹം എഴുതുന്നു: 'മദീനാ മാര്‍ക്കറ്റിന്റെ നിയന്ത്രണവും വിധിത്തീര്‍പ്പും ശിഫാഇനെയാണ് ഉമറുല്‍ ഫാറൂഖ് ഏല്‍പിച്ചിരുന്നത്. അവരവിടെ നിയമലംഘനങ്ങള്‍ തടയുകയും നീതി നടപ്പാക്കുകയും വിധി പ്രസ്താവിക്കുകയും ചെയ്തത് ആണ്‍പെണ്‍ഭേദമന്യേ എല്ലാവര്‍ക്കുമിടയിലായിരുന്നു. ഏതെങ്കിലും വിധത്തിലുള്ള നിയന്ത്രണങ്ങള്‍ ഖലീഫ അവരുടെ മേല്‍ ഏര്‍പ്പെടുത്തിയിരുന്നില്ല.
സ്ത്രീകളെ രാഷ്ട്രനേതൃത്വമോ ഭരണാധികാരമോ ഏല്‍പിച്ചുകൊടുക്കാന്‍ മോഹിച്ചു നടക്കുന്ന കൂട്ടത്തിലൊന്നുമല്ല ഞാന്‍. ഒരു കാര്യമേ നാം ആഗ്രഹിക്കുന്നുള്ളൂ. സമൂഹത്തിലെ ഏറ്റവും അനുയോജ്യനായ ആള്‍ രാജ്യത്തിന്റെയും രാജ്യഭരണത്തിന്റെയും നേതൃത്വത്തില്‍ വരണം.
'സ്ത്രീയെ അധികാരമേല്‍പിച്ച ജനത പരാജയപ്പെട്ടിരിക്കുന്നു' എന്ന പ്രവാചകവചനമുണ്ടല്ലോ. അപ്പോള്‍ ഏതെങ്കിലും വിധത്തിലുള്ള അധികാരം സ്ത്രീയെ ഏല്‍പിക്കുന്നത് പരാജയകാരണമാവില്ലേയെന്ന് ചോദിച്ചേക്കാം. ഇവിടെ ഈ പ്രവാചകവചനത്തെ സംബന്ധിച്ച് അല്‍പം ആഴത്തില്‍ ആലോചിക്കാന്‍ നാമാഗ്രഹിക്കുന്നു. പ്രവാചകവചനം സ്വീകാര്യം തന്നെ. എന്നാല്‍ അതിന്റെ ആശയം എന്തായിരിക്കും?
ഇസ്‌ലാമിക മുന്നേറ്റത്തിനു മുമ്പില്‍ പേര്‍ഷ്യന്‍ സാമ്രാജ്യം നിലംപൊത്തിക്കൊണ്ടിരുന്നപ്പോള്‍ അവിടെ ഭരണം നടത്തിയിരുന്നത് അഭിശപ്തമായ രാജവാഴ്ചയും ഏകാധിപത്യവുമായിരുന്നു. വിഗ്രഹാരാധനയിലധിഷ്ഠിതമായ മതം, കൂടിയാലോചന അചിന്ത്യമായ രാജകുടുംബം, മരണം വിധിക്കപ്പെട്ട അഭിപ്രായ വിമര്‍ശനസ്വാതന്ത്യ്രം, തമ്മിലടിക്കുന്ന രാജകുടുംബങ്ങള്‍, മകന്‍ പിതാവിനെയും സഹോദരന്‍ സഹോദരനെയും വകവരുത്തുന്ന അധികാരക്കൊതി, സര്‍വോപരി എല്ലാം സഹിച്ച് അടങ്ങിയൊതുങ്ങി കഴിയുന്ന പ്രതികരണശേഷി അറിയാത്ത ജനത!
മുസ്‌ലിം മുന്നേറ്റത്തിനു മുമ്പില്‍ പേര്‍ഷ്യന്‍ സൈന്യം പിടിച്ചുനില്‍ക്കാനാവാതെ പിന്തിരിയുകയും രാജ്യാതിര്‍ത്തി ചുരുങ്ങിച്ചുരുങ്ങി വരികയുമായിരുന്നു. അപ്പോഴുമവര്‍ക്ക് പ്രാപ്തനായ ഒരാളെ രാജ്യഭരണം ഏല്‍പിക്കാനായില്ല. രാജഭരണത്തിന്റെ ഭാഗമായി അവിവേകിയായ ഒരു സ്ത്രീയെ അധികാരത്തില്‍ വാഴിക്കുകയായിരുന്നു. ഇത് ആ രാജ്യത്തിന്റെയും സാമ്രാജ്യത്തിന്റെയും എന്നന്നേക്കുമായുള്ള തിരോധാനം വിളിച്ചറിയിക്കുന്നതായിരുന്നു. ദീര്‍ഘവീക്ഷകനും സൂക്ഷ്മജ്ഞാനിയുമായ പ്രവാചകന്‍ ഇതേ പ്പറ്റിയുള്ള തന്റെ വിലയിരുത്തല്‍ സത്യസന്ധമായി രേഖപ്പെടുത്തുകയായിരുന്നു. 'സ്ത്രീയെ അധികാരമേല്‍പിച്ച ജനത പരാജയമടഞ്ഞതുതന്നെ.'
നേരെ മറിച്ച് ആ അവസരം പേര്‍ഷ്യന്‍ഭരണം കൂടിയാലോചനയിലധിഷ്ഠിതവും ഭരണാധികാരിയായ വനിത ഗോള്‍ഡാമീറിനെപ്പോലൊരാളാവുകയും സൈനിക തീരുമാനങ്ങള്‍ ഉത്തരവാദപ്പെട്ടവരുടെ കരങ്ങളില്‍ ആയിരിക്കുകയും ചെയ്തിരുന്നുവെങ്കില്‍ പ്രവാചകന്റെ വിലയിരുത്തല്‍ മറ്റൊരു വിധത്തിലായിരുന്നേനെ.''
ഉപര്യുക്ത പ്രവാചകവചനത്തെ സംബന്ധിച്ച് ഡോ. ജമാല്‍ ബദവി എഴുതുന്നു: 'ഈ നബിവചനം സ്ത്രീകളെ ഭരണനേതൃത്വത്തില്‍നിന്നും മാറ്റിനിര്‍ത്താനുള്ള തെളിവായി വ്യാഖ്യാനിക്കപ്പെട്ടുവരുന്നുണ്ടെങ്കിലും പല പണ്ഡിതന്‍മാരും ഇതിനോട് യോജിക്കുന്നില്ല. നബിയുടെ കാലത്തെ പേര്‍ഷ്യന്‍ ഭരണാധികാരികള്‍ പ്രവാചകനോടും അദ്ദേഹം അവരുടെ അടുത്തേക്കയച്ച ദൂതനോടും കൊടിയ ശത്രുത കാണിച്ചവരായിരുന്നു. അതിനാല്‍ പേര്‍ഷ്യക്കാര്‍ ഖുസ്രുവിന്റെ പുത്രിയെ തങ്ങളുടെ ഭരണാധികാരിയായി അംഗീകരിച്ച വാര്‍ത്തയോടുള്ള പ്രതികരണത്തെ, രാഷ്ട്രത്തിന്റെ ഭരണനേതൃത്വവുമായി ബന്ധപ്പെട്ട ലിംഗപ്രശ്‌നത്തിന്റെ വിശദീകരണമായല്ല, ആ മര്‍ദകസാമ്രാജ്യത്തിന്റെ ആസന്നപതനത്തെ സംബന്ധിച്ച പ്രവചനമെന്ന നിലയിലാണ് കാണേണ്ടത്. നബിയുടെ പ്രവചനം പിന്നീട് പുലരുകയും ചെയ്തു... അതിനാല്‍ ഈ നബിവചനം സ്ത്രീകളെ രാഷ്ട്രത്തിന്റെ ഭരണനേതൃത്വത്തില്‍നിന്നും ഒഴിച്ചുനിര്‍ത്തിയേ പറ്റൂ എന്നുള്ളതിന് തെളിവാകുന്നില്ല.''
അതിനാല്‍ സ്ത്രീയെ ഭരണനേതൃത്വത്തില്‍നിന്ന് വിലക്കുന്ന വ്യക്തവും ഖണ്ഡിതവുമായ ഖുര്‍ആന്‍ വാക്യമോ പ്രവാചക വചനമോ ഇല്ല. എങ്കിലും വളരെ അനിവാര്യ സാഹചര്യമില്ലെങ്കില്‍ അതൊഴിവാക്കുന്നതാണുത്തമം. ഡോ. ജമാല്‍ ബദവി എഴുതുന്നു: 'ഇസ്‌ലാമിന്റെ രാഷ്ട്രീയ വ്യവസ്ഥ വിശദീകരിക്കുന്നതില്‍ പ്രസിദ്ധിയാര്‍ജിച്ച പ്രമുഖ നിയമപണ്ഡിതന്‍ അബൂയഅ്‌ലാ രാഷ്ട്രത്തലവന്റെ യോഗ്യതകളില്‍ 'പുരുഷനായിരിക്കുക' എന്നൊരു വ്യവസ്ഥ ഉള്‍പ്പെടുത്തിയിട്ടില്ലെന്ന് അല്‍ഖാസിമി നിരീക്ഷിക്കുന്നു. ഇവിടെ പക്ഷേ, ഒരു കാര്യം പ്രത്യേകം ശ്രദ്ധിക്കേണ്ടതുണ്ട്. ഇസ്‌ലാമിലെ രാഷ്ട്രത്തലവന്‍ വെറുമൊരു ചടങ്ങു തലവനല്ല. അദ്ദേഹം നമസ്‌കാരത്തിനു നേതൃത്വം കൊടുക്കുന്നു. ചിലപ്പോള്‍ നിരന്തരം യാത്രചെയ്യുന്നു. ഇതര രാഷ്ട്രത്തലവന്‍മാരുമായി അവര്‍ പലപ്പോഴും പുരുഷന്‍മാരായിരിക്കും കൂടിയാലോചന നടത്തുന്നു; പലപ്പോഴും രഹസ്യ സംഭാഷണങ്ങളും. സ്ത്രീകള്‍ക്ക് ഇത്തരം ബന്ധങ്ങളും ബാധ്യതകളും ദുര്‍വഹമായിരിക്കുമെന്നതില്‍ സംശയമില്ല. മാത്രമല്ല, സ്ത്രീപുരുഷന്‍മാര്‍ക്കിടയിലെ ശരിയായ പരസ്പരബന്ധങ്ങളെ സംബന്ധിച്ച് ഇസ്‌ലാമിക മാര്‍ഗ നിര്‍ദേശങ്ങളോട് അവ പൊരുത്തപ്പെടുകയില്ല.''
ഇത് രാഷ്ട്രത്തിന്റെ പരമോന്നത പദവിയെ സംബന്ധിച്ചു മാത്രം ബാധകമാകുന്ന കാര്യമാണ്. സ്ത്രീ അതിനുതാഴെയുള്ള സ്ഥാനങ്ങള്‍ വഹിക്കുന്നതിനെയോ സജീവ രാഷ്ട്രീയത്തില്‍ ഇടപെടുന്നതിനെയോ ഇസ്‌ലാംവിലക്കുന്നില്ല. ഇറാന്‍ ഇസ്‌ലാമിക് റിപ്പബ്‌ളിക്കിലെ വൈസ് പ്രസിഡന്റുമാരിലൊരാളും അവിടത്തെ എം.പി.മാരില്‍ പതിനാലു പേരും വനിതകളാണ്. ഇതര മുസ്‌ലിം നാടുകളുടെ സ്ഥിതിയും ഭിന്നമല്ല.
എങ്കിലും സ്ത്രീകളുടെ പ്രഥമവും പ്രധാനവുമായ ബാധ്യത വീടിന്റെ ഭരണവും മാതൃത്വപരമായ ഉത്തരവാദിത്തങ്ങളുടെ ശരിയായ നിര്‍വഹണവുമാണെന്ന കാര്യം വിസ്മരിക്കാവതല്ല. വരുംതലമുറകളെ യഥാവിധി വാര്‍ത്തും വളര്‍ത്തിയുമെടുക്കുകയെന്നതിനെ ഇസ്‌ലാം ഒട്ടും നിസ്സാരമായിക്കാണുന്നില്ലെന്നു മാത്രമല്ല, അതിപ്രധാന കൃത്യമായി പരിഗണിക്കുകയും ചെയ്യുന്നു. മാതൃത്വം ഭൂമിയില്‍ ഏറ്റവും ആദരണീയവും മഹിതവുമാവാനുള്ള കാരണവും അതുതന്നെ.
\chapter{പൊതുജീവിതത്തിലെ സ്ത്രീ പങ്കാളിത്തം }
 \section{ സ്ത്രീകളെ പൊതുജീവിതത്തില്‍നിന്ന് മാറ്റിനിര്‍ത്തുകയും അടുക്കളയില്‍ തളച്ചിടുകയുമല്ലേ ഇസ്‌ലാം ചെയ്യുന്നത്?}
   
പ്രകൃതിപരമായ പ്രത്യേകതകള്‍ പരിഗണിക്കുമ്പോള്‍ സ്ത്രീയുടെ പ്രധാന പ്രവര്‍ത്തനരംഗം വീടുതന്നെയാണ്. മഹിതമായ കൃത്യം മാതൃത്വവും. എന്നാല്‍ സ്ത്രീ പൊതുജീവിതത്തില്‍ ഇടപെടുന്നതിനെയോ സജീവപങ്കാളിത്തം വഹിക്കുന്നതിനെയോ ഇസ്‌ലാം വിലക്കുന്നില്ല. എന്നല്ല; അതനുവദിക്കുകയും അനിവാര്യസന്ദര്‍ഭങ്ങളില്‍ പ്രോത്സാഹിപ്പിക്കുകയും ചെയ്യുന്നു.
അധ്യയനത്തിലും അധ്യാപനത്തിലും പ്രവാചകന്റെ കാലംതൊട്ടുതന്നെ സ്ത്രീകളും പുരുഷന്മാരെപ്പോലെ സജീവമായി പങ്കെടുത്തുപോന്നിട്ടുണ്ട്. പ്രവാചകസന്നിധിയില്‍ വന്ന് കാര്യങ്ങള്‍ പഠിച്ചു മനസ്സിലാക്കുന്നതിലും അത് മറ്റുള്ളവരെ പഠിപ്പിക്കുന്നതിലും അവരൊട്ടും പിന്നിലായിരുന്നില്ല. പ്രവാചകചര്യയുടെ നിവേദകരില്‍ പ്രമുഖരായ വനിതകള്‍ ഉണ്ടാവാനുള്ള കാരണവും അതത്രെ.
പ്രവാചകപത്‌നി ആഇശയുടെ പാണ്ഡിത്യം സുവിദിതമാണ്. ഇമാം സുഹ്രി പറയുന്നു: 'ആഇശ ജനങ്ങളില്‍ ഏറ്റവും അറിവുള്ള വ്യക്തിയായിരുന്നു. പ്രവാചകന്റെ അനുചര•ാരില്‍ പ്രമുഖര്‍ പോലും അവരോട് ചോദിച്ച് പഠിക്കാറുണ്ടായിരുന്നു''. സുബൈറിന്റെ മകന്‍ ഉര്‍വ രേഖപ്പെടുത്തുന്നു: 'ഖുര്‍ആന്‍, അനന്തരാവകാശ നിയമങ്ങള്‍, വിജ്ഞാനം, കവിത, കര്‍മശാസ്ത്രം, അനുവദനീയം, നിഷിദ്ധം, വൈദ്യം, അറബികളുടെ പുരാതന വൃത്താന്തങ്ങള്‍, ഗോത്രചരിത്രം എന്നിവയില്‍ ആഇശയേക്കാള്‍ അറിവുള്ള ആരെയും ഞാന്‍ കണ്ടിട്ടില്ല''.
ലബീദിന്റെ മകന്‍ മഹ്മൂദ് പറയുന്നു: 'പ്രവാചകപത്‌നിമാരെല്ലാം ഹദീസുകള്‍ മനഃപാഠമാക്കിയിരുന്നു. എന്നാല്‍ ആഇശയോടും ഉമ്മുസല്‍മയോടുമൊപ്പമെത്തിയിരുന്നില്ല മറ്റുള്ളവര്‍.''
പ്രവാചകപത്‌നിമാരില്‍ ആഇശ മാത്രം 2210 ഹദീസുകള്‍ നിവേദനം ചെയ്തിട്ടുണ്ട്. ഉമ്മുസല്‍മയും നിരവധി ഹദീസുകള്‍ നിവേദനം ചെയ്യുകയുണ്ടായി. സ്ത്രീകള്‍ മാത്രമല്ല, ധാരാളം പുരുഷന്മാരും അവരില്‍നിന്ന് അറിവ് നേടിയിരുന്നു. വൈജ്ഞാനികരംഗത്തെന്നപോലെ ഇസ്‌ലാമികപ്രബോധന പ്രവര്‍ത്തനങ്ങളിലും സ്ത്രീകള്‍ സജീവ പങ്കുവഹിച്ചു. അതിനാല്‍ പുരുഷന്മാരെപ്പോലെത്തന്നെ അവരും കൊടിയ പീഡനങ്ങള്‍ക്കിരയായി. ഇസ്‌ലാമിലെ ആദ്യത്തെ രക്തസാക്ഷിപോലും സുമയ്യ എന്ന സ്ത്രീയാണ്. ജന്മനാട്ടില്‍ ജീവിതം ദുസ്സഹമായി പലായനം അനിവാര്യമായപ്പോള്‍ സ്ത്രീകളുമതില്‍ പങ്കാളികളായി.
പ്രവാചകന്റെയോ സച്ചരിതരായ ഖലീഫമാരുടെയോ കാലത്ത് പൊതുജീവിതത്തില്‍നിന്ന് സ്ത്രീകള്‍ മാറ്റിനിര്‍ത്തപ്പെട്ടിരുന്നില്ല. യുദ്ധരംഗത്തുപോലും സ്ത്രീകളുടെ സജീവ സാന്നിധ്യമുണ്ടായിരുന്നു. ഉഹ്ദ് യുദ്ധത്തില്‍ ഭടന്മാര്‍ക്ക് വെള്ളമെത്തിക്കാനും മുറിവേറ്റവരെ ശുശ്രൂഷിക്കാനും നേതൃത്വം നല്‍കിയത് പ്രവാചകപത്‌നി ആഇശയായിരുന്നു. ഉമ്മു സുലൈമും ഉമ്മു സലീത്തും ഈ സാഹസത്തില്‍ പങ്കുചേരുകയുണ്ടായി.
ഖൈബര്‍ യുദ്ധത്തില്‍ പങ്കെടുത്ത പട്ടാളക്കാര്‍ക്ക് ആഹാരമൊരുക്കിക്കൊടുക്കുകയും മുറിവേറ്റവരെ ശുശ്രൂഷിക്കുകയും ചെയ്ത വനിതകള്‍ക്ക് പ്രവാചകന്‍ സമരാര്‍ജിത സമ്പത്തില്‍നിന്ന് വിഹിതം നല്‍കുകയുണ്ടായി. ഉഹ്ദ് യുദ്ധത്തില്‍ മുറിവേറ്റവരെയും രക്തസാക്ഷികളെയും മദീനയിലേക്കെത്തിക്കുന്ന ചുമതല നിര്‍വഹിച്ചത് മുഅവ്വിദിന്റെ പുത്രി റുബയ്യഉം സഹപ്രവര്‍ത്തകരുമായിരുന്നു. ഉമ്മു അതിയ്യ ഏഴു യുദ്ധങ്ങളില്‍ സംബന്ധിക്കുകയുണ്ടായി. അനസുബ്‌നു മാലിക്കിന്റെ മാതാവ് ഉമ്മു സുലൈമും നിരവധി യുദ്ധങ്ങളില്‍ പ്രവാചകനെ അനുഗമിക്കുകയുണ്ടായി. ഖന്‍ദഖ് യുദ്ധത്തില്‍ സ്ത്രീകളെയും കുട്ടികളെയും അക്രമിക്കാനെത്തിയ ശത്രുവെ കൂര്‍ത്ത കമ്പെടുത്ത് കുത്തിക്കൊന്നത് സ്വഫിയ്യയാണ്. ഉഹ്ദ് യുദ്ധത്തില്‍ പ്രവാചകന്റെ പരിരക്ഷയ്ക്കായി പൊരുതിയ പ്രമുഖരിലൊരാള്‍ ഉമ്മു അമ്മാറയാണ്. അവരുടെ ശരീരത്തില്‍ നിരവധി മുറിവുകളേല്‍ക്കുകയുണ്ടായി. ഒന്നാം ഖലീഫ അബൂബക്ര്! സിദ്ദീഖിന്റെ കാലത്തു നടന്ന യമാമ യുദ്ധത്തില്‍ പങ്കെടുത്ത അവരുടെ ശരീരത്തില്‍ പന്ത്രണ്ടു മുറിവുകളുണ്ടായിരുന്നു. ഈവിധം രണാങ്കണത്തില്‍ ധീരമായി പൊരുതിയ നിരവധി വനിതകളെ ഇസ്‌ലാമികചരിത്രം പരിചയപ്പെടുത്തുന്നുണ്ട്.
ആധുനികലോകത്തും മുസ്‌ലിം സ്ത്രീകള്‍ സമരരംഗത്ത് സജീവമായി പങ്കെടുത്തതിന് നിരവധി ഉദാഹരണങ്ങളുണ്ട്. റിദാ ഷാ പഹ്ലവിയുടെ ഏകാധിപത്യ മര്‍ദക ഭരണത്തിനെതിരെ ഖുമൈനി നയിച്ച പോരാട്ടത്തിലും റഷ്യന്‍ അധിനിവേശത്തിനെതിരെ അഫ്ഗാന്‍ ജനത നടത്തിയ ചെറുത്തുനില്‍പിലും സ്ത്രീകള്‍ ധീരോജ്ജ്വലമായ സേവനങ്ങളര്‍പ്പിക്കുകയുണ്ടായി. ഇസ്‌ലാമിക സമരനിരയിലെ സ്ത്രീസാന്നിധ്യം പാശ്ചാത്യ മാധ്യമങ്ങള്‍പോലും എടുത്തുകാണിക്കാന്‍ നിര്‍ബന്ധിതമാകുംവിധം അവഗണിക്കാനാവാത്തതാണ്.
സ്ത്രീ വീടിന് പുറത്തുപോയി തൊഴിലിലേര്‍പ്പെടുന്നതിനെയോ സേവനവൃത്തികളില്‍ വ്യാപൃതമാവുന്നതിനെയോ പൊതു പ്രവര്‍ത്തനങ്ങളില്‍ പങ്കാളിയാവുന്നതിനെയോ ഇസ്‌ലാം ഒരു നിലയ്ക്കും വിലക്കുന്നില്ലെന്ന് ഇതും ഇതുപോലുള്ളവയുമായ സംഭവങ്ങള്‍ അസന്ദിഗ്ധമായി വ്യക്തമാക്കുന്നു. രണ്ടാം ഖലീഫ ഉമറുല്‍ ഫാറൂഖിന്റെ ഭരണകാലത്ത് കടകമ്പോളങ്ങളുടെ മേല്‍നോട്ടത്തിന്റെ ഉത്തരവാദിത്വം ഏല്‍പിച്ചിരുന്നത് ശിഫാ ബിന്‍തു അബ്ദില്ലാ എന്ന സ്ത്രീയെയായിരുന്നുവെന്നത് പ്രത്യേകം പ്രസ്താവ്യമത്രെ. നമ്മുടെ കാലത്തെ ഉപഭോക്തൃസംരക്ഷണ വകുപ്പിന്റെ ഡയറക്ടര്‍ പദവിക്കു സമാനമായ സ്ഥാനമാണത്.
സമകാലീന സമൂഹത്തിലും മുസ്‌ലിം സ്ത്രീകള്‍ ഇസ്‌ലാമിക പ്രവര്‍ത്തനങ്ങളിലെന്നപോലെ സാമൂഹിക സേവനരംഗത്തും പൊതുജീവിതത്തിലും സജീവമായി പങ്കുവഹിച്ചുവരുന്നുണ്ട്. ശാസ്ത്ര സാങ്കേതികവിദ്യാഭ്യാസ മേഖലകളിലെല്ലാം ഇറാനിയന്‍ സ്ത്രീകള്‍ സ്തുത്യര്‍ഹമായ സേവനമാണ് നിര്‍വഹിച്ചുകൊണ്ടിരിക്കുന്നത്. അവിടത്തെ യൂനിവേഴ്‌സിറ്റികളിലെ പ്രഫസര്‍മാരിലും മറ്റു ജീവനക്കാരിലും നാല്‍പതു ശതമാനത്തോളം സ്ത്രീകളാണ്. നഴ്‌സറി സ്‌കൂള്‍ തൊട്ട് ഹൈസ്‌കൂളിലെ അവസാനവര്‍ഷം വരെ അധ്യാപനവൃത്തി പൂര്‍ണമായും നിര്‍വഹിക്കുന്നത് സ്ത്രീകളാണ്. ഇറാനിലെ നല്ലൊരു വിഭാഗം സ്ത്രീകള്‍ സ്വന്തമായി കച്ചവടസ്ഥാപനങ്ങള്‍ നടത്തുന്നവരാണ്. സ്വകാര്യസ്ഥാപനങ്ങളിലും സര്‍ക്കാര്‍ ഓഫീസുകളിലും ഉന്നത ജോലിയിലേര്‍പ്പെട്ടവരും നിരവധിയാണ്. ധാരാളം വനിതാ അഡ്വക്കേറ്റുമാരും ഡോക്ടര്‍മാരും എഞ്ചിനീയര്‍മാരും അവിടെ ജോലി ചെയ്യുന്നുണ്ട്. വിദ്യാഭ്യാസ ആരോഗ്യവകുപ്പുകളിലായി പതിനൊന്ന് ലക്ഷത്തിലേറെ സ്ത്രീകള്‍ സേവനമനുഷ്ഠിച്ചു വരുന്നു. മാധ്യമരംഗത്തും ഇറാനില്‍ സ്ത്രീകള്‍ക്ക് മെച്ചപ്പെട്ട പ്രാതിനിധ്യമുണ്ട്. ടെലിവിഷന്‍ മേഖലയില്‍ മുപ്പത്തഞ്ചു ശതമാനം സ്ത്രീകളാണ്.
ഈജിപ്ത്, സുഡാന്‍ തുടങ്ങി ഇതര മുസ്‌ലിം നാടുകളിലും ഇസ്‌ലാമിക മര്യാദകള്‍ പൂര്‍ണമായും പാലിച്ചുകൊണ്ടു തന്നെ ലക്ഷക്കണക്കിന് സ്ത്രീകള്‍ പൊതുരംഗത്ത് പ്രവര്‍ത്തിച്ചുവരുന്നു. മതപണ്ഡിതന്മാരോ ഇസ്‌ലാമിക പ്രസ്ഥാനങ്ങളോ ഇതിനെ എതിര്‍ക്കാറില്ലെന്നതും ശ്രദ്ധേയമത്രെ. ഈ വിധം പൊതു ജീവിതത്തില്‍ സജീവമായി പങ്കുവഹിക്കാന്‍ അനുവദിക്കുമ്പോഴും സ്ത്രീയുടെ പ്രഥമവും പ്രധാനവുമായ ചുമതല ഗൃഹഭരണവും കുട്ടികളുടെ സംരക്ഷണവുമാണെന്ന കാര്യം ഇസ്‌ലാം ഊന്നിപ്പറയുന്നു. അതവഗണിക്കുന്നത് അത്യന്തം അപകടകരവും ദൂരവ്യാപകമായ വിപത്തുകള്‍ക്ക് നിമിത്തവുമാണെന്ന് ഉണര്‍ത്തുകയും ചെയ്യുന്നു.
\chapter{സ്ത്രീയുടെ സാക്ഷ്യം }
 \section{ സാക്ഷ്യത്തിന് ഒരാണിനു പകരം രണ്ട് സ്ത്രീ വേണമെന്നാണല്ലോ ഇസ്‌ലാമിക നിയമം. ഇത് സ്ത്രീയോടുള്ള അനീതിയും വിവേചനവും പുരുഷമേധാവിത്വപരമായ സമീപനവുമല്ലേ?}

A ഒരു പുരുഷനു പകരം രണ്ട് സ്ത്രീയെന്നത് സാക്ഷ്യത്തിനുള്ള ഇസ്‌ലാമിന്റെ പൊതു നിയമമല്ല; സാമ്പത്തിക ഇടപാടുകളില്‍ മാത്രം ബാധകമായ കാര്യമാണ്. സ്ത്രീകള്‍ സാധാരണ ഗതിയില്‍ സാമ്പത്തിക ഇടപാടുകള്‍ നടത്തുന്നവരും കൊള്ളക്കൊടുക്കകളില്‍ ഏര്‍പ്പെടുന്നവരുമല്ലാത്തതിനാല്‍ പണമിടപാടുകളുടെ സാക്ഷ്യത്തില്‍ അബദ്ധം സംഭവിക്കാതിരിക്കാനും സൂക്ഷ്മത പാലിക്കാനുമായി നിശ്ചയിക്കപ്പെട്ട നിബന്ധന മാത്രമാണിത്. സ്ത്രീക്കെതിരെ സദാചാര ലംഘനം ആരോപിക്കപ്പെട്ടാല്‍ സ്വീകരിക്കേണ്ട സ്വയം സാക്ഷ്യത്തിന്റെയും സത്യം ചെയ്യലിന്റെയും കാര്യത്തില്‍ സ്ത്രീപുരുഷ വ്യത്യാസമൊട്ടുമില്ലെന്ന് ഖുര്‍ആന്‍ തന്നെ സംശയത്തിനിടമില്ലാത്ത വിധം വ്യക്തമാക്കിയിട്ടുണ്ട് (അധ്യായം 24, വാക്യം 69).
ഇതര സാക്ഷ്യങ്ങളുടെ സ്ഥിതിയും ഇതുതന്നെ. വിവാഹമോചനത്തെക്കുറിച്ച് ഖുര്‍ആന്‍ പറയുന്നു: 'ഇനി അവരുടെ ദീക്ഷാവധി സമാപിച്ചാലോ, ഒന്നുകില്‍ അവരെ മാന്യമായി കൂടെ നിര്‍ത്തുകയോ അല്ലെങ്കില്‍ മാന്യമായ നിലയില്‍ വേര്‍പിരിയുകയോ ചെയ്യുക. നിങ്ങളില്‍ നീതിമാന്‍മാരായ രണ്ടാളുകളെ സാക്ഷികളാക്കുകയും ചെയ്യുക. അവര്‍ അല്ലാഹുവിനുവേണ്ടി നീതിപൂര്‍വം സാക്ഷ്യം വഹിക്കട്ടെ.''(65: 2). ഒസ്യത്തിനെ സംബന്ധിച്ച് ഖുര്‍ആന്‍ പറയുന്നു: 'വിശ്വസിച്ചവരേ, നിങ്ങളിലൊരുവന്ന് മരണമാസന്നമാവുകയും അയാള്‍ ഒസ്യത്ത് ചെയ്യുകയുമാണെങ്കില്‍ അതിനുള്ള സാക്ഷ്യത്തിന്റെ മാനം ഇപ്രകാരമത്രെ. നിങ്ങളില്‍നിന്നുള്ള രണ്ടു നീതിമാന്മാര്‍ സാക്ഷ്യം വഹിക്കണം. അല്ലെങ്കില്‍ നിങ്ങള്‍ യാത്രാവസ്ഥയിലായിരിക്കുകയും അവിടെ മരണവിപത്ത് അഭിമുഖീകരിക്കുകയുമാണെങ്കില്‍ അപ്പോള്‍ മറ്റു ജനത്തില്‍നിന്നു രണ്ടാളുകളെ സാക്ഷികളാക്കണം.''(5: 106)
ആര്‍ത്തവം, പ്രസവം, തുടങ്ങി പുരുഷന്മാര്‍ക്ക് സാക്ഷികളാകാന്‍ പ്രയാസമുള്ള കാര്യങ്ങളില്‍ സ്ത്രീകളുടെ മാത്രം സാക്ഷ്യമാണ് സ്വീകാര്യമാവുകയെന്നതില്‍ ഇസ്‌ലാമിക പണ്ഡിതന്മാര്‍ ഏകാഭിപ്രായക്കാരാണ്.
ഇസ്‌ലാമില്‍ സാമ്പത്തികവും സാമൂഹികവും രാഷ്ട്രീയവും ഭരണപരവുമായ എല്ലാ നിയമങ്ങളുടെയും അടിസ്ഥാന പ്രമാണങ്ങളിലൊന്ന് പ്രവാചക ചര്യയാണ്. ഈ പ്രവാചക ചര്യയുടെ നിവേദനത്തിന്റെ സ്വീകാര്യതയില്‍ പുരുഷന്റേതു പോലെത്തന്നെ സ്ത്രീയുടേതും പ്രാമാണികവും പ്രബലവുമത്രേ. ലിംഗഭേദത്തിന്റെ അടിസ്ഥാനത്തിലിവിടെ ഒരുവിധ വിവേചനവുമില്ല. അതുകൊണ്ടു തന്നെ പ്രബലമായ ഹദീസ് ഗ്രന്ഥങ്ങളില്‍ പുരുഷന്‍മാരെന്ന പോലെ സ്ത്രീകള്‍ നിവേദനം ചെയ്തവയും ധാരാളമായി കാണാവുന്നതാണ്. എല്ലാ സാക്ഷ്യങ്ങളുടെയും സാക്ഷ്യമായ അടിസ്ഥാന പ്രമാണത്തിന്റെ കാര്യത്തില്‍ പുരുഷന്റെ പദവി തന്നെ സ്ത്രീക്കും കല്‍പിച്ച ഇസ്‌ലാം ഇടപാടുകളുടെ കാര്യത്തില്‍ വ്യത്യസ്തമായ നിലപാട് സ്വീകരിച്ചത് വിവേചനപരമോ അവഗണനയോ അനീതിയോ അല്ലെന്നും മറിച്ച്, അബദ്ധം സംഭവിക്കാതിരിക്കാനുള്ള സൂക്ഷ്മത മാത്രമാണെന്നും വ്യക്തമത്രേ.
ഇമാം അബൂഹനീഫ, ത്വബരി പോലുള്ള പണ്ഡിതന്‍മാര്‍ സ്ത്രീകള്‍ക്ക് ന്യായാധിപസ്ഥാനം വരെ വഹിക്കാമെന്ന് അഭിപ്രായപ്പെട്ടിട്ടുണ്ട്. നിയമത്തിന്റെയും നീതിയുടെയും കാര്യത്തില്‍ വിവേചനമുണ്ടെങ്കില്‍ നിയമനടത്തിപ്പിന്റെ പരമോന്നത പദവിയായ ന്യായാധിപസ്ഥാനം സ്ത്രീക്ക് ആവാമെന്ന് പ്രാമാണിക പണ്ഡിതന്മാര്‍ പറയുകയില്ലല്ലോ.
\chapter{സ്ത്രീയുടെ അനുവാദമില്ലാതെ അവളുടെ വിവാഹം }
 \section{സ്ത്രീയുടെ അനുവാദമോ സമ്മതമോ ഇല്ലാതെ അവളെ കല്യാണം കഴിച്ചുകൊടുക്കാന്‍ രക്ഷിതാക്കള്‍ക്ക് ഇസ്‌ലാം അനുവാദം നല്‍കുന്നുണ്ടെന്ന് പറയുന്നത് ശരിയാണോ?}

 അല്ല. സ്ത്രീയുടെ അനുവാദമോ സമ്മതമോ ഇല്ലാതെ അവളെ വിവാഹം ചെയ്തുകൊടുക്കാന്‍ ഇസ്‌ലാം അനുവദിക്കുന്നില്ല. അഥവാ, അങ്ങനെ ചെയ്താല്‍ അത് സ്വീകരിക്കാനെന്ന പോലെ നിരസിക്കാനും സ്ത്രീക്ക് അവകാശമുണ്ട്.
പ്രവാചകന്‍ (സ) പറയുന്നു: 'കന്യകയല്ലാത്ത സ്ത്രീയെ സംബന്ധിച്ചേടത്തോളം അവള്‍ക്കുതന്നെയാണ് രക്ഷിതാവിനേക്കാള്‍ സ്വന്തം കാര്യം തീരുമാനിക്കാനുള്ള അവകാശം. കന്യകയെ കല്യാണം കഴിച്ചു കൊടുക്കാന്‍ അവളുടെ അനുവാദം ആരായേണ്ടതാണ്. അവളുടെ മൗനം അനുവാദമായി ഗണിക്കും''(മുസ്‌ലിം, തിര്‍മിദി, അബൂദാവൂദ്, നസാഈ, ഇബ്‌നുമാജ).
ഒരിക്കല്‍ ഒരു പെണ്‍കുട്ടി പ്രവാചക സന്നിധിയില്‍ വന്ന്, പിതാവ് തന്റെ സമ്മതം കൂടാതെ ഒരാളെ വിവാഹം കഴിക്കാന്‍ തന്നെ നിര്‍ബന്ധിക്കുന്നതായി പരാതിപ്പെട്ടു. അപ്പോള്‍ അവള്‍ക്ക് സ്വന്തം ഇഷ്ടപ്രകാരം പ്രവര്‍ത്തിക്കാന്‍ പ്രവാചകന്‍ അനുമതി നല്‍കി.
ഒരു യുവതി നബിയുടെ സന്നിധിയില്‍ വന്നു പറഞ്ഞു: 'എന്റെ പിതാവ് സ്വന്തം സഹോദരപുത്രനെക്കൊണ്ട്, എന്നിലൂടെ അദ്ദേഹത്തിന്റെ പോരായ്മ പരിഹരിക്കാനായി, എന്നെ കല്യാണം കഴിച്ചുകൊടുത്തിരിക്കുന്നു.' അങ്ങനെ നബി കാര്യം തീരുമാനിക്കാനുള്ള അവകാശം അവള്‍ക്ക് നല്‍കി. അപ്പോള്‍ അവള്‍ പറഞ്ഞു: പിതാവിന്റെ പ്രവൃത്തി ഞാന്‍ അംഗീകരിച്ചിരിക്കുന്നു. ഇക്കാര്യത്തില്‍ പിതാക്കന്മാര്‍ക്ക് ഒരു അധികാരവുമില്ലെന്ന് സ്ത്രീകളെ പഠിപ്പിച്ചുകൊടുക്കലാണ് എന്റെ ഉദ്ദേശ്യം.
വിവാഹക്കാര്യത്തില്‍ തീരുമാനമെടുക്കാനുള്ള അധികാരവും അവകാശവും സ്ത്രീകള്‍ക്കാണെന്ന് ഇതൊക്കെയും അസന്ദിഗ്ധമായി തെളിയിക്കുന്നു.
\chapter{സ്ത്രീക്ക് പുരുഷന്റെ പാതി സ്വത്ത് }
  \section{സ്ത്രീക്ക് പുരുഷന്റെ പാതിസ്വത്തല്ലേ ഇസ്‌ലാം അനുവദിക്കുന്നുള്ളൂ. ഇത് കടുത്ത അനീതിയും കൊടിയ വിവേചനവുമല്ലേ?}

ഇസ്‌ലാമിക നിയമമനുസരിച്ച് ഏതവസരത്തിലും സ്ത്രീക്ക് ഒരു വിധ സാമ്പത്തിക ബാധ്യതകളുമില്ല. അവകാശങ്ങളേയുള്ളൂ. എല്ലാ ബാധ്യതകളും ഏതു സാഹചര്യത്തിലും പുരുഷനു മാത്രമാണ്. വിവാഹവേളയില്‍ വരന്റേയും വധുവിന്റെയും വസ്ത്രങ്ങളുള്‍പ്പെടെ എല്ലാ വിധ ചെലവുകളും വഹിക്കേണ്ടത് പുരുഷനാണ്. അതോടൊപ്പം വധുവിന് നിര്‍ബന്ധമായും മഹ്ര്! നല്‍കുകയും വേണം. തുടര്‍ന്ന് സ്ത്രീയുടെയും കുട്ടികളുടെയും സംരക്ഷണോത്തരവാദിത്വം പൂര്‍ണമായും പുരുഷന്നാണ്. രണ്ടുപേരും ഒരേപോലെ വരുമാനമുള്ള ഡോക്ടര്‍മാരോ അധ്യാപകരോ ആരായിരുന്നാലും ശരി, സ്ത്രീ താന്‍ ഉള്‍പ്പെടെ ആരുടെയും സാമ്പത്തിക ചെലവുകള്‍ വഹിക്കേണ്ടതില്ല. ഉദ്യോഗസ്ഥയായ ഭാര്യയുടെ പോലും ഭക്ഷണവും വസ്ത്രവും ചികിത്സയുമുള്‍പ്പെടെയുള്ള ചെലവുകളൊക്കെ നിര്‍വഹിക്കേണ്ടത് ഭര്‍ത്താവാണ്. അഥവാ ഭര്‍ത്താവ് മരണപ്പെട്ടാല്‍ അയാള്‍ക്ക് സ്വത്തില്ലെങ്കില്‍ അയാളുടെ അനാഥക്കുട്ടികളെ സംരക്ഷിക്കേണ്ടത് പിതാവ്, സഹോദരന്മാര്‍, സഹോദരമക്കള്‍, പിതൃവ്യര്‍ തുടങ്ങി മരിച്ചയാള്‍ക്ക് മക്കളില്ലെങ്കില്‍ അയാളുടെ സ്വത്തിന്റെ ശിഷ്ടാവകാശികളാകാനിടയുള്ളവരാണ്. സ്ത്രീ വിവാഹിതയാണെങ്കില്‍ ഭര്‍ത്താവും അവിവാഹിതയെങ്കില്‍ പിതാവും പിതാവില്ലെങ്കില്‍ സഹോദരന്മാരുമാണ് അവളെ സംരക്ഷിക്കേണ്ടത്. മാതാവിന്റെ സംരക്ഷണച്ചുമതല മക്കള്‍ക്കാണ്. അതിനാല്‍ ഏതവസ്ഥയിലും നിയമപരമായി സ്ത്രീക്ക് ഒരുവിധ സാമ്പത്തിക ബാധ്യതയുമില്ല. പരസ്പര ബന്ധത്തിന്റെയും സ്‌നേഹത്തിന്റെയും പേരില്‍ ചെലവഴിക്കുന്നുവെങ്കില്‍ അത് മറ്റൊരു കാര്യമാണ്. എന്നിട്ടും ഇസ്‌ലാം സ്ത്രീക്ക് സ്വത്തവകാശമനുവദിച്ചതും മഹ്ര്! നിര്‍ബന്ധമാക്കിയതും അവരുടെ അന്തസ്സും സുരക്ഷിതത്വവും ഉറപ്പുവരുത്താനാണ്. സ്വന്തം സ്വത്ത് സൂക്ഷിക്കാനും സംരക്ഷിക്കാനും വര്‍ധിപ്പിക്കാനും സ്ത്രീക്ക് സ്വാതന്ത്യ്രവും അവകാശവുമുണ്ട്. മാതാവ്, മകള്‍, ഭാര്യ, സഹോദരി പോലുള്ള ഏതവസ്ഥയിലും സംരക്ഷണം പൂര്‍ണമായും ഉറപ്പുവരുത്തപ്പെട്ട ശേഷവും ഭൗതിക മാനദണ്ഡമനുസരിച്ച് സ്വത്ത് ആവശ്യമില്ലാതിരുന്നിട്ടും ഇസ്‌ലാം അത് അനുവദിച്ചത് സ്ത്രീത്വത്തിന്റെ മഹത്ത്വത്തിനും ആദരവിനും ഒരുവിധ പോറലുമേല്‍ക്കാതിരിക്കാനാണ്. 
\chapter{പര്‍ദ നിര്‍ബന്ധമാക്കുക വഴി ഇസ്ലാം സ്ത്രീകളെ പീഡിപ്പിക്കുകയല്ലേ? }
\section{പര്‍ദ നിര്‍ബന്ധമാക്കുക വഴി ഇസ്‌ലാം സ്ത്രീകളെ പീഡിപ്പിക്കുകയല്ലേ ചെയ്യുന്നത്? മുസ്‌ലിം സ്ത്രീകളുടെ പിന്നാക്കാവസ്ഥക്ക് പ്രധാന കാരണം പര്‍ദയല്ലേ?}

 സ്ത്രീ മുഖവും മുന്‍കൈയുമൊഴിച്ചുള്ള ശരീരഭാഗങ്ങള്‍ മറയ്ക്കണമെന്നതാണ് ഇസ്‌ലാമിക ശാസന. ഇത് പുരോഗതിക്ക് തടസ്സമല്ലെന്നു മാത്രമല്ല; സഹായകവുമാണ്. സ്ത്രീക്ക് പര്‍ദ പീഡനമല്ല; സുരക്ഷയാണ് നല്‍കുന്നത്.
ഇന്ന് ലോകത്തിന്റെ പല ഭാഗത്തും പര്‍ദയണിഞ്ഞ സ്ത്രീകള്‍ ശാസ്ത്രജ്ഞകളായും വൈമാനികരായും സാഹിത്യകാരികളായും പത്രപ്രവര്‍ത്തകകളായും പാര്‍ലമെന്റംഗങ്ങളായും പ്രശസ്ത സേവനം നിര്‍വഹിച്ചുവരുന്നു. ഇറാനിലെ അഞ്ചു വൈസ് പ്രസിഡന്റുമാരിലൊരാളായ മഅ്‌സൂമാ ഇബ്തികാര്‍ പര്‍ദാധാരിണിയാണ്. അമേരിക്കയില്‍ ഉപരിപഠനം നിര്‍വഹിച്ച മഅ്‌സൂമാ തെഹ്‌റാന്‍ യൂനിവേഴ്‌സിറ്റിയിലെ അസോസിയേറ്റ് പ്രഫസറും പ്രശസ്ത പത്രപ്രവര്‍ത്തകയുമാണ്. അറിയപ്പെടുന്ന സാമൂഹിക പ്രവര്‍ത്തകയായ അവര്‍ നിരവധി അന്തര്‍ദേശീയ സമ്മേളനങ്ങളില്‍ സംബന്ധിച്ചിട്ടുണ്ട്. ഇറാനിലെ പ്രമുഖ വനിതാ പ്രസിദ്ധീകരണമായ 'മഹ്ജൂബാഹി'ന്റെ പത്രാധിപയും തെഹ്‌റാന്‍ സര്‍വകലാശാലാ പ്രഫസറുമായ ടുറാന്‍ ജംശീദ്യാന്‍, നാഷനല്‍ ഒളിംപിക് വൈസ് പ്രസിഡന്റും പാര്‍ലമെന്റ് ഉപാധ്യക്ഷയുമായിരുന്ന ഫസീഹ് ഹാശ്മി, വനിതാ ക്ഷേമവകുപ്പിന്റെ ഉപദേശകയായിരുന്ന ശഹ്ലാ ഹബീബി, മലേഷ്യന്‍ രാഷ്ട്രീയത്തില്‍ നിറഞ്ഞുനില്‍ക്കുന്ന വാന്‍ അസീസ പോലുള്ള വിഖ്യാതരായ വനിതകളെല്ലാം പര്‍ദാധാരിണികളാണ്. ആധുനിക ലോകത്ത് രണാങ്കണത്തില്‍ ധീരമായി പൊരുതിയ വീരവനിതകള്‍ ഇറാനിലെയും അഫ്ഗാനിസ്താനിലെയും പര്‍ദാധാരിണികളാണെന്ന വസ്തുത വിസ്മരിക്കാവതല്ല. വോളിബോള്‍, ബാസ്‌കറ്റ് ബോള്‍, ഷൂട്ടിംഗ്, സൈക്‌ളിംഗ്, ടെന്നീസ്, ജിംനാസ്‌റിക്, കുതിരയോട്ടം, ജൂഡോ, കരാട്ടെ, ചെസ്സ് തുടങ്ങിയ വിനോദങ്ങളിലും കായികാഭ്യാസങ്ങളിലും മികവ് പ്രകടിപ്പിക്കുന്ന ഇറാനിയന്‍ വനിതകളുടെ പുരോഗതിയുടെ പാതയില്‍ പര്‍ദ ഒരുവിധ വിഘാതവും സൃഷ്ടിക്കുന്നില്ല. പര്‍ദക്ക് സമാനമായ വസ്ത്രമണിഞ്ഞത് മദര്‍ തെരേസയുടെ സേവനവൃത്തികള്‍ക്കൊട്ടും വിഘാതം സൃഷ്ടിച്ചില്ല. വിവിധ മേഖലകളില്‍ സേവനമനുഷ്ഠിക്കുന്ന കന്യാസ്ത്രീകളുടെ വേഷം ഇസ്‌ലാമിലെ പര്‍ദക്കു സമാനമാണല്ലോ. ശരീരഭാഗം മറയ്ക്കുന്നത് ഗോളാന്തരയാത്രക്കോ പരീക്ഷണനിരീക്ഷണങ്ങള്‍ക്കോ ഒട്ടും തടസ്സം സൃഷ്ടിക്കില്ലെന്ന് ചാന്ദ്രയാത്രികരുടെ അനുഭവം തെളിയിക്കുന്നു. പാന്റ്‌സും ഷര്‍ട്ടും ടൈയും ഓവര്‍കോട്ടും സോക്‌സും ഷൂവും കേപ്പുമണിയുന്ന പാശ്ചാത്യരാജ്യങ്ങളിലെ പുരുഷന്മാര്‍ ഇസ്‌ലാം സ്ത്രീകളോട് മറയ്ക്കാനാവശ്യപ്പെട്ട ശരീരഭാഗങ്ങളിലേറെയും മറയ്ക്കുന്നവരാണ്. സ്ത്രീകള്‍ മറിച്ചാണെങ്കിലും.
സ്ത്രീയെന്നാല്‍ അവളുടെ ശരീരവും രൂപലാവണ്യവുമാണെന്നും അവളുടെ വ്യക്തിത്വം അതിന്റെ മോടി പിടിപ്പിക്കലിനനുസൃതമാണെന്നുമുള്ള ധാരണ സൃഷ്ടിക്കുന്നതില്‍ പുത്തന്‍ മുതലാളിത്ത സാമ്രാജ്യത്വവും അതിന്റെ സൃഷ്ടിയായ കമ്പോള സംസ്‌കാരവും വന്‍വിജയം വരിച്ചതാണ്, പര്‍ദ പുരോഗതിക്കും പരിഷ്‌കാരത്തിനും തടസ്സമാണെന്ന ധാരണ വളരാന്‍ കാരണം. മാംസളമായ ശരീരഭാഗങ്ങള്‍ പ്രദര്‍ശിപ്പിച്ച് പട്ടണങ്ങളിലും പൊതുസ്ഥലങ്ങളിലും ചുറ്റിക്കറങ്ങലാണ് പുരോഗതിയെന്ന് പ്രചരിപ്പിക്കുന്ന സ്ത്രീകള്‍, പുരുഷന്മാര്‍ തങ്ങളുടെ ശരീരസൗന്ദര്യം കണ്ടാസ്വദിക്കുന്നതില്‍ നിര്‍വൃതിയടയുന്ന ഒരുതരം മനോവൈകൃതത്തിനടിപ്പെട്ടവരത്രെ.
സമൂഹത്തിലെ സ്ത്രീകളുടെയൊക്കെ സൗന്ദര്യം കണ്ടാസ്വദിക്കാന്‍ കാമാതുരമായ കണ്ണുകളുമായി കാത്തിരിക്കുന്നവരില്‍നിന്ന് സ്വന്തം ശരീരം മറച്ചുവയ്ക്കലാണ് മാന്യത. സ്ത്രീയുടെ സുരക്ഷിതത്വത്തിന് ഏറെ സഹായകവും അതത്രെ. പര്‍ദയണിയാന്‍ ഇസ്‌ലാം ആവശ്യപ്പെടാനുള്ള കാരണവും അതുതന്നെ. അതോടൊപ്പം അത് പുരോഗതിയെ ഒട്ടും പ്രതികൂലമായി ബാധിക്കുന്നുമില്ല. ഇറാന്‍ സന്ദര്‍ശിച്ചശേഷം എം.പി. വീരേന്ദ്രകുമാര്‍ എഴുതിയ വരികള്‍ ശ്രദ്ധേയമത്രെ: 'ഇറാനിയന്‍ സ്ത്രീകള്‍ പര്‍ദ ധരിക്കുന്നു. മുഖം മൂടാറില്ല. തല മൂടും. ഏത് പിക്‌നിക് സ്‌പോട്ടില്‍ ചെന്നാലും നൂറുകണക്കിന് സ്ത്രീകളെ കാണാം. ഇറാനിയന്‍ വാര്‍ത്താ ഏജന്‍സിയായ 'ഇര്‍ന'യുടെ കേന്ദ്രകമ്മിറ്റി ഓഫീസില്‍ ചെന്നപ്പോള്‍ അവിടെ പ്രവര്‍ത്തിച്ചിരുന്നവരില്‍ ഭൂരിഭാഗവും സ്ത്രീകളായിരുന്നു. സ്ത്രീകള്‍ തെഹ്‌റാനിലൂടെ കാറോടിക്കുന്നു. ഒരു കുഴപ്പവുമില്ല. പക്ഷേ ഒന്നുണ്ട്. ന്യൂയോര്‍ക്കിലൊക്കെ പോയാല്‍ കാണുന്നതുപോലെ സ്ത്രീകളെ സെക്‌സ് സിംബലാക്കി മാറ്റാന്‍ ഇറാനികള്‍ അനുവദിക്കുകയില്ല.''(ബോധനം വാരിക, 1993 നവംബര്‍ 6).
ശ്രീമതി കല്‍പനാ ശര്‍മയുടെ ചോദ്യം പ്രസക്തമത്രെ: 'വിദ്യയഭ്യസിക്കാനും പുറത്തിറങ്ങി ജോലി ചെയ്യുവാനും ദാമ്പത്യ ബന്ധം പൊറുപ്പിച്ചുകൂടാതെ വരുമ്പോള്‍ വിവാഹമോചനമാവശ്യപ്പെട്ട് കോടതിയില്‍ പോവാനും അവകാശമുള്ള ഇറാനിലെ സ്ത്രീകള്‍ക്കെതിരെ വിധിപറയാന്‍ നാം ശക്തരാണോ? പര്‍ദയണിയുന്ന ഇറാനിലെ സഹോദരിമാരേക്കാള്‍ എന്തു മഹത്ത്വമാണ് നമ്മുടെ നാട്ടിലെ സ്ത്രീകള്‍ക്കുള്ളത്?''(Kalpana Sharma- Behind The Veil- The Hindu 20- 7-'97).

\section{'സ്ത്രീകള്‍ക്ക് പര്‍ദ നിര്‍ബന്ധമാക്കുകയും പുരുഷന്മാരെ അതില്‍നിന്നൊഴിവാക്കുകയും ചെയ്തത് തികഞ്ഞ വിവേചനമല്ലേ?''}
ഈ വിവേചനം പ്രകൃതിപരമാണ്. സ്ത്രീകളുടെയും പുരുഷന്മാരുടെയും ശരീരപ്രകൃതി ഒരുപോലെയല്ലല്ലോ. ഏതൊരു കരുത്തനായ പുരുഷനും സ്ത്രീയെ അവളുടെ അനുവാദമില്ലാതെ തന്നെ ബലാത്സംഗം ചെയ്യാന്‍ സാധിക്കും. എന്നാല്‍ സ്ത്രീ എത്ര കരുത്തയായാലും പുരുഷന്റെ അനുമതിയില്ലാതെ അയാളെ ലൈംഗികമായി കീഴ്‌പ്പെടുത്താനാവില്ല. ഈ അന്തരത്തിന്റെ അനിവാര്യമായ താല്‍പര്യമാണ് വസ്ത്രത്തിലെ വ്യത്യാസം. അതിനാലാണല്ലോ ഇന്ത്യയുള്‍പ്പെടെ ലോകമെങ്ങും സ്ത്രീയുടെ സംരക്ഷണത്തിന് പ്രത്യേക നിയമം അനിവാര്യമായത്. സ്ത്രീപീഡനത്തിന് കഠിനശിക്ഷ നിയമം മൂലം നിശ്ചയിച്ച നാടുകളിലൊന്നും പുരുഷ പീഡനത്തിനെതിരെ ഇവ്വിധം നിയമനിര്‍മാണം നടത്തിയിട്ടില്ലല്ലോ. ശാരീരിക വ്യത്യാസങ്ങളാല്‍ കൂടുതല്‍ സുരക്ഷിതത്വവും മുന്‍കരുതലും ആവശ്യമുള്ളത് സ്ത്രീകള്‍ക്കാണെന്ന് ഇത് സുതരാം വ്യക്തമാക്കുന്നു. അവള്‍ തന്റെ ശരീരസൗന്ദര്യം പരപുരുഷന്മാരുടെ മുമ്പില്‍ പ്രകടിപ്പിക്കരുതെന്ന് ഇസ്‌ലാം ആവശ്യപ്പെടാനുള്ള കാരണവും അതത്രെ. അതിനാല്‍ പര്‍ദ സ്ത്രീകള്‍ക്ക് സുരക്ഷയും സൗകര്യവുമാണ്. അസൗകര്യമോ പീഡനമോ അല്ല.
\chapter{എന്തുകൊണ്ട് സ്ത്രീപ്രവാചകന്മാരില്ല? }
 \section{ ദൈവത്തിങ്കല്‍ ലിംഗവിവേചനമില്ലെങ്കില്‍ എന്തുകൊണ്ട് സ്ത്രീ പ്രവാചകന്മാരെ നിയോഗിച്ചില്ല?}

ദൈവിക ജീവിതവ്യവസ്ഥ സമൂഹത്തിന് സമര്‍പ്പിക്കലും അതിന് കര്‍മപരമായ സാക്ഷ്യം വഹിക്കലും പ്രായോഗിക മാതൃക കാണിച്ചുകൊടുക്കലുമാണല്ലോ പ്രവാചകന്‍മാരുടെ പ്രധാന ദൗത്യം. അതിനാല്‍ ആരാധനാകാര്യങ്ങളിലും സാമൂഹിക, സാമ്പത്തിക, സാംസ്‌കാരിക, രാഷ്ട്രീയ, ഭരണരംഗങ്ങളിലും യുദ്ധം, സന്ധി പോലുള്ളവയിലും സമൂഹത്തിന് അവര്‍ മാതൃകയാവേണ്ടതുണ്ട്. മാസത്തില്‍ ഏതാനും ദിവസം ആരാധനാകര്‍മങ്ങള്‍ക്ക് നേതൃത്വം നല്‍കാനും ഗര്‍ഭധാരണം, പ്രസവം പോലുള്ള ഘട്ടങ്ങളില്‍ നായകത്വപരമായ പങ്കുവഹിക്കാനും സ്ത്രീകള്‍ക്ക് സാധ്യമാവാതെ വരുന്നു. അതുകൊണ്ടുതന്നെ സമൂഹത്തിന് സദാ സകല മേഖലകളിലും നേതൃത്വം നല്‍കേണ്ട പ്രവാചകത്വബാധ്യതയില്‍നിന്ന് സ്ത്രീകള്‍ ഒഴിവാക്കപ്പെട്ടിരിക്കുന്നു. എങ്കിലും മൂസാനബിയുടെ മാതാവ് സ്വന്തം ജീവിതത്തില്‍ പകര്‍ത്തേണ്ട നിര്‍ദേശങ്ങള്‍ ദൈവത്തില്‍നിന്ന് നേരിട്ട് സ്വീകരിച്ചതായി വിശുദ്ധ ഖുര്‍ആന്‍ വ്യക്തമാക്കുന്നു. അല്ലാഹു അറിയിക്കുന്നു: 'നാം മൂസായുടെ മാതാവിനു ബോധനം നല്‍കി. അവനെ മുലയൂട്ടിക്കൊള്ളുക. അവന്റെ ജീവനില്‍ ആശങ്കയുണ്ടായാല്‍ അവനെ നദിയിലെറിയുക. ഒന്നും ഭയപ്പെടേണ്ടതില്ല. ഒട്ടും ദുഃഖിക്കേണ്ടതില്ല. നാം അവനെ നിന്റെ അടുക്കലേക്കുതന്നെ തിരികെ കൊണ്ടുവരുന്നതാകുന്നു. അവനെ ദൈവദൂതന്മാരിലുള്‍പ്പെടുത്തുകയും ചെയ്യും''(28: 7).
പ്രവാചകന്‍മാര്‍ക്ക് ലഭിക്കും വിധം യേശുവിന്റെ മാതാവ് മര്‍യമിന് മലക്കില്‍നിന്ന് സന്ദേശം ലഭിച്ചതായും ഖുര്‍ആന്‍ പ്രസ്താവിക്കുന്നു: 'അങ്ങനെ മര്‍യം ആ കുഞ്ഞിനെ ഗര്‍ഭം ധരിച്ചു. ഗര്‍ഭവുമായി അവള്‍ അകലെയുള്ള ഒരു സ്ഥലത്ത് ചെന്നെത്തി. പിന്നെ പ്രസവവേദന അവളെ ഒരു ഈന്തപ്പനയുടെ ചുവട്ടിലെത്തിച്ചു. അവള്‍ കേണുകൊണ്ടിരുന്നു. ഹാ കഷ്ടം! ഇതിനു മുമ്പുതന്നെ ഞാന്‍ മരിക്കുകയും എന്റെ പേരും കുറിയും വിസ്മൃതമാവുകയും ചെയ്തിരുന്നെങ്കില്‍! അപ്പോള്‍ താഴെനിന്ന് മലക്ക് അവളെ വിളിച്ചറിയിച്ചു. വ്യസനിക്കാതിരിക്കുക! നിന്റെ നാഥന്‍ നിനക്കു താഴെ ഒരു അരുവി ഒഴുക്കിയിരിക്കുന്നു. നീ ആ ഈന്തപ്പനയുടെ തടിയൊന്നു കുലുക്കി നോക്കുക. അതു നിനക്ക് പുതിയ ഈത്തപ്പഴം വീഴ്ത്തിത്തരും. അതു തിന്നുകയും കുടിക്കുകയും കണ്‍കുളിര്‍ക്കുകയും ചെയ്തുകൊള്ളുക. പിന്നെ, വല്ല മനുഷ്യരെയും കാണുകയാണെങ്കില്‍ അവരോടു പറഞ്ഞേക്കുക: ഞാന്‍ കാരുണികനായ ദൈവത്തിനുവേണ്ടി വ്രതം നേര്‍ന്നിരിക്കുകയാണ്. അതിനാല്‍ ഞാനിന്ന് ആരോടും സംസാരിക്കുന്നതല്ല''(19: 2226).
പ്രകൃതിപരമായ കാരണങ്ങളാല്‍ സ്ത്രീകള്‍ പ്രവാചകരായി നിയോഗിതരായിട്ടില്ലെങ്കിലും പ്രവാചകന്മാര്‍ക്കെന്നപോലെ അവര്‍ക്കും ദിവ്യബോധനം ലഭിച്ചിരുന്നതായി ഈ വേദവാക്യങ്ങള്‍ വ്യക്തമാക്കുന്നു.
\chapter{ഇസ്‌ലാം പുരുഷമേധാവിത്വത്തിന്റെ മതമോ? }
\section{  ഇസ്‌ലാം പുരുഷമേധാവിത്വപരമല്ലേ? ഖുര്‍ആന്‍ നാലാം അധ്യായം 34ാം വാക്യം തന്നെ ഇതിനു തെളിവാണല്ലോ?}

 ഖുര്‍ആന്‍ നാലാം അധ്യായം മുപ്പത്തിനാലാം വാക്യം കുടുംബഘടനയെ സംബന്ധിച്ച ദൈവികനിര്‍ദേശമാണ്. അത് ഈ വിധമത്രെ: 'പുരുഷന്മാര്‍ സ്ത്രീകളുടെ രക്ഷാധികാരികളാകുന്നു. അല്ലാഹു അവരില്‍ ചിലരെ മറ്റു ചിലരേക്കാള്‍ കഴിവുറ്റവരാക്കിയതിനാലും പുരുഷന്മാര്‍ തങ്ങളുടെ ധനം ചെലവഴിക്കുന്നതിനാലുമാണത്.''
ഇവിടെ പുരുഷന്മാരെ സംബന്ധിച്ച് ഖുര്‍ആന്‍ പ്രയോഗിച്ച പദം ഖവ്വാം എന്നാണ്. 'ഒരു വ്യക്തിയുടെയോ സ്ഥാപനത്തിന്റെയോ സംഘടനയുടെയോ കാര്യങ്ങള്‍ നല്ല നിലയില്‍ കൊണ്ടുനടത്താനും മേല്‍നോട്ടം വഹിക്കാനും അതിനാവശ്യമായത് സജ്ജീകരിക്കാനും ഉത്തരവാദപ്പെട്ട വ്യക്തിക്കാണ് അറബിയില്‍ ഖവ്വാം അല്ലെങ്കില്‍ ഖയ്യിം എന്നു പറയുക''(തഫ്ഹീമുല്‍ ഖുര്‍ആന്‍, ഭാഗം 1, പുറം 310. 56ാം അടിക്കുറിപ്പ്).
മേല്‍നോട്ടക്കാരനും രക്ഷാധികാരിയുമില്ലാതെ ഏതൊരു സ്ഥാപനവും സംരംഭവും വിജയകരമായി നിലനില്‍ക്കുകയില്ല. സമൂഹത്തിന്റെ അടിസ്ഥാന ഘടകമായ കുടുംബം ഭദ്രമായും സുരക്ഷിതമായും നിലനില്‍ക്കേണ്ട സ്ഥാപനമത്രെ. കൈകാര്യകര്‍ത്താവില്ലാതെ അത് സാധ്യമല്ല. അത് ആരായിരിക്കണമെന്നത് ഓരോ കുടുംബത്തിലും സ്ത്രീ പുരുഷന്മാര്‍ക്കിടയില്‍ വിവാദ വിഷയമായാല്‍ കുടുംബഭദ്രത നഷ്ടമാവുകയും ഛിദ്രത അനിവാര്യമാവുകയും ചെയ്യും. അതിനാല്‍ ശാരീരികവും മാനസികവും വൈകാരികവുമായ പ്രത്യേകതകള്‍ പരിഗണിച്ച് ഇസ്‌ലാം അത് പുരുഷനെ ഏല്‍പിച്ചു. കുടുംബത്തിന്റെ സംരക്ഷണമെന്നത് ഒരു അവകാശമോ അധികാരമോ അല്ല. ഭാരിച്ച ഒരു ഉത്തരവാദിത്തമാണ്. ജീവിതവുമായി മല്ലിടാന്‍ ഏറ്റവും പ്രാപ്തനും കരുത്തനും പുരുഷനായതിനാലാണ് കടുത്ത ആ ചുമതല പുരുഷനെ ഏല്‍പിച്ചത്. അതിനാല്‍ ഇസ്‌ലാമിക വീക്ഷണത്തില്‍ പുരുഷന്‍ കുടുംബമെന്ന കൊച്ചു രാഷ്ട്രത്തിലെ പ്രധാനമന്ത്രിയും സ്ത്രീ ആഭ്യന്തരമന്ത്രിയുമാണ്. വീട്ടിനകത്തെ കാര്യങ്ങളൊക്കെ തീരുമാനിക്കുന്നതും നിര്‍വഹിക്കുന്നതും സ്ത്രീയാണ്.
സ്ത്രീപുരുഷന്മാര്‍ക്കിടയിലെ ബന്ധം ഭരണാധികാരിഭരണീയ ബന്ധമല്ല. അതിനാലാണ് ഇസ്‌ലാം ദമ്പതികളെ ഭാര്യാഭര്‍ത്താക്കന്മാരെന്ന് വിളിക്കുകയോ വിശേഷിപ്പിക്കുകയോ ചെയ്യാത്തത്. ഇണകളെന്നാണ് ഇസ്‌ലാം ദമ്പതികളെ വിശേഷിപ്പിക്കുന്നത്. 'സ്ത്രീകള്‍ പുരുഷന്മാരുടെയും പുരുഷന്മാര്‍ സ്ത്രീകളുടെയും വസ്ത്രമാണെ'ന്ന് (2: 187) വിശുദ്ധ ഖുര്‍ആന്‍ പറയാനുള്ള കാരണവും അതുതന്നെ.
രാജ്യത്തെ ഭരണാധികാരി ഭരണീയരോടും സമൂഹത്തിലെ നേതാവ് അനുയായികളോടുമെന്നപോലെ ഗൃഹനാഥന്‍ വീട്ടുകാരോട് കൂടിയാലോചിച്ചു മാത്രമായിരിക്കണം തീരുമാനങ്ങളെടുക്കുന്നതും നടപ്പാക്കുന്നതും. 'തങ്ങളുടെ കാര്യങ്ങള്‍ പരസ്പരം കൂടിയാലോചിച്ചു നടത്തുന്നവരാണവര്‍''(ഖുര്‍ആന്‍: 42: 38).
അതിനാല്‍ പുരുഷന്‍ വീട്ടിലെ സ്വേഛാധിപതിയോ സര്‍വാധികാരിയോ അല്ല. എല്ലാ ദൈവികപരിധികളും പാലിക്കാനയാള്‍ ബാധ്യസ്ഥനാണ്; കുടുംബത്തിന്റെ സംരക്ഷണം മാന്യമായും മര്യാദയോടെയും നിര്‍വഹിക്കാന്‍ കടപ്പെട്ടവനും. സ്ത്രീയുടെ അവകാശങ്ങളെല്ലാം പൂര്‍ണമായും പാലിച്ചുകൊണ്ടായിരിക്കണം അത് നിര്‍വഹിക്കുന്നത്. അല്ലാഹു അറിയിക്കുന്നു: 'സ്ത്രീകള്‍ക്ക് ചില ബാധ്യതകളുള്ളതുപോലെത്തന്നെ ന്യായമായ ചില അവകാശങ്ങളുമുണ്ട്''(2: 228).
പ്രവാചകന്‍(സ) പറയുന്നു: 'മാന്യന്മാരല്ലാതെ സ്ത്രീകളെ മാനിക്കുകയില്ല. നീചനല്ലാതെ അവരെ നിന്ദിക്കുകയില്ല.''
'കുടുംബത്തോട് കാരുണ്യം കാണിക്കാത്തവനും അഹങ്കരിക്കുന്നവനും സ്വര്‍ഗത്തില്‍ പ്രവേശിക്കുകയില്ല''(അബൂദാവൂദ്).
'വിശ്വാസികളില്‍ വിശ്വാസപരമായി ഏറ്റവുമധികം പൂര്‍ണത വരിച്ചവന്‍ അവരില്‍ ഏറ്റം നല്ല സ്വഭാവമുള്ളവനാണ്. നിങ്ങളിലേറ്റവും നല്ലവര്‍ സ്വന്തം സഹധര്‍മിണിമാരോട് നന്നായി വര്‍ത്തിക്കുന്നവരാണ്''(തിര്‍മിദി).
സ്വകുടുംബത്തിന്റെ നിലനില്‍പിനും പ്രതിരോധത്തിനും ജീവിതാവശ്യങ്ങള്‍ക്കും ഗുണകരമായ എല്ലാറ്റിനും പങ്കുവഹിക്കുന്നതിനുവേണ്ടി ജീവിതം നീക്കിവച്ചത് പുരുഷനായതിനാല്‍ വീട്ടിലെ അവസാനവാക്ക് ചര്‍ച്ചക്കും കൂടിയാലോചനക്കും ശേഷം അവന്റേതാണ്. എന്നാലത് നന്മക്ക് എതിരോ അവകാശനിഷേധമോ അവിവേകപൂര്‍വമോ ആകാവതല്ല. ഭര്‍ത്താവിന് തെറ്റ് സംഭവിച്ചാല്‍ തിരുത്താനും അയാളുടെ അന്യായമായ തീരുമാനങ്ങള്‍ അവഗണിക്കാനും വേണ്ടിവന്നാല്‍ തദാവശ്യാര്‍ഥം തന്റെയോ അയാളുടെയോ കുടുംബത്തെയോ അധികാരകേന്ദ്രങ്ങളെയോ സമീപിക്കാനും സ്ത്രീക്ക് അവകാശമുണ്ടായിരിക്കും. പുരുഷനാവട്ടെ, അപ്പോള്‍ ദൈവികപരിധികള്‍ പാലിച്ചു നടപ്പാക്കാന്‍ ബാധ്യസ്ഥനുമാണ്.
പുറത്തുപോയി പണിയെടുക്കാന്‍ സ്ത്രീയേക്കാള്‍ കൂടുതല്‍ കഴിയുക പുരുഷന്നാണ്. പ്രതിയോഗികളുടെ പരാക്രമങ്ങളെ പ്രതിരോധിക്കാന്‍ സാധിക്കുന്നതും അവനുതന്നെ. എന്തുതന്നെയായാലും സ്ത്രീക്ക് എല്ലാ സമയത്തും ഒരുപോലെ പാടത്തും പറമ്പിലും ഫാക്ടറിയിലും പുറത്തും പോയി ജോലി ചെയ്തു സമ്പാദിക്കുക സാധ്യമല്ല. മനുഷ്യരാശി നിലനില്‍ക്കണമെങ്കില്‍ സ്ത്രീ ഗര്‍ഭം ധരിക്കുകയും പ്രസവിക്കുകയും മുലയൂട്ടുകയുമൊക്കെ വേണമല്ലോ. ഈ ശാരീരികമായ പ്രത്യേകതകളാലാണ് ഇസ്‌ലാം കുടുംബത്തിന്റെ സാമ്പത്തികബാധ്യതകളും സംരക്ഷണോത്തരവാദിത്തങ്ങളും നേതൃപദവിയും പുരുഷനെ ഏല്‍പിച്ചത്.
\chapter{സ്ത്രീയുടെ പദവി ഇസ്‌ലാമില്‍ }
  \section{ഇസ്‌ലാമില്‍ സ്ത്രീപുരുഷ സമത്വമുണ്ടോ? സ്ത്രീയുടെ പദവി പുരുഷന്റേതിനേക്കാള്‍ വളരെ താഴെയല്ലേ?}

A മനുഷ്യര്‍ പല തരക്കാരാണ്. മനുഷ്യരിലെ അവസ്ഥാവ്യത്യാസമനുസരിച്ച് അവരുടെ സ്ഥാനപദവികളിലും അവകാശബാധ്യതകളിലും അന്തരമുണ്ടാവുക സ്വാഭാവികവും അനിവാര്യവുമത്രെ. അതുപോലെ സ്ത്രീപുരുഷന്മാര്‍ക്കിടയിലും ശാരീരികവും മാനസികവുമായ അന്തരമുണ്ട്. പുരുഷന്‍ എത്രതന്നെ ആഗ്രഹിച്ചാലും ശ്രമിച്ചാലും ഗര്‍ഭം ധരിക്കാനും പ്രസവിക്കാനും മുലയൂട്ടാനും സാധ്യമല്ലല്ലോ. സ്ത്രീ, പുരുഷനില്‍നിന്ന് വ്യത്യസ്തമായി മാസത്തില്‍ നിശ്ചിത ദിവസങ്ങളില്‍ ആര്‍ത്തവവും അതിന്റെ അനിവാര്യതയായ ശാരീരികവും മാനസികവുമായ പ്രശ്‌നങ്ങളും അനുഭവിക്കുന്നു. കായികമായി പുരുഷന്‍ സ്ത്രീയേക്കാള്‍ കരുത്തനും ഭാരിച്ച ജോലികള്‍ ചെയ്യാന്‍ കഴിവുറ്റവനുമാണ്.
പുരുഷന്റെ ഏതാണ്ട് എല്ലാ ശാരീരികാവയവങ്ങളും സ്ത്രീയുടേതില്‍നിന്ന് തീര്‍ത്തും വ്യത്യസ്തമാണ്. പ്രമുഖ ശരീര ശാസ്ത്രജ്ഞനായ ഹാവ്‌ലോക് എല്ലിസ് പറയുന്നു: 'പുരുഷന്‍ അവന്റെ കൈവിരല്‍ത്തുമ്പു വരെ പുരുഷന്‍ തന്നെയാണ്. സ്ത്രീ കാല്‍വിരല്‍ത്തുമ്പുവരെ സ്ത്രീയും''.
ശരീരഘടനയിലെ അന്തരം മാനസികവും വൈകാരികവുമായ അവസ്ഥയിലും പ്രകടമത്രെ. അതിനാല്‍ സ്ത്രീപുരുഷന്മാര്‍ക്കിടയില്‍ ശാരീരികമോ മാനസികമോ ആയ സമത്വമോ തുല്യതയോ ഇല്ല. അതുകൊണ്ടുതന്നെ അവര്‍ക്കിടയിലെ സമ്പൂര്‍ണ സമത്വം അപ്രായോഗികമാണ്, പ്രകൃതിവിരുദ്ധവും.
മനുഷ്യരാശിയുടെ സ്രഷ്ടാവും സംരക്ഷകനുമായ ദൈവം നല്കിയ ജീവിതവ്യവസ്ഥയാണ് ഇസ്‌ലാം. അതിനാലത് മനുഷ്യപ്രകൃതിയോട് പൂര്‍ണമായും ഇണങ്ങുന്നതും പൊരുത്തപ്പെടുന്നതുമത്രെ. ഇസ്‌ലാം സ്ത്രീയെയും പുരുഷനെയും അവകാശബാധ്യതകളുടെ പേരില്‍ പരസ്പരം കലഹിക്കുന്ന രണ്ടു ശത്രുവര്‍ഗമായല്ല കാണുന്നത്. ഒരേ വര്‍ഗത്തിലെ അന്യോന്യം സഹകരിച്ചും ഇണങ്ങിയും കഴിയുന്ന, കഴിയേണ്ട രണ്ട് അംഗങ്ങളായാണ്. അല്ലാഹു അറിയിക്കുന്നു: 'നിങ്ങളെല്ലാവരും ഒരേ വര്‍ഗത്തില്‍പെട്ടവരാണ്''(ഖുര്‍ആന്‍ 4:25). പ്രവാചകന്‍ പറഞ്ഞു: 'സ്ത്രീകള്‍ പുരുഷന്മാരുടെ ഭാഗം തന്നെയാണ്''(അബൂദാവൂദ്).
അതിനാല്‍ സ്ത്രീപുരുഷന്മാരുടെ പദവികളെ ഗണിത ശാസ്ത്രപരമായി വിശകലനം ചെയ്യുക സാധ്യമല്ല. ചില കാര്യങ്ങളില്‍ പുരുഷന്മാര്‍ക്കാണ് മുന്‍ഗണനയെങ്കില്‍ മറ്റു ചിലതില്‍ സ്ത്രീകള്‍ക്കാണ്. തദ്‌സംബന്ധമായ ഇസ്‌ലാമിന്റെ സമീപനം ഇങ്ങനെ സംഗ്രഹിക്കാം:
1. അല്ലാഹുവിങ്കല്‍ സ്ത്രീപുരുഷന്മാര്‍ക്കിടയില്‍ എന്തെങ്കിലും അന്തരമോ വിവേചനമോ ഇല്ല. അവന്റെയടുക്കല്‍ സമ്പൂര്‍ണ സമത്വവും തുല്യതയും വാഗ്ദാനം ചെയ്യപ്പെട്ടിരിക്കുന്നു.
'പുരുഷനാവട്ടെ സ്ത്രീയാവട്ടെ, ആരു സത്യവിശ്വാസമുള്‍ക്കൊണ്ട് സല്‍ക്കര്‍മമനുഷ്ഠിക്കുന്നുവോ, ഈ ലോകത്ത് അവര്‍ക്ക് നാം വിശുദ്ധജീവിതം നല്‍കും. പരലോകത്ത് അവരുടെ ശ്രേഷ്ഠവൃത്തികളുടെ അടിസ്ഥാനത്തില്‍ നാമവര്‍ക്ക് പ്രതിഫലം നല്‍കുകയും ചെയ്യും''(16: 97). 'പുരുഷനാവട്ടെ സ്ത്രീയാവട്ടെ, സത്യവിശ്വാസമുള്‍ക്കൊണ്ട് സല്‍ക്കര്‍മമനുഷ്ഠിക്കുന്നതാരോ, അവര്‍ സ്വര്‍ഗാവകാശികളായിരിക്കും''(40: 40). 'അവരുടെ നാഥന്‍ അവരോട് ഉത്തരമേകി: സ്ത്രീയാവട്ടെ, പുരുഷനാവട്ടെ, നിങ്ങളിലാരുടെയും കര്‍മത്തെ നാം നിഷ്ഫലമാക്കുന്നതല്ല. നിങ്ങളെല്ലാവരും ഒരേ വര്‍ഗത്തില്‍ പെട്ടവരാണല്ലോ''(3:195).
2. ഭൂമിയില്‍ ഏറ്റവുമധികം ആദരവ് അര്‍ഹിക്കുന്നത് സ്ത്രീയാണ്. മാതൃത്വത്തോളം മഹിതമായി മറ്റൊന്നും ലോകത്തില്ല. തലമുറകള്‍ക്ക് ജന്മം നല്‍കുന്നത് അവരാണ്. ആദ്യ ഗുരുക്കന്മാരും അവര്‍തന്നെ. മനുഷ്യന്റെ ജനനത്തിലും വളര്‍ച്ചയിലും ഏറ്റവുമധികം പങ്കുവഹിക്കുന്നതും പ്രയാസമനുഭവിക്കുന്നതും മാതാവാണ്. അതുകൊണ്ടുതന്നെ മനുഷ്യന്‍ ഭൂമിയില്‍ ഏറ്റവുമധികം അനുസരിക്കുകയും ആദരിക്കുകയും അംഗീകരിക്കുകയും ചെയ്യേണ്ടത് മാതാവിനെയാണ്.
ഒരാള്‍ പ്രവാചകസന്നിധിയില്‍ വന്ന് ചോദിച്ചു: 'ദൈവദൂതരേ, എന്റെ ഏറ്റവും മെച്ചപ്പെട്ട സഹവാസത്തിന് അര്‍ഹന്‍ ആരാണ്?'' അവിടന്ന് അരുള്‍ ചെയ്തു: 'നിന്റെ മാതാവ്''. അയാള്‍ ചോദിച്ചു: 'പിന്നെ ആരാണ്?'' പ്രവാചകന്‍ പ്രതിവചിച്ചു: 'നിന്റെ മാതാവ്''. അയാള്‍ വീണ്ടും ചോദിച്ചു: 'പിന്നെ ആരാണ്'' നബി അറിയിച്ചു: 'നിന്റെ മാതാവ് തന്നെ.'' അയാള്‍ ചോദിച്ചു: 'പിന്നെ ആരാണ്?'' പ്രവാചകന്‍ പറഞ്ഞു: 'നിന്റെ പിതാവ്''(ബുഖാരി, മുസ്‌ലിം).
വിശുദ്ധ ഖുര്‍ആന്‍ മാതാപിതാക്കളെ ഒരുമിച്ച് പരാമര്‍ശിച്ച ഒന്നിലേറെ സ്ഥലങ്ങളില്‍ എടുത്തുപറഞ്ഞത് മാതാവിന്റെ സേവനമാണ്. 'മനുഷ്യനോട് അവന്റെ മാതാപിതാക്കളുടെ കാര്യം നാം ഉപദേശിച്ചിരിക്കുന്നു. കടുത്ത ക്ഷീണത്തോടെയാണ് മാതാവ് അവനെ ഗര്‍ഭം ചുമക്കുന്നത്. അവന്റെ മുലകുടി നിര്‍ത്താന്‍ രണ്ടു വര്‍ഷം വേണം. അതിനാല്‍ നീ എന്നോടും നിന്റെ മാതാപിതാക്കളോടും നന്ദികാണിക്കണം''(31:14). 'മാതാപിതാക്കള്‍ക്ക് നന്മ ചെയ്യണമെന്ന് മനുഷ്യനോട് നാം കല്‍പിച്ചിരിക്കുന്നു. ഏറെ പ്രയാസത്തോടെയാണ് മാതാവ് അവനെ ഗര്‍ഭം ചുമന്നത്. കടുത്ത പാരവശ്യത്തോടെയാണ് പ്രസവിച്ചത്''(46:15). അതിനാല്‍ ഇസ്‌ലാമികവീക്ഷണത്തില്‍ പ്രഥമസ്ഥാനവും പരിഗണനയും മാതാവെന്ന സ്ത്രീക്കാണ്.
3. പുരുഷന്മാരേക്കാള്‍ വിവേകപൂര്‍വവും ഉചിതവുമായ സമീപനം സ്വീകരിക്കാന്‍ സാധിക്കുന്ന സ്ത്രീകളുണ്ടെന്ന് ഖുര്‍ആന്‍ ഉദ്ധരിക്കുന്ന ശേബാരാജ്ഞിയുടെ ചരിത്രം അസന്ദിഗ്ധമായി വ്യക്തമാക്കുന്നു. സുലൈമാന്‍ നബിയുടെ സന്ദേശം ലഭിച്ചപ്പോള്‍ അവര്‍ തന്റെ കൊട്ടാരത്തിലുള്ളവരുമായി എന്തുവേണമെന്ന് കൂടിയാലോചിച്ചു. വ്യക്തവും യുക്തവുമായ നിലപാട് സ്വീകരിക്കാനവര്‍ക്ക് സാധിച്ചില്ല. പക്വമായ അഭിപ്രായവും സമീപനവുമുണ്ടായത് ശേബാ രാജ്ഞിയുടെ ഭാഗത്തുനിന്നു തന്നെയായിരുന്നുവെന്ന് ഖുര്‍ആന്‍ വ്യക്തമാക്കുന്നു. (27: 29,44).
4. ആദ്യപാപത്തിന്റെ കാരണക്കാരി സ്ത്രീയാണെന്ന ജൂതെ്രെകസ്തവ സങ്കല്‍പത്തെ ഇസ്‌ലാം തീര്‍ത്തും നിരാകരിക്കുന്നു. ദൈവശാസന ലംഘിച്ച് ആദമും ഹവ്വായും വിലക്കപ്പെട്ട കനി ഭക്ഷിച്ചിട്ടുണ്ടെന്നു വ്യക്തമാക്കുന്ന ഖുര്‍ആന്‍ ഇതിന്റെ കുറ്റം പ്രധാനമായും ചുമത്തുന്നത് ആദമിലാണ്, ഹവ്വയിലല്ല. 'അങ്ങനെ ആദമും പത്‌നിയും ആ വൃക്ഷത്തിന്റെ ഫലം ഭക്ഷിച്ചു. തദ്ഫലമായി അപ്പോള്‍തന്നെ അവരുടെ നഗ്‌നത പരസ്പരം വെളിവായി. ഇരുവരും തോട്ടത്തിലെ ഇലകള്‍കൊണ്ട് തങ്ങളെ മറയ്ക്കാന്‍ തുടങ്ങി. ആദം തന്റെ നാഥനെ ധിക്കരിച്ചു. നേര്‍വഴിയില്‍നിന്ന് വ്യതിചലിച്ചു. പിന്നീട് ആദമിനെ തന്റെ നാഥന്‍ തെരഞ്ഞെടുത്തു. അദ്ദേഹത്തിന്റെ പശ്ചാത്താപം സ്വീകരിക്കുകയും സന്മാര്‍ഗമേകുകയും ചെയ്തു''(20:121122).
'നാം ഇതിനു മുമ്പ് ആദമിന് കല്‍പന കൊടുത്തിരുന്നു. പക്ഷേ, ആദം അത് മറന്നു. നാം അയാളില്‍ നിശ്ചയദാര്‍ഢ്യം കണ്ടില്ല''(20:115). ആദിപാപത്തിന്റെ കാരണക്കാരി പെണ്ണാണെന്ന പരമ്പരാഗതധാരണയെ ഖുര്‍ആനിവിടെ പൂര്‍ണമായും തിരുത്തുന്നു.
5. ഭോഗാസക്തിക്കടിപ്പെട്ട ഭൗതിക സമൂഹങ്ങള്‍ പലപ്പോഴും സ്ത്രീകളെ കൊടിയ പീഡനങ്ങള്‍ക്കിരയാക്കുക മാത്രമല്ല, പെണ്‍കുഞ്ഞുങ്ങളെ ക്രൂരമായി കൊലപ്പെടുത്താറുമുണ്ട്. സമകാലീന സമൂഹത്തിലെ സ്ഥിതിയും ഭിന്നമല്ല. ഇക്കാര്യത്തില്‍ ഇന്ത്യയില്‍ പ്രഥമസ്ഥാനം തമിഴ്‌നാടിനാണ്. അവിടത്തെ സേലം ജില്ലയിലെ ഉശിലാംപെട്ടി ശിശുഹത്യക്ക് കുപ്രസിദ്ധിയാര്‍ജിച്ച സ്ഥലമാണ്. അവിടെ പ്രസവിച്ചത് പെണ്ണിനെയാണെന്നറിഞ്ഞാല്‍ ഭര്‍ത്താവ് കാണാന്‍ പോലും വരില്ല. കൊന്നുകളഞ്ഞിട്ടുവാ എന്ന അറിയിപ്പാണ് കിട്ടുക. അതിനാല്‍ വീട്ടുകാര്‍ കുട്ടിയുടെ കഥ കഴിച്ചശേഷം തള്ളയെ ഭര്‍ത്തൃഭവനത്തിലേക്കയക്കുന്നു.
പെണ്‍കുഞ്ഞുങ്ങള്‍ പിറക്കാതിരിക്കാന്‍ അവരെ ഗര്‍ഭപാത്രത്തില്‍വെച്ച് കൊലപ്പെടുത്തുന്ന രീതി ലോകത്തിന്റെ ഇതര ഭാഗങ്ങളിലെന്നപോലെ ഇന്ത്യയിലും വളരെ വ്യാപകമത്രെ. 'ഇന്ത്യയില്‍ ഭ്രൂണഹത്യക്കിരയായി ഒരു വര്‍ഷം അമ്പത് ലക്ഷം പെണ്‍കുഞ്ഞുങ്ങള്‍ മരിക്കുന്നുണ്ടെന്ന് ഇന്ത്യന്‍ മെഡിക്കല്‍ അസോസിയേഷന്‍ ഭാരവാഹികള്‍ വ്യക്തമാക്കുന്നു. ഔദ്യോഗിക കണക്കനുസരിച്ച് ഇത് ഇരുപത് ലക്ഷമാണ്.
ഈ ക്രൂരകൃത്യം നബിതിരുമേനിയുടെ ആഗമന കാലത്ത് അറേബ്യയിലെ ചില ഗോത്രങ്ങളിലും നിലനിന്നിരുന്നു. ഖുര്‍ആന്‍ പറയുന്നു: 'അവരിലൊരാള്‍ക്ക് പെണ്‍കുട്ടി പിറന്നതായി സുവാര്‍ത്ത ലഭിച്ചാല്‍ കൊടിയ ദുഃഖം കടിച്ചിറക്കി അവന്റെ മുഖം കറുത്തിരുളുന്നു. അവന്‍ ജനങ്ങളില്‍നിന്ന് ഒളിച്ചു നടക്കുന്നു; ഈ ചീത്ത വാര്‍ത്ത അറിഞ്ഞശേഷം ആരെയും അഭിമുഖീകരിക്കാതിരിക്കാന്‍. മാനഹാനി സഹിച്ച് അതിനെ വളര്‍ത്തേണമോ അതോ അവളെ ജീവനോടെ കുഴിച്ചുമൂടേണമോ എന്ന് അയാള്‍ ആലോചിച്ചുകൊണ്ടിരിക്കുന്നു''(16: 58, 59).
ഇസ്‌ലാം ഇതിനെ കഠിനമായി വിലക്കുകയും ഗുരുതരമായ കുറ്റമായി പ്രഖ്യാപിക്കുകയും ചെയ്തു. 'ജീവനോടെ കുഴിച്ചുമൂടപ്പെട്ട പെണ്‍കുഞ്ഞിനോട്, അവളെന്ത് അപരാധത്തിന്റെ പേരിലാണ് വധിക്കപ്പെട്ടതെന്ന് ചോദിക്കപ്പെടുന്ന' വിചാരണനാളിനെ അഭിമുഖീകരിക്കേണ്ടി വരുമെന്ന താക്കീത് നല്‍കുകയും ചെയ്തു (81: 8,9).
ഇങ്ങനെ ഇസ്‌ലാം സ്ത്രീകളുടെ ജീവിക്കാനുള്ള അവകാശം ഉറപ്പുവരുത്തി. പെണ്‍കുഞ്ഞുങ്ങളെ ഹനിക്കുന്ന ഹീനവൃത്തിക്ക് അറുതിവരുത്തി.
6. എക്കാലത്തും എവിടെയും അവഗണിക്കപ്പെടുന്ന വിഭാഗമാണ് സ്ത്രീകളെന്നതിനാല്‍ ഇസ്‌ലാം അവര്‍ക്ക് പ്രത്യേക പരിഗണനയും പ്രാധാന്യവും നല്‍കി. പ്രവാചകന്‍ പറഞ്ഞു: 'ഒരാള്‍ക്ക് രണ്ടു പെണ്‍കുട്ടികളുണ്ടാവുകയും അയാളവരെ നല്ലനിലയില്‍ പരിപാലിക്കുകയും ചെയ്താല്‍ അവര്‍ കാരണമായി അയാള്‍ സ്വര്‍ഗാവകാശിയായിത്തീരും''(ബുഖാരി).
'മൂന്നു പെണ്‍മക്കളോ സഹോദരിമാരോ കാരണമായി പ്രാരാബ്ധമനുഭവിക്കുന്നവന് സ്വര്‍ഗം ലഭിക്കാതിരിക്കില്ല'' (ത്വഹാവി). 'നിങ്ങള്‍ നിങ്ങളുടെ മക്കള്‍ക്കിടയില്‍ ദാനത്തില്‍ തുല്യത പുലര്‍ത്തുക. ഞാന്‍ ആര്‍ക്കെങ്കിലും പ്രത്യേകത കല്‍പിക്കുന്നവനായിരുന്നുവെങ്കില്‍ സ്ത്രീകള്‍ക്ക് മുന്‍ഗണന നല്‍കുമായിരുന്നു'' (ത്വബ്‌റാനി).
ഈ വിധം പ്രകൃതിപരമായ പ്രത്യേകതകള്‍ പൂര്‍ണമായും പരിഗണിച്ചുള്ള സ്ഥാനപദവികളും അവകാശബാധ്യതകളുമാണ് ഇസ്‌ലാം സ്ത്രീപുരുഷന്മാര്‍ക്ക് കല്‍പിക്കുന്നത്. അതുകൊണ്ടുതന്നെ മാതൃത്വത്തിന് മഹത്വമേകുകയും അതിനെ അത്യധികം ആദരിക്കുകയും ചെയ്യുന്നു. സ്ത്രീയുടെ ഏറ്റവും വലിയ സവിശേഷതയും മാതൃത്വമത്രെ. അമേരിക്കന്‍ മനശ്ശാസ്ത്ര വിദഗ്ധനായ തിയോഡര്‍ റൈക്ക് 'സ്ത്രീ പുരുഷന്മാര്‍ക്കിടയിലെ വൈകാരിക വൈജാത്യങ്ങള്‍' എന്ന ഗ്രന്ഥത്തില്‍ മാതൃത്വത്തിലഭിമാനിക്കുന്ന ഒരു സ്ത്രീയുടെ വാക്കുകള്‍ ഇങ്ങനെ ഉദ്ധരിക്കുന്നു: 'ധൈഷണികരംഗത്തും ഇതര മേഖലകളിലുമുള്ള പുരുഷന്റെ പ്രത്യേകത സങ്കോചലേശമന്യേ ഞങ്ങളംഗീകരിക്കുന്നു. പക്ഷേ, ഞങ്ങള്‍ സ്ത്രീകള്‍ അതിനേക്കാള്‍ എത്രയോ പ്രധാനപ്പെട്ട ഒന്നുകൊണ്ട് അനുഗൃഹീതരാണ്. ഞങ്ങളില്ലെങ്കില്‍ മനുഷ്യരാശി വേരറ്റുപോകും. മക്കള്‍ക്ക് ജന്മം നല്‍കുന്നത് ഞങ്ങളാണ്. വരുംതലമുറകളുടെ സാന്നിധ്യം അതുവഴി ഞങ്ങള്‍ ഉറപ്പുവരുത്തുന്നു''.
എന്നാല്‍ പെണ്ണ് പോലും ഇന്ന് മാതൃത്വത്തിന്റെ മഹത്വത്തെപ്പറ്റി ബോധവതിയല്ല. താന്‍ ഇത്ര കുട്ടികളെ ഗര്‍ഭം ചുമക്കുകയും പ്രസവിക്കുകയും പോറ്റിവളര്‍ത്തുകയും ചെയ്തിരിക്കുന്നുവെന്ന് അഭിമാനത്തോടെ പറയുന്ന സ്ത്രീകളിന്ന് വളരെ വിരളമാണ്. മനുഷ്യന്റെ വിലയിടിവാണിതിനു കാരണം. മനുഷ്യന്‍ ഈ വിധം തീരേ വില കുറഞ്ഞ വസ്തുവായി മാറിയതിനാല്‍ അവനെ ഗര്‍ഭം ചുമക്കുന്നതും പ്രസവിക്കുന്നതും ഹോട്ടലിലെ റിസപ്ഷനിസ്‌റിന്റെ ജോലിയേക്കാള്‍ തരംതാണതായി സ്ത്രീകള്‍ക്ക് തോന്നി. അതിനാല്‍ പലര്‍ക്കുമിന്ന് ഗര്‍ഭം ചുമക്കാനും പ്രസവിക്കാനും മടിയാണ്. അഥവാ ഒന്നോ രണ്ടോ കുഞ്ഞിന് ജന്മം നല്‍കിയാല്‍ തന്നെ വേണ്ട രീതിയില്‍ വളര്‍ത്താനവര്‍ ഒരുക്കമല്ല. അവരുടെ സംരക്ഷണം ആയമാരെ ഏല്‍പിക്കുന്നു. അവര്‍ കുഞ്ഞുങ്ങളെ സംരക്ഷിക്കുന്നത് മാസംതോറും ലഭിക്കുന്ന വേതനം പ്രതീക്ഷിച്ചാണ്. അലിജാ അലി ഇസ്സത്ത് ബെഗോവിച്ച് പറഞ്ഞപോലെ മാതാപിതാക്കള്‍ക്ക് മക്കള്‍ വ്യക്തിത്വമുള്ള അസ്തിത്വമാണ്; ആയമാര്‍ക്ക് സാധനങ്ങളില്‍ ഒരു സാധനവും. അതിനാല്‍ ആയമാര്‍ അവരെ കളിപ്പിക്കുന്നത് യന്ത്രങ്ങളുടെ ചക്രങ്ങള്‍ തിരിക്കുന്നതുപോലെയും കുളിപ്പിക്കുന്നത് യന്ത്രങ്ങള്‍ തേച്ചുമിനുക്കുന്നതുപോലെയും തീര്‍ത്തും നിര്‍വികാരമായിരിക്കും.
മാതൃത്വം അവഗണിക്കപ്പെട്ടതിന്റെ അനിവാര്യമായ ദുരന്തം ലോകമെങ്ങുമിന്ന് പ്രകടമാണ്. സോവിയറ്റ് ഭരണാധികാരിയായിരുന്ന മിഖായേല്‍ ഗോര്‍ബച്ചേവ് തന്റെ വിശ്വവിഖ്യാതമായ പെരിസ്‌ത്രോയിക്കയില്‍ എഴുതുന്നു: 'ഞങ്ങളുടെ വിഷമകരവും വീരോചിതവുമായ ചരിത്രത്തിന്റെ വര്‍ഷങ്ങളില്‍ അമ്മയെന്ന നിലയിലും ഗൃഹനായികയെന്ന നിലയിലും സ്വന്തം കുട്ടികളെ വിദ്യാഭ്യാസം ചെയ്യിക്കുകയെന്ന ഒഴിച്ചുകൂടാനാവാത്ത ജോലിയും സ്ത്രീകള്‍ക്കുള്ള സ്ഥാനത്തുനിന്ന് ഉയര്‍ന്നുവരുന്ന സ്ത്രീകളുടെ പ്രത്യേകാവകാശങ്ങള്‍ക്കും ആവശ്യങ്ങള്‍ക്കും പരിഗണന നല്‍കുന്നതില്‍ ഞങ്ങള്‍ പരാജയപ്പെട്ടു. ശാസ്ത്രീയ ഗവേഷണങ്ങളിലേര്‍പ്പെടുകയും നിര്‍മാണ സ്ഥലങ്ങളിലും ഉല്‍പാദനങ്ങളിലും സേവനതുറകളിലും പണിയെടുക്കുകയും സര്‍ഗാത്മക പ്രവര്‍ത്തനങ്ങളില്‍ പങ്കെടുക്കുകയും ചെയ്യുന്നതിനാല്‍ സ്ത്രീകള്‍ക്ക് വീട്ടില്‍ അവരുടെ ദൈനംദിന കടമകള്‍ നിര്‍വഹിക്കുന്നതിന് വീട്ടു ജോലി, കുട്ടികളെ വളര്‍ത്തല്‍, നല്ല കുടുംബാന്തരീക്ഷം സൃഷ്ടിക്കല്‍മതിയായ സമയം കിട്ടാതെ വരുന്നു. ഞങ്ങളുടെ പല പ്രശ്‌നങ്ങള്‍ക്കും കുട്ടികളുടെയും യുവജനങ്ങളുടെയും പെരുമാറ്റങ്ങളുടെ ധാര്‍മിക മൂല്യങ്ങളിലും സംസ്‌കാരത്തിലും ഉല്‍പാദനത്തിലുമുള്ള പ്രശ്‌നങ്ങള്‍ക്ക്  ഭാഗികമായ കാരണം ദുര്‍ബലമാകുന്ന കുടുംബബന്ധങ്ങളും കുടുംബപരമായ ഉത്തരവാദിത്തങ്ങളോടുള്ള തണുത്ത സമീപനങ്ങളുമാണെന്ന് ഞങ്ങള്‍ കണ്ടെത്തിയിരിക്കുന്നു. എല്ലാ കാര്യത്തിലും സ്ത്രീയെ പുരുഷനു തുല്യമാക്കണമെന്ന ഞങ്ങളുടെ ആത്മാര്‍ഥവും രാഷ്ട്രീയമായി നീതീകരിക്കത്തക്കതുമായ ആഗ്രഹത്തിന്റെ ഫലമാണ് ഈ വിരോധാഭാസം. ഇപ്പോള്‍ പെരിസ്‌ത്രോയിക്കയുടെ പ്രക്രിയയില്‍ ഈ കുറവ് ഞങ്ങള്‍ തരണം ചെയ്യാന്‍ തുടങ്ങിയിരിക്കുന്നു. സ്ത്രീകള്‍ക്ക് സ്ത്രീകളെന്ന നിലയ്ക്കുള്ള അവരുടെ തനിയായ ദൗത്യത്തിലേക്ക് മടങ്ങാന്‍ സാധ്യമാക്കുന്നതിന് എന്തു ചെയ്യണമെന്ന പ്രശ്‌നത്തെപ്പറ്റി, പത്രങ്ങളിലും പൊതു സംഘടനകളിലും തൊഴില്‍ സ്ഥലത്തും വീട്ടിലും ഇപ്പോള്‍ ചൂടുപിടിച്ച വാദപ്രതിവാദങ്ങള്‍ നടക്കുന്നത് അതിനാലാണ്''.
സ്ത്രീയുടെ ശാരീരിക സവിശേഷതകള്‍ പരിഗണിക്കുകയോ മാതൃത്വത്തിന്റെ മഹത്വം അംഗീകരിക്കുകയോ ചെയ്യാത്ത പ്രകൃതിവിരുദ്ധമായ ഏതു വ്യവസ്ഥിതിയിലും മനുഷ്യന്റെ വിലയിടിയുകയും മനസ്സിന്റെ സ്വാസ്ഥ്യം നഷ്ടപ്പെടുകയും കുടുംബഘടന ശിഥിലമാവുകയും സമൂഹത്തില്‍നിന്ന് സമാധാനം വിടപറയുകയും വ്യക്തികള്‍ ആള്‍ക്കൂട്ടത്തിലും ഒറ്റപ്പെട്ട് ഏകാന്തതയുടെ കൊടിയ വ്യഥക്ക് വിധേയരാവുകയും ചെയ്യുക അനിവാര്യമാണ്. പ്രകൃതിപരമായ പ്രത്യേകതകള്‍ പൂര്‍ണമായും പരിഗണിക്കുന്ന ഇസ്‌ലാമിക വ്യവസ്ഥ ഇത്തരം ന്യൂനതകളില്‍നിന്ന് തീര്‍ത്തും മുക്തവും പ്രതിസന്ധികള്‍ക്കിടവരുത്താത്തതുമത്രെ.
\chapter{ജിസ്‌യ }
\section{ ഇസ്‌ലാമികരാഷ്ട്രം അമുസ്‌ലിം പൗരന്മാരില്‍നിന്ന് ജിസ്‌യ എന്ന മതനികുതി ഈടാക്കാറില്ലേ? അത് കടുത്ത വിവേചനവും അനീതിയുമല്ലേ?}


ജിസ് യ മതനികുതിയല്ല. ആണെന്ന ധാരണ അബദ്ധമാണ്. ഏറെ തെറ്റിദ്ധരിക്കപ്പെട്ട വിഷയമെന്ന നിലയില്‍ ഇത് അല്‍പം വിശദീകരിക്കുന്നത് ഗുണകരമായിരിക്കുമെന്ന് പ്രതീക്ഷിക്കുന്നു. മുസ്‌ലിംകള്‍ തങ്ങളുടെ കാര്‍ഷികവരുമാനത്തിന്റെ പത്തു ശതമാനവും ഇതര സാമ്പത്തിക വരുമാനങ്ങളുടെ രണ്ടര ശതമാനവും പൊതു ഖജനാവില്‍ അടക്കാന്‍ ബാധ്യസ്ഥരാണ്. ഇത് മതപരമായ ആരാധനാകര്‍മം കൂടിയായതിനാല്‍ ഇതര മതവിഭാഗക്കാരുടെ മേല്‍ നിര്‍ബന്ധമാക്കാന്‍ നിര്‍വാഹമില്ല. കാരണം അതവരുടെ മതസ്വാതന്ത്യ്രത്തിന്റെ നിഷേധമായിരിക്കും. അതിനാല്‍ സമൂഹത്തില്‍ സാമ്പത്തിക സന്തുലിതത്വം നിലനിര്‍ത്താനായി അമുസ്‌ലിം പൗരന്മാരുടെ മേല്‍ ഇസ്‌ലാമിലെ മതചടങ്ങുകളുമായി ബന്ധമില്ലാത്ത മറ്റൊരു നികുതി ചുമത്തുകയാണുണ്ടായത്. അതാണ് ജിസ് യ. മുസ്‌ലിംകളില്‍നിന്ന് രാഷ്ട്രം നിര്‍ബന്ധമായും പിരിച്ചെടുക്കുന്ന സകാത്തിനു പകരമുള്ള നികുതിയാണത്.
സമ്പത്തുള്ള മുസ്‌ലിംകളെല്ലാം സകാത്ത് നല്‍കാന്‍ ബാധ്യസ്ഥരാണ്. സ്ത്രീകളും കുട്ടികളും വൃദ്ധന്മാരും രോഗികളും ഉള്‍പ്പെടെ ആരും തന്നെ അതില്‍നിന്ന് മുക്തരല്ല. എന്നാല്‍ അതിനെയപേക്ഷിച്ച് ജിസ് യ യില്‍ ഒട്ടേറെ ആനുകൂല്യങ്ങളും ഇളവുകളുമുണ്ട്. സ്ത്രീകള്‍, കുട്ടികള്‍, അന്ധന്മാര്‍, വൃദ്ധന്മാര്‍, ഭ്രാന്തന്മാര്‍, മാറാരോഗികള്‍, മഠങ്ങളിലെ സന്യാസിമാര്‍, പുരോഹിതന്മാര്‍ പോലുള്ളവരില്‍നിന്നൊന്നും ജിസ്യ പിരിക്കുന്നതല്ല. അതിനാല്‍ ജിസ്യ അമുസ്‌ലിം പൗരന്മാരോടുള്ള വിവേചനമോ അനീതിയോ അല്ല. അവര്‍ക്ക് സാമ്പത്തികമായി ഇളവ് ലഭിക്കാനുള്ള ഉപാധിയാണ്.
ആരെങ്കിലും മുസ്‌ലിംകളെപ്പോലെ സകാത്ത് നല്‍കാന്‍ സ്വയം സന്നദ്ധമായി മുന്നോട്ടു വരികയാണെങ്കില്‍ അവരെ ഇസ്‌ലാമിക രാഷ്ട്രം ജിസ് യ യില്‍നിന്ന് ഒഴിവാക്കുന്നതാണ്. ചരിത്രത്തിലിതിന് ഏറെ ഉദാഹരണങ്ങള്‍ കാണാം. ഒന്നിവിടെ ഉദ്ധരിക്കാം: സര്‍ തോമസ് ആര്‍ണള്‍ഡ് എഴുതുന്നു: 'അവരോട് (തഗ്ലിബ് ഗോത്രം) അമുസ്‌ലിം ഗോത്രങ്ങള്‍ക്ക് നല്‍കുന്ന സംരക്ഷണത്തിനു പകരമായി ചുമത്തുന്ന കരം ജിസ് യ അടക്കാനും അദ്ദേഹം (ഉമറുല്‍ ഫാറൂഖ്) ആവശ്യപ്പെട്ടു. എന്നാല്‍ ജിസ് യ കൊടുക്കുന്നത് അപമാനമായി കരുതിയ തഗ്ലിബ് ഗോത്രം തങ്ങളെ മുസ്‌ലിംകളെപ്പോലെ നികുതി (സകാത്ത്) അടക്കാന്‍ അനുവദിക്കണമെന്നാവശ്യപ്പെട്ടു. ഖലീഫ അതനുവദിക്കുകയും അവര്‍ മുസ്‌ലിംകളെപ്പോലെ ജിസ് യയുടെ ഇരട്ടി വരുന്ന സംഖ്യ ഖജനാവിലേക്കടക്കുകയും ചെയ്തു'' (ഇസ്‌ലാം പ്രബോധനവും പ്രചാരവും, പുറം 62).
ഇസ്‌ലാമികരാഷ്ട്രത്തിലെ മുഴുവന്‍ പൗരന്മാരെയും സംരക്ഷിക്കാന്‍ ഭരണകൂടം ബാധ്യസ്ഥമാണ്. അതിനാല്‍ മുസ്‌ലിംകളുടെ മാത്രമല്ല, അമുസ്‌ലിംകളുടെയും ജീവനും സ്വത്തും അഭിമാനവും സംരക്ഷിക്കാന്‍ നിര്‍ബന്ധ സൈനികസേവനം നിര്‍വഹിക്കാന്‍ മുസ്‌ലിംകള്‍ ബാധ്യസ്ഥരായിരുന്നു. ഈ വിധം സംരക്ഷണം ഉറപ്പു നല്‍കുന്നതിനും പട്ടാളസേവനത്തില്‍നിന്ന് ഒഴിവാക്കുന്നതിനും പകരമായാണ് അവരില്‍നിന്ന് ജിസ് യ ഈടാക്കിയിരുന്നത്. സൈനികസേവനത്തിന് അക്കാലത്ത് ശമ്പളമുണ്ടായിരുന്നില്ലെന്നത് പ്രത്യേകം പ്രസ്താവ്യമത്രെ. എപ്പോഴെങ്കിലും രാജ്യനിവാസികള്‍ക്ക് സംരക്ഷണം നല്‍കാന്‍ ഭരണകൂടത്തിന് സാധിക്കാതെ വന്നാല്‍ ജിസ് യ തിരിച്ചുനല്‍കുക പതിവായിരുന്നു. അപ്രകാരം തന്നെ സൈനികസേവനത്തിന് സ്വയം സന്നദ്ധരായി മുന്നോട്ടുവരുന്നവരെയും ജിസ് യ യില്‍നിന്ന് പൂര്‍ണമായും ഒഴിവാക്കിയിരുന്നു. സര്‍ തോമസ് ആര്‍ണള്‍ഡ് എഴുതുന്നു: 'ചിലര്‍ നമ്മെ വിശ്വസിപ്പിക്കുവാന്‍ ശ്രമിക്കുന്നതുപോലെ മുസ്‌ലിം വിശ്വാസം സ്വീകരിക്കാന്‍ വിസമ്മതിച്ചതിന്റെ ശിക്ഷയായി െ്രെകസ്തവരുടെ മേല്‍ ചുമത്തപ്പെടുന്നതല്ല ഈ നികുതി. എല്ലാ അമുസ്‌ലിം പൗരന്മാരും അടക്കേണ്ടതായിരുന്നു അത്. മതപരമായ കാരണങ്ങളാല്‍ നിര്‍ബന്ധ സൈനിക സേവനത്തില്‍നിന്ന് അവര്‍ ഒഴിവാക്കപ്പെട്ടിരുന്നു. മുസ്‌ലിംകള്‍ നല്‍കിയിരുന്ന സംരക്ഷണത്തിന് പകരമായാണ് അവര്‍ ജിസ് യ കൊടുക്കേണ്ടി വന്നത്....
'തുര്‍ക്കീ ഭരണകാലത്ത് സൈന്യത്തില്‍ സേവനമനുഷ്ഠിച്ചിരുന്ന ക്രിസ്ത്യാനികളും ജിസ് യ യില്‍നിന്ന് ഒഴിവാക്കപ്പെട്ടിരുന്നതായി കാണാം. കൊറിന്‍ത് കരയിടുക്കിലേക്ക് നയിക്കുന്ന സിത്തിറോണ്‍, ഗറാനിയ ചുരങ്ങള്‍ കാക്കാന്‍ ഒരു സംഘം സായുധരെ നല്‍കാമെന്ന വ്യവസ്ഥയില്‍ അല്‍ബേനിയന്‍ െ്രെകസ്തവവര്‍ഗമായ മെഗാരികളെ തുര്‍ക്കികള്‍ ജിസ് യ യില്‍നിന്ന് ഒഴിവാക്കിയിരുന്നു. തുര്‍ക്കീ സൈന്യത്തിന്റെ മുമ്പേ പോയി നിരത്തുകളും പാലങ്ങളും നന്നാക്കിയിരുന്ന ക്രിസ്തീയ സംഘത്തില്‍നിന്ന് ജിസ് യ ഈടാക്കിയിരുന്നില്ലെന്നു മാത്രമല്ല, കരം വാങ്ങാതെ അവര്‍ക്ക് ഭൂമി പതിച്ചുകൊടുക്കുകകൂടി ചെയ്തിരുന്നു. ഹൈസ്രയിലെ ക്രിസ്ത്യാനികള്‍ സുല്‍ത്താന് ജിസ് യ നല്‍കിയിരുന്നില്ല. പകരമായി അവര്‍ 250 ദൃഢഗാത്രരായ നാവികരെ തുര്‍ക്കിപ്പടക്കു നല്‍കി.
'ആര്‍മത്തോളി എന്നു വിളിക്കപ്പെടുന്ന തെക്കന്‍ റുമാനിയക്കാരാണ് പതിനാറും പതിനേഴും നൂറ്റാണ്ടുകളില്‍ തുര്‍ക്കി സൈന്യത്തില്‍ മുഖ്യഘടകമായിരുന്നത്. സ്‌കൂട്ടാരിക്കു വടക്കുള്ള പര്‍വതനിരകളില്‍ വസിച്ചിരുന്ന മിര്‍ദികള്‍ എന്ന അല്‍ബേനിയന്‍ കത്തോലിക്കര്‍ കരത്തില്‍നിന്നൊഴിവാക്കപ്പെട്ടിരുന്നു. യുദ്ധവേളയില്‍ സായുധ സംഘത്തെ നല്‍കാമെന്നായിരുന്നു അവരുടെ പ്രതിജ്ഞ. അതേപോലെ ഗ്രീക്ക് ക്രിസ്ത്യാനികളെയും ജിസ് യ യില്‍നിന്നൊഴിവാക്കി. കോണ്‍സ്‌റാന്റിനോപ്പിളിലേക്ക് ശുദ്ധജലം കൊണ്ടുവന്നിരുന്ന കല്‍ക്കുഴലുകള്‍ അവരായിരുന്നു സംരക്ഷിച്ചിരുന്നത്. നഗരത്തിലെ വെടിമരുന്നുശാലക്ക് കാവലിരുന്നവരേയും കരത്തില്‍നിന്നൊഴിവാക്കിയിരുന്നു. എന്നാല്‍, ഈജിപ്തിലെ ഗ്രാമീണ കര്‍ഷകര്‍ സൈനികസേവനത്തില്‍നിന്ന് ഒഴിവാക്കപ്പെട്ടപ്പോള്‍ അവരുടെ മേല്‍ ക്രിസ്ത്യാനികളെപ്പോലെ കരം ചുമത്തുകയും ചെയ്തു.''(സര്‍ തോമസ് ആര്‍ണള്‍ഡ്, ഇസ്‌ലാം: പ്രബോധനവും പ്രചാരവും, പുറം 7376).
നബിതിരുമേനിയുടെ കാലത്ത് മദീനയിലെ അമുസ്‌ലിം വിഭാഗങ്ങള്‍ രാഷ്ട്രത്തിന്റെ പ്രതിരോധപ്രവര്‍ത്തനങ്ങളില്‍ പങ്കാളികളായിരുന്നതിനാല്‍ അവരില്‍നിന്ന് ജിസ് യ ഈടാക്കിയിരുന്നില്ല.
ചുരുക്കത്തില്‍, പാശ്ചാത്യര്‍ പ്രചരിപ്പിക്കുകയും അവരുടെ മെഗാഫോണുകളായി മാറിയ മറ്റുള്ളവര്‍ ഏറ്റുപിടിക്കുകയും ചെയ്യുന്നതുപോലെ ജിസ് യ ഒരു മതനികുതിയല്ല. ഒരിക്കലും അങ്ങനെ ആയിരുന്നിട്ടുമില്ല. യഥാര്‍ഥത്തിലിത് യുദ്ധനികുതിയാണ്. കഴിവും കായികബലവും ഉണ്ടായിരുന്നിട്ടും സൈനികസേവനമനുഷ്ഠിക്കാന്‍ സന്നദ്ധമാവാതെ മാറിനിന്നവരാണ് അത് നല്‍കേണ്ടിവന്നിരുന്നത്. നിര്‍ബന്ധ സൈനിക സേവനം നിലനിന്നിരുന്ന ഘട്ടത്തില്‍ അതില്‍നിന്നൊഴിവാക്കുകയും അതോടൊപ്പം ശാരീരികവും സാമ്പത്തികവുമായ സുരക്ഷിതത്വം അനുഭവിക്കുകയും ചെയ്തിരുന്നതിന്റെ പ്രതിഫലമായിരുന്നു അത്. എന്നാല്‍ ഇസ്‌ലാമികരാഷ്ട്രത്തിലെ മുസ്‌ലിം പൗരന്മാര്‍ നിര്‍ബന്ധ സൈനിക സേവനമനുഷ്ഠിച്ചാലും ഭരണകൂടത്തിനു സകാത്ത് നല്‍കാന്‍ ബാധ്യസ്ഥരാണ്. എക്കാലത്തും സകാത്ത് സംഖ്യ ജിസ് യ യേക്കാള്‍ വളരെ കൂടുതലായിരുന്നുവെന്ന വസ്തുത വിസ്മരിക്കാവതല്ല. അമുസ്‌ലിം പൗരന്മാര്‍ സകാത്ത് നല്‍കുകയോ സൈനികസേവനമനുഷ്ഠിക്കാന്‍ സന്നദ്ധമാവുകയോ ചെയ്തപ്പോഴെല്ലാം അവരെ ജിസ് യയില്‍നിന്നൊഴിവാക്കിയിരുന്നു. സൈനികവൃത്തി വേതനമുള്ള തൊഴിലായി മാറിയ ഇക്കാലത്തും ഇസ്‌ലാമിക രാഷ്ട്രം അമുസ്‌ലിം പൗരന്മാരുടെ മേല്‍ ജിസ് യ ചുമത്തുന്നതല്ല. അതിനാല്‍ മതന്യൂനപക്ഷങ്ങള്‍ ഇസ്‌ലാമികരാഷ്ട്രത്തില്‍ ഒരുവിധ വിവേചനവും അനുഭവിക്കുകയില്ലെന്നു മാത്രമല്ല, സകാത്തില്‍നിന്ന് ഒഴിവാക്കപ്പെടുന്നതിനാല്‍ മുസ്‌ലിംകളേക്കാള്‍ സാമ്പത്തിക സൗകര്യവും ആനുകൂല്യവും അനുഭവിക്കുകയും ചെയ്യുന്നു
\chapter{ഗള്‍ഫുനാടുകളിലെ ക്ഷേത്രവിലക്ക് }
 \section{ഗള്‍ഫ്‌നാടുകളില്‍ ഹിന്ദുക്കളെ ക്ഷേത്രം പണിയാന്‍ അനുവദിക്കാത്തതെന്തുകൊണ്ടാണ്?}

 ഗള്‍ഫ് നാടുകളില്‍ ഹിന്ദുപൌരന്മാരില്ല. അവിടെയുള്ള ഹിന്ദുക്കള്‍ ജോലിയാവശ്യാര്‍ഥം ചെന്ന വിദേശികളാണ്. മുസ്‌ലിംകളുള്‍പ്പെടെ വിദേശികളായ ആര്‍ക്കും അവിടെ സ്ഥലം വാങ്ങാനോ ആരാധനാലയം പണിയാനോ അനുവാദമില്ല. അഥവാ, വിദേശമുസ്‌ലിംകള്‍ക്ക് പള്ളിയുണ്ടാക്കാനും അവിടങ്ങളിലെ നിയമം അനുവദിക്കുന്നില്ല. വിദേശികള്‍ക്ക് ഇന്ത്യയിലോ ഇതര സെക്യുലര്‍ നാടുകളിലോ സ്വന്തമായ ആരാധനാലയം പണിയാന്‍ അനുവാദമില്ലാത്തതുപോലെത്തന്നെ.
എന്നിട്ടും ദീര്‍ഘകാലമായി ഗള്‍ഫ് നാടുകളില്‍ കഴിഞ്ഞുകൂടുന്ന സിന്ധി ഹിന്ദുക്കള്‍ക്ക് ആരാധനാലയം നിര്‍മിക്കാന്‍ ഭരണകൂടം പ്രത്യേകം അനുമതി നല്‍കുകയുണ്ടായി. മലയാളത്തിലെ ആര്‍.എസ്.എസ്. വാരിക കേസരി എഴുതുന്നു: 'മസ്‌കത്ത്, ബഹ്‌റൈന്‍, ദുബൈ എന്നീ ഗള്‍ഫു രാജ്യങ്ങള്‍ വിദേശക്കോയ്മക്ക് കീഴില്‍ ഉള്ളതുമുതല്‍ ഭാരതീയരും ഹിന്ദുക്കളുമായ സിന്ധികള്‍ ഇവിടങ്ങളില്‍ കുടിയേറിപ്പാര്‍ക്കുക കാരണം പ്രസ്തുത രാജ്യങ്ങളുടെ അഭിവൃദ്ധിയില്‍ ഗണ്യമായ പങ്കുവഹിച്ചിരുന്നു. പ്രസ്തുത യാഥാര്‍ഥ്യം മനസ്സിലാക്കി അവിടങ്ങളിലെ ഭരണാധിപന്മാര്‍ അവര്‍ക്ക് പ്രത്യേക പരിഗണനകള്‍ കൊടുത്തുപോന്നു''(6-4-1986).
'ദുബൈ ശൈഖ് അവര്‍ക്ക് വേണ്ടുന്ന എല്ലാ ആനുകൂല്യങ്ങളും നല്‍കുന്നതില്‍ പ്രത്യേകം ശ്രദ്ധിച്ചുപോന്നു. സിന്ധികളുടെ ആവശ്യാര്‍ഥം ഹിന്ദുക്കളുടെ ഒരു ക്ഷേത്രവും ഇഷ്ടദേവതകളെ വച്ചു പൂജിക്കാനുള്ള അനുവാദവും അതുപോലെ ഉത്സവാദികള്‍ കൊണ്ടാടാനുള്ള അംഗീകാരവും നല്‍കുകയുണ്ടായി. ഹിന്ദുക്കള്‍ക്ക് ശവസംസ്‌കാരം ചെയ്യാനുള്ള ഒരു പ്രത്യേക ശ്മശാനത്തിനുള്ള സ്ഥലവും കൂടി അനുവദിക്കുകയുണ്ടായി'' (കേസരി 5-1-1986). 
\chapter{മുസ്‌ലിം രാജ്യത്തില്‍ ക്ഷേത്രങ്ങള്‍ക്ക് അനുവാദമുണ്ടോ? }
\section{ മതേതരനാടുകളിലേതുപോലെയോ കൂടുതലായോ മതന്യൂനപക്ഷങ്ങള്‍ക്ക് ഇസ്‌ലാമികരാഷ്ട്രത്തില്‍ മതസ്വാതന്ത്യ്രമുണ്ടാകുമെന്ന് താങ്കള്‍ അവകാശപ്പെടുകയുണ്ടായല്ലോ? എന്നാല്‍ ലോകത്ത് ഏതെങ്കിലും മുസ്‌ലിം രാജ്യത്ത് ഹിന്ദുക്കള്‍ക്ക് ക്ഷേത്രം നിര്‍മിക്കാന്‍ അനുവാദമുണ്ടോ? പാകിസ്താനിലും ബംഗ്‌ളാദേശിലുമെല്ലാം ഉള്ളവതന്നെ തകര്‍ക്കുകയല്ലേ ചെയ്യുന്നത്?}

 സമൂഹത്തില്‍ നിലനില്‍ക്കുന്ന ഗുരുതരമായ തെറ്റിദ്ധാരണകളാണ് ഇത്തരം സംശയങ്ങള്‍ക്ക് കാരണം. മുസ്‌ലിംകള്‍ ഭൂരിപക്ഷമുള്ള നാടുകളെല്ലാം മതാധിഷ്ഠിത ഇസ്‌ലാമികരാഷ്ട്രങ്ങളാണെന്ന ധാരണ ശരിയല്ല. ഇസ്‌ലാമികവ്യവസ്ഥ യഥാവിധി നടപ്പാക്കപ്പെടുന്ന രാജ്യങ്ങള്‍ മാത്രമേ ഇസ്‌ലാമികരാഷ്ട്രമെന്ന വിശേഷണത്തിന് അര്‍ഹമാവുകയുള്ളൂ. നിലവിലുള്ള മുസ്‌ലിം നാടുകള്‍ അവ്വിധം ചെയ്യാത്തതിനാലാണ് ലോകത്തെവിടെയും മാതൃകാ ഇസ്‌ലാമിക രാഷ്ട്രങ്ങള്‍ കാണപ്പെടാത്തത്. ഭാഗികമായി ഇസ്‌ലാമികവ്യവസ്ഥ നടപ്പാക്കുന്ന നാടുകളുണ്ട്. അവ അത്രത്തോളമേ ഇസ്‌ലാമികമാവുകയുള്ളൂ.
ഇസ്‌ലാമികരാഷ്ട്രത്തിലെ അമുസ്‌ലിം പൗരന്മാര്‍ക്ക് തങ്ങളുടെ വിശ്വാസങ്ങള്‍ വച്ചുപുലര്‍ത്താനും ആരാധനാനുഷ്ഠാനങ്ങള്‍ നിര്‍വഹിക്കാനും ആചാരങ്ങള്‍ പിന്തുടരാനും സ്വാതന്ത്യ്രമുണ്ടായിരിക്കും. ആരുടെ മേലും ഇസ്‌ലാമിനെ അടിച്ചേല്‍പിക്കുകയോ ആരെയെങ്കിലും മതം മാറാന്‍ നിര്‍ബന്ധിക്കുകയോ ഇല്ല. അങ്ങനെ ചെയ്യുന്നത് ഇസ്‌ലാം കണിശമായി വിലക്കിയിരിക്കുന്നു: 'മതത്തില്‍ ഒരുവിധ നിര്‍ബന്ധവുമില്ല. സന്മാര്‍ഗം മിഥ്യാധാരണകളില്‍ നിന്ന് വേര്‍തിരിഞ്ഞ് കഴിഞ്ഞിരിക്കുന്നു.''(ഖു. 2: 256).
'നീ വിളംബരം ചെയ്യുക: ഇത് നിങ്ങളുടെ നാഥനില്‍നിന്നുള്ള സത്യമാകുന്നു. ഇഷ്ടമുള്ളവര്‍ക്കിത് സ്വീകരിക്കാം. ഇഷ്ടമുള്ളവര്‍ക്ക് നിഷേധിക്കാം.'' (ഖു. 18:29).
ദൈവദൂതന്മാര്‍ക്കുപോലും മതം സ്വീകരിക്കാന്‍ ആരെയും നിര്‍ബന്ധിക്കാന്‍ അനുവാദമുണ്ടായിരുന്നില്ല. അല്ലാഹു അറിയിക്കുന്നു: 'ജനങ്ങള്‍ വിശ്വാസികളാകാന്‍ നീ അവരെ നിര്‍ബന്ധിക്കുകയോ? ദൈവഹിതമില്ലാതെ ഒരാള്‍ക്കും വിശ്വസിക്കുക സാധ്യമല്ല''(ഖു. 6: 69).
'നബിയേ, നീ ഉദ്‌ബോധിപ്പിച്ചുകൊണ്ടിരിക്കുക. നീ ഉദ്‌ബോധകന്‍ മാത്രമാകുന്നു. നീ അവരെ നിര്‍ബന്ധിച്ച് വഴിപ്പെടുത്തുന്നവനൊന്നുമല്ല.''(ഖു. 88: 21,22).
ലോകത്തിലെ ആദ്യത്തെ ഇസ്‌ലാമികരാഷ്ട്രമായ മദീനയില്‍, അതിന്റെ സ്ഥാപകനായ നബിതിരുമേനി മതന്യൂനപക്ഷങ്ങള്‍ക്ക് അനുവദിച്ചതുപോലുള്ള സ്വാതന്ത്യ്രവും സൗകര്യവും മറ്റേതെങ്കിലും മതാധിഷ്ഠിത നാടുകളിലോ മതനിരപേക്ഷ രാജ്യങ്ങളിലോ കാണപ്പെടുമോയെന്ന് ന്യായമായും സംശയിക്കേണ്ടിയിരിക്കുന്നു. മദീന അംഗീകരിച്ച് പ്രഖ്യാപിച്ച പ്രമാണത്തില്‍ ഇങ്ങനെ കാണാം: 'നമ്മുടെ ഭരണസാഹോദര്യസീമയില്‍പെടുന്ന ജൂതന്മാര്‍ക്ക് വര്‍ഗാടിസ്ഥാനത്തിലുള്ള പക്ഷപാതപരമായ പെരുമാറ്റങ്ങളില്‍നിന്നും ദ്രോഹങ്ങളില്‍നിന്നും രക്ഷ നല്‍കും. നമ്മുടെ സഹായത്തിനും ദയാനിരതമായ സംരക്ഷണത്തിനും മുസ്‌ലിം സമുദായാംഗങ്ങളെപ്പോലെ അവര്‍ക്കും അവകാശമുണ്ട്. മുസ്‌ലിംകളുമായി ചേര്‍ന്ന് അവര്‍ ഏക ഘടനയുള്ള ഒരു രാഷ്ട്രമായിത്തീരും. മുസ്‌ലിംകളെപ്പോലെത്തന്നെ അവര്‍ക്കും സ്വതന്ത്രമായി തങ്ങളുടെ മതം ആചരിക്കാവുന്നതാണ്.''
സര്‍ തോമസ് ആര്‍ണള്‍ഡ് എഴുതുന്നു: 'മുഹമ്മദ് പല അറബ്‌െ്രെകസ്തവ ഗോത്രങ്ങളുമായും സന്ധിയിലേര്‍പ്പെട്ടിരുന്നു. അവര്‍ക്കദ്ദേഹം സംരക്ഷണവും സ്വന്തം മതമാചരിക്കാനുള്ള സ്വാതന്ത്യ്രവും നല്‍കി''(ഇസ്‌ലാം: പ്രബോധനവും പ്രചാരവും, പുറം 60).
പ്രവാചകന്റെ പാത പിന്തുടര്‍ന്ന് മുഴുവന്‍ മുസ്‌ലിംഭരണാധികാരികളും സ്വീകരിച്ച സമീപനവും ഇതുതന്നെ. ഇന്ത്യാ ചരിത്രത്തില്‍ ഏറെ തെറ്റിദ്ധരിക്കപ്പെടുകയും വിമര്‍ശനവിധേയനാവുകയും ചെയ്ത ഔറംഗസീബിന്റെ മതസമീപനത്തില്‍ പ്രകടമായിരുന്ന സഹിഷ്ണുതയെ സംബന്ധിച്ച് അലക്‌സാണ്ടര്‍ ഹാമില്‍ട്ടണ്‍ എഴുതുന്നു: 'ഹിന്ദുക്കള്‍ക്ക് പരിപൂര്‍ണമായ മതസ്വാതന്ത്യ്രം ലഭിക്കുന്നുണ്ടെന്നതിന് പുറമെ ഹൈന്ദവ രാജാക്കന്മാരുടെ കീഴിലായിരിക്കുമ്പോഴൊക്കെ അവര്‍ നടത്തിയിരുന്ന വ്രതങ്ങളും ഉത്സവങ്ങളും ആഘോഷിക്കാനുള്ള സൗകര്യങ്ങള്‍ ഉണ്ടായിരുന്നു. മീററ്റ് നഗരത്തില്‍ മാത്രം ഹൈന്ദവവിഭാഗത്തില്‍ നൂറില്‍പരം വ്യത്യസ്ത വിഭാഗങ്ങളുണ്ടെങ്കിലും അവര്‍ തമ്മില്‍ പ്രാര്‍ഥനകളുടെയോ സിദ്ധാന്തങ്ങളുടെയോ പേരില്‍ യാതൊരു വിധ വിവാദവും ഉണ്ടായിരുന്നില്ല. ഏതൊരാള്‍ക്കും അയാളാഗ്രഹിക്കുന്ന വിധം ദൈവാര്‍ച്ചനകള്‍ നടത്തുവാനും ആരാധിക്കാനുമുള്ള സ്വാതന്ത്യ്രമുണ്ടായിരുന്നു. മതധ്വംസനങ്ങള്‍ അജ്ഞാതമത്രെ'' (Alaxander Hamilton, A new Account of the East Indies, Vol. 1, PP. 159, 162, 163).).
മുസ്‌ലിം ഭരണാധികാരികള്‍ മതപരിവര്‍ത്തനത്തിന് നിര്‍ബന്ധിക്കുകയോ സമ്മര്‍ദം ചെലുത്തുകയോ ചെയ്തിരുന്നുവെങ്കില്‍ നീണ്ട നിരവധി നൂറ്റാണ്ടുകളുടെ ഭരണത്തിനു ശേഷവും മുസ്‌ലിംകളിവിടെ ന്യൂനപക്ഷമാകുമായിരുന്നില്ലെന്ന് സുവിദിതമാണല്ലോ.
ഇസ്‌ലാമിക ഭരണത്തില്‍ മതന്യൂനപക്ഷങ്ങള്‍ക്ക് പൂര്‍ണമായ ആരാധനാസ്വാതന്ത്യ്രം നല്‍കപ്പെട്ടിരുന്നു. നജ്‌റാനിലെ ക്രിസ്ത്യാനികളുമായി പ്രവാചകനുണ്ടാക്കിയ സന്ധിവ്യവസ്ഥകളില്‍ ഇങ്ങനെ കാണാം. 'നജ്‌റാനിലെ െ്രെകസ്തവര്‍ക്കും അവരുടെ സഹവാസികള്‍ക്കും ദൈവത്തിന്റെ അഭയവും ദൈവദൂതനായ മുഹമ്മദിന്റെ സംരക്ഷണോത്തരവാദിത്വവുമുണ്ട്. അവരുടെ ജീവന്‍, മതം, ഭൂമി, ധനം എന്നിവയ്ക്കും അവരില്‍ ഹാജറുള്ളവന്നും ഇല്ലാത്തവന്നും അവരുടെ ഒട്ടകങ്ങള്‍ക്കും നിവേദകസംഘങ്ങള്‍ക്കും കുരിശ്, ചര്‍ച്ച് പോലുള്ള മതചിഹ്നങ്ങള്‍ക്കും വേണ്ടിയാണിത്. നിലവിലുള്ള അവസ്ഥയില്‍ ഒരു മാറ്റവും വരുത്തുന്നതല്ല. അവരുടെ യാതൊരവകാശവും ഒരു മതചിഹ്നവും മാറ്റപ്പെടുന്നതല്ല. അവരുടെ പാതിരിയോ പുരോഹിതനോ ചര്‍ച്ച് സേവകനോ തന്റെ സ്ഥാനത്ത് നിന്ന് നീക്കപ്പെടുന്നതല്ല.''
ഒന്നാം ഖലീഫ അബൂബക്ര്! സ്വിദ്ദീഖ്(റ) ഹീറാവാസികളുമായി ഒപ്പുവച്ച സന്ധിവ്യവസ്ഥകളിലിങ്ങനെ പറയുന്നു: 'അവരുടെ ആരാധനാലയങ്ങളും കനീസുകളും ശത്രുക്കളില്‍നിന്ന് രക്ഷ നേടുന്ന കോട്ടകളും പൊളിച്ചുമാറ്റപ്പെടുന്നതല്ല. മണിയടിക്കുന്നതോ പെരുന്നാളിന് കുരിശ് എഴുന്നള്ളിക്കുന്നതോ തടയപ്പെടുന്നതുമല്ല.''
മധ്യപൂര്‍വദേശത്തെ െ്രെകസ്തവവിശ്വാസികള്‍ മുസ്‌ലിം ഭരണത്തിന് കീഴില്‍ സമ്പൂര്‍ണ മതസ്വാതന്ത്യ്രം അനുവദിച്ചിരുന്നതിനാല്‍ വളരെയേറെ സംതൃപ്തരായിരുന്നു. നീണ്ട അഞ്ചു നൂറ്റാണ്ടുകാലം ഇസ്‌ലാമികാധിപത്യം അനുഭവിച്ചശേഷവും ഈ അവസ്ഥയിലൊരു മാറ്റവും സംഭവിച്ചിരുന്നില്ലെന്ന് പന്ത്രണ്ടാം നൂറ്റാണ്ടിന്റെ രണ്ടാം പകുതിയില്‍ അന്തോക്യയിലെ യാക്കോബായ പാത്രിയാര്‍ക്കീസായിരുന്ന വലിയ മൈക്കലിന്റെ പ്രസ്താവന അസന്ദിഗ്ധമായി വ്യക്തമാക്കുന്നു. റോമന്‍ ഭരണാധികാരിയായിരുന്ന ഹിരാക്‌ളിയസിന്റെ മര്‍ദന കഥകള്‍ വിവരിച്ചുകൊണ്ട് അദ്ദേഹം എഴുതുന്നു: 'ഇങ്ങനെയാണ് സര്‍വശക്തനും മനുഷ്യരുടെ സാമ്രാജ്യങ്ങള്‍ തന്റെ ഹിതത്തിനനുസരിച്ച് മാറ്റിമറിക്കുന്നവനും താനിഛിക്കുന്നവര്‍ക്ക് സാമ്രാജ്യം നല്‍കുന്നവനും പാവങ്ങളെ ഉദ്ധരിക്കുന്നവനുമായ പ്രതികാരത്തിന്റെ ദൈവം ഇശ്‌മേലിന്റെ സന്താനങ്ങളെ റോമന്‍ കരങ്ങളില്‍നിന്നും നമ്മെ രക്ഷിക്കാനായി തെക്കുനിന്നു കൊണ്ടുവന്നത്. റോമക്കാര്‍ നമ്മുടെ ചര്‍ച്ചുകളും മഠങ്ങളും കവര്‍ച്ച ചെയ്യുന്നതും നമ്മെ നിര്‍ദയം മര്‍ദിക്കുന്നതും ദൈവം നോക്കിക്കാണുകയായിരുന്നു. യഥാര്‍ഥത്തില്‍ നമുക്കല്‍പം നഷ്ടം സംഭവിച്ചിട്ടുണ്ടെങ്കില്‍ അത് കാല്‍സിഡോണിയന്‍ പക്ഷത്തുനിന്ന് ഏല്‍പിക്കപ്പെട്ട നമ്മുടെ ചര്‍ച്ചുകള്‍ അവരുടെ കൈയില്‍ തന്നെ ശേഷിക്കുന്നതുകൊണ്ട് ഉണ്ടായതു മാത്രമാണ്. അറബികള്‍ നഗരങ്ങള്‍ അധീനപ്പെടുത്തിയപ്പോള്‍ ഓരോരുത്തരുടെയും കൈവശമുള്ള ചര്‍ച്ചുകള്‍ അങ്ങനെത്തന്നെ നിലനിര്‍ത്തി. ഏതായിരുന്നാലും റോമക്കാരുടെ ക്രൂരതയില്‍നിന്നും നീചത്വത്തില്‍നിന്നും രോഷത്തില്‍നിന്നും മതാവേശത്തില്‍നിന്നും രക്ഷ പ്രാപിക്കുകയും നാം സമാധാനത്തില്‍ കഴിയുകയും ചെയ്യുന്നുവെന്നത് ഒട്ടും നിസ്സാരകാര്യമല്ല'''(Michael the elder Vol. 2, PP. 412, 413. ഉദ്ധരണം: സര്‍ തോമസ് ആര്‍ണള്‍ഡ്, ഇസ്ലാം പ്രബോധനവും പ്രചാരവും, പുറം 67).
ഇന്ത്യയിലെ മുസ്‌ലിം ആധിപത്യത്തെ സംബന്ധിച്ച് ശ്രീ. ഈശ്വരി പ്രസാദ് പറയുന്നു: 'മുസ്‌ലിംകള്‍ കീഴടക്കപ്പെട്ട ജനതയ്ക്ക് ആരാധനാ സ്വാതന്ത്യ്രം അനുവദിക്കുകയും അവരോട് സഹിഷ്ണുതാപൂര്‍വം പെരുമാറുകയുമുണ്ടായി.'''(History of Muslim Rule, Page 46).
ഡോക്ടര്‍ താരാചന്ദ് എഴുതുന്നു: 'മുസ്‌ലിം ജേതാക്കള്‍ പരാജിതരോട് വളരെ നന്നായി പെരുമാറി. ഹിന്ദു പണ്ഡിതന്മാര്‍ക്കും പൂജാരിമാര്‍ക്കും തങ്ങളുടെ ദേവാലയങ്ങള്‍ക്കും ചട്ടപ്പടിയുള്ള അവകാശം നല്‍കാന്‍ കര്‍ഷകരെ അനുവദിച്ചു.''(Ibid, Page 49).
മകന്‍ ഹുമയൂണിന് ബാബര്‍ ചക്രവര്‍ത്തി നല്‍കിയ അന്ത്യോപദേശങ്ങളില്‍ ഹിന്ദുസഹോദരന്മാരോട് അത്യുദാരമായി പെരുമാറാനാവശ്യപ്പെടുകയുണ്ടായി. ഡോ. രാജേന്ദ്രപ്രസാദ് ഉള്‍പ്പെടെ ഉദ്ധരിച്ച പ്രസ്തുത അന്ത്യോപദേശങ്ങളില്‍ ഇങ്ങനെ കാണാം: 'ഇന്ത്യ മതവൈവിധ്യങ്ങളുടെ നാടാണ്. അതില്‍ നീ നന്ദി രേഖപ്പെടുത്തണം. അല്ലാഹു നിനക്ക് അധികാരം നല്‍കിയാല്‍ നീ മതപക്ഷപാതിത്വം കാണിക്കരുത്. ഹൈന്ദവരുടെ ഹൃദയം വ്രണപ്പെടും വിധം പശുക്കളെ അറുക്കരുത്. അതു ചെയ്താല്‍ ജനം നിന്നെ വെറുക്കും. ക്ഷേത്രങ്ങളും ആരാധനാലയങ്ങളും തകര്‍ക്കരുത്. ഭരണാധികാരി ഭരണീയരെയും ഭരണീയര്‍ ഭരണകര്‍ത്താവിനെയും സ്‌നേഹിക്കുന്ന സാഹചര്യം സൃഷ്ടിക്കുക. ദയാഹൃദയം കൊണ്ടാണ് ഇസ്‌ലാമിനെ ധന്യമാക്കേണ്ടത്. അടിച്ചമര്‍ത്തലിലൂടെയല്ല''(ഉദ്ധരണം: മിസിസ് നിലോഫര്‍ അഹ്മദ്, ഇന്‍സ്‌റിറ്റിയൂട്ട് ഓഫ് ഒബ്ജക്ടീവ് സ്‌റഡീസ്, ന്യൂഡല്‍ഹി).
ആലംഗീര്‍ നാമയിലിങ്ങനെ വായിക്കാം: 'ഔറംഗസീബ് ബംഗാളിലും ആസാമിലും ചില ഹിന്ദു ക്ഷേത്രങ്ങള്‍ നിര്‍മിക്കുകയും ബുദ്ധഗയക്ക് വമ്പിച്ച ഭൂസ്വത്ത് രാജകീയശാസന വഴി നല്‍കുകയുമുണ്ടായി'''(ഉദ്ധരണം:Illustrated weekly, 5.10.'75).
പണ്ഡിറ്റ് സുന്ദര്‍ലാല്‍ പറയുന്നു: 'അക്ബര്‍, ജഹാംഗീര്‍, ഷാജഹാന്‍ എന്നിവരുടെ കാലത്തും ഔറംഗസീബിന്റെയും പിന്‍ഗാമികളുടെയും കാലത്തും ഹിന്ദുക്കളോടും മുസ്‌ലിംകളോടും ഒരേ സമീപനമാണ് സ്വീകരിച്ചിരുന്നത്. രണ്ടു മതങ്ങളും തുല്യമായി ആദരിക്കപ്പെട്ടു. മതത്തിന്റെ പേരില്‍ ആരോടും ഒരുവിധ വിവേചനവും കാണിച്ചിരുന്നില്ല. എല്ലാ ചക്രവര്‍ത്തിമാരും നിരവധി ക്ഷേത്രങ്ങള്‍ക്ക് ഒട്ടേറെ ഭൂസ്വത്തുക്കള്‍ നല്‍കുകയുണ്ടായി. ഇന്നും ഇന്ത്യയിലെ വിവിധ ക്ഷേത്രപൂജാരികളുടെ വശം ഔറംഗസീബിന്റെ ഒപ്പുള്ള രാജകല്‍പന നിലവിലുണ്ട്. അവ അദ്ദേഹം പാരിതോഷികങ്ങളും ഭൂസ്വത്തുക്കളും നല്‍കിയതിന്റെ സ്മരണികയത്രെ. ഇത്തരം രണ്ടു കല്‍പനകള്‍ ഇപ്പോഴും ഇലഹാബാദിലുണ്ട്. അവയിലൊന്ന് സോമനാഥ ക്ഷേത്രത്തിലെ പൂജാരിയുടെ വശമാണ്''.
ഇതര മതവിഭാഗങ്ങളുടെ ആരാധനാലയങ്ങളുടെ സംരക്ഷണത്തിലും ഇസ്‌ലാമികരാഷ്ട്രം എന്നും എവിടെയും തികഞ്ഞ ശ്രദ്ധയും ജാഗ്രതയും പുലര്‍ത്തിപ്പോന്നിട്ടുണ്ട്. ഒന്നാം ഖലീഫയായ അബൂബക്ര്! സ്വിദ്ദീഖിനോട് രാജ്യത്തെ െ്രെകസ്തവ വിശ്വാസികള്‍, തങ്ങള്‍ പുതുതായി നിര്‍മിച്ച ചര്‍ച്ച് ഉദ്ഘാടനം ചെയ്യാനാവശ്യപ്പെടുകയും അത് ഇസ്‌ലാമികാരാധനയായ നമസ്‌കാരം നിര്‍വഹിച്ച് നടത്തിയാല്‍ മതിയെന്ന് നിര്‍ദേശിക്കുകയും ചെയ്തു. അപ്പോള്‍ അദ്ദേഹം പറഞ്ഞു: 'ഞാനത് ഉദ്ഘാടനം ചെയ്താല്‍ എന്റെ കാലശേഷം യാഥാര്‍ഥ്യമറിയാത്തവര്‍ ഞങ്ങളുടെ ഖലീഫ നമസ്‌കരിച്ച സ്ഥലമാണെന്ന് അതിന്റെ പേരില്‍ അവകാശവാദമുന്നയിക്കുകയും അത് കുഴപ്പങ്ങള്‍ക്കിടവരുത്തുകയും ചെയ്‌തേക്കാം.'' ഖലീഫയുടെ ആശങ്ക ശരിയാണെന്നു ബോധ്യമായ െ്രെകസ്തവ സഹോദരന്മാര്‍ തങ്ങളുടെ ഉദ്യമത്തില്‍നിന്ന് പിന്‍മാറി.
ഫലസ്തീന്‍ സന്ദര്‍ശിക്കവെ നമസ്‌കാരസമയമായപ്പോള്‍ രണ്ടാം ഖലീഫ ഉമറുല്‍ ഫാറൂഖിനോട് അവിടത്തെ പാത്രിയാര്‍ക്കീസ് സ്വഫര്‍നിയൂസ്, തങ്ങളുടെ ചര്‍ച്ചില്‍വച്ച് നമസ്‌കാരം നിര്‍വഹിക്കാനാവശ്യപ്പെട്ടു. എന്നാല്‍ ആ നിര്‍ദേശം ഖലീഫ നന്ദിപൂര്‍വം നിരസിക്കുകയാണുണ്ടായത്. അതിന് അദ്ദേഹം പറഞ്ഞ കാരണം, താനവിടെ വച്ച് നമസ്‌കരിച്ചാല്‍ പില്‍ക്കാലത്ത് അവിവേകികളായ മുസ്‌ലിംകളാരെങ്കിലും അതിന്റെ പേരില്‍ അവകാശവാദമുന്നയിക്കുകയും അത് പള്ളിയാക്കി മാറ്റാന്‍ ശ്രമിക്കുകയും ചെയ്‌തേക്കുമെന്നായിരുന്നു. അത്തരമൊരവിവേകത്തിന് അവസരമുണ്ടാവരുതെന്ന് നിര്‍ബന്ധമുണ്ടായിരുന്ന ഉമറുല്‍ ഫാറൂഖ് ചര്‍ച്ചിനു പുറത്തുള്ള ഒരു ഒഴിഞ്ഞ സ്ഥലത്ത് വസ്ത്രം വിരിച്ച് നമസ്‌കരിക്കുകയാണുണ്ടായത്.
അമുസ്‌ലിംകള്‍ക്ക് അവരുടെ വ്യക്തിനിയമങ്ങളനുസരിച്ച് ജീവിക്കാന്‍ പൂര്‍ണ സ്വാതന്ത്യ്രം നല്‍കുന്ന ഇസ്‌ലാം അവരുടെ ഒരവകാശവും ഹനിക്കാന്‍ അനുവദിക്കുന്നില്ല. പ്രവാചകന്‍ തിരുമേനി അരുള്‍ ചെയ്യുന്നു: 'സൂക്ഷിച്ചുകൊള്ളുക, അമുസ്‌ലിം പൗരന്മാരെ വല്ലവരും അടിച്ചമര്‍ത്തുകയോ അവരുടെ മേല്‍ കഴിവിനതീതമായ നികുതിഭാരം ചുമത്തുകയോ അവരോട് ക്രൂരമായി പെരുമാറുകയോ അവരുടെ അവകാശങ്ങള്‍ വെട്ടിക്കുറയ്ക്കുകയോ ചെയ്യുകയാണെങ്കില്‍ അന്ത്യവിധി നാളില്‍ അവര്‍ക്കെതിരെ ഞാന്‍ സ്വയം തന്നെ പരാതി ബോധിപ്പിക്കുന്നതാണ്.''(അബൂദാവൂദ്)
'ആര്‍ അമുസ്‌ലിം പൗരനെ അപായപ്പെടുത്തുന്നുവോ അവന്‍ സ്വര്‍ഗത്തിന്റെ ഗന്ധം പോലും അനുഭവിക്കുകയില്ല'' (അബൂയൂസുഫ്, കിതാബുല്‍ ഖറാജ്, പേജ് 71).
വിവാഹം, വിവാഹമോചനം, അനന്തരാവകാശം തുടങ്ങിയ വ്യക്തിനിയമങ്ങളുമായി ബന്ധപ്പെട്ട പ്രശ്‌നങ്ങളെല്ലാം ബന്ധപ്പെട്ട കക്ഷികളുടെ മതാചാരപ്രകാരമാണ് ഇസ്‌ലാമിക കോടതികള്‍ തീര്‍പ്പ് കല്‍പിക്കുക. ശരീഅത്ത് വിധികള്‍ അവരുടെ മേല്‍ നടപ്പിലാക്കുകയില്ല. നബിതിരുമേനിയുടെ കാലത്ത് ജൂതന്മാരുടെ കേസുകള്‍ വിചാരണയ്ക്കു വന്നാല്‍ മദീനയിലെ 'ബൈത്തുല്‍ മിദ്‌റാസ്' എന്ന ജൂതസെമിനാരിയുമായി ബന്ധപ്പെട്ട് അവിടത്തെ പുരോഹിതന്മാരോട് തോറയിലെ വിധികള്‍ അന്വേഷിച്ച് പഠിച്ച ശേഷമേ അവിടന്നു തീര്‍പ്പ് കല്‍പിച്ചിരുന്നുള്ളൂ. (ഇബ്‌നു ഹിശാം, സീറത്തുന്നബി, വാള്യം 2, പുറം 201).
സര്‍ തോമസ് ആര്‍ണള്‍ഡ് എഴുതുന്നു: 'അമുസ്‌ലിം സമൂഹങ്ങള്‍ മിക്കവാറും പൂര്‍ണമായ സ്വയംഭരണാവകാശം അനുഭവിക്കുകയുണ്ടായി. എന്തുകൊണ്ടെന്നാല്‍ തങ്ങളുടെ ആഭ്യന്തരകാര്യങ്ങള്‍ കൈകാര്യം ചെയ്യാനുള്ള സ്വാതന്ത്യ്രം ഭരണകൂടം അവരുടെ കരങ്ങളില്‍ തന്നെ ഏല്‍പിച്ചിരുന്നു. മതപരമായ വിഷയങ്ങളില്‍ തീരുമാനമെടുക്കുവാനുള്ള അധികാരം അവരുടെ പുരോഹിതന്മാര്‍ക്ക് ലഭിച്ചു. അവരുടെ ചര്‍ച്ചുകളും മഠങ്ങളും യാതൊരു ഊനവും തട്ടാതെ നിലനിര്‍ത്താനനുവദിച്ചു''(ഇസ്‌ലാം: പ്രബോധനവും പ്രചാരവും, പുറം 78).
എന്നാല്‍ ഇസ്‌ലാമികരാഷ്ട്രത്തില്‍ അനീതികളും വിവേചനങ്ങളും നടക്കുകയില്ല. ജാതിമതകക്ഷി ഭേദമന്യേ നിഷ്‌കൃഷ്ടമായ നീതി നടപ്പിലാക്കപ്പെടും. ഖുര്‍ആന്‍ കല്‍പിക്കുന്നു: 'വിശ്വസിച്ചവരേ, നിങ്ങള്‍ അല്ലാഹുവിനു വേണ്ടി നേര്‍മാര്‍ഗത്തില്‍ നിലകൊള്ളുന്നവരും നീതിക്ക് സാക്ഷ്യം വഹിക്കുന്നവരുമാകുക. ഒരു ജനതയോടുള്ള വിരോധം നീതി നടപ്പിലാക്കപ്പെടാതിരിക്കാന്‍ നിങ്ങളെ പ്രേരിപ്പിക്കാതിരിക്കട്ടെ. നീതി പാലിക്കുവിന്‍. അതാണ് ദൈവഭക്തിക്ക് ഏറ്റവും അനുയോജ്യം''(5:8).
ഇസ്‌ലാമികരാഷ്ട്രത്തില്‍ അമുസ്‌ലിം പൗരന്മാര്‍ക്കെതിരെ കലാപം നടത്തിയാല്‍ ശിക്ഷാ നടപടികള്‍ സ്വീകരിക്കപ്പെടും. വധിച്ചാല്‍ പ്രതിക്രിയ നടപ്പാക്കപ്പെടും. ആരാധനാലയങ്ങള്‍ തകര്‍ക്കപ്പെട്ടാല്‍ തകര്‍ത്തവര്‍ക്കെതിരെ കടുത്ത ശിക്ഷ വിധിക്കുകയും തകര്‍ക്കപ്പെട്ടവ പുനര്‍നിര്‍മിക്കുകയും ചെയ്യും. ആരാധനാലയങ്ങളെപ്പോലെത്തന്നെ മതന്യൂനപക്ഷങ്ങളുടെ ജീവനും സ്വത്തും വിദ്യാസ്ഥാപനങ്ങളും വ്യക്തിനിയമങ്ങളും പൂര്‍ണമായും സുരക്ഷിതമായിരിക്കും. അവയുടെയൊന്നും നേരെ ഒരുവിധ കൈയേറ്റവും അനുവദിക്കപ്പെടുന്നതല്ല. അതുകൊണ്ടുതന്നെ ഇസ്‌ലാമിക ഭരണത്തില്‍ മതന്യൂനപക്ഷങ്ങള്‍ സ്വാതന്ത്യ്രവും സുരക്ഷയും അനുവദിക്കുമെന്നതിലൊട്ടും സംശയമില്ല. അവരൊരിക്കലും അല്‍പവും അനീതിക്കോ കൈയേറ്റങ്ങള്‍ക്കോ അവഹേളനങ്ങള്‍ക്കോ ഇരയാവുകയില്ല.
രാജ്യത്തെ ആദിവാസികളും പിന്നോക്ക ജാതിക്കാരുമെല്ലാം തങ്ങള്‍ ഹിന്ദുക്കളല്ലെന്ന് ശക്തമായി വാദിച്ചുകൊണ്ടിരിക്കെ, ചോദ്യത്തിലുന്നയിച്ച, ഇന്ത്യയിലെ ഭൂരിപക്ഷം ഹിന്ദുക്കളാണെന്ന പരാമര്‍ശം സൂക്ഷ്മമോ വസ്തുനിഷ്ഠമോ അല്ലെന്നുകൂടി പറയേണ്ടതുണ്ട്.
\chapter{ഇസ്‌ലാമിക ഭരണത്തിലെ മതന്യൂനപക്ഷങ്ങള്‍ }
 \section{ ഇന്ത്യയില്‍ ഭൂരിപക്ഷം ഹിന്ദുക്കളായതിനാലല്ലേ എല്ലാ മതക്കാര്‍ക്കും തുല്യാവകാശം നല്‍കുന്ന സെക്കുലരിസം ഇവിടെ നിലനില്‍ക്കുന്നത്? ഭൂരിപക്ഷം മുസ്‌ലിംകളായിരുന്നുവെങ്കില്‍ ഇന്ത്യയും മതാധിഷ്ഠിത ഇസ്‌ലാമിക രാഷ്ട്രമാവുകയും മറ്റു മതാനുയായികള്‍ രണ്ടാംതരം പൌരന്‍മാരാവുകയോ മതംമാറാന്‍ നിര്‍ബന്ധിതരാവുകയോ ചെയ്യുമായിരുന്നില്ലേ?}
  

A സമൂഹത്തില്‍ നിലനില്‍ക്കുന്ന ഗുരുതരമായ തെറ്റിദ്ധാരണകളാണ് ഇത്തരം സംശയങ്ങള്‍ക്ക് കാരണം. മുസ്‌ലിംകള്‍ ഭൂരിപക്ഷമുള്ള നാടുകളെല്ലാം മതാധിഷ്ഠിത ഇസ്‌ലാമികരാഷ്ട്രങ്ങളാണെന്ന ധാരണ ശരിയല്ല. ഇസ്‌ലാമികവ്യവസ്ഥ യഥാവിധി നടപ്പാക്കപ്പെടുന്ന രാജ്യങ്ങള്‍ മാത്രമേ ഇസ്‌ലാമികരാഷ്ട്രമെന്ന വിശേഷണത്തിന് അര്‍ഹമാവുകയുള്ളൂ. നിലവിലുള്ള മുസ്‌ലസ്‌ലിം നാടുകള്‍ അവ്വിധം ചെയ്യാത്തതിനാലാണ് ലോകത്തെവിടെയും മാതൃകാ ഇസ്‌ലാമിക രാഷ്ട്രങ്ങള്‍ കാണപ്പെടാത്തത്. ഭാഗികമായി ഇസ്‌ലാമികവ്യവസ്ഥ നടപ്പാക്കുന്ന നാടുകളുണ്ട്. അവ അത്രത്തോളമേ ഇസ്‌ലാമികമാവുകയുള്ളൂ.
ഇസ്‌ലാമികരാഷ്ട്രത്തിലെ അമുസ്‌ലിം പൌരന്മാര്‍ക്ക് തങ്ങളുടെ വിശ്വാസങ്ങള്‍ വച്ചുപുലര്‍ത്താനും ആരാധനാനുഷ്ഠാനങ്ങള്‍ നിര്‍വഹിക്കാനും ആചാരങ്ങള്‍ പിന്തുടരാനും സ്വാതന്ത്യ്രമുണ്ടായിരിക്കും. ആരുടെ മേലും ഇസ്‌ലാമിനെ അടിച്ചേല്‍പിക്കുകയോ ആരെയെങ്കിലും മതം മാറാന്‍ നിര്‍ബന്ധിക്കുകയോ ഇല്ല. അങ്ങനെ ചെയ്യുന്നത് ഇസ്‌ലാം കണിശമായി വിലക്കിയിരിക്കുന്നു: 'മതത്തില്‍ ഒരുവിധ നിര്‍ബന്ധവുമില്ല. സന്മാര്‍ഗം മിഥ്യാധാരണകളില്‍ നിന്ന് വേര്‍തിരിഞ്ഞ് കഴിഞ്ഞിരിക്കുന്നു.''(ഖു. 2: 256).
'നീ വിളംബരം ചെയ്യുക: ഇത് നിങ്ങളുടെ നാഥനില്‍നിന്നുള്ള സത്യമാകുന്നു. ഇഷ്ടമുള്ളവര്‍ക്കിത് സ്വീകരിക്കാം. ഇഷ്ടമുള്ളവര്‍ക്ക് നിഷേധിക്കാം.'' (ഖു. 18:29).
ദൈവദൂതന്മാര്‍ക്കുപോലും മതം സ്വീകരിക്കാന്‍ ആരെയും നിര്‍ബന്ധിക്കാന്‍ അനുവാദമുണ്ടായിരുന്നില്ല. അല്ലാഹു അറിയിക്കുന്നു: 'ജനങ്ങള്‍ വിശ്വാസികളാകാന്‍ നീ അവരെ നിര്‍ബന്ധിക്കുകയോ? ദൈവഹിതമില്ലാതെ ഒരാള്‍ക്കും വിശ്വസിക്കുക സാധ്യമല്ല''(ഖു. 6: 69).
'നബിയേ, നീ ഉദ്‌ബോധിപ്പിച്ചുകൊണ്ടിരിക്കുക. നീ ഉദ്‌ബോധകന്‍ മാത്രമാകുന്നു. നീ അവരെ നിര്‍ബന്ധിച്ച് വഴിപ്പെടുത്തുന്നവനൊന്നുമല്ല.''(ഖു. 88: 21,22).
ലോകത്തിലെ ആദ്യത്തെ ഇസ്‌ലാമികരാഷ്ട്രമായ മദീനയില്‍, അതിന്റെ സ്ഥാപകനായ നബിതിരുമേനി മതന്യൂനപക്ഷങ്ങള്‍ക്ക് അനുവദിച്ചതുപോലുള്ള സ്വാതന്ത്യ്രവും സൌകര്യവും മറ്റേതെങ്കിലും മതാധിഷ്ഠിത നാടുകളിലോ മതനിരപേക്ഷ രാജ്യങ്ങളിലോ കാണപ്പെടുമോയെന്ന് ന്യായമായും സംശയിക്കേണ്ടിയിരിക്കുന്നു. മദീന അംഗീകരിച്ച് പ്രഖ്യാപിച്ച പ്രമാണത്തില്‍ ഇങ്ങനെ കാണാം: 'നമ്മുടെ ഭരണസാഹോദര്യസീമയില്‍പെടുന്ന ജൂതന്മാര്‍ക്ക് വര്‍ഗാടിസ്ഥാനത്തിലുള്ള പക്ഷപാതപരമായ പെരുമാറ്റങ്ങളില്‍നിന്നും ദ്രോഹങ്ങളില്‍നിന്നും രക്ഷ നല്‍കും. നമ്മുടെ സഹായത്തിനും ദയാനിരതമായ സംരക്ഷണത്തിനും മുസ്‌ലിം സമുദായാംഗങ്ങളെപ്പോലെ അവര്‍ക്കും അവകാശമുണ്ട്. മുസ്‌ലിംകളുമായി ചേര്‍ന്ന് അവര്‍ ഏക ഘടനയുള്ള ഒരു രാഷ്ട്രമായിത്തീരും. മുസ്‌ലിംകളെപ്പോലെത്തന്നെ അവര്‍ക്കും സ്വതന്ത്രമായി തങ്ങളുടെ മതം ആചരിക്കാവുന്നതാണ്.''
സര്‍ തോമസ് ആര്‍ണള്‍ഡ് എഴുതുന്നു: 'മുഹമ്മദ് പല അറബ്‌െ്രെകസ്തവ ഗോത്രങ്ങളുമായും സന്ധിയിലേര്‍പ്പെട്ടിരുന്നു. അവര്‍ക്കദ്ദേഹം സംരക്ഷണവും സ്വന്തം മതമാചരിക്കാനുള്ള സ്വാതന്ത്യ്രവും നല്‍കി''(ഇസ്‌ലാം: പ്രബോധനവും പ്രചാരവും, പുറം 60).
പ്രവാചകന്റെ പാത പിന്തുടര്‍ന്ന് മുഴുവന്‍ മുസ്‌ലിംഭരണാധികാരികളും സ്വീകരിച്ച സമീപനവും ഇതുതന്നെ. ഇന്ത്യാ ചരിത്രത്തില്‍ ഏറെ തെറ്റിദ്ധരിക്കപ്പെടുകയും വിമര്‍ശനവിധേയനാവുകയും ചെയ്ത ഔറംഗസീബിന്റെ മതസമീപനത്തില്‍ പ്രകടമായിരുന്ന സഹിഷ്ണുതയെ സംബന്ധിച്ച് അലക്‌സാണ്ടര്‍ ഹാമില്‍ട്ടണ്‍ എഴുതുന്നു: 'ഹിന്ദുക്കള്‍ക്ക് പരിപൂര്‍ണമായ മതസ്വാതന്ത്യ്രം ലഭിക്കുന്നുണ്ടെന്നതിന് പുറമെ ഹൈന്ദവ രാജാക്കന്മാരുടെ കീഴിലായിരിക്കുമ്പോഴൊക്കെ അവര്‍ നടത്തിയിരുന്ന വ്രതങ്ങളും ഉത്സവങ്ങളും ആഘോഷിക്കാനുള്ള സൌകര്യങ്ങള്‍ ഉണ്ടായിരുന്നു. മീററ്റ് നഗരത്തില്‍ മാത്രം ഹൈന്ദവവിഭാഗത്തില്‍ നൂറില്‍പരം വ്യത്യസ്ത വിഭാഗങ്ങളുണ്ടെങ്കിലും അവര്‍ തമ്മില്‍ പ്രാര്‍ഥനകളുടെയോ സിദ്ധാന്തങ്ങളുടെയോ പേരില്‍ യാതൊരു വിധ വിവാദവും ഉണ്ടായിരുന്നില്ല. ഏതൊരാള്‍ക്കും അയാളാഗ്രഹിക്കുന്ന വിധം ദൈവാര്‍ച്ചനകള്‍ നടത്തുവാനും ആരാധിക്കാനുമുള്ള സ്വാതന്ത്യ്രമുണ്ടായിരുന്നു. മതധ്വംസനങ്ങള്‍ അജ്ഞാതമത്രെ'' (അഹമഃമിറലൃ ഒമാശഹീേി, അ ിലം അരരീൗി േീള വേല ഋമേെ കിറശല,െ ഢീഹ. 1, ജജ. 159, 162, 163).).
മുസ്‌ലിം ഭരണാധികാരികള്‍ മതപരിവര്‍ത്തനത്തിന് നിര്‍ബന്ധിക്കുകയോ സമ്മര്‍ദം ചെലുത്തുകയോ ചെയ്തിരുന്നുവെങ്കില്‍ നീണ്ട നിരവധി നൂറ്റാണ്ടുകളുടെ ഭരണത്തിനു ശേഷവും മുസ്‌ലിംകളിവിടെ ന്യൂനപക്ഷമാകുമായിരുന്നില്ലെന്ന് സുവിദിതമാണല്ലോ.
ഇസ്‌ലാമിക ഭരണത്തില്‍ മതന്യൂനപക്ഷങ്ങള്‍ക്ക് പൂര്‍ണമായ ആരാധനാസ്വാതന്ത്യ്രം നല്‍കപ്പെട്ടിരുന്നു. നജ്‌റാനിലെ ക്രിസ്ത്യാനികളുമായി പ്രവാചകനുണ്ടാക്കിയ സന്ധിവ്യവസ്ഥകളില്‍ ഇങ്ങനെ കാണാം. 'നജ്‌റാനിലെ െ്രെകസ്തവര്‍ക്കും അവരുടെ സഹവാസികള്‍ക്കും ദൈവത്തിന്റെ അഭയവും ദൈവദൂതനായ മുഹമ്മദിന്റെ സംരക്ഷണോത്തരവാദിത്വവുമുണ്ട്. അവരുടെ ജീവന്‍, മതം, ഭൂമി, ധനം എന്നിവയ്ക്കും അവരില്‍ ഹാജറുള്ളവന്നും ഇല്ലാത്തവന്നും അവരുടെ ഒട്ടകങ്ങള്‍ക്കും നിവേദകസംഘങ്ങള്‍ക്കും കുരിശ്, ചര്‍ച്ച് പോലുള്ള മതചിഹ്നങ്ങള്‍ക്കും വേണ്ടിയാണിത്. നിലവിലുള്ള അവസ്ഥയില്‍ ഒരു മാറ്റവും വരുത്തുന്നതല്ല. അവരുടെ യാതൊരവകാശവും ഒരു മതചിഹ്നവും മാറ്റപ്പെടുന്നതല്ല. അവരുടെ പാതിരിയോ പുരോഹിതനോ ചര്‍ച്ച് സേവകനോ തന്റെ സ്ഥാനത്ത് നിന്ന് നീക്കപ്പെടുന്നതല്ല.''
ഒന്നാം ഖലീഫ അബൂബക്ര്! സ്വിദ്ദീഖ്(റ) ഹീറാവാസികളുമായി ഒപ്പുവച്ച സന്ധിവ്യവസ്ഥകളിലിങ്ങനെ പറയുന്നു: 'അവരുടെ ആരാധനാലയങ്ങളും കനീസുകളും ശത്രുക്കളില്‍നിന്ന് രക്ഷ നേടുന്ന കോട്ടകളും പൊളിച്ചുമാറ്റപ്പെടുന്നതല്ല. മണിയടിക്കുന്നതോ പെരുന്നാളിന് കുരിശ് എഴുന്നള്ളിക്കുന്നതോ തടയപ്പെടുന്നതുമല്ല.''
മധ്യപൂര്‍വദേശത്തെ െ്രെകസ്തവവിശ്വാസികള്‍ മുസ്‌ലിം ഭരണത്തിന് കീഴില്‍ സമ്പൂര്‍ണ മതസ്വാതന്ത്യ്രം അനുവദിച്ചിരുന്നതിനാല്‍ വളരെയേറെ സംതൃപ്തരായിരുന്നു. നീണ്ട അഞ്ചു നൂറ്റാണ്ടുകാലം ഇസ്‌ലാമികാധിപത്യം അനുഭവിച്ചശേഷവും ഈ അവസ്ഥയിലൊരു മാറ്റവും സംഭവിച്ചിരുന്നില്ലെന്ന് പന്ത്രണ്ടാം നൂറ്റാണ്ടിന്റെ രണ്ടാം പകുതിയില്‍ അന്തോക്യയിലെ യാക്കോബായ പാത്രിയാര്‍ക്കീസായിരുന്ന വലിയ മൈക്കലിന്റെ പ്രസ്താവന അസന്ദിഗ്ധമായി വ്യക്തമാക്കുന്നു. റോമന്‍ ഭരണാധികാരിയായിരുന്ന ഹിരാക്‌ളിയസിന്റെ മര്‍ദന കഥകള്‍ വിവരിച്ചുകൊണ്ട് അദ്ദേഹം എഴുതുന്നു: 'ഇങ്ങനെയാണ് സര്‍വശക്തനും മനുഷ്യരുടെ സാമ്രാജ്യങ്ങള്‍ തന്റെ ഹിതത്തിനനുസരിച്ച് മാറ്റിമറിക്കുന്നവനും താനിഛിക്കുന്നവര്‍ക്ക് സാമ്രാജ്യം നല്‍കുന്നവനും പാവങ്ങളെ ഉദ്ധരിക്കുന്നവനുമായ പ്രതികാരത്തിന്റെ ദൈവം ഇശ്‌മേലിന്റെ സന്താനങ്ങളെ റോമന്‍ കരങ്ങളില്‍നിന്നും നമ്മെ രക്ഷിക്കാനായി തെക്കുനിന്നു കൊണ്ടുവന്നത്. റോമക്കാര്‍ നമ്മുടെ ചര്‍ച്ചുകളും മഠങ്ങളും കവര്‍ച്ച ചെയ്യുന്നതും നമ്മെ നിര്‍ദയം മര്‍ദിക്കുന്നതും ദൈവം നോക്കിക്കാണുകയായിരുന്നു. യഥാര്‍ഥത്തില്‍ നമുക്കല്‍പം നഷ്ടം സംഭവിച്ചിട്ടുണ്ടെങ്കില്‍ അത് കാല്‍സിഡോണിയന്‍ പക്ഷത്തുനിന്ന് ഏല്‍പിക്കപ്പെട്ട നമ്മുടെ ചര്‍ച്ചുകള്‍ അവരുടെ കൈയില്‍ തന്നെ ശേഷിക്കുന്നതുകൊണ്ട് ഉണ്ടായതു മാത്രമാണ്. അറബികള്‍ നഗരങ്ങള്‍ അധീനപ്പെടുത്തിയപ്പോള്‍ ഓരോരുത്തരുടെയും കൈവശമുള്ള ചര്‍ച്ചുകള്‍ അങ്ങനെത്തന്നെ നിലനിര്‍ത്തി. ഏതായിരുന്നാലും റോമക്കാരുടെ ക്രൂരതയില്‍നിന്നും നീചത്വത്തില്‍നിന്നും രോഷത്തില്‍നിന്നും മതാവേശത്തില്‍നിന്നും രക്ഷ പ്രാപിക്കുകയും നാം സമാധാനത്തില്‍ കഴിയുകയും ചെയ്യുന്നുവെന്നത് ഒട്ടും നിസ്സാരകാര്യമല്ല''(ങശരവമലഹ വേല ലഹറലൃ ഢീഹ. 2, ജജ. 412, 413. ഉദ്ധരണം: സര്‍ തോമസ് ആര്‍ണള്‍ഡ്, ഇസ്‌ലാം പ്രബോധനവും പ്രചാരവും, പുറം 67).
ഇന്ത്യയിലെ മുസ്‌ലിം ആധിപത്യത്തെ സംബന്ധിച്ച് ശ്രീ. ഈശ്വരി പ്രസാദ് പറയുന്നു: 'മുസ്‌ലിംകള്‍ കീഴടക്കപ്പെട്ട ജനതയ്ക്ക് ആരാധനാ സ്വാതന്ത്യ്രം അനുവദിക്കുകയും അവരോട് സഹിഷ്ണുതാപൂര്‍വം പെരുമാറുകയുമുണ്ടായി.''(ഒശേെീൃ്യ ീള ങൗഹെശാ ഞൗഹല, ജമഴല 46).
ഡോക്ടര്‍ താരാചന്ദ് എഴുതുന്നു: 'മുസ്‌ലിം ജേതാക്കള്‍ പരാജിതരോട് വളരെ നന്നായി പെരുമാറി. ഹിന്ദു പണ്ഡിതന്മാര്‍ക്കും പൂജാരിമാര്‍ക്കും തങ്ങളുടെ ദേവാലയങ്ങള്‍ക്കും ചട്ടപ്പടിയുള്ള അവകാശം നല്‍കാന്‍ കര്‍ഷകരെ അനുവദിച്ചു.''(കയശറ, ജമഴല 49).
മകന്‍ ഹുമയൂണിന് ബാബര്‍ ചക്രവര്‍ത്തി നല്‍കിയ അന്ത്യോപദേശങ്ങളില്‍ ഹിന്ദുസഹോദരന്മാരോട് അത്യുദാരമായി പെരുമാറാനാവശ്യപ്പെടുകയുണ്ടായി. ഡോ. രാജേന്ദ്രപ്രസാദ് ഉള്‍പ്പെടെ ഉദ്ധരിച്ച പ്രസ്തുത അന്ത്യോപദേശങ്ങളില്‍ ഇങ്ങനെ കാണാം: 'ഇന്ത്യ മതവൈവിധ്യങ്ങളുടെ നാടാണ്. അതില്‍ നീ നന്ദി രേഖപ്പെടുത്തണം. അല്ലാഹു നിനക്ക് അധികാരം നല്‍കിയാല്‍ നീ മതപക്ഷപാതിത്വം കാണിക്കരുത്. ഹൈന്ദവരുടെ ഹൃദയം വ്രണപ്പെടും വിധം പശുക്കളെ അറുക്കരുത്. അതു ചെയ്താല്‍ ജനം നിന്നെ വെറുക്കും. ക്ഷേത്രങ്ങളും ആരാധനാലയങ്ങളും തകര്‍ക്കരുത്. ഭരണാധികാരി ഭരണീയരെയും ഭരണീയര്‍ ഭരണകര്‍ത്താവിനെയും സ്‌നേഹിക്കുന്ന സാഹചര്യം സൃഷ്ടിക്കുക. ദയാഹൃദയം കൊണ്ടാണ് ഇസ്‌ലാമിനെ ധന്യമാക്കേണ്ടത്. അടിച്ചമര്‍ത്തലിലൂടെയല്ല''(ഉദ്ധരണം: മിസിസ് നിലോഫര്‍ അഹ്മദ്, ഇന്‍സ്‌റിറ്റിയൂട്ട് ഓഫ് ഒബ്ജക്ടീവ് സ്‌റഡീസ്, ന്യൂഡല്‍ഹി).
ആലംഗീര്‍ നാമയിലിങ്ങനെ വായിക്കാം: 'ഔറംഗസീബ് ബംഗാളിലും ആസാമിലും ചില ഹിന്ദു ക്ഷേത്രങ്ങള്‍ നിര്‍മിക്കുകയും ബുദ്ധഗയക്ക് വമ്പിച്ച ഭൂസ്വത്ത് രാജകീയശാസന വഴി നല്‍കുകയുമുണ്ടായി''(ഉദ്ധരണം:കഹഹൗേെൃമലേറ ംലലസഹ്യ, 5.10.'75).
പണ്ഡിറ്റ് സുന്ദര്‍ലാല്‍ പറയുന്നു: 'അക്ബര്‍, ജഹാംഗീര്‍, ഷാജഹാന്‍ എന്നിവരുടെ കാലത്തും ഔറംഗസീബിന്റെയും പിന്‍ഗാമികളുടെയും കാലത്തും ഹിന്ദുക്കളോടും മുസ്‌ലിംകളോടും ഒരേ സമീപനമാണ് സ്വീകരിച്ചിരുന്നത്. രണ്ടു മതങ്ങളും തുല്യമായി ആദരിക്കപ്പെട്ടു. മതത്തിന്റെ പേരില്‍ ആരോടും ഒരുവിധ വിവേചനവും കാണിച്ചിരുന്നില്ല. എല്ലാ ചക്രവര്‍ത്തിമാരും നിരവധി ക്ഷേത്രങ്ങള്‍ക്ക് ഒട്ടേറെ ഭൂസ്വത്തുക്കള്‍ നല്‍കുകയുണ്ടായി. ഇന്നും ഇന്ത്യയിലെ വിവിധ ക്ഷേത്രപൂജാരികളുടെ വശം ഔറംഗസീബിന്റെ ഒപ്പുള്ള രാജകല്‍പന നിലവിലുണ്ട്. അവ അദ്ദേഹം പാരിതോഷികങ്ങളും ഭൂസ്വത്തുക്കളും നല്‍കിയതിന്റെ സ്മരണികയത്രെ. ഇത്തരം രണ്ടു കല്‍പനകള്‍ ഇപ്പോഴും ഇലഹാബാദിലുണ്ട്. അവയിലൊന്ന് സോമനാഥ ക്ഷേത്രത്തിലെ പൂജാരിയുടെ വശമാണ്''.
ഇതര മതവിഭാഗങ്ങളുടെ ആരാധനാലയങ്ങളുടെ സംരക്ഷണത്തിലും ഇസ്‌ലാമികരാഷ്ട്രം എന്നും എവിടെയും തികഞ്ഞ ശ്രദ്ധയും ജാഗ്രതയും പുലര്‍ത്തിപ്പോന്നിട്ടുണ്ട്. ഒന്നാം ഖലീഫയായ അബൂബക്ര്! സ്വിദ്ദീഖിനോട് രാജ്യത്തെ െ്രെകസ്തവ വിശ്വാസികള്‍, തങ്ങള്‍ പുതുതായി നിര്‍മിച്ച ചര്‍ച്ച് ഉദ്ഘാടനം ചെയ്യാനാവശ്യപ്പെടുകയും അത് ഇസ്‌ലാമികാരാധനയായ നമസ്‌കാരം നിര്‍വഹിച്ച് നടത്തിയാല്‍ മതിയെന്ന് നിര്‍ദേശിക്കുകയും ചെയ്തു. അപ്പോള്‍ അദ്ദേഹം പറഞ്ഞു: 'ഞാനത് ഉദ്ഘാടനം ചെയ്താല്‍ എന്റെ കാലശേഷം യാഥാര്‍ഥ്യമറിയാത്തവര്‍ ഞങ്ങളുടെ ഖലീഫ നമസ്‌കരിച്ച സ്ഥലമാണെന്ന് അതിന്റെ പേരില്‍ അവകാശവാദമുന്നയിക്കുകയും അത് കുഴപ്പങ്ങള്‍ക്കിടവരുത്തുകയും ചെയ്‌തേക്കാം.'' ഖലീഫയുടെ ആശങ്ക ശരിയാണെന്നു ബോധ്യമായ െ്രെകസ്തവ സഹോദരന്മാര്‍ തങ്ങളുടെ ഉദ്യമത്തില്‍നിന്ന് പിന്‍മാറി.
ഫലസ്തീന്‍ സന്ദര്‍ശിക്കവെ നമസ്‌കാരസമയമായപ്പോള്‍ രണ്ടാം ഖലീഫ ഉമറുല്‍ ഫാറൂഖിനോട് അവിടത്തെ പാത്രിയാര്‍ക്കീസ് സ്വഫര്‍നിയൂസ്, തങ്ങളുടെ ചര്‍ച്ചില്‍വച്ച് നമസ്‌കാരം നിര്‍വഹിക്കാനാവശ്യപ്പെട്ടു. എന്നാല്‍ ആ നിര്‍ദേശം ഖലീഫ നന്ദിപൂര്‍വം നിരസിക്കുകയാണുണ്ടായത്. അതിന് അദ്ദേഹം പറഞ്ഞ കാരണം, താനവിടെ വച്ച് നമസ്‌കരിച്ചാല്‍ പില്‍ക്കാലത്ത് അവിവേകികളായ മുസ്‌ലിംകളാരെങ്കിലും അതിന്റെ പേരില്‍ അവകാശവാദമുന്നയിക്കുകയും അത് പള്ളിയാക്കി മാറ്റാന്‍ ശ്രമിക്കുകയും ചെയ്‌തേക്കുമെന്നായിരുന്നു. അത്തരമൊരവിവേകത്തിന് അവസരമുണ്ടാവരുതെന്ന് നിര്‍ബന്ധമുണ്ടായിരുന്ന ഉമറുല്‍ ഫാറൂഖ് ചര്‍ച്ചിനു പുറത്തുള്ള ഒരു ഒഴിഞ്ഞ സ്ഥലത്ത് വസ്ത്രം വിരിച്ച് നമസ്‌കരിക്കുകയാണുണ്ടായത്.
അമുസ്‌ലിംകള്‍ക്ക് അവരുടെ വ്യക്തിനിയമങ്ങളനുസരിച്ച് ജീവിക്കാന്‍ പൂര്‍ണ സ്വാതന്ത്യ്രം നല്‍കുന്ന ഇസ്‌ലാം അവരുടെ ഒരവകാശവും ഹനിക്കാന്‍ അനുവദിക്കുന്നില്ല. പ്രവാചകന്‍ തിരുമേനി അരുള്‍ ചെയ്യുന്നു: 'സൂക്ഷിച്ചുകൊള്ളുക, അമുസ്‌ലിം പൌരന്മാരെ വല്ലവരും അടിച്ചമര്‍ത്തുകയോ അവരുടെ മേല്‍ കഴിവിനതീതമായ നികുതിഭാരം ചുമത്തുകയോ അവരോട് ക്രൂരമായി പെരുമാറുകയോ അവരുടെ അവകാശങ്ങള്‍ വെട്ടിക്കുറയ്ക്കുകയോ ചെയ്യുകയാണെങ്കില്‍ അന്ത്യവിധി നാളില്‍ അവര്‍ക്കെതിരെ ഞാന്‍ സ്വയം തന്നെ പരാതി ബോധിപ്പിക്കുന്നതാണ്.''(അബൂദാവൂദ്)
'ആര്‍ അമുസ്‌ലിം പൌരനെ അപായപ്പെടുത്തുന്നുവോ അവന്‍ സ്വര്‍ഗത്തിന്റെ ഗന്ധം പോലും അനുഭവിക്കുകയില്ല'' (അബൂയൂസുഫ്, കിതാബുല്‍ ഖറാജ്, പേജ് 71).
വിവാഹം, വിവാഹമോചനം, അനന്തരാവകാശം തുടങ്ങിയ വ്യക്തിനിയമങ്ങളുമായി ബന്ധപ്പെട്ട പ്രശ്‌നങ്ങളെല്ലാം ബന്ധപ്പെട്ട കക്ഷികളുടെ മതാചാരപ്രകാരമാണ് ഇസ്‌ലാമിക കോടതികള്‍ തീര്‍പ്പ് കല്‍പിക്കുക. ശരീഅത്ത് വിധികള്‍ അവരുടെ മേല്‍ നടപ്പിലാക്കുകയില്ല. നബിതിരുമേനിയുടെ കാലത്ത് ജൂതന്മാരുടെ കേസുകള്‍ വിചാരണയ്ക്കു വന്നാല്‍ മദീനയിലെ 'ബൈത്തുല്‍ മിദ്‌റാസ്' എന്ന ജൂതസെമിനാരിയുമായി ബന്ധപ്പെട്ട് അവിടത്തെ പുരോഹിതന്മാരോട് തോറയിലെ വിധികള്‍ അന്വേഷിച്ച് പഠിച്ച ശേഷമേ അവിടന്നു തീര്‍പ്പ് കല്‍പിച്ചിരുന്നുള്ളൂ. (ഇബ്‌നു ഹിശാം, സീറത്തുന്നബി, വാള്യം 2, പുറം 201).
സര്‍ തോമസ് ആര്‍ണള്‍ഡ് എഴുതുന്നു: 'അമുസ്‌ലിം സമൂഹങ്ങള്‍ മിക്കവാറും പൂര്‍ണമായ സ്വയംഭരണാവകാശം അനുഭവിക്കുകയുണ്ടായി. എന്തുകൊണ്ടെന്നാല്‍ തങ്ങളുടെ ആഭ്യന്തരകാര്യങ്ങള്‍ കൈകാര്യം ചെയ്യാനുള്ള സ്വാതന്ത്യ്രം ഭരണകൂടം അവരുടെ കരങ്ങളില്‍ തന്നെ ഏല്‍പിച്ചിരുന്നു. മതപരമായ വിഷയങ്ങളില്‍ തീരുമാനമെടുക്കുവാനുള്ള അധികാരം അവരുടെ പുരോഹിതന്മാര്‍ക്ക് ലഭിച്ചു. അവരുടെ ചര്‍ച്ചുകളും മഠങ്ങളും യാതൊരു ഊനവും തട്ടാതെ നിലനിര്‍ത്താനനുവദിച്ചു''(ഇസ്‌ലാം: പ്രബോധനവും പ്രചാരവും, പുറം 78).
എന്നാല്‍ ഇസ്‌ലാമികരാഷ്ട്രത്തില്‍ അനീതികളും വിവേചനങ്ങളും നടക്കുകയില്ല. ജാതിമതകക്ഷി ഭേദമന്യേ നിഷ്‌കൃഷ്ടമായ നീതി നടപ്പിലാക്കപ്പെടും. ഖുര്‍ആന്‍ കല്‍പിക്കുന്നു: 'വിശ്വസിച്ചവരേ, നിങ്ങള്‍ അല്ലാഹുവിനു വേണ്ടി നേര്‍മാര്‍ഗത്തില്‍ നിലകൊള്ളുന്നവരും നീതിക്ക് സാക്ഷ്യം വഹിക്കുന്നവരുമാകുക. ഒരു ജനതയോടുള്ള വിരോധം നീതി നടപ്പിലാക്കപ്പെടാതിരിക്കാന്‍ നിങ്ങളെ പ്രേരിപ്പിക്കാതിരിക്കട്ടെ. നീതി പാലിക്കുവിന്‍. അതാണ് ദൈവഭക്തിക്ക് ഏറ്റവും അനുയോജ്യം''(5:8).
ഇസ്‌ലാമികരാഷ്ട്രത്തില്‍ അമുസ്‌ലിം പൌരന്മാര്‍ക്കെതിരെ കലാപം നടത്തിയാല്‍ ശിക്ഷാ നടപടികള്‍ സ്വീകരിക്കപ്പെടും. വധിച്ചാല്‍ പ്രതിക്രിയ നടപ്പാക്കപ്പെടും. ആരാധനാലയങ്ങള്‍ തകര്‍ക്കപ്പെട്ടാല്‍ തകര്‍ത്തവര്‍ക്കെതിരെ കടുത്ത ശിക്ഷ വിധിക്കുകയും തകര്‍ക്കപ്പെട്ടവ പുനര്‍നിര്‍മിക്കുകയും ചെയ്യും. ആരാധനാലയങ്ങളെപ്പോലെത്തന്നെ മതന്യൂനപക്ഷങ്ങളുടെ ജീവനും സ്വത്തും വിദ്യാസ്ഥാപനങ്ങളും വ്യക്തിനിയമങ്ങളും പൂര്‍ണമായും സുരക്ഷിതമായിരിക്കും. അവയുടെയൊന്നും നേരെ ഒരുവിധ കൈയേറ്റവും അനുവദിക്കപ്പെടുന്നതല്ല. അതുകൊണ്ടുതന്നെ ഇസ്‌ലാമിക ഭരണത്തില്‍ മതന്യൂനപക്ഷങ്ങള്‍ സ്വാതന്ത്യ്രവും സുരക്ഷയും അനുവദിക്കുമെന്നതിലൊട്ടും സംശയമില്ല. അവരൊരിക്കലും അല്‍പവും അനീതിക്കോ കൈയേറ്റങ്ങള്‍ക്കോ അവഹേളനങ്ങള്‍ക്കോ ഇരയാവുകയില്ല.
രാജ്യത്തെ ആദിവാസികളും പിന്നോക്ക ജാതിക്കാരുമെല്ലാം തങ്ങള്‍ ഹിന്ദുക്കളല്ലെന്ന് ശക്തമായി വാദിച്ചുകൊണ്ടിരിക്കെ, ചോദ്യത്തിലുന്നയിച്ച, ഇന്ത്യയിലെ ഭൂരിപക്ഷം ഹിന്ദുക്കളാണെന്ന പരാമര്‍ശം സൂക്ഷ്മമോ വസ്തുനിഷ്ഠമോ അല്ലെന്നുകൂടി പറയേണ്ടതുണ്ട്.
\chapter{ഇസ്‌ലാമിന്റെ പ്രചാരണവും ആയുധപ്രയോഗവും} 
 \section{ ഇന്ത്യയില്‍ മുസ്‌ലിംകള്‍ ന്യൂനപക്ഷമായതിനാലല്ലേ ഞങ്ങള്‍ക്കെതിരെ ജിഹാദ് നടത്താത്തത്? ഇസ്‌ലാം കാഫിറുകള്‍ക്കെതിരെ ജിഹാദ് ചെയ്യാന്‍ കല്‍പിക്കുന്നില്ലേ?}
 ഇന്ത്യയിലെ മുസ്‌ലിംകള്‍ ജിഹാദ് നടത്തുന്നില്ലെന്ന ധാരണ ശരിയല്ല. കാരണം മുഴുവന്‍ വിശ്വാസികളും ജിഹാദ് നിര്‍വഹിക്കാന്‍ ബാധ്യസ്ഥരാണ്. അതില്‍നിന്ന് മാറിനില്‍ക്കാന്‍ ആര്‍ക്കും അനുവാദമില്ല. നരകശിക്ഷയില്‍നിന്ന് രക്ഷനേടാനും സ്വര്‍ഗലബ്ധിക്കും അതനിവാര്യമാണ്. ഖുര്‍ആന്‍ പറയുന്നു: 'വിശ്വസിച്ചവരേ, വേദനയേറിയ ശിക്ഷയില്‍നിന്ന് നിങ്ങളെ മോചിപ്പിക്കുന്ന ഒരു വ്യാപാരത്തെക്കുറിച്ച് നിങ്ങള്‍ക്കു ഞാനറിയിച്ചു തരട്ടെയോ? നിങ്ങള്‍ ദൈവത്തിലും അവന്റെ ദൂതനിലും വിശ്വസിക്കുക. നിങ്ങളുടെ ജീവധനാദികളാല്‍ ദൈവമാര്‍ഗത്തില്‍ ജിഹാദ് നടത്തുക. അതാണ് നിങ്ങള്‍ക്കുത്തമം. നിങ്ങള്‍ അറിയുന്നവരെങ്കില്‍.'' (61: 1011)
'ജിഹാദ് ചെയ്യുന്നവരും ക്ഷമ അവലംബിക്കുന്നവരുമാണെന്ന് വ്യക്തമാക്കപ്പെടാതെ സ്വര്‍ഗപ്രവേശം സാധ്യമല്ലെന്നു' ഖുര്‍ആന്‍ പ്രഖ്യാപിക്കുന്നു (3: 142). മുസ്‌ലിംസമുദായത്തിന്റെ നിയോഗലക്ഷ്യം തന്നെ ജിഹാദ് നിര്‍വഹിക്കലത്രെ. 'ദൈവമാര്‍ഗത്തില്‍ യഥാവിധി ജിഹാദ് ചെയ്യുക. തന്റെ ദൗത്യത്തിനുവേണ്ടി നിങ്ങളെ നിയോഗിച്ചത് അവനാണ്. മതത്തില്‍ നിങ്ങള്‍ക്കൊരു ക്‌ളിഷ്ടതയും അവനുണ്ടാക്കിയിട്ടില്ല.''(22: 78)
ജിഹാദ് ആര്‍ക്കെങ്കിലും എതിരാണെന്ന ധാരണ പരമാബദ്ധമാണ്. സത്യസംസ്ഥാപനത്തിനുള്ള നിരന്തരയത്‌നമാണത്. മോഹങ്ങളെ മെരുക്കിയെടുത്തും ഇഛകളെ നിയന്ത്രിച്ചും ആഗ്രഹങ്ങളുടെ മേല്‍ മേധാവിത്വം പുലര്‍ത്തിയും സ്വന്തം ജീവിതത്തെ ദൈവനിര്‍ദേശങ്ങള്‍ക്കനുരൂപമാക്കി, യഥാര്‍ഥ സത്യവിശ്വാസിയാവാന്‍ നടത്തുന്ന ശ്രമം പോലും ജിഹാദാണ്. യുദ്ധരംഗത്തുനിന്ന് മടങ്ങവേ ഒരിക്കല്‍ പ്രവാചകന്‍ പറഞ്ഞു: 'നാം ഏറ്റവും ചെറിയ ജിഹാദില്‍നിന്ന് ഏറ്റവും വലിയ ജിഹാദിലേക്ക് തിരിച്ചുവന്നിരിക്കുന്നു''. പ്രവാചക ശിഷ്യന്‍മാര്‍ ചോദിച്ചു: 'ഏതാണ് ഏറ്റവും വലിയ ആ ജിഹാദ്?'' നബിതിരുമേനി അരുള്‍ ചെയ്തു: 'മനസ്സിനോടുള്ള സമരവും സ്വന്തത്തോടുള്ള ജിഹാദുമാണത്.''
കുടുംബത്തിന്റെ ഇസ്‌ലാമീകരണത്തിനായി നടത്തപ്പെടുന്ന വിദ്യാഭ്യാസം, സംസ്‌കരണം, സദുപദേശം, ശിക്ഷണം തുടങ്ങിയവയെല്ലാം ജിഹാദിലുള്‍പ്പെടുന്നു. സത്യസംസ്ഥാപനത്തിനും നന്‍മയുടെ പ്രചാരണത്തിനും ധര്‍മത്തിന്റെ ഉന്നതിക്കുമായുള്ള എഴുത്തും പ്രസംഗവും സംഭാഷണവും ചര്‍ച്ചയും വിദ്യാഭ്യാസ പ്രചാരണവുമെല്ലാം അതില്‍പെടുന്നു. സമുദായത്തിന്റെ അഭ്യുന്നതി ലക്ഷ്യംവെച്ചുള്ള ശാസ്ത്രസാങ്കേതികസാമ്പത്തികസാംസ്‌കാരികകലാസാഹിത്യ പ്രവര്‍ത്തനങ്ങളും ജിഹാദുതന്നെ. ദൈവിക സന്‍മാര്‍ഗത്തിന്റെ സംസ്ഥാപനത്തിനായി നടത്തപ്പെടുന്ന സകല ശ്രമങ്ങളും ദൈവമാര്‍ഗത്തിലെ ജിഹാദാണ്. സമഗ്രമായൊരു പദമാണത്. ബുദ്ധിപരമായും ചിന്താപരമായുമുള്ള വിപ്‌ളവങ്ങളുണ്ടാക്കാനും ജനങ്ങളുടെ വികാരങ്ങളും താല്‍പര്യങ്ങളും സംസ്‌കരിക്കാനും, അവരുടെ വീക്ഷണം ദൈവിക സന്‍മാര്‍ഗത്തിനനുരൂപമാക്കി മാറ്റാനും നടത്തപ്പെടുന്ന വാചികവും ലിഖിതവുമായ സംരംഭങ്ങള്‍തൊട്ട് സത്യത്തിന്റെ ശത്രുക്കളോടുള്ള സായുധ സമരം വരെ അത് വ്യാപിച്ചുകിടക്കുന്നു. വ്യക്തി, തന്റെ അഭിമാനവും ജീവനും സ്വത്തും സംരക്ഷിക്കാന്‍ ശ്രമിക്കുന്നതും ജിഹാദ്തന്നെ; ആ മാര്‍ഗത്തിലെ മരണം ദൈവസരണിയിലെ രക്തസാക്ഷിത്വവും.
അതിനാല്‍, മുസ്‌ലിംകള്‍ ന്യൂനാല്‍ ന്യൂനപക്ഷമായാലും മഹാഭൂരിപക്ഷമായാലും ജിഹാദ് നിര്‍ബന്ധമാണ്. സാഹചര്യമാണ് അതിന്റെ രീതി നിശ്ചയിക്കുക. അത് അമുസ്‌ലിംകള്‍ക്കെതിരെയുള്ള സായുധ സമരമോ യുദ്ധമോ അല്ല. ദൈവികമാര്‍ഗത്തിലെ ത്യാഗപരിശ്രമമാണ്.
ചോദ്യത്തില്‍ സൂചിപ്പിക്കപ്പെട്ടപോലെ അമുസ്‌ലിംകളെല്ലാം കാഫിറുകളല്ല. കാഫിര്‍ എന്നത് ഇസ്‌ലാമിലെ ഒരു സാങ്കേതികപദമാണ്. സത്യവും സന്‍മാര്‍ഗവും യഥാവിധി മനസ്സിലാക്കിയ ശേഷവും ബോധപൂര്‍വം അതിനെ നിഷേധിക്കുന്നവനാണ് കാഫിര്‍. കാഫിറുകളോടും അനിവാര്യമായ കാരണമില്ലാതെ ആയുധമെടുത്ത് അടരാടാന്‍ ഇസ്‌ലാം അനുവദിക്കുന്നില്ല. അവര്‍ മുസ്‌ലിംഭൂരിപക്ഷ പ്രദേശത്തായാലും ന്യൂനപക്ഷ നാടുകളിലായാലും ശരി.

'മുഹമ്മദ് നബി തന്നെ നിരവധി യുദ്ധം നടത്തിയിട്ടില്ലേ? ഇസ്‌ലാമിന്റെ പ്രചാരണത്തില്‍ മുഖ്യ പങ്കുവഹിച്ചത് ആയുധപ്രയോഗമല്ലേ?''
മുഹമ്മദ് നബി പ്രവാചകത്വ ലബ്ധിക്കുശേഷം നീണ്ട പതിമൂന്നു വര്‍ഷം മക്കയില്‍ ഇസ്‌ലാമിക പ്രബോധന പ്രവര്‍ത്തനങ്ങള്‍ നടത്തി. ആ ഘട്ടത്തില്‍ ഇസ്‌ലാമിന്റെ ശത്രുക്കള്‍ പ്രവാചകനെയും അനുയായികളെയും നിര്‍ദയം മര്‍ദിച്ചു. കൊടിയ പീഡനങ്ങള്‍ക്കിരയാക്കി. അസഹ്യമാംവിധം അവഹേളിക്കുകയും നാട്ടില്‍നിന്ന് ബഹിഷ്‌കരിക്കുകയും ചെയ്തു. അപ്പോഴൊന്നും പ്രവാചകന്‍ അവയെ പ്രതിരോധിക്കുകയോ പ്രതികാര നടപടികള്‍ സ്വീകരിക്കുകയോ പ്രത്യാക്രമണങ്ങള്‍ നടത്തുകയോ ചെയ്തില്ല. അനുയായികള്‍ തിരിച്ചടിക്കാന്‍ അനുവാദം ആരാഞ്ഞെങ്കിലും നബിതിരുമേനി അംഗീകരിച്ചില്ല. 'കൈകള്‍ അടക്കിവയ്ക്കുക. നമസ്‌കാരം നിഷ്ഠയോടെ നിര്‍വഹിക്കുക. സകാത്ത് നല്‍കുക' എന്നതായിരുന്നു അവരോടുള്ള ദൈവശാസന.
മക്കയില്‍ ജീവിതം ദുസ്സഹമായപ്പോള്‍ നബിതിരുമേനിയും അനുയായികളും നാടുവിട്ടു. അവര്‍ മദീനയില്‍ അഭയം തേടി. അവിടെ അവര്‍ സ്ഥാപിച്ച ഇസ്‌ലാമിക സമൂഹത്തിനും രാഷ്ട്രത്തിനുമെതിരെ ശത്രുക്കള്‍ തങ്ങളുടെ എതിര്‍പ്പും അതിക്രമവും തുടര്‍ന്നു. അപ്പോള്‍ മാത്രമാണ് പ്രവാചകന്നും അനുചരന്‍മാര്‍ക്കും തിരിച്ചടിക്കാന്‍ അനുവാദം ലഭിച്ചത്. അല്ലാഹു അറിയിച്ചു: 'തീര്‍ച്ചയായും അല്ലാഹു സത്യവിശ്വാസികള്‍ക്കുവേണ്ടി പ്രതിരോധിക്കുന്നു. അല്ലാഹു കൃതഘ്‌നരായ വഞ്ചകരെയാരെയും ഇഷ്ടപ്പെടുകയില്ല. ആര്‍ക്കെതിരെ യുദ്ധം നടത്തപ്പെടുന്നുവോ ആ അക്രമത്തിനിരയാകുന്നവര്‍ക്ക് തിരിച്ചടിക്കാന്‍ അനുമതി നല്‍കപ്പെട്ടിരിക്കുന്നു. കാരണം അവര്‍ മര്‍ദിക്കപ്പെട്ടിരിക്കുന്നു. നിശ്ചയമായും അവരെ സഹായിക്കാന്‍ കഴിവുറ്റവനത്രെ അല്ലാഹു. ഞങ്ങളുടെ നാഥന്‍ ദൈവമാണ് എന്നു പറഞ്ഞതല്ലാതെ യാതൊരു ന്യായവുമില്ലാതെ സ്വന്തം വീടുകളില്‍നിന്ന് പുറത്താക്കപ്പെട്ടവരാണവര്‍. അല്ലാഹു ചിലരെക്കൊണ്ട് മറ്റു ചിലരെ തടഞ്ഞിരുന്നില്ലെങ്കില്‍ മഠങ്ങളും ദേവാലയങ്ങളും പ്രാര്‍ഥനാ മന്ദിരങ്ങളും ദൈവനാമം ധാരാളമായി സ്മരിക്കപ്പെടുന്ന പള്ളികളും തകര്‍ക്കപ്പെടുമായിരുന്നു. തന്നെ സഹായിക്കുന്നവരെ അല്ലാഹുവും സഹായിക്കുകതന്നെ ചെയ്യും. തീര്‍ച്ചയായും ശക്തനും അജയ്യനുമാകുന്നു അല്ലാഹു.'' (22: 3840)
ഇസ്‌ലാം യുദ്ധം അനുവദിച്ചത് എല്ലാവരുടെയും ആരാധനാ സ്വാതന്ത്യ്രം സംരക്ഷിക്കാനാണെന്ന് ഈ വിശുദ്ധ വചനം അസന്ദിഗ്ധമായി വ്യക്തമാക്കുന്നു.
അതിക്രമം കാണിക്കാത്തവരാരെയും അക്രമിക്കാന്‍ ഇസ്‌ലാം അനുവദിക്കുന്നില്ല. ആരോടാണ്, എപ്പോഴാണ് യുദ്ധം അനുവദിക്കപ്പെട്ടതും ആജ്ഞാപിക്കപ്പെട്ടതുമെന്ന് വ്യക്തമാക്കുന്ന നിരവധി ഖുര്‍ആന്‍ വാക്യങ്ങളുണ്ട്. ചിലതു മാത്രമിവിടെ ഉദ്ധരിക്കാം:
'നിങ്ങളോട് യുദ്ധം ചെയ്യുന്നവരോട് അല്ലാഹുവിന്റെ മാര്‍ഗത്തില്‍ നിങ്ങളും യുദ്ധം ചെയ്യുക. നിങ്ങള്‍ അതിക്രമം പ്രവര്‍ത്തിക്കരുത്. അതിക്രമകാരികളെ ഒരിക്കലും അല്ലാഹു ഇഷ്ടപ്പെടുകയില്ല. യുദ്ധത്തില്‍ ഏറ്റുമുട്ടുമ്പോള്‍ അവരെ കണ്ടിടത്തുവച്ച് നിങ്ങള്‍ കൊന്നുകളയുക. അവര്‍ നിങ്ങളെ പുറംതള്ളിയ മാര്‍ഗത്തിലൂടെ നിങ്ങള്‍ അവരെയും പുറന്തള്ളുക. കുഴപ്പം കൊലപാതകത്തെക്കാള്‍ ഗുരുതരമത്രെ. മക്കയിലെ പള്ളിയുടെ പരിസരത്തുവച്ച് നിങ്ങളവരോട് യുദ്ധം ചെയ്യരുത്, അവര്‍ അവിടെ വച്ച് നിങ്ങളോട് യുദ്ധം ചെയ്യുന്നതുവരെ. ഇനി, നിങ്ങളോടവര്‍ യുദ്ധം ചെയ്താല്‍ ആ യുദ്ധത്തില്‍ നിങ്ങള്‍ക്കവരെ വധിക്കാം. അപ്രകാരമാണ് നിഷേധികളുടെ പ്രതിഫലം. അഥവാ, അവര്‍ വിരമിച്ചാല്‍ പൊറുക്കുന്നവനും കരുണാനിധിയുമാകുന്നു അല്ലാഹു. നാശം ഇല്ലാതാവുകയും വിധേയത്വം ദൈവത്തിനാവുകയും ചെയ്യുന്നതുവരെ നിങ്ങളവരോട് യുദ്ധം ചെയ്യുക. അഥവാ അവര്‍ വിരമിച്ചാല്‍ പിന്നെ അക്രമികളോടല്ലാതെ ശത്രുതയില്ല.''(2: 190192)
'അവര്‍ സമാധാനത്തിലേക്ക് തിരിഞ്ഞാല്‍ നീയും അതിലേക്കു തിരിയുക''(8: 61). ഇങ്ങോട്ടു യുദ്ധത്തിനൊരുങ്ങിയ ശത്രുക്കളോട് യുദ്ധത്തിന് ഈ വിധം അനുമതിയും ആജ്ഞയും നല്‍കുമ്പോഴും അതിക്രമമോ അനീതിയോ അരുതെന്ന് മതമനുശാസിക്കുന്നു. 'നിങ്ങളെ പുണ്യഭവനത്തില്‍ നിന്ന് തടഞ്ഞുവെന്നതിന്റെ പേരില്‍ ഒരു ജനതയോടുള്ള വിദ്വേഷം, അതിക്രമം ചെയ്യാന്‍ നിങ്ങളെ പ്രേരിപ്പിക്കരുത്. പുണ്യത്തിനും സൂക്ഷ്മതയോടു കൂടിയ ജീവിതത്തിനും വേണ്ടി നിങ്ങള്‍ പരസ്പരം സഹകരിക്കുക. പാപത്തിനും ശത്രുതയ്ക്കും വേണ്ടി നിങ്ങള്‍ സഹകരിക്കരുത്. നിങ്ങള്‍ അല്ലാഹുവെ സൂക്ഷിക്കുക. തീര്‍ച്ചയായും കഠിനമായി ശിക്ഷിക്കുന്നവനാകുന്നു അല്ലാഹു.'' (5: 2)
'ഒരു ജനതയോടുള്ള വിരോധം നിങ്ങളെ നീതിയില്‍നിന്ന് വ്യതിചലിപ്പിക്കാന്‍ പാടില്ലാത്തതാകുന്നു. നീതി പാലിക്കുവിന്‍. അതാണ് ഭക്തിക്ക് അനുയോജ്യം. ദൈവഭക്തിയുള്ളവരായി വര്‍ത്തിക്കുക. നിങ്ങള്‍ പ്രവര്‍ത്തിച്ചുകൊണ്ടിരിക്കുന്നതൊക്കെയും അല്ലാഹു സൂക്ഷ്മമായി അറിയുന്നുണ്ട്.'' (5: 8)
തങ്ങളോട് യുദ്ധത്തിനു വരാത്തവരോട് സ്വീകരിക്കേണ്ട സമീപനമെന്തെന്ന് വിശുദ്ധ ഖുര്‍ആന്‍ അസന്ദിഗ്ധമായി വ്യക്തമാക്കുന്നുണ്ട്: 'മതത്തിന്റെ പേരില്‍ നിങ്ങളോട് യുദ്ധം ചെയ്യാതിരിക്കുകയും നിങ്ങളുടെ ഭവനങ്ങളില്‍നിന്ന് നിങ്ങളെ പുറത്താക്കാതിരിക്കുകയും ചെയ്തവരോട് നിങ്ങള്‍ നന്മ ചെയ്യുന്നതിനെയോ നീതി കാണിക്കുന്നതിനെയോ അല്ലാഹു വിലക്കുന്നില്ല. നിശ്ചയമായും അല്ലാഹു നീതിമാന്മാരെ ഇഷ്ടപ്പെടുന്നു. മതത്തിന്റെ പേരില്‍ നിങ്ങളോടു യുദ്ധം ചെയ്യുകയും നിങ്ങളുടെ ഭവനങ്ങളില്‍നിന്ന് നിങ്ങളെ പുറത്താക്കുകയും നിങ്ങളെ പുറംതള്ളാന്‍ ശത്രുക്കളെ സഹായിക്കുകയും ചെയ്തവരെ മിത്രങ്ങളാക്കുന്നതിനെ മാത്രമേ അല്ലാഹു നിരോധിക്കുന്നുള്ളൂ. അക്കൂട്ടരെ ആരെങ്കിലും മിത്രങ്ങളാക്കിയാല്‍ അവരത്രെ അക്രമികള്‍''(60:89).
വിശുദ്ധ ഖുര്‍ആന്റെ ആജ്ഞകള്‍ക്കും അധ്യാപനങ്ങള്‍ക്കുമനുസരിച്ച് പ്രവര്‍ത്തിച്ച പ്രവാചകന്‍(സ) ഇസ്‌ലാമിനെയും മുസ്‌ലിംകളെയും അക്രമിക്കുകയും തങ്ങളോട് യുദ്ധം നടത്തുകയും ചെയ്തവരോട് മാത്രമാണ് അടരാടിയത്. നബിതിരുമേനിയുടെ കാലത്ത് ആകെ എണ്‍പത്തൊന്ന് പോരാട്ടങ്ങളാണ് നടന്നത്. അതില്‍ 27 എണ്ണത്തിലാണ് പ്രവാചകന്‍ നേരിട്ടു പങ്കെടുത്തത്. മറ്റ് 54 എണ്ണം അനുയായികളുടെ സംഘങ്ങള്‍ നയിച്ച സംഘട്ടനങ്ങളായിരുന്നു. ഈ 81 യുദ്ധങ്ങളിലുമായി ആകെ വധിക്കപ്പെട്ടത് 1018 പേരാണ്. 259 മുസ്‌ലിംകളും 759 ശത്രുക്കളും. അനിവാര്യമായ സാഹചര്യത്തില്‍ യുദ്ധം നയിച്ച് നബിതിരുമേനി കൂട്ടക്കൊലകള്‍ പരമാവധി ഒഴിവാക്കാനാണ് ശ്രമിച്ചിരുന്നതെന്ന് ഈ കണക്കുകള്‍ സംശയാതീതമായി വ്യക്തമാക്കുന്നു.
മര്‍ദിതരുടെ മോചനത്തിനായി മര്‍ദകരോട് പൊരുതാന്‍ ഇസ്‌ലാം അതിന്റെ അനുയായികളെ ആഹ്വാനം ചെയ്യുന്നു: 'അല്ലാഹുവിന്റെ മാര്‍ഗത്തില്‍ എന്തുകൊണ്ട് നിങ്ങള്‍ യുദ്ധം ചെയ്യുന്നില്ല? 'ഞങ്ങളുടെ നാഥാ, അക്രമികളുടേതായ ഈ നാട്ടില്‍നിന്ന് ഞങ്ങളെ നീ മോചിപ്പിക്കേണമേ, നിന്നില്‍നിന്നുള്ള ഒരു രക്ഷകനെ നീ ഞങ്ങള്‍ക്കു നല്‍കേണമേ, നിന്നില്‍നിന്നുള്ള ഒരു സഹായിയെ നീ ഞങ്ങള്‍ക്കു തരേണമേ!' എന്നു പ്രാര്‍ഥിച്ചുകൊണ്ടിരുന്ന ദുര്‍ബലരായ പുരുഷന്മാരുടെയും സ്ത്രീകളുടെയും കുട്ടികളുടെയും മോചനമാര്‍ഗത്തിലും നിങ്ങളെന്തുകൊണ്ട് പൊരുതുന്നില്ല?''(4: 75).
ഈ ശാസന പാലിച്ച് മുസ്‌ലിംകള്‍ റോമാപേര്‍ഷ്യന്‍ സാമ്രാജ്യങ്ങളുടെ പിടിയില്‍പെട്ട് അടിയാളരായി നരകയാതനയനുഭവിച്ച ജനവിഭാഗങ്ങളെ മോചിപ്പിക്കാന്‍ ശ്രമിക്കുകയുണ്ടായി. തദ്ഫലമായി പ്രസ്തുത സാമ്രാജ്യശക്തികള്‍ ഇസ്‌ലാമുമായി ഏറ്റുമുട്ടി. അവ പരാജയമടഞ്ഞ് തകര്‍ന്നപ്പോള്‍ കോളനികളായിരുന്ന ഇറാന്‍, ഇറാഖ്, സിറിയ, ജോര്‍ദാന്‍, ഫലസ്ത്വീന്‍, ഈജിപ്ത് പോലുള്ള രാജ്യങ്ങള്‍ സ്വതന്ത്രമാവുകയും ഇസ്‌ലാമിക രാഷ്ട്രത്തിന്റെ ഭാഗമാവുകയും ചെയ്തു. റോമാപേര്‍ഷ്യന്‍ സാമ്രാജ്യങ്ങളെ കഠിനമായി വെറുത്തിരുന്ന തദ്ദേശീയര്‍ മുസ്‌ലിംകളുടെ ആഗമനം ആവശ്യപ്പെടുകയും സഹര്‍ഷം സ്വാഗതം ചെയ്യുകയുമായിരുന്നു. ഈ വിധം ഇസ്‌ലാമിന്റെ കീഴില്‍ വന്ന നിവാസികളിലാരെയും മതംമാറ്റത്തിന് മുസ്‌ലിംകള്‍ നിര്‍ബന്ധിച്ചിരുന്നില്ല. അതിന് ഇസ്‌ലാം ആരെയും അനുവദിക്കുന്നുമില്ല.
ഇസ്‌ലാമില്‍ ആകൃഷ്ടരായി സ്വയം അത് സ്വീകരിക്കാന്‍ സന്നദ്ധരായവര്‍ മുസ്‌ലിംകളാവുകയും അവശേഷിക്കുന്നവര്‍ പൂര്‍വിക മതത്തില്‍തന്നെ തുടരുകയുമാണുണ്ടായത്. അതുകൊണ്ടുതന്നെ ഇസ്‌ലാമിന്റെ പ്രബോധനവും പ്രചാരണവും ആയുധപ്രയോഗത്തിലൂടെയോ അധികാരശക്തിയുടെ നിര്‍ബന്ധം വഴിയോ ആയിരുന്നില്ല. നിഷ്പക്ഷരായ ചരിത്രകാരന്മാരെല്ലാം ഇക്കാര്യം അസന്ദിഗ്ധമായി വ്യക്തമാക്കിയിട്ടുണ്ട്. ഇസ്‌ലാം പ്രചരിച്ചത് വാളുകൊണ്ടാണെന്ന വ്യാജപ്രചാരണത്തെ ഖണ്ഡിച്ചുകൊണ്ട് സര്‍ തോമസ് ആര്‍ണള്‍ഡ് ബൃഹത്തായ ഒരു ഗ്രന്ഥം തന്നെ രചിക്കുകയുണ്ടായി. അദ്ദേഹത്തിന്റെ 'ഇസ്‌ലാം: പ്രബോധനവും പ്രചാരവും' എന്ന കൃതി വിവിധ നാടുകളില്‍ ഇസ്‌ലാം പ്രചരിച്ച പശ്ചാത്തലം വിശദമായി വിവരിച്ച് പാശ്ചാത്യരുടെ ആരോപണങ്ങള്‍ തീര്‍ത്തും അടിസ്ഥാനരഹിതമാണെന്ന് സമര്‍ഥിക്കുന്നു. മധ്യപൂര്‍വദേശത്തെ ഇസ്‌ലാമിന്റെ വ്യാപനം വിവരിച്ചുകൊണ്ട് ആര്‍ണള്‍ഡ് എഴുതുന്നു: 'മേലുദ്ധരിച്ച ഉദാഹരണങ്ങളില്‍നിന്ന് ക്രൈസ്തവ ഗോത്രങ്ങള്‍ സ്വമനസ്സാലെ ഇസ്‌ലാം സ്വീകരിക്കുകയാണുണ്ടായതെന്ന് നമുക്ക് ഊഹിക്കാം. ജേതാക്കളായ മുസ്‌ലിംകളും തുടര്‍ന്നുള്ള മുസ്‌ലിം തലമുറകളും അവരോടു കാണിച്ച സഹിഷ്ണുത അങ്ങനെ ഉറപ്പിക്കാനാണ് നമ്മെ പ്രേരിപ്പിക്കുന്നത്. മുഹമ്മദീയരുടെ ഇടയില്‍ ഇപ്പോഴും ജീവിക്കുന്ന ക്രൈസ്തവ അറബികള്‍ ഈ സഹിഷ്ണുതയുടെ ജീവനുള്ള തെളിവുകളാണ്'' (പേജ് 64). അദ്ദേഹം തന്നെ തുടരുന്നു: 'ക്രൈസ്തവരോട് മുസ്‌ലിം ഭരണത്തിന്റെ ആദ്യകാലത്ത് കാണിച്ച സഹിഷ്ണുതയുടെ വെളിച്ചത്തില്‍ വാളാണ് മതപരിവര്‍ത്തനത്തിന്റെ മുഖ്യ ഉപകരണം എന്ന വാദം ഒട്ടും തൃപ്തികരമല്ലാത്തതാകുന്നു''(പേജ് 83).
ഇന്ത്യയിലെ ഇസ്‌ലാമിക പ്രചാരണത്തെ സംബന്ധിച്ച് ഡോ. ഈശ്വരി പ്രസാദ് പറയുന്നു: 'ഉദ്യോഗമേന്മയ്ക്കും ഫ്യൂഡല്‍ സ്ഥാനങ്ങള്‍ക്കും കൊതിച്ചിരുന്ന കുറേയേറെ സവര്‍ണര്‍ മുസ്‌ലിം ഭരണത്തോടുകൂടി ഇസ്‌ലാമിലേക്ക് വന്നു. എന്നാല്‍ കൂടുതലും താഴ്ന്ന ജാതിയില്‍ പെട്ട അധഃകൃതരായിരുന്നു. അതിനു കാരണം ക്രൂരമായ ജാതിവ്യവസ്ഥ ഹിന്ദുമതത്തില്‍ അവര്‍ക്ക് മൃഗതുല്യത മാത്രമാണ് നല്‍കിയതെന്നും ഇസ്‌ലാം സമത്വസാഹോദര്യം സംഭാവന ചെയ്തുവെന്നതുമാണ്''(ഞമാ ഏീുമഹ, കിറശമി ങൗഹെശാ െഅ ജീഹശശേരമഹ ഒശേെീൃ്യ, ജ. 2).
ഡോ. റോയ് ചൌധരി എഴുതുന്നു: 'ബംഗാളിലെ അടിച്ചമര്‍ത്തപ്പെട്ട അധഃകൃത വര്‍ഗത്തിന് ഇസ്‌ലാം ഒരാശ്വാസവും സാന്ത്വനവുമായിരുന്നു. സവര്‍ണപീഡയില്‍നിന്നും ഒരു വരം കിട്ടിയിട്ടെന്നപോലെ അവര്‍ പിടഞ്ഞെഴുന്നേറ്റു'' ((ഉൃ. ഞീ്യ ഇവീൗറമൃ്യ, ഒശേെീ്യ ീള ങൗഹെശാ ഞൗഹല, ജ. 14 ഉദ്ധരണം: ചരിത്രപാഠങ്ങള്‍, പേജ് 344).
ദീര്‍ഘകാലം ബംഗാള്‍ സിവില്‍ സര്‍വീസിലായിരുന്ന സര്‍ ഹെന്‍ട്രി കോട്ടന്‍ കിറശമ മിറ ഒീാല മളളമശൃ െഎന്ന ഗ്രന്ഥത്തിലെഴുതുന്നു: 'കിഴക്കന്‍ ബംഗാളിലെ മുസ്‌ലിംകള്‍ മുഴുക്കെ അധഃകൃതരും അധഃസ്ഥിതരുമായ ഹിന്ദുക്കളുടെ സന്താനപരമ്പരയാണ്. സാമൂഹികനീതിക്കുവേണ്ടി മതപരിവര്‍ത്തനം ചെയ്തവരാണ് അവരുടെ പിതാക്കന്മാര്‍''(ഉദ്ധരണം: കയശറ പേജ്:335).
ഹര്‍ബന്‍സ് മുഖിയ എഴുതുന്നു: 'മതപരിവര്‍ത്തനം നടന്നിരുന്നുവെന്ന കാര്യം ആരും നിഷേധിക്കുന്നില്ല. ബഹുജനതലത്തിലധികവും സംഭവിച്ചത് സ്വമേധയാ ആയിരുന്നു. ജനങ്ങളുടെ ഇടയില്‍ താമസിക്കുകയും അവരോട് അവരുടെ ഭാഷയില്‍ സംസാരിക്കുകയും ചെയ്തുപോന്ന സ്വൂഫീസന്യാസിമാരുടെ സ്വാധീനശക്തികൊണ്ടാവാം അങ്ങനെ സംഭവിച്ചത്''(വര്‍ഗീയതയും പ്രാചീന ഭാരത ചരിത്രരചനയും, പേജ് 48).
മഹാത്മാഗാന്ധി എഴുതുന്നു: 'ഇന്നു മനുഷ്യവര്‍ഗത്തിലെ ജനങ്ങളുടെ ഹൃദയങ്ങളില്‍, നിര്‍വിവാദമായ ആധിപത്യം പുലര്‍ത്തുന്ന ഒരാളുടെ ജീവിതത്തിന്റെ ഏറ്റവും ഉത്തമമായ വശം അറിയാന്‍ ഞാന്‍ ശ്രമിച്ചു. അക്കാലത്ത് ജീവിതസരണിയില്‍ ഇസ്‌ലാമിന് സ്ഥാനം നേടിക്കൊടുത്തത് വാളായിരുന്നില്ലെന്ന് മുമ്പെന്നത്തേക്കാളും എനിക്കു ബോധ്യമായിരിക്കുന്നു. പ്രവാചകന്റെ അചഞ്ചലമായ ലാളിത്യവും ഉദാത്തമായ ആത്മബലവും പ്രതിജ്ഞകളോടുള്ള ദൃഢമായ പ്രതിബദ്ധതയും കൂട്ടുകാരോടും അനുയായികളോടുമുള്ള അതിരറ്റ അര്‍പ്പണവും നിര്‍ഭയത്വവും ദൈവത്തിലും തന്റെ ദൗത്യത്തിലുമുള്ള പരമമായ വിശ്വാസവുമായിരുന്നു, വാളായിരുന്നില്ല എല്ലാറ്റിനെയും അവരുടെ മുമ്പിലെത്തിച്ചതും എല്ലാ തടസ്സങ്ങളെയും അതിജീവിക്കാന്‍ സഹായിച്ചതും''(യങ്ങ് ഇന്ത്യ 1691924).
പാശ്ചാത്യന്‍ ഗ്രന്ഥകാരനായ മിസ്‌റര്‍ റോബെര്‍ട്ട് സെന്‍ എഴുതുന്നു: 'മതാവേശവും ഇതര മതസ്ഥരോടുള്ള സഹിഷ്ണുതയുടെ ചൈതന്യവും ഒരുപോലെ നിലനിര്‍ത്തിയവര്‍ മുസ്‌ലിംകള്‍ മാത്രമാകുന്നു. അവര്‍ വാളെടുത്തതോടൊപ്പം തന്നെ ഇസ്‌ലാമില്‍ താല്‍പര്യമില്ലാത്തവരെ അവരുടെ മതം മുറുകെ പിടിക്കാന്‍ അനുവദിക്കുകയുണ്ടായി''(ഇശ്ശഹശ്വമശേീി ീള അൃമയ,െ ഉദ്ധരണം: ഇസ്‌ലാമും മതസഹിഷ്ണുതയും, പേജ് 68).
പ്രവാചകന്‍ പങ്കെടുത്തതും നയിച്ചതുമായ യുദ്ധങ്ങള്‍ അനിവാര്യ സാഹചര്യത്തില്‍ അതിക്രമങ്ങളെ പ്രതിരോധിക്കാനും അനീതിയും അധര്‍മവും അവസാനിപ്പിക്കാനും വേണ്ടിയായിരുന്നുവെന്നും ഇസ്‌ലാം പ്രചരിച്ചത് വാളാലാണെന്ന വാദം വ്യാജാരോപണം മാത്രമാണെന്നും ഈ വിധം വസ്തുതകളെ വിശകലനം ചെയ്യാന്‍ സന്നദ്ധരായവരെല്ലാം സംശയലേശമന്യേ വ്യക്തമാക്കിയ വസ്തുതയത്രെ.

\chapter{ഖുര്‍ആന്‍ വാക്യങ്ങളില്‍ വൈരുധ്യമോ? }
\section{  കാര്യങ്ങളെല്ലാം നടക്കുക ദൈവവിധിയനുസരിച്ചാണെന്ന് കാണിക്കുന്ന കുറേ ഖുര്‍ആന്‍ വാക്യങ്ങളും മനുഷ്യകര്‍മങ്ങള്‍ക്കനുസൃതമായ ഫലമാണുണ്ടാവുകയെന്ന് വ്യക്തമാക്കുന്ന നിരവധി വചനങ്ങളും തെളിയിക്കുന്നത്, വിധിവിശ്വാസത്തെ സംബന്ധിച്ച ഖുര്‍ആന്‍ വാക്യങ്ങളില്‍ പരസ്പര വൈരുധ്യമുണ്ടെന്നല്ലേ?}
 വിധിവിശ്വാസത്തെ സംബന്ധിച്ച വിശുദ്ധ ഖുര്‍ആന്‍ വചനങ്ങളില്‍ ഒരുവിധ വൈരുധ്യവുമില്ല. മാത്രമല്ല, അവ പരസ്പരം വ്യാഖ്യാനിക്കുന്നവയും വിശദീകരിക്കുന്നവയുമാണ്. ഒരു ഉദാഹരണത്തിലൂടെ ഇത് വ്യക്തമാക്കാം.
നല്ല നിലയില്‍ ഉയര്‍ന്ന നിലവാരത്തോടെ അച്ചടക്കപൂര്‍ണമായി നടത്തപ്പെടുന്ന ഒരു മാതൃകാവിദ്യാലയം. സമര്‍ഥനായ പ്രധാനാധ്യാപകന്‍. ആത്മാര്‍ഥതയുള്ള സഹപ്രവര്‍ത്തകര്‍. യോഗ്യരായ വിദ്യാര്‍ഥികള്‍. കുട്ടികളുടെ കാര്യം ജാഗ്രതയോടെ ശ്രദ്ധിക്കുന്ന രക്ഷിതാക്കള്‍. അങ്ങനെ എല്ലാവരും വിദ്യാലയത്തിന്റെ നന്മയിലും ഉയര്‍ച്ചയിലും നിര്‍ണായകമായ പങ്കുവഹിക്കുന്നു. സ്ഥാപനം നന്നാവുകയെന്ന സംഭവത്തിനു പിന്നില്‍ ഒന്നിലേറെ കാരണങ്ങളും നിരവധി ഘടകങ്ങളുമുണ്ടെന്നര്‍ഥം. ഇത്തരമൊരവസ്ഥയില്‍ വിദ്യാലയം മാതൃകായോഗ്യമാകാന്‍ കാരണം പ്രധാനാധ്യാപകനാണെന്നു പറയാം. അധ്യാപകരാണെന്നും വിദ്യാര്‍ഥികളാണെന്നും രക്ഷിതാക്കളാണെന്നുമൊക്കെ പറയാം. ഇതിലേതു പറഞ്ഞാലും കളവാകില്ല. ഒരിക്കലൊന്നും മറ്റൊരിക്കലൊന്നും പറഞ്ഞാല്‍ പരസ്പര വിരുദ്ധവുമാവുകയില്ല. ആവശ്യാനുസൃതം ഓരോ കാരണവും എടുത്തു കാണിക്കുകയാണെങ്കില്‍ സന്ദര്‍ഭാനുസൃതമായ സത്യപ്രസ്താവം മാത്രമേ ആവുകയുള്ളൂ. എന്നാല്‍ ഇതില്‍ ഏതെങ്കിലും ഒരു ഘടകം മാത്രമാണ് സംഭവത്തിനു പിന്നിലെന്ന് വിശ്വസിക്കുന്നവരോട് അത് നിഷേധിക്കേണ്ടിയും വരും.
മനുഷ്യകര്‍മങ്ങളുടെ സ്ഥിതിയും ഈവിധം തന്നെ. ഒരാള്‍ നടന്നുപോകവെ വഴിയിലൊരു വൃദ്ധന്‍ വീണുകിടക്കുന്നത് കാണാനിടയാകുന്നു. അയാള്‍ക്ക് വേണമെങ്കില്‍ വൃദ്ധനെ കാണാത്തവിധം നടന്നുനീങ്ങാം. അങ്ങനെ ചെയ്യാതെ, അയാള്‍ വൃദ്ധനെ താങ്ങിയെടുത്ത് ആശുപത്രിയിലെത്തിക്കുകയും ശുശ്രൂഷിക്കുകയും ചെയ്യുന്നു. അങ്ങനെ ചെയ്യുമ്പോള്‍ തന്നെ വ്യത്യസ്തമായ ഉദ്ദേശ്യത്തോടെ അത് നിര്‍വഹിക്കാവുന്നതാണ്. വൃദ്ധന്റെയോ അയാളുടെ ബന്ധുക്കളുടെയോ നന്ദിയും പ്രത്യുപകാരവും പ്രതിഫലവും പ്രതീക്ഷിക്കാം. അത്തരമൊന്നുമാഗ്രഹിക്കാതെ വൃദ്ധനോടുള്ള സ്‌നേഹകാരുണ്യ വാത്സല്യഗുണകാംക്ഷാ വികാരത്തോടെയും അതു ചെയ്യാം. അപ്പോള്‍ ഈ സംഭവത്തില്‍ മനുഷ്യന്റെ ഭാഗത്തുനിന്ന് അയാളെടുക്കുന്ന തീരുമാനത്തിനും അതിന്റെ പിന്നിലെ ഉദ്ദേശ്യത്തിനും തുടര്‍ന്നുള്ള പ്രവര്‍ത്തനത്തിനുമെല്ലാം അനല്‍പമായ പങ്കുണ്ട്. അതിനാല്‍ വൃദ്ധനെ ആശുപത്രിയിലെത്തിച്ചതും ശുശ്രൂഷിച്ചതും ആ മനുഷ്യനാണെന്ന് പറയുന്നതില്‍ തെറ്റോ അസാംഗത്യമോ ഇല്ല. അതേസമയം, വൃദ്ധനെ താങ്ങിയെടുക്കാനുപയോഗിച്ച കൈകളും ശരീരവും ആരോഗ്യവും കഴിവും കരുത്തുമൊക്കെ നല്‍കിയത് ദൈവമാണ്. വൃദ്ധനോട് അലിവ് തോന്നുകയും ശുശ്രൂഷിക്കാന്‍ തീരുമാനമെടുക്കുകയും ചെയ്ത മനസ്സും ദൈവത്തിന്റെ ദാനം തന്നെ. അതിനാല്‍ ദൈവമാണ് വൃദ്ധനെ രക്ഷിച്ചതെന്ന പ്രസ്താവവും സത്യനിഷ്ഠവും വസ്തുതാപരവുമത്രെ. അപ്പോള്‍ ഇത്തരം സംഭവങ്ങളെ മനുഷ്യനോട് ചേര്‍ത്തുപറയാം, ദൈവത്തോട് ചേര്‍ത്തു പറയാം; മനുഷ്യനോടും ദൈവത്തോടും ഒരേസമയം ചേര്‍ത്തു പറയാം. അപ്രകാരം തന്നെ മനുഷ്യകര്‍മം മാത്രമാണെന്ന് ധരിക്കുന്നവരോട് അതിനെ നിഷേധിക്കാം. ദൈവത്തിന്റെ പങ്ക് ഊന്നിപ്പറയാം. ഈ രീതികളെല്ലാം ഖുര്‍ആന്‍ സ്വീകരിച്ചിട്ടുണ്ട്. മനസ്സിന്റെ തീരുമാനത്തിലും ഉദ്ദേശ്യത്തിലും മനുഷ്യന്റെ പങ്ക് എത്രയെന്ന് ലോകത്ത് ആര്‍ക്കും അറിയുകയില്ല. മനുഷ്യന്റെ ഗ്രാഹ്യവരുതിക്ക് അതീതമായതിനാല്‍ ദൈവം വിശദമായി പറഞ്ഞുതന്നിട്ടുമില്ല.
സംഭവങ്ങളെയും കര്‍മങ്ങളെയും അവയ്ക്കു പിന്നിലെ തീരുമാനങ്ങളെയും വിശുദ്ധ ഖുര്‍ആന്‍ ദൈവവുമായും മനുഷ്യനുമായും ബന്ധപ്പെടുത്തിയതായി കാണാം. കപടവിശ്വാസികള്‍ സ്വീകരിച്ചിരുന്ന സമീപനം തിരുത്തി ഖുര്‍ആന്‍ പറയുന്നു: 'അവര്‍ക്ക് വല്ല നേട്ടവും കിട്ടിയാല്‍ അത് ദൈവത്തിങ്കല്‍നിന്നാണെന്ന് അവര്‍ പറയും. വല്ല വിപത്തും ബാധിച്ചാലോ, നിന്നെ കുറ്റപ്പെടുത്തുകയും ചെയ്യും. പറയുക: എല്ലാം ദൈവത്തിങ്കല്‍ നിന്നുതന്നെ. ഇവര്‍ക്കെന്തു പറ്റി? ഇവര്‍ ഗ്രഹിക്കുന്നില്ലല്ലോ''(ഖുര്‍ആന്‍ 4: 78).
നന്മയും തിന്മയും ദൈവത്തിങ്കല്‍ നിന്നാണെന്ന് വ്യക്തമാക്കുന്ന ഖുര്‍ആന്‍ ഇവിടെ കപടവിശ്വാസിയുടെ തെറ്റായ സമീപനത്തിന് കാരണക്കാര്‍ അവര്‍ തന്നെയാണെന്ന് പറയുകയും അതിന്റെ പേരില്‍ അവരെ കുറ്റപ്പെടുത്തുകയും ചെയ്യുന്നു. ശരിയായ സമീപനം സ്വീകരിക്കാന്‍ സാധ്യതയുണ്ടായിട്ടും മറിച്ചൊരു നിലപാട് അവലംബിച്ചതിനാലാണ് ഖുര്‍ആന്‍ അവരെ ആക്ഷേപിക്കുന്നത്.
ഗുണദോഷങ്ങളില്‍ മനുഷ്യകര്‍മം പോലെ ദൈവവിധിക്കും പങ്കുള്ളതിനാല്‍ അതിന്റെ കാരണത്തെ സ്രഷ്ടാവിലേക്ക് ചേര്‍ത്തുപറയുന്ന സമീപനം സ്വീകരിച്ചതിന് ഖുര്‍ആനില്‍ നിരവധി ഉദാഹരണങ്ങള്‍ കാണാം.
'അല്ലാഹു നിനക്ക് വല്ല ദോഷവും വരുത്തുകയാണെങ്കില്‍ അത് പരിഹരിക്കാന്‍ അവനല്ലാതെ മറ്റാര്‍ക്കും സാധ്യമല്ല. അഥവാ, അവന്‍ നിനക്ക് വല്ല ദോഷവും വരുത്തുകയാണെങ്കില്‍ എല്ലാറ്റിനും കഴിവുള്ളവനത്രെ അവന്‍''(6:17). 'അല്ലാഹു താനിഛിക്കുന്നവരെ സന്മാര്‍ഗത്തിലാക്കുന്നു. താനിഛിക്കുന്നവരെ ദുര്‍മാര്‍ഗത്തിലാക്കുന്നു. പ്രതാപശാലിയും യുക്തിജ്ഞനുമത്രെ അവന്‍''(14: 4).
അതോടൊപ്പം സന്മാര്‍ഗദുര്‍മാര്‍ഗ പ്രാപ്തിയില്‍ മനുഷ്യന്റെ പങ്കും ഖുര്‍ആന്‍ ഊന്നിപ്പറയുന്നു: 'ആര്‍ അണുമണിത്തൂക്കം നന്മ ചെയ്യുന്നുവോ അവന്‍ അത് കണ്ടെത്തുകതന്നെ ചെയ്യും. ആര്‍ അണുമണിത്തൂക്കം തിന്മ ചെയ്യുന്നുവോ അവന്‍ അതും കണ്ടെത്തും''(99: 7,8). 'ഒരുത്തനും മറ്റൊരുത്തന്റെ ഭാരം ചുമക്കുന്നില്ല. മനുഷ്യന് അവന്‍ പ്രവര്‍ത്തിച്ചതല്ലാതെ ഇല്ല'' (53: 38). 'അല്ലാഹു ആരെയും അവന്റെ കഴിവിനതീതമായതിന് കല്‍പിക്കുകയില്ല. ഓരോരുത്തര്‍ക്കും അവര്‍ പ്രവര്‍ത്തിച്ചതിനുള്ള പ്രതിഫലവും ശിക്ഷയുമാണുണ്ടാവുക''(2: 286). 'നിനക്ക് വല്ല ദോഷവും ബാധിച്ചിട്ടുണ്ടെങ്കില്‍ അത് നിന്റെ ഭാഗത്തുനിന്നു തന്നെയുള്ളതാണ്''(4: 79). 'എന്നാല്‍ വിശ്വസിക്കുകയും സല്‍ക്കര്‍മങ്ങളനുഷ്ഠിക്കുകയും ചെയ്തവര്‍ക്ക് അവരുടെ പ്രതിഫലം പൂര്‍ണമായി ലഭിക്കുന്നതാണ്. അക്രമികളെ അല്ലാഹു ഇഷ്ടപ്പെടുന്നില്ല''(3: 57). 'അവിശ്വസിക്കുകയും നമ്മുടെ ദൃഷ്ടാന്തങ്ങള്‍ കളവാക്കുകയും ചെയ്തവരാരോ അവരത്രെ നരകാവകാശികള്‍. അവരതില്‍ നിത്യവാസികളത്രെ''(2: 39). 'സത്യനിഷേധികളേ, ഇന്ന് നിങ്ങള്‍ ഒഴികഴിവ് ബോധിപ്പിക്കേണ്ട. നിങ്ങള്‍ പ്രവര്‍ത്തിച്ചതിന്റെ ഫലം തന്നെയാണ് നിങ്ങളിന്ന് അനുഭവിക്കുന്നത്''(66: 7).
നന്മതിന്മകളില്‍ മനുഷ്യന്റെ പങ്കും ദൈവവിധിയും എവ്വിധം ബന്ധപ്പെടുന്നുവെന്നും ഖുര്‍ആന്‍ വ്യക്തമാക്കുന്നു: 'പറയുക: ദൈവം ഇഛിക്കുന്നവരെ അവന്‍ ദുര്‍മാര്‍ഗത്തിലാക്കുകയും തന്നിലേക്ക് ഖേദിച്ചു മടങ്ങുന്നവരെ സന്മാര്‍ഗത്തിലാക്കുകയും ചെയ്യുന്നു''(13: 27). 'സന്മാര്‍ഗം സ്പഷ്ടമായിക്കഴിഞ്ഞശേഷവും ആരെങ്കിലും ദൈവദൂതനെ ധിക്കരിക്കുകയും സത്യവിശ്വാസികളുടേതല്ലാത്ത മാര്‍ഗം പിന്തുടരുകയുമാണെങ്കില്‍ അവന്‍ സ്വയം തിരിഞ്ഞുകളഞ്ഞ ഭാഗത്തേക്കു തന്നെ നാം അവനെ തിരിക്കുന്നതാണ്''(4: 115).
ചുരുക്കത്തില്‍, കര്‍മങ്ങളുടെ ഫലത്തെ സംബന്ധിച്ച് മനസ്സിലാക്കാന്‍ സാധിക്കുന്ന മനുഷ്യന് ചിന്തിക്കാനും തീരുമാനിക്കാനും അതനുസരിച്ച് പ്രവര്‍ത്തിക്കാനും സ്വാതന്ത്യ്രം നല്‍കപ്പെട്ടിരിക്കുന്നു. എന്നാല്‍ ഈ സ്വാതന്ത്യ്രം അപരിമിതമോ അനിയന്ത്രിതമോ അല്ല. ദൈവേഛയ്ക്കും വിധിക്കും വിധേയമാണ്. ഈ പരിമിതിയുടെ പരിധിക്കുള്ളില്‍ നല്‍കപ്പെട്ട സ്വാതന്ത്യ്രത്തിന്റെ തോതനുസരിച്ച ബാധ്യത മാത്രമേ മനുഷ്യന്റെ മേല്‍ ചുമത്തപ്പെട്ടിട്ടുള്ളൂ. ആ ബാധ്യതയുടെ നിര്‍വഹണവും ലംഘനവുമാണ് ജീവിതത്തിന്റെ ജയാപജയങ്ങളും സ്വര്‍ഗനരകങ്ങളും തീരുമാനിക്കുക. അതിനാല്‍ കഴിവിനതീതമായ ഒന്നിനും അല്ലാഹു ആരെയും നിര്‍ബന്ധിക്കുന്നില്ല. ആരോടും അനീതി കാണിക്കുന്നുമില്ല. വിധിയുടെയും മനുഷ്യസ്വാതന്ത്യ്രത്തിന്റെയും അവസ്ഥയും അവയ്ക്കിടയിലെ ബന്ധവും മനുഷ്യന് മനസ്സിലാക്കാന്‍ സാധിക്കുംവിധം വിവരിക്കുന്ന വിശുദ്ധ ഖുര്‍ആന്‍ വാക്യങ്ങളില്‍ ഒട്ടും വൈരുധ്യവുമില്ല.
ദൈവവിധിയുടെയും മനുഷ്യ സ്വാതന്ത്യ്രത്തിന്റെയും അവസ്ഥയെ സംബന്ധിച്ച ഇസ്‌ലാമിക വീക്ഷണം യഥാവിധി അനാവരണം ചെയ്യുന്ന ഒരു സംഭവം രണ്ടാം ഖലീഫ ഉമറുല്‍ ഫാറൂഖിന്റെ കാലത്ത് നടക്കുകയുണ്ടായി. ഫലസ്ത്വീനില്‍ പ്‌ളേഗ് ബാധിച്ചു. ഏറെ കഴിയും മുമ്പേ അത് സിറിയയിലേക്കും പടര്‍ന്നുപിടിച്ചു. അതിവേഗം ആളിപ്പടര്‍ന്ന രോഗം അത് സ്പര്‍ശിക്കുന്നവരെയെല്ലാം കൊന്നൊടുക്കി. മരുന്നും ചികിത്സയുമൊന്നും ഫലിച്ചില്ല. ഒരൊറ്റ മാസത്തിനകം പതിനയ്യായിരം പേര്‍ മരിച്ചു. ഈ വിപത്തിനെ സംബന്ധിച്ച് വിവരമറിഞ്ഞ ഉമറുല്‍ ഫാറൂഖ് ഒരു സംഘം സൈനികരോടൊപ്പം സിറിയയിലേക്ക് പുറപ്പെട്ടു. വഴിമധ്യേ തന്റെ അടുത്ത അനുയായികളുമായി, എന്തു ചെയ്യണമെന്ന് കൂടിയാലോചിച്ചു. അനന്തര നടപടികള്‍ക്ക് നിര്‍ദേശം നല്‍കിയശേഷം, രോഗബാധിത പ്രദേശത്തേക്ക് പോകേണ്ടതില്ലെന്ന് തീരുമാനിച്ചു. പകര്‍ച്ചവ്യാധി ബാധിച്ചേടത്തേക്കുള്ള യാത്ര അപകടം വരുത്തിയേക്കുമെന്നതിനാല്‍ എല്ലാവരും തടസ്സം നില്‍ക്കുകയായിരുന്നു. വിവരമറിഞ്ഞ അബൂ ഉബൈദ ഖലീഫയോട് രോഷത്തോടെ ചോദിച്ചു: 'ദൈവവിധിയില്‍ നിന്ന് ഓടിപ്പോകയോ?'' ആ സേനാനായകന്റെ അന്തര്‍ഗതം വായിച്ചറിഞ്ഞ ഉമറുല്‍ ഫാറൂഖ് പറഞ്ഞു: 'അതെ, ഒരു ദൈവവിധിയില്‍ നിന്ന് മറ്റൊരു ദൈവവിധിയിലേക്ക്.'' അല്‍പസമയത്തെ മൌനത്തിനുശേഷം അദ്ദേഹം തുടര്‍ന്നു: 'ഒരാള്‍ ഒരു സ്ഥലത്ത് ചെന്നിറങ്ങി. അവിടെ അയാള്‍ക്ക് രണ്ടു താഴ്വരകളുണ്ട്. ഫലസമൃദ്ധമായതും അല്ലാത്തതും. ഫലസമൃദ്ധമായത് സംരക്ഷിക്കുന്നവനും അല്ലാഹുവിന്റെ വിധിയനുസരിച്ചല്ലേ പ്രവര്‍ത്തിക്കുന്നത്? അല്ലാത്തത് നോക്കി നടത്തുന്നവനും അല്ലാഹുവിന്റെ വിധിയനുസരിക്കുകയല്ലേ ചെയ്യുന്നത്?''
ഇസ്‌ലാമിലെ വിധിവിശ്വാസം വിപത്തുകളുടെയും വിനാശത്തിന്റെയും താക്കോലാവരുതെന്നും പുരോഗതിയുടെയും നേട്ടങ്ങളുടെയും വിജയത്തിന്റെയും വഴിയില്‍ വിഘാതം വരുത്തരുതെന്നും വ്യക്തമായ ഭാഷയില്‍ സമൂഹത്തെ പഠിപ്പിക്കുകയായിരുന്നു ഉമറുല്‍ ഫാറൂഖ്. ദൈവവിധിയുടെ വരുതിയില്‍ മനുഷ്യനു നല്‍കപ്പെട്ട സ്വാതന്ത്യ്രത്തെ പരമാവധി പ്രയോജനപ്പെടുത്തണമെന്ന ഉദ്‌ബോധനവും അതുള്‍ക്കൊള്ളുന്നു.
\chapter{വിധിവിശ്വാസം ഭൗതികവാദത്തിലും ഇസ്‌ലാമിലും} 
   \section{ദൈവം സര്‍വശക്തനും സര്‍വജ്ഞനുമാണല്ലോ. എങ്കില്‍ ലോകത്തിലെ മനുഷ്യരെല്ലാം എവ്വിധമായിരിക്കുമെന്നും എങ്ങനെയാണ് ജീവിക്കുകയെന്നും ദൈവത്തിന് മുന്‍കൂട്ടി അറിയില്ലേ? ദിവ്യജ്ഞാനത്തിന് തെറ്റുപറ്റാനിടയില്ലാത്തതിനാല്‍ എല്ലാം മുന്‍കൂട്ടി ദൈവം തീരുമാനിച്ചു വച്ചിട്ടുണ്ടെന്നും അതിലൊട്ടും മാറ്റം വരുത്താനാര്‍ക്കും സാധ്യമല്ലെന്നുമല്ലേ ഇതിനര്‍ഥം? ദൈവവിധിക്കനുസൃതമായി ചരിക്കുന്ന മനുഷ്യന്‍ തീര്‍ത്തും അസ്വതന്ത്രനല്ലേ? പിന്നെ, നിര്‍ബന്ധിതമായി ചെയ്യുന്ന കര്‍മങ്ങളുടെ പേരില്‍ അവനെ ശിക്ഷിക്കുന്നത് നീതിയാണോ?}

   
മതവിശ്വാസികളും നിഷേധികളും ഈ വിഷയകമായി നിരന്തരം സംശയങ്ങളുന്നയിക്കുക പതിവാണ്. അതിനാല്‍ ഈ വിഷയം അല്‍പം വിശദമായി തന്നെ പരാമര്‍ശിക്കുന്നത് ഫലപ്രദമായിരിക്കുമെന്ന് പ്രതീക്ഷിക്കുന്നു.
ദൈവവിധിയെയും മനുഷ്യസ്വാതന്ത്യ്രത്തെയും സംബന്ധിച്ച ഇസ്‌ലാമിക വീക്ഷണം വിശദീകരിക്കുന്നതിനു മുമ്പ് മനുഷ്യാവസ്ഥയെ സംബന്ധിച്ച ഭൗതിക സങ്കല്‍പം പരിശോധിക്കുന്നത് നന്നായിരിക്കും.


1. ഭൗതികവാദികള്‍ അപ്രമാദിത്വം കല്‍പിച്ച ജനിതകശാസ്ത്രമനുസരിച്ച് മനുഷ്യന്‍ തീര്‍ത്തും അസ്വതന്ത്രനാണ്. അവന്റെ വികാര വിചാരങ്ങളും തീരുമാനങ്ങളും കര്‍മങ്ങളുമെല്ലാം ശരീരഘടനയുടെ സൃഷ്ടിയാണ്. ജൈവവസ്തുക്കള്‍ രൂപംകൊള്ളുന്നത് ജീവകോശങ്ങളില്‍ നിന്നാണ്. അവയ്ക്കുള്ളിലെ ക്രോമസോമുകളിലെ ജീനുകളിലുള്ള ജനിതകകോഡുകളാണ് ജീവികളുടെ സ്വഭാവം നിര്‍ണയിക്കുന്നത്. മനുഷ്യന്റെ സ്ഥിതിയും ഇതുതന്നെ. അതിനാല്‍ കരുണയും ക്രൂരതയും സല്‍സ്വഭാവവും ദുഃസ്വഭാവവും സത്കര്‍മങ്ങളും ദുഷ്‌കര്‍മങ്ങളുമെല്ലാം ജനിതക കോഡുകള്‍ക്കനുസരിച്ചാണുണ്ടാകുന്നത്. മനുഷ്യന്റെ ശരീരപ്രകൃതം മുതല്‍ വികാര വിചാരങ്ങള്‍ വരെ അവയെ അന്ധമായി അനുധാവനം ചെയ്യുകയാണ്. ജീവിതത്തിലെ മുഴുവന്‍ കാര്യങ്ങളും കര്‍മങ്ങളും ഈ വിധം ജനിതക കോഡുകളില്‍ രേഖപ്പെടുത്തപ്പെട്ടിട്ടുണ്ടെന്ന് അതുസംബന്ധമായ ശാസ്ത്രം അവകാശപ്പെടുന്നു. അവയില്‍നിന്ന് അണുഅളവ് തെറ്റാനോ അവയെ ലംഘിക്കാനോ ആര്‍ക്കും സാധ്യമല്ല. അഥവാ മനുഷ്യന്‍ തീര്‍ത്തും തന്റെ ശരീരഘടനക്ക് വിധേയനാണ്. മസ്തിഷ്‌കത്തിന്റെയും നാഡീവ്യൂഹങ്ങളുടെയും പദാര്‍ഥപരമായ ഘടനയാണ് അവന്റെ ഭാഗധേയം പരിപൂര്‍ണമായും തീരുമാനിക്കുന്നതും നിയന്ത്രിക്കുന്നതും. അതില്‍ ആര്‍ക്കും ഇടപെടാനോ ഏതെങ്കിലും വിധത്തില്‍ പങ്കുവഹിക്കാനോ സാധ്യമല്ല. അതിനാല്‍ ആധുനിക ഭൗതിക ജനിതക ശാസ്ത്രമനുസരിച്ച് മനുഷ്യന്‍ പ്രകൃതിവിധിക്ക് വിധേയനാണ്. അതില്‍നിന്ന് പുറത്തുകടക്കാനാകാത്തവിധം പൂര്‍ണമായും അസ്വതന്ത്രനും. നാം ശരി, തെറ്റ്, നന്മ, തിന്മ, ധര്‍മം, അധര്‍മം, വിനയം, അഹങ്കാരം, കനിവ്, ക്രൂരത എന്നൊക്കെ പറയുന്നത് ജീനുകളിലെ ജനിതകകോഡുകളുടെ ഫലമായുണ്ടാകുന്നവയാണ്. ശാരീരികാരോഗ്യം പോലെ തന്നെയാണ് ജീവിത വിശുദ്ധിയും. മ്‌ളേഛത അനാരോഗ്യം പോലെയും. അതിനാല്‍ മഹല്‍കൃത്യങ്ങളുടെ പേരില്‍ ആളുകളെ വാഴ്ത്തുന്നത് ശാരീരികാരോഗ്യത്തിന്റെ പേരില്‍ അഭിനന്ദിക്കുന്നതുപോലെ അര്‍ഥശൂന്യമത്രെ. ഹീനകൃത്യങ്ങളുടെ പേരില്‍ ഇകഴ്ത്തുന്നത് അനാരോഗ്യത്തിന്റെ പേരില്‍ അപലപിക്കുന്നതുപോലെയും.


2. ചാള്‍സ് ഡാര്‍വിന്റെ പരിണാമസിദ്ധാന്തം മുന്നോട്ടുവയ്ക്കുന്ന ശക്തമായ ഒരാശയമാണ് 'പാരമ്പര്യനിയമം'. അതനുസരിച്ച് മനുഷ്യന്റെ സ്വഭാവവും പെരുമാറ്റവും ജീവിതരീതിയുമെല്ലാം യുഗാന്തരങ്ങളിലൂടെ തലമുറ തലമുറകളായി തുടര്‍ന്നു വരുന്നവയാണ്. പൈതൃകത്തിന്റെ പിടിയില്‍നിന്ന് കുതറിമാറാനാര്‍ക്കും സാധ്യമല്ല. മനുഷ്യന്‍ ചെയ്യുന്ന നന്മയുടെയും തിന്മയുടെയും ബീജങ്ങള്‍ തന്റെ പൂര്‍വപരമ്പരയിലെ ഏതോ പ്രപിതാവിനാല്‍ നിക്ഷേപിക്കപ്പെട്ടതായിരിക്കും. അയാളിലതുണ്ടായത് മുന്‍ഗാമികളിലെ മുതുമുത്തച്ഛന്‍ നിക്ഷേപിച്ചതിന്റെ ഫലവും. അതിനാല്‍ അണ്ടിയില്‍ നിന്ന് മാവുണ്ടാവുന്നതുപോലെ അനിവാര്യമായും പാരമ്പര്യത്തില്‍ നിന്നവ പിറവിയെടുക്കും. അതുകൊണ്ടുതന്നെ ആര്‍ക്കും തങ്ങളുടെ സ്വഭാവരീതികളും പെരുമാറ്റ സമ്പ്രദായങ്ങളും കര്‍മപരിപാടികളും തീരുമാനിക്കുന്നതിലൊരു പങ്കുമില്ല. എല്ലാം അലംഘനീയമായ പൈതൃകത്തിനും പാരമ്പര്യത്തിനും വിധേയമാണ്. മനുഷ്യന്‍ കാലത്തിന്റെ കൈകളിലെ കളിപ്പാവ മാത്രം.
3. കാറല്‍മാര്‍ക്‌സും ഫ്രെഡറിക് ഏംഗല്‍സുമുള്‍പ്പെടെയുള്ള ഭൗതിക ദാര്‍ശനികരുടെ വീക്ഷണത്തില്‍, മനുഷ്യന്‍ സാമൂഹികാവസ്ഥകളുടെയും സാമ്പത്തിക ഘടനയുടെയും സാംസ്‌കാരിക സാഹചര്യത്തിന്റെയും സൃഷ്ടിയാണ്. കേവല ഭൗതിക തലത്തില്‍നിന്ന് ചരിത്രത്തെ വ്യാഖ്യാനിച്ചവര്‍ ഈ വീക്ഷണം സമര്‍ഥിക്കാനായി സമാനമായ സാഹചര്യത്തില്‍ സമൂഹത്തില്‍ കാണപ്പെട്ട വിദൂര സാദൃശ്യങ്ങളെ പോലും തേടിപ്പിടിക്കുകയുണ്ടായി. മനുഷ്യന്റെ സ്വഭാവവും പെരുമാറ്റവും വികാര വിചാരങ്ങളും കര്‍മ സമ്പ്രദായങ്ങളും ജീവിത സമീപനങ്ങളുമെല്ലാം ബാഹ്യമായ കാരണങ്ങളാല്‍ സംഭവിക്കുന്നതാണെന്ന് അവരവകാശപ്പെടുന്നു. ജനങ്ങളിലെ നന്മതിന്മകള്‍ അവര്‍ ജീവിക്കുന്ന സാഹചര്യങ്ങളുടെ സ്വാധീനതയാല്‍ സംഭവിക്കുന്നതാണെന്ന വീക്ഷണമംഗീകരിച്ച ഈ വിഭാഗം ശരിയും തെറ്റും ധര്‍മവും അധര്‍മവും ന്യായവും അന്യായവുമൊക്കെ സാഹചര്യങ്ങളുടെ സൃഷ്ടിയാണെന്ന് വാദിക്കുന്നു. ഇവിടെയും വ്യക്തി തീര്‍ത്തും അസ്വതന്ത്രനാണ്; സാഹചര്യങ്ങളുടെ വിധിക്ക് പൂര്‍ണവിധേയനും.
മതനിരാസത്തിന്റെ മുഖമുദ്രയണിഞ്ഞ ഭൗതിക വാദം അനിഷേധ്യമായി കരുതുന്ന ഉപര്യുക്ത വീക്ഷണങ്ങളെല്ലാം മനുഷ്യന്‍ തീര്‍ത്തും അസ്വതന്ത്രനാണെന്ന് വിളംബരം ചെയ്യുന്നു. സ്വന്തവും സ്വതന്ത്രവുമായ ഇഛയോ തീരുമാനശേഷിയോ ആര്‍ക്കുമില്ല. പ്രകൃതിനിയമങ്ങളാല്‍ ബന്ധിതനാണവന്‍. അലംഘനീയമായ വിധിയുടെ ഇര. ഇവാന്‍ പാവ്‌ലോവ് എഴുതുന്നു: 'പ്രകൃതിയിലുള്ള മറ്റെല്ലാ വ്യവസ്ഥകളെയും പോലെ മനുഷ്യന്‍ ഒരു വ്യവസ്ഥയാണ്. പ്രകൃതിയുടെ അലംഘനീയവും സാമാന്യവുമായ നിയമങ്ങള്‍ക്ക് മനുഷ്യന്‍ കീഴ്‌പെട്ടിരിക്കുന്നു'' (ജ്യെരവീഹീഴശര ഋഃുലൃശാലിമേഹ, ഉദ്ധരണം: അലിജാ അലി ഇസ്സത്ത് ബെഗോവിച്ച് 'ഇസ്‌ലാം രാജമാര്‍ഗം', പുറം 39).
ഫ്രെഡറിക് ഏംഗല്‍സ് പറയുന്നു: 'മനുഷ്യന്‍ അവന്റെ പരിതോവസ്ഥയുടെയും ജോലിയുടെയും ഉല്‍പന്നമാണ്''(ഉദ്ധരണം കയശറ, പുറം 39).
മനുഷ്യനൊഴിച്ചുള്ള ജീവജാലങ്ങളെ സംബന്ധിച്ചേടത്തോളം ഇതെല്ലാം ശരിയാണ്. എന്നാല്‍ മനുഷ്യന്‍ തീര്‍ത്തും അസ്വതന്ത്രനോ പ്രകൃതിനിയമങ്ങളാല്‍ ബന്ധിതനോ അല്ല.
1. പക്ഷികളും മൃഗങ്ങളും ഇഴജീവികളും ജലജീവികളുമെല്ലാം അവയുടെ ജന്മവാസനകള്‍ക്കനുസൃതമായാണ് നിലകൊള്ളുന്നത്. അവയെ ലംഘിക്കാനോ മറികടക്കാനോ അവയ്ക്ക് സാധ്യമല്ല. ജന്മവാസനകളുടെ കാര്യത്തില്‍ മനുഷ്യനേക്കാള്‍ മികച്ചുനില്‍ക്കുന്ന നിരവധി ജീവജാലങ്ങളുണ്ട്. ആ ജന്മവാസനകളാണ് അവയുടെ നിലനില്‍പും സുരക്ഷയും ഉറപ്പുവരുത്തുന്നത്. എന്നാല്‍ ജന്മവാസനകള്‍ നിയന്ത്രിക്കാനവയ്ക്കാവില്ല. വിശന്നുവലഞ്ഞ പശുവിന് പുല്ല് കിട്ടിയാല്‍ തിന്നാതിരിക്കാന്‍ സാധ്യമല്ല. ദാഹിച്ചു വിവശയായ പട്ടിക്ക് വെള്ളം കിട്ടിയാല്‍ കുടിക്കാതിരിക്കാന്‍ കഴിയില്ല. കൊത്താന്‍ വരുന്ന കോഴിയുടെ മുമ്പിലകപ്പെട്ട പൂവന്‍ കോഴി ക്ഷമിക്കുകയോ മാപ്പു പറയുകയോ ചെയ്യാറില്ല. തേനീച്ചക്ക് നിര്‍ണിതമായ രൂപത്തിലല്ലാതെ അതിന്റെ കൂട് നിര്‍മിക്കാനാവില്ല. ദേശാടനപക്ഷികള്‍ക്ക് വിദൂരദേശങ്ങളെ പ്രാപിക്കാതിരിക്കാന്‍ സാധ്യമല്ല. എന്നാല്‍ മനുഷ്യന്റെ സ്ഥിതി മറിച്ചാണ്. കഠിനമായ ദാഹമുള്ളപ്പോള്‍ വെള്ളം കിട്ടിയാല്‍ കുടിക്കാന്‍ കൊതിയുണ്ടെങ്കിലും കുടിക്കാതിരിക്കാനവന് കഴിയും. അതിശക്തമായ വിശപ്പുള്ളപ്പോള്‍ ആഹാരം കിട്ടിയാല്‍ അത് കഴിക്കാനാഗ്രഹമുണ്ടെങ്കിലും തിന്നാതിരിക്കാനവന് സാധിക്കും. 'എടാ' എന്നു വിളിക്കുന്നവനോട് 'പോടാ' എന്നു പറയാന്‍ പൂതിയുണ്ടെങ്കിലും പ്രതിയോഗിയെ പേടിച്ചോ ആര്‍ജിതമായ സംസ്‌കാരം കൊണ്ടോ അങ്ങനെ പറയാതിരിക്കാനവന് ഒട്ടും പ്രയാസമില്ല. സ്വന്തം ജീവിതരീതി നിര്‍ണയിക്കാനും തീരുമാനിക്കാനുമവന് സാധിക്കും. മഹിതമായ ആദര്‍ശങ്ങള്‍ക്കായി ആത്മാര്‍പ്പണം നടത്തുന്ന വിപ്‌ളവകാരി തികഞ്ഞ ഇഛാശക്തി കൊണ്ടാണത് ചെയ്യുന്നത്. മോഹങ്ങളെ മെരുക്കിയെടുക്കാനും ആഗ്രഹങ്ങളെ നിയന്ത്രിക്കാനും ഇഛകളുടെ മേല്‍ മേധാവിത്വം പുലര്‍ത്താനും സാധിക്കുന്ന മനുഷ്യന്‍ ജന്മവാസനകളുടെ അടിമയല്ല; അവയുടെ യജമാനനാണ്. ശരീരഘടനയുടെയോ പാരമ്പര്യത്തിന്റെയോ സാഹചര്യത്തിന്റെയോ സൃഷ്ടിയുമല്ല. മറിച്ച്, സ്വന്തം ഭാഗധേയം തീരുമാനിക്കാന്‍ സാധിക്കുന്ന സ്വതന്ത്രമായ അസ്തിത്വത്തിന്റെയും വ്യക്തിത്വത്തിന്റെയും ഉടമയാണ്.\\
2. നിര്‍ബന്ധിതമായി ചെയ്യുന്ന കര്‍മങ്ങളുടെ പേരില്‍ ആരും വാഴ്ത്തപ്പെടുകയോ ഇകഴ്ത്തപ്പെടുകയോ ഇല്ല. സമ്മാനാര്‍ഹനോ ശിക്ഷാര്‍ഹനോ ആവുകയില്ല. യജമാനന്റെ നിര്‍ബന്ധത്തിനു വഴങ്ങി മോഷണം നടത്തുന്ന അടിമയോ വഴിയാത്രക്കാരെ കല്ലെറിയുന്ന മനോരോഗിയോ കുറ്റക്കാരനായി ഗണിക്കപ്പെടുകയില്ല. എന്നാല്‍ മറ്റാരെങ്കിലുമത് ചെയ്താല്‍ പാപം ചുമത്തപ്പെടുകയും ശിക്ഷിക്കപ്പെടുകയും ചെയ്യും. അതിനാല്‍ ഭൂമിയില്‍ പ്രശംസയും ആക്ഷേപവും ശിക്ഷയും രക്ഷയുമൊക്കെ നല്‍കപ്പെടുന്നുവെന്നത് തന്നെ മനുഷ്യന്‍ നിര്‍ബന്ധിതാവസ്ഥയാല്‍ ബന്ധിതനല്ലെന്നതിനും ഒട്ടൊക്കെ സ്വതന്ത്രനാണെന്നതിനും മതിയായ തെളിവാണ്. അവന്‍ തീര്‍ത്തും അസ്വതന്ത്രനും പ്രകൃതിനിയമങ്ങളാല്‍ വരിഞ്ഞുമുറുക്കപ്പെട്ടവനുമാണെങ്കില്‍ ഹീനകൃത്യങ്ങളുടെ പേരില്‍ ആക്ഷേപിക്കപ്പെടുകയോ ശിക്ഷിക്കപ്പെടുകയോ ചെയ്യുന്നതിലര്‍ഥമില്ല. മഹല്‍കൃത്യങ്ങളുടെ പേരിലുള്ള പ്രശംസയും പ്രതിഫലം നല്‍കലും ആ വിധം തന്നെ.\\
3. ഭൗതിക വാദികള്‍ വിശ്വസിക്കുന്ന പോലെ മനുഷ്യന്‍ തീര്‍ത്തും പ്രകൃതിയുടെ വിധിക്കു വിധേയനെങ്കില്‍ നിയമങ്ങള്‍ക്കും വ്യവസ്ഥകള്‍ക്കും ചട്ടങ്ങള്‍ക്കും ചിട്ടകള്‍ക്കും ഒട്ടും പ്രസക്തി ഇല്ല. മസ്തിഷ്‌കത്തിലെയും നാഡീവ്യൂഹത്തിലെയും ജീനുകളിലെ ജനിതക കോഡുകള്‍ക്കും പാരമ്പര്യത്തിനും സാമൂഹികാവസ്ഥക്കുമൊത്തവന്‍ ചരിച്ചുകൊള്ളുമല്ലോ. എന്നാല്‍ നിയമവും വ്യവസ്ഥയും ഭരണഘടനയും നീതിന്യായ സംവിധാനങ്ങളുമില്ലാത്ത സമൂഹങ്ങളോ രാഷ്ട്രങ്ങളോ ഇല്ല. മനുഷ്യന് പരിമിതമായ സ്വാതന്ത്യ്രമെങ്കിലുമുണ്ടെന്നും സ്വന്തം കര്‍മങ്ങള്‍ക്ക് ഓരോരുത്തരും ഉത്തരവാദികളാണെന്നും പ്രയോഗതലത്തില്‍ എല്ലാവരും അംഗീകരിക്കുന്നുണ്ടെന്നാണിത് വ്യക്തമാക്കുന്നത്.\\
ചുരുക്കത്തില്‍, ഏതു മതവിശ്വാസിയേക്കാളും കടുത്ത വിധിവിശ്വാസികളാണ് ഭൗതിക വാദികള്‍. അതോടൊപ്പം പ്രകൃതി നിയമങ്ങളുടെ പിടിയില്‍ നിന്ന് കുതറിമാറാനാവാതെ തീര്‍ത്തും അസ്വതന്ത്രനായി കഴിയുന്ന യന്ത്രസമാനമായ ജന്തുവാണ് മനുഷ്യനെന്ന അവരുടെ വീക്ഷണം വസ്തുനിഷ്ഠമല്ല. ഇതര ജീവജാലങ്ങളില്‍നിന്ന് വ്യത്യസ്തമായ അസ്തിത്വവും വ്യക്തിത്വവും അവന് അംഗീകരിച്ചുകൊടുക്കാത്ത ഏതു ദര്‍ശനവും പ്രത്യയ ശാസ്ത്രവും അബദ്ധപൂര്‍ണമത്രെ.

\section{വിധിവിശ്വാസം ഇസ്‌ലാമിക വീക്ഷണത്തില്‍}

നാം ആരായിരിക്കണം? വെളുത്തവരോ, സുന്ദരരോ, വിരൂപരോ, പുരുഷന്മാരോ, സ്ത്രീകളോ, കുറിയവരോ, നീണ്ടവരോ? എവിടെ ജനിക്കണം? ഇന്ത്യയിലോ ഇന്തോനേഷ്യയിലോ? മരുഭൂമിയിലോ മലമ്പ്രദേശത്തോ? പട്ടിണിക്കാരന്റെ കൂരയിലോ പണക്കാരന്റെ കൊട്ടാരത്തിലോ? ഏകാധിപത്യ വ്യവസ്ഥയിലോ ജനാധിപത്യ സംവിധാനത്തിലോ? കാലമേതായിരിക്കണം? പതിനൊന്നാം നൂറ്റാണ്ടിലോ ഇരുപത്തൊന്നാം നൂറ്റാണ്ടിലോ?
വര്‍ഗം, വര്‍ണം, ദേശം, ഭാഷ, കാലം, കോലം, കുലം, ലിംഗം പോലുള്ള ഇത്തരം കാര്യങ്ങളൊന്നും തീരുമാനിക്കുന്നത് നാമാരുമല്ല. നമ്മോട് ആലോചിക്കാതെയാണ് എല്ലാം നിശ്ചയിക്കപ്പെട്ടത്. അപ്രകാരം തന്നെയാണ് നമ്മുടെ മരണവും. അതെപ്പോള്‍, എവിടെവച്ച് എങ്ങനെയാവണമെന്ന് തീരുമാനിക്കുന്നതും നാമല്ല, എല്ലാം ദൈവനിശ്ചിതം. അവന്റെ വിധിയോ, നിര്‍ണിതം; അലംഘനീയവും. ഇത്തരം കാര്യങ്ങളിലൊക്കെയും നാം ദൈവത്തിന്റെ വിധിനിഷേധങ്ങള്‍ക്ക് നിന്നുകൊടുക്കാന്‍ നിര്‍ബന്ധിതരാണ്. നമ്മുടെ ആഗ്രഹാഭിലാഷങ്ങളും ഇഛകളും വെറും നോക്കുകുത്തികള്‍ മാത്രം. അല്ലാഹു അറിയിക്കുന്നു: 'ഗര്‍ഭാശയങ്ങളില്‍ താനുദ്ദേശിക്കുന്നവിധം നിങ്ങളെ രൂപപ്പെടുത്തുന്നത് അല്ലാഹുവാണ്. അജയ്യനും യുക്തിജ്ഞനുമായ അവനല്ലാതെ ദൈവമില്ല''(ഖുര്‍ആന്‍ 3: 6). 'അവന്‍ നിങ്ങളെ സൃഷ്ടിച്ചു. പിന്നീട് രൂപപ്പെടുത്തുകയും ചെയ്തു'' (7:11). 'മനുഷ്യാ, ഉദാരനായ നിന്റെ നാഥനെക്കുറിച്ച് നിന്നെ വഞ്ചിതനാക്കിയത് എന്തൊന്നാണ്? ഉദ്ദിഷ്ടരൂപത്തില്‍ നിന്റെ ഘടകങ്ങളവന്‍ കൂട്ടിയിണക്കി. സന്തുലിതമായ ആകാരവടിവ് നല്‍കി നിന്നെ സൃഷ്ടിച്ച നിന്റെ നാഥനാണവന്‍''(82: 68).
പുരുഷന്റെ ഏതു ബീജമാണ് സ്ത്രീയുടെ അണ്ഡവുമായി സംയോജിക്കുന്നതെന്ന് തീരുമാനിക്കുന്നത് മനുഷ്യനല്ല, ദൈവമാണ്. സംയോജിക്കുന്ന ബീജത്തിന്റെ മാറ്റത്തിനനുസരിച്ച് മനുഷ്യന്റെ പ്രകൃതത്തിലും രൂപത്തിലും സ്വഭാവത്തിലുമെല്ലാം വ്യത്യാസം സംഭവിക്കുന്നുവെന്നറിയുമ്പോഴാണ് ദൈവനിശ്ചയത്തിന്റെ പ്രാധാന്യം മനസ്സിലാവുക. അല്ലാഹു ചോദിക്കുന്നു: 'നിങ്ങള്‍ നിക്ഷേപിക്കുന്ന ബീജത്തെക്കുറിച്ച് നിങ്ങള്‍ ചിന്തിച്ചുനോക്കിയിട്ടുണ്ടോ? നിങ്ങളാണോ അതിനെ സൃഷ്ടിക്കുന്നത്; അതോ നാമോ?''(56: 58, 59).
ലക്ഷക്കണക്കില്‍ പുരുഷബീജങ്ങളില്‍ അല്ലാഹു തനിക്കിഷ്ടമുള്ള ഒന്നിനെ ഗര്‍ഭാശയത്തിലെ അണ്ഡവുമായി ബന്ധിപ്പിക്കുന്നു. അതിന്റെ സ്വഭാവമനുസരിച്ച് കുഞ്ഞ് ആണോ പെണ്ണോ ആയിത്തീരുന്നു. അല്ലാഹു അറിയിക്കുന്നു: 'ആകാശങ്ങളുടെ ആധിപത്യം അല്ലാഹുവിന്നാണ്. താനുദ്ദേശിക്കുന്നത് അവന്‍ സൃഷ്ടിക്കുന്നു. താനുദ്ദേശിക്കുന്നവര്‍ക്ക് അവന്‍ പെണ്‍മക്കളെ പ്രദാനം ചെയ്യുന്നു. താനുദ്ദേശിക്കുന്ന മറ്റു ചിലര്‍ക്ക് ആണ്‍മക്കളെയും. അല്ലെങ്കില്‍ ആ സന്താനങ്ങളെ ആണും പെണ്ണുമായി ഇടകലര്‍ത്തുന്നു. അപ്രകാരം തന്നെ താനുദ്ദേശിക്കുന്നവനെ അവന്‍ സന്താനരഹിതനാക്കുന്നു. അവന്‍ അഗാധജ്ഞനും സര്‍വജ്ഞനുമത്രെ.''(42: 49,50). 'അല്ലാഹു അവനെ ബീജത്തില്‍ നിന്ന് സൃഷ്ടിക്കുകയും രൂപപ്പെടുത്തുകയും ചെയ്തു''(80:19).
മനുഷ്യന്റെ ഭാഗധേയം രൂപപ്പെടുത്തുന്നതില്‍ നമ്മുടെ ജീവിതപശ്ചാത്തലത്തിനും കുടുംബപാരമ്പര്യത്തിനും ശാരീരിക പ്രകൃതത്തിനും അനല്‍പമായ പങ്കുണ്ട്. ആ അര്‍ഥത്തില്‍ നമ്മുടെ ഭാഗധേയം നിര്‍ണിതമത്രെ; ദൈവനിശ്ചിതവും. ദൈവഹിതത്തെ മറികടക്കാനാര്‍ക്കുമാവില്ല. അല്ലാഹു പറയുന്നു: 'നിങ്ങളെയും നിങ്ങള്‍ ഉണ്ടാക്കുന്നതിനെയും സൃഷ്ടിച്ചത് അല്ലാഹുവാകുന്നു''(37: 96). 'അല്ലാഹു താനിഛിക്കുന്നവരെ ദുര്‍മാര്‍ഗത്തിലും താനിഛിക്കുന്നവരെ നേര്‍മാര്‍ഗത്തിലുമാക്കുന്നു'' (6: 39). 'അല്ലാഹു വഴിതെറ്റിച്ചവരെ നേര്‍വഴിയിലാക്കാന്‍ നിങ്ങള്‍ ഉദ്ദേശിക്കുന്നുവോ? ആരെ അല്ലാഹു വഴിതെറ്റിച്ചുവോ അവന് നീ ഒരു വഴിയും കണ്ടെത്തുകയില്ല''(4: 88). 'നിന്റെ നാഥന്‍ ഉദ്ദേശിച്ചിരുന്നുവെങ്കില്‍ ഭൂതലത്തിലുള്ളവര്‍ മുഴുക്കെ വിശ്വാസികളാകുമായിരുന്നു. എന്നിട്ടും ജനങ്ങളെ വിശ്വാസികളാക്കാന്‍ നീ നിര്‍ബന്ധിക്കുകയാണോ? എന്നാല്‍ ദൈവത്തിന്റെ അനുമതിയില്ലാതെ ആര്‍ക്കും വിശ്വാസിയാവുക സാധ്യമല്ല''(10: 99). 'അവിശ്വസിച്ചവരോ, യഥാര്‍ഥത്തില്‍ നീ അവരെ താക്കീതു ചെയ്യുന്നതും ചെയ്യാതിരിക്കുന്നതും സമമാണ്. അവരുടെ ഹൃദയങ്ങളും ശ്രവണേന്ദ്രിയങ്ങളും അല്ലാഹു മുദ്രവച്ചിരിക്കുന്നു. അവരുടെ ദൃഷ്ടികളുടെ മേല്‍ ആവരണം വീണിരിക്കുന്നു. അവര്‍ക്ക് കഠിനമായ ശിക്ഷയുണ്ട്''(2: 67). 'ഭൂമിയിലോ, നിങ്ങള്‍ക്കുതന്നെയോ ഭവിക്കുന്ന ഒരാപത്തുമില്ല, നാമത് സൃഷ്ടിക്കുന്നതിനു മുമ്പ് ഒരു ഗ്രന്ഥത്തില്‍ രേഖപ്പെടുത്തിവച്ചിട്ടല്ലാതെ. അവ്വിധം ചെയ്യുക അല്ലാഹുവിന് അതിലളിതമാകുന്നു''(57: 22).
എന്നാല്‍ എല്ലാം ദൈവനിശ്ചയമനുസരിച്ചാണ് സംഭവിക്കുകയെന്ന് പ്രഖ്യാപിക്കുന്ന വിശുദ്ധ ഖുര്‍ആന്‍ മനുഷ്യനു നല്‍കപ്പെട്ട സ്വാതന്ത്യ്രത്തെ എടുത്തു കാണിക്കുകയും ഊന്നിപ്പറയുകയും ചെയ്യുന്നു. ജന്മനാ തന്നെ മനുഷ്യനില്‍ നന്മതിന്മാവിവേചനശേഷി നിക്ഷേപിക്കപ്പെട്ടിട്ടുണ്ട്. അല്ലാഹു പറയുന്നു: 'അല്ലാഹു ആത്മാവിനെ സന്തുലിതമാക്കി അതിന് ധര്‍മാധര്‍മബോധനം നല്‍കി''(91: 7,8). 'നാമവന് കണ്ണിണകളും നാവും രണ്ടു ചുണ്ടുകളും നല്‍കിയില്ലേ? വ്യക്തമായ രണ്ടു വഴികള്‍ കാണിച്ചു കൊടുക്കുകയും ചെയ്തില്ലയോ''? (90: 810).
അതിനാല്‍ മനുഷ്യര്‍ക്കെല്ലാം നന്മയും തിന്മയും തെറ്റും ശരിയും ധര്‍മവും അധര്‍മവും തെരഞ്ഞെടുത്ത് സന്മാര്‍ഗിയും ദുര്‍മാര്‍ഗിയുമാകാനുള്ള സ്വാതന്ത്യ്രമുണ്ട്. അവ്വിധം സ്വയം തീരുമാനിച്ച് ജീവിക്കുന്നതിനനുസരിച്ചായിരിക്കും മരണാനന്തരമുള്ള രക്ഷാശിക്ഷകള്‍. വിശുദ്ധ ഖുര്‍ആന്‍ പറയുന്നു: 'ആര്‍ സന്മാര്‍ഗം സ്വീകരിക്കുന്നുവോ, അത് അവന്റെ തന്നെ ഗുണത്തിനുവേണ്ടിയാണ്. ആര്‍ ദുര്‍മാര്‍ഗമവലംബിക്കുന്നുവോ അതിന്റെ ദോഷവും അവനുതന്നെ. ഭാരം വഹിക്കുന്നവരാരും മറ്റാരുടെയും ഭാരം ചുമക്കുകയില്ല''(17:15). 'മതത്തില്‍ ഒരുവിധ ബലപ്രയോഗവുമില്ല. സന്മാര്‍ഗം മിഥ്യാധാരണകളില്‍നിന്ന് വേര്‍തിരിഞ്ഞ് വ്യക്തമായിക്കഴിഞ്ഞിരിക്കുന്നു. അതിനാല്‍ ആര്‍ ദൈവേതര ശക്തികളെ നിഷേധിച്ച് അല്ലാഹുവില്‍ വിശ്വസിക്കുന്നുവോ അവന്‍ ബലിഷ്ഠമായ അവലംബപാശത്തെ മുറുകെ പിടിച്ചിരിക്കുന്നു. അത് ഒരിക്കലും അറ്റുപോകുന്നതല്ല''(2: 256). 'പറയുക, ഇത് നിന്റെ നാഥനില്‍നിന്നുള്ള സത്യമാണ്. ഇഷ്ടമുള്ളവര്‍ക്കിത് സ്വീകരിക്കാം. ഇഷ്ടമുള്ളവര്‍ക്ക് നിഷേധിക്കാം. അക്രമികള്‍ക്ക് നാം നരകം സജ്ജമാക്കിവച്ചിട്ടുണ്ട്''(18: 29). 'ഒരു ജനതയെയും അവര്‍ സ്വയം മാറ്റുന്നതുവരെ അല്ലാഹു പരിവര്‍ത്തിപ്പിക്കുന്നില്ല.''(13: 11). 'ആര്‍ തെറ്റ് ചെയ്യുന്നുവോ അവനതിന്റെ ഫലമനുഭവിക്കും''(4: 123). 'ഓരോ വ്യക്തിയും തന്റെ പ്രവര്‍ത്തനത്തിന് കടപ്പെട്ടിരിക്കുന്നു''(52:21). 'ഇന്ന് നിങ്ങള്‍ പ്രവര്‍ത്തിച്ചതിന് പ്രതിഫലം നല്‍കപ്പെടും''(45: 28). 'നിങ്ങള്‍ പ്രവര്‍ത്തിച്ചതല്ലാതെ നിങ്ങള്‍ക്ക് പകരം നല്‍കപ്പെടുമോ?''(27: 90). 'നിന്റെ നാഥന്‍ ദാസന്‍മാരോട് ഒരിക്കലും അക്രമം പ്രവര്‍ത്തിക്കുന്നതല്ല'' (41: 46). 'ജനങ്ങളുടെ കരങ്ങള്‍ പ്രവര്‍ത്തിച്ചതിന്റെ ഫലമായി കരയിലും കടലിലും കുഴപ്പം പ്രത്യക്ഷപ്പെട്ടിരിക്കുന്നു'' (30: 41). 'നിങ്ങള്‍ക്ക് വല്ല വിപത്തും സംഭവിച്ചിട്ടുണ്ടെങ്കില്‍ അത് നിങ്ങളുടെ കരങ്ങള്‍ പ്രവര്‍ത്തിച്ചതിന്റെ ഫലമാണ്''(42: 30). 'നിശ്ചയം, അല്ലാഹു മനുഷ്യരോട് ഒട്ടും അക്രമം പ്രവര്‍ത്തിക്കുന്നില്ല. പ്രത്യുത, മനുഷ്യര്‍ സ്വയം തന്നെ അക്രമം പ്രവര്‍ത്തിക്കുകയാണ്''(10: 44). 'വല്ലവനും നേര്‍വഴി സ്വീകരിച്ചാല്‍ അവന്‍ തനിക്കുവേണ്ടി തന്നെയാണ് സ•ാര്‍ഗം സ്വീകരിക്കുന്നത്'' (10:108). 'ഒരു നാടിനെയും അതിലെ നിവാസികള്‍ അക്രമികളായെങ്കിലല്ലാതെ നാം നശിപ്പിക്കുന്നില്ല''(28: 59). 'ആര്‍ നമുക്കുവേണ്ടി കഠിനാധ്വാനം ചെയ്യുന്നുവോ അവര്‍ക്കു നാം നമ്മുടെ മാര്‍ഗങ്ങള്‍ കാണിച്ചുകൊടുക്കും'' (29: 69). 'എന്റെ ഉദ്‌ബോധനത്തില്‍നിന്ന് മുഖം തിരിക്കുന്നതാരാണോ അവന് ഈ ലോകത്ത് കുടുസ്സായ ജീവിതമാണുള്ളത്. പുനരുത്ഥാന നാളിലോ, അവനെ നാം അന്ധനായി എഴുന്നേല്‍പിക്കും. അപ്പോള്‍ അവന്‍ പറയും: നാഥാ, നീ എന്നെ അന്ധനായി എഴുന്നേല്‍പിച്ചതെന്ത്? ഭൂമിയില്‍ ഞാന്‍ കാഴ്ചയുള്ളവനായിരുന്നല്ലോ. അല്ലാഹു അരുള്‍ ചെയ്യും: ശരിയാണ്, നമ്മുടെ ദൃഷ്ടാന്തങ്ങള്‍ നിന്റെയടുക്കല്‍ വന്നപ്പോള്‍ നീ അവയെ മറന്നല്ലോ. അതേവിധം ഇന്ന് നീയും വിസ്മരിക്കപ്പെടുകയാകുന്നു. ഇപ്രകാരമത്രെ തങ്ങളുടെ നാഥന്റെ സൂക്തങ്ങളംഗീകരിക്കാതെ അതിരുകവിഞ്ഞവര്‍ക്ക് നാം ഈ ലോകത്ത് പ്രതിഫലം നല്‍കുന്നത്. പരലോക ശിക്ഷയോ അതികഠിനവും ഏറെ ദീര്‍ഘവുമത്രെ''(20: 124127). 'ഭാരം ചുമക്കുന്ന ആരും മറ്റൊരുവന്റെ ഭാരം പേറുകയില്ല. മനുഷ്യന് താന്‍ പ്രയത്‌നിച്ചതല്ലാതെ യാതൊന്നുമില്ല. താന്‍ പ്രയത്‌നിച്ചത് താമസിയാതെ അവന് കാണിക്കപ്പെടുന്നതാണ്. അനന്തരം അവന് തികവോടെ പ്രതിഫലം നല്‍കപ്പെടുകയും ചെയ്യും''(53: 3841). 'ആര്‍ അണുമണിത്തൂക്കം നന്മ ചെയ്തുവോ അവനത് കണ്ടെത്തുക തന്നെ ചെയ്യും. ആര്‍ അണുമണിത്തൂക്കം തിന്മ ചെയ്‌തോ അവനുമത് കണ്ടെത്തും''(99: 7,8).
ഇസ്‌ലാമിക വീക്ഷണത്തില്‍ മനുഷ്യന്‍ തീര്‍ത്തും അസ്വതന്ത്രനോ കേവലം വിധിയുടെ കൈയിലെ പാവയോ ഉപകരണമോ അല്ലെന്ന് ഇവിടെ ഉദ്ധരിച്ചവയും അതുപോലുള്ളവയുമായ നിരവധി വിശുദ്ധ വാക്യങ്ങള്‍ അസന്ദിഗ്ധമായി വ്യക്തമാക്കുന്നു. തന്റെ ജീവിത മാര്‍ഗം തീരുമാനിക്കാനും തെരഞ്ഞെടുക്കാനും ഓരോരുത്തര്‍ക്കും അനുവാദവും സ്വാതന്ത്യ്രവും നല്‍കപ്പെട്ടിട്ടുണ്ട്. അതിനാല്‍ മനുഷ്യജീവിതത്തിന് രണ്ടു വശമുണ്ട്. നാട്, തറവാട്, ദേശം, ഭാഷ, കാലം, കോലം, ലിംഗം, ജനനം, മരണം പോലുള്ളവയെല്ലാം അവന്റെ നിയന്ത്രണത്തിലോ പരിധിയിലോ അല്ല. ഇതാണ് ഒരുവശം. മറുവശത്ത് താന്‍ എന്ത് കുടിക്കണം, കുടിക്കരുത്, തിന്നണം, തിന്നരുത്, എന്ത് കാണണം, കാണരുത്, കേള്‍ക്കണം, കേള്‍ക്കരുത്, എന്ത് പറയണം, പറയരുത്, എന്ത് ചെയ്യണം, എന്ത് ചെയ്യരുത്, എങ്ങനെ ജീവിക്കണം, ജീവിക്കരുത് തുടങ്ങിയ കാര്യങ്ങള്‍ തീരുമാനിക്കാനും അതനുസരിച്ച് ചരിക്കാനും മനുഷ്യന് ബാധ്യതയും സ്വാതന്ത്യ്രവുമുണ്ട്. എന്നാല്‍ ഈ സ്വാതന്ത്യ്രം അപരിമിതമോ തീര്‍ത്തും അനിയന്ത്രിതമോ അല്ല. അതിനുപയോഗിക്കുന്ന കണ്ണും കാതും നാക്കും മൂക്കും കൈയും കാലും ശരീരവും ജീവനും പൂര്‍ണമായും മനുഷ്യനിയന്ത്രണത്തിലല്ലല്ലോ. അവയൊക്കെയും ദൈവാധീനത്തിലും നിയന്ത്രണത്തിലുമാണ്. അതോടൊപ്പം അവ ഇഷ്ടാനുസൃതം ഉപയോഗിക്കാനുള്ള സാധ്യതയും സ്വാതന്ത്യ്രവും മനുഷ്യന് ദൈവം തന്നെ നല്‍കിയിട്ടുണ്ട്. ഇവ്വിധം ലഭ്യമായ സ്വാതന്ത്യ്രത്തിന്റെ തോതനുസരിച്ചാണ് മനുഷ്യന്റെ മേല്‍ ബാധ്യത ചുമത്തപ്പെട്ടിട്ടുള്ളത്. അഥവാ ഓരോ മനുഷ്യനും തനിക്ക് സ്വാതന്ത്യ്രം ലഭിച്ച മേഖലയെ എവ്വിധം ഉപയോഗിക്കുന്നുവെന്നതിനനുസരിച്ചായിരിക്കും ജീവിത വിജയവും പരാജയവും രക്ഷയും ശിക്ഷയുമുണ്ടാകുക. ഈ സ്വാതന്ത്യ്രം കിട്ടിയ വശങ്ങളില്‍ സ്വയം സ്വീകരിക്കാനും നടപ്പാക്കാനുമുള്ള ജീവിത പദ്ധതിയാണ് ദൈവം പ്രവാചക•ാരിലൂടെ നല്‍കിയത്. ഉള്ള സാധ്യതയും സ്വാതന്ത്യ്രവും ഉപയോഗിച്ച് അതനുസരിച്ച് ജീവിക്കുന്നവര്‍ വിജയികളായി സ്വര്‍ഗാവകാശികളായിത്തീരും. അതിനെ ധിക്കരിച്ചും നിഷേധിച്ചും താന്തോന്നികളായി ജീവിക്കുന്നവര്‍ പരാജിതരായി നരകശിക്ഷക്ക് അര്‍ഹരാവുകയും ചെയ്യും. സ•ാര്‍ഗം സ്വീകരിച്ച് നല്ലവരായി ജീവിക്കാന്‍ തീരുമാനിച്ച് അതിനായി ശ്രമിക്കുന്ന ആരെയും അല്ലാഹു നിര്‍ബന്ധിച്ച് ദുര്‍മാര്‍ഗികളാക്കുകയില്ല. മറിച്ചും അവ്വിധം തന്നെ.
മനുഷ്യന് നല്‍കപ്പെട്ട സ്വാതന്ത്യ്രത്തിനും സാധ്യതക്കുമപ്പുറമൊരു കാര്യവും ദൈവം അവനോടാവശ്യപ്പെടുകയോ ആജ്ഞാപിക്കുകയോ ചെയ്യുന്നില്ല. അല്ലാഹു അറിയിക്കുന്നു: 'അല്ലാഹു ആരോടും അവന്റെ കഴിവിന്നതീതമായത് കല്പിക്കുകയില്ല. ഓരോരുത്തരും പ്രവര്‍ത്തിച്ചതിനനുസരിച്ചുള്ള രക്ഷയും ശിക്ഷയും അവര്‍ക്കുണ്ട്''(2: 286).
അതിനാല്‍ ദൈവം ആരോടും അനീതി കാണിക്കുന്നില്ല. 'നിന്റെ നാഥന്‍ തന്റെ ദാസന്മാരോട് ഒട്ടും അനീതി കാണിക്കുന്നവനല്ല''(41: 46).
മനുഷ്യന്റെ തീരുമാനങ്ങളും കര്‍മങ്ങളും ദൈവത്തിന്റെ അറിവിനും വിധിക്കും വിധേയമല്ലേ? മനുഷ്യസ്വാതന്ത്യ്രത്തെയും ദൈവനിശ്ചയത്തെയും ബന്ധിപ്പിക്കുന്ന കണ്ണി ഏത്? എങ്ങനെ? തുടങ്ങിയ ചോദ്യങ്ങളുന്നയിക്കപ്പെടാവുന്നതാണ്. എന്നാല്‍ മനുഷ്യനിവിടെ തന്റെ പരിധിയും പരിമിതിയും നന്നായി മനസ്സിലാക്കുക തന്നെ വേണം. അനന്ത വിസ്തൃതമായ ഈ മഹാപ്രപഞ്ചത്തിലെ ഒരു കൊച്ചു ഗോളമാണ് ഭൂമി. അതിലെ പരകോടി സൃഷ്ടികളിലൊന്നു മാത്രമാണ് മനുഷ്യന്‍. അവനിന്നോളം സ്വന്തത്തെക്കുറിച്ചുപോലും പൂര്‍ണവും കണിശവുമായി മനസ്സിലാക്കാന്‍ കഴിഞ്ഞിട്ടില്ല. ജീവന്‍, ബുദ്ധി, ഓര്‍മശക്തി, ആത്മാവ് തുടങ്ങിയവയെക്കുറിച്ചൊന്നും സൂക്ഷ്മവും ഖണ്ഡിതവുമായ അറിവ് ആര്‍ക്കുമില്ല. താന്‍ ജീവിക്കുന്ന ഭൂമിയെയും അതുള്‍ക്കൊള്ളുന്ന പ്രപഞ്ചത്തെയും സംബന്ധിച്ച അറിവോ, നന്നെ പരിമിതവും. ആ വിജ്ഞാന സാമ്രാജ്യത്തെ സമുദ്രത്തോട് തുലനം ചെയ്താല്‍ അതിലെ ഒരു തുള്ളിപോലും സ്വായത്തമാക്കാന്‍ മനുഷ്യനിന്നോളം സാധിച്ചിട്ടില്ല. നമ്മുടെ അറിവിന്റെ പരിമിതിയാണിതിനു കാരണം. അതിനാല്‍ നാം പ്രപഞ്ചത്തിന്റെ സ്രഷ്ടാവായ ദൈവത്തെയും അവന്റെ കര്‍മങ്ങളെയും സംബന്ധിച്ച് അറിയണമെന്ന് ശഠിക്കുന്നതിലൊട്ടും അര്‍ഥമില്ല. ദൈവവുമായി ബന്ധപ്പെട്ട എല്ലാം പഞ്ചേന്ദ്രിയങ്ങള്‍ക്കും അവയുടെ പരിമിതികള്‍ക്കും വിധേയമായ മനുഷ്യബുദ്ധിക്കും ചിന്തക്കും അപ്രാപ്യമായ അഭൗതിക ജ്ഞാനത്തില്‍പെട്ടവയാണ്. അല്ലാഹു തന്റെ ദൂതന്മാരിലൂടെ അറിയിച്ചു കൊടുത്തതിനപ്പുറം അതിനെക്കുറിച്ച് ആര്‍ക്കും ഒന്നുമറിയില്ല. അതിനാല്‍ ദൈവവിധിയെക്കുറിച്ച് ദിവ്യബോധനങ്ങളിലൂടെ ലഭ്യമായതല്ലാതെ മനസ്സിലാക്കാന്‍ നമുക്ക് സാധ്യമല്ല. വേദഗ്രന്ഥത്തിലൂടെ ലഭ്യമായ ജ്ഞാനമനുസരിച്ച് ദൈവവിധിയും ജ്ഞാനവും പ്രപഞ്ചത്തെയും അതിലുള്ള സകലതിനെയും ചൂഴ്ന്നുനില്‍ക്കുന്നു. മനുഷ്യനും അതിന്നതീതനല്ല. അതേസമയം സ്വയം തീരുമാനമെടുക്കാനും അതനുസരിച്ച് ജീവിക്കാനും അവന് സ്വാതന്ത്യ്രവും സാധ്യതയും നല്‍കപ്പെട്ടിട്ടുണ്ട്. അത് അപരിമിതമോ അനിയന്ത്രിതമോ അല്ല. സ്വാതന്ത്യ്രം ലഭിച്ച മേഖലകളില്‍ ദൈവത്തെ അനുസരിച്ച് അവന്‍ നിശ്ചയിച്ച ജീവിതപാത പിന്തുടരണമെന്ന് അല്ലാഹു അനുശാസിച്ചിരിക്കുന്നു. അങ്ങനെ ചെയ്യുന്നവര്‍ക്ക് ജീവിതവിജയവും പരലോകരക്ഷയും വാഗ്ദാനം ചെയ്തിട്ടുമുണ്ട്. ദൈവത്തെ ധിക്കരിക്കാനുള്ള സാധ്യതയും സ്വാതന്ത്യ്രവും മനുഷ്യന് നല്‍കപ്പെട്ടിരിക്കുന്നു. അങ്ങനെ ചെയ്യുന്നവര്‍ക്ക് പരാജയവും പരലോകത്ത് കടുത്ത ശിക്ഷയുമാണുണ്ടാവുകയെന്ന് താക്കീത് നല്‍കുകയും ചെയ്തിരിക്കുന്നു. ചുരുക്കത്തില്‍, ഓരോരുത്തര്‍ക്കും നല്‍കപ്പെട്ട കഴിവിന്റെയും സ്വാതന്ത്യ്രത്തിന്റെയും തോതനുസരിച്ചാണ് അവരവരുടെ ബാധ്യത. അതിന്റെ നിര്‍വഹണത്തിന്റെയും ലംഘനത്തിന്റെയും അടിസ്ഥാനത്തിലായിരിക്കും മരണാനന്തരം മറുലോകത്തെ രക്ഷാശിക്ഷകള്‍. അതിനാല്‍ നമ്മുടെ സ്വര്‍ഗവും നരകവും നേടുന്നത് നാം തന്നെയാണ്. സ്വന്തം തീരുമാനങ്ങളിലൂടെയും കര്‍മങ്ങളിലൂടെയും രണ്ടിലൊന്നിന്റെ അവകാശിയായിത്തീരുന്നു. അതിനാല്‍ ആരും തീരെ അനീതിക്കിരയാവുന്നില്ല. സത്യം കവിവാക്യം തന്നെ:
'നമിക്കിലുയരാം, നടുകില്‍ തിന്നാം\\
നല്‍കുകില്‍ നേടിടാം \\
നമുക്കുനാമേ പണിവതുനാകം \\
നരകവുമതുപോലെ.'

\section{വിധിവിശ്വാസവും മനഃശാന്തിയും}

കര്‍മങ്ങള്‍ക്ക് അവയിലേര്‍പ്പെടുന്നവര്‍ ലക്ഷ്യം വച്ച ഫലം തന്നെ ഉണ്ടാവണമെന്നില്ല. ലാഭമഭിലഷിച്ച് വ്യാപാരത്തിലേര്‍പ്പെടുന്നവര്‍ നഷ്ടം സഹിക്കേണ്ടിവരുന്നു. വരുമാനം പ്രതീക്ഷിച്ച് വയലേലകളില്‍ വേല ചെയ്യുന്നവര്‍ വിളനഷ്ടത്തിന്നിരയാവുന്നു. നാം നമ്മുടെ അറിവിന്റെയും പരിചയത്തിന്റെയും കഴിവിന്റെയും പരിമിതികളില്‍ നിന്നുകൊണ്ട് കാര്യങ്ങള്‍ കണക്കുകൂട്ടി അതിനനുസരിച്ച് പ്രവര്‍ത്തിക്കുന്നു. ചിലപ്പോള്‍ നമ്മുടെ സ്വപ്നങ്ങളും സങ്കല്‍പങ്ങളും നിമിഷനേരംകൊണ്ട് നിലംപൊത്തുന്നു. അപ്രതീക്ഷിതമായ ആഘാതങ്ങളെ അഭിമുഖീകരിക്കേണ്ടിവരുന്നു. എങ്കിലും അത്തരം വിപത്സാധ്യതകള്‍ ആരെയും നിഷ്‌ക്രിയരാക്കാറില്ല. ആക്കാവുന്നതുമല്ല. വിജയവും നേട്ടവും പ്രതീക്ഷിച്ച് കര്‍മങ്ങളിലേര്‍പ്പെടാറാണ് പതിവ്. മോഹങ്ങളൊക്കെയും പൂവണിയാറില്ലെന്നതുകൊണ്ടു മാത്രം ആരും ഒന്നും ആഗ്രഹിക്കാതിരിക്കാറില്ല. മനുഷ്യന്‍ കൊതിച്ചതല്ല കിട്ടുക; ദൈവം വിധിച്ചതാണ്.
'ആകാശഭൂമികളുടെ താക്കോല്‍ അല്ലാഹുവില്‍ നിക്ഷിപ്തമത്രെ! അവനിഛിക്കുന്നവര്‍ക്ക് അവന്‍ വിഭവസമൃദ്ധി നല്‍കുന്നു. ഇഛിക്കുന്നവര്‍ക്ക് ക്‌ളിപ്തപ്പെടുത്തിക്കൊടുക്കുകയും ചെയ്യുന്നു. നിശ്ചയം, എല്ലാം അറിയുന്നവനത്രെ അവന്‍''(ഖുര്‍ആന്‍ 42: 12).
'നിങ്ങള്‍ ചിന്തിക്കുന്നില്ലേ? നിങ്ങള്‍ വിതയ്ക്കുന്ന വിത്ത്, അതില്‍ നിന്ന് വിള മുളപ്പിക്കുന്നത് നിങ്ങളോ? അതോ നാമോ? നാം ഇഛിക്കുകയാണെങ്കില്‍ ഈ വിളകളെ ഉണങ്ങിയ താളുകളാക്കുക തന്നെ ചെയ്യും! അപ്പോഴോ നിങ്ങള്‍ പലതും പറഞ്ഞുകൊണ്ടിരിക്കും. നമ്മുടെ മേല്‍ കടം കുടുങ്ങിയല്ലോ, നാം നിര്‍ഭാഗ്യവാ•ാരായിപ്പോയല്ലോ എന്നിങ്ങനെ''(56: 6367).
നമ്മുടെ മരണം എവിടെവച്ച്; എങ്ങനെ എന്ന് ആര്‍ക്കുമറിയില്ല. എല്ലാം അലംഘനീയമായ ദൈവവിധിക്കു വിധേയം. അതിനെ തട്ടിമാറ്റാനാര്‍ക്കും സാധ്യമല്ല. ആയുസ്സിന്റെ ഹ്രസ്വദൈര്‍ഘ്യതയൊക്കെയും അല്ലാഹുവിന്റെ തീരുമാനമനുസരിച്ചാണ്.
'ആരും അറിയുന്നില്ല; നാളെ താന്‍ എന്തൊക്കെ നേടുമെന്ന്; ഏതു മണ്ണിലാണ് തന്റെ മരണമെന്നറിയുന്നവരുമില്ല''(31: 34).
'ജീവനുള്ളവയ്‌ക്കൊന്നും ദൈവഹിതമന്യേ മരിക്കുക സാധ്യമല്ല. മരണസമയമാകട്ടെ ലിഖിതവും''(31: 145). 'യഥാര്‍ഥത്തില്‍ ജീവിപ്പിക്കുകയും മരിപ്പിക്കുകയും ചെയ്യുന്നത് അല്ലാഹു മാത്രമാകുന്നു''(3: 156). 'മരണമാകട്ടെ, നിങ്ങള്‍ എവിടെയായിരുന്നാലും അത് നിങ്ങളെ പിടികൂടുകതന്നെ ചെയ്യും. നിങ്ങള്‍ എത്ര ഭദ്രമായ കോട്ടകളിലായിരുന്നാലും''(4: 78).
മനുഷ്യജീവിതത്തിലെ നന്മതിന്മകള്‍, സന്മാര്‍ഗദുര്‍മാര്‍ഗങ്ങള്‍, ജയാപചയങ്ങള്‍ തുടങ്ങിയവപോലെത്തന്നെ ഭൗതിക ജീവിതത്തിലെ കാര്യങ്ങളും ദൈവവിധിക്കു വിധേയമാണ്. അതോടൊപ്പം ദൈവവിധിയുണ്ടെങ്കില്‍ ആഹാരം ലഭിച്ചുകൊള്ളുമെന്നു കരുതി ആരും അധ്വാനിക്കാതെ അടങ്ങിയൊതുങ്ങിക്കഴിയാറില്ല. ദൈവഹിതമനുസരിച്ചാണ് മരണമുണ്ടാവുകയെന്നതിനാല്‍ രോഗമാകുമ്പോള്‍ ദൈവേഛയുണ്ടെങ്കില്‍ സുഖമായിക്കൊള്ളുമെന്ന് സമാധാനിച്ച് ചികിത്സ നടത്താതിരിക്കാറുമില്ല. കാര്യബോധമുള്ളവരാരും ജീവിതത്തിന്റെ ഭാഗധേയം വിധിക്ക് വിട്ടുകൊടുത്ത് ആലസ്യത്തില്‍ ആമഗ്‌നരാവുകയില്ല. ശരിയായ പാത പരതിയെടുത്ത് അതിലൂടെ ജീവിതപുരോഗതിക്കും ലക്ഷ്യപ്രാപ്തിക്കും പരമാവധി പ്രയത്‌നിക്കാറാണ് പതിവ്. മതമനുശാസിക്കുന്നതും അതുതന്നെ. ദൈവവിധി പുലരുക മനുഷ്യകര്‍മങ്ങളിലൂടെയാണല്ലോ.
'എല്ലാവര്‍ക്കും തങ്ങളുടെ പ്രവര്‍ത്തനങ്ങള്‍ക്കനുസൃതമായ പദവികള്‍ ലഭിക്കും. അവരുടെ പ്രവര്‍ത്തനങ്ങള്‍ അല്ലാഹു അവര്‍ക്ക് പൂര്‍ത്തിയാക്കിക്കൊടുക്കും. അവരോട് അനീതി കാണിക്കുകയില്ല''(46: 19). 'സല്‍ക്കര്‍മം ചെയ്യുക. നാം നിങ്ങള്‍ ചെയ്യുന്നതൊക്കെയും കാണുന്നവനല്ലോ''(34: 11).
പ്രവാചകന്‍ പറയുന്നു: 'നിങ്ങള്‍ പ്രഭാതപ്രാര്‍ഥന പൂര്‍ത്തിയാക്കിയാല്‍ ആഹാരത്തിനായി അധ്വാനിക്കാതെ ഉറങ്ങരുത്''(ത്വബ്‌റാനി). 'ലോകാവസാനം ആസന്നമായ ഘട്ടത്തില്‍പോലും നിങ്ങളിലാരുടെയെങ്കിലുമടുക്കല്‍ ഒരു ഈന്തപ്പനത്തൈ ഉണ്ടെങ്കില്‍ അവനത് നട്ടുകൊള്ളട്ടെ. അവന്നതില്‍ പ്രതിഫലമുണ്ട്.''
ഒരിക്കല്‍ നബിതിരുമേനി രണ്ടുപേര്‍ തമ്മിലുണ്ടായ കേസില്‍ വിധി നടത്തി. പിരിഞ്ഞുപോയപ്പോള്‍ കേസില്‍ പരാജയപ്പെട്ടയാള്‍ പറഞ്ഞു: 'എനിക്ക് അല്ലാഹു മതി. അവന്‍ ഏറ്റവും നല്ല സംരക്ഷകനത്രെ''. ഇതു കേള്‍ക്കാനിടയായ പ്രവാചകന്‍ അയാളോടു പറഞ്ഞു: 'ദൌര്‍ബല്യത്തെ ദൈവം വെറുക്കുന്നു. അതിനാല്‍ ശക്തിയും തന്റേടവും നേടുക. എന്നിട്ടും പരാജിതരായാല്‍ അപ്പോള്‍ നിങ്ങള്‍ക്കു പറയാം: 'എനിക്ക് അല്ലാഹു മതി. അവന്‍ ഏറ്റവും നല്ല സംരക്ഷകനത്രെ.''
മനുഷ്യന്‍ തന്റെ സാധ്യതകളൊക്കെയും സ്വരൂപിച്ച് പ്രതിബന്ധങ്ങളോടും പ്രതികൂല പരിതഃസ്ഥിതികളോടും പൊരുതാന്‍ ബാധ്യസ്ഥനാണ്. വിജയം വരിക്കാന്‍ അത് അനിവാര്യമാണെന്ന് മതമോതുന്നു. ദൃഢനിശ്ചയം, സ്ഥിരചിത്തത, ത്യാഗമനസ്ഥിതി, സ്ഥിരോത്സാഹം തുടങ്ങിയവയൊക്കെ സ്വായത്തമാക്കണമെന്ന് അതാവശ്യപ്പെടുന്നു. പണിയെടുക്കാതെ അലസമായിരുന്ന് പതനങ്ങളും പാളിച്ചകളും പിഴവുകളുമൊക്കെ വിധിയുടെ മേല്‍ വച്ചുകെട്ടുന്നത് കടുത്ത കുറ്റമത്രെ. അശ്രദ്ധയുടെയും അനാസ്ഥയുടെയും അനിവാര്യഫലങ്ങള്‍ക്ക് ദൈവവിധിയുടെ പരിവേഷമണിയിക്കുന്നത് അത്യന്തം അപലപനീയമാണ്. അധ്വാനിക്കാതെ ഫലം പ്രതീക്ഷിക്കുന്നതും കഷ്ടതകള്‍ സഹിക്കാതെ നേട്ടം കൊതിക്കുന്നതും മടയത്തമാണ്. കര്‍മത്തിനേ അല്ലാഹു പ്രതിഫലം നല്‍കുകയുള്ളൂ. നിഷ്‌ക്രിയത്വം നാശനിമിത്തമത്രെ.
ഒരാള്‍ പ്രവാചകനോട് ചോദിച്ചു: 'ദൈവദൂതരേ, ഞാന്‍ ഒട്ടകത്തെ ബന്ധിച്ച ശേഷം അല്ലാഹുവില്‍ ഭരമേല്‍പിക്കണമോ? അതോ അതിനെ അഴിച്ചുവിട്ടു കൊണ്ടോ?'' പ്രവാചകന്‍ പ്രതിവചിച്ചു: 'അതിനെ കെട്ടിയിടുക. എന്നിട്ട് ദൈവത്തില്‍ ഭരമേല്‍പിക്കുക.''
ആഹാരം കഴിക്കാതെ വിശപ്പുമാറില്ല. വെള്ളം കുടിക്കാതെ ദാഹം ശമിക്കുകയില്ല. വിത്തിറക്കാതെ വിളവുണ്ടാവില്ല. പണിയെടുക്കാതെ പണവും. വന്‍ വിജയങ്ങള്‍ക്ക് മര്‍മമറിഞ്ഞ കര്‍മം അനിവാര്യമത്രെ. ഇത് മാറ്റമില്ലാത്ത ദൈവിക നിയമമാണ്. ഇതറിയുന്ന യഥാര്‍ഥ വിശ്വാസി സദാ കര്‍മനിരതനായിരിക്കും. അതിനാല്‍ വിധിവിശ്വാസം ആലസ്യമല്ല; ഔത്സുക്യമാണ് വളര്‍ത്തുക.
ദൈവ വിധിയും ദിവ്യജ്ഞാനവും പ്രത്യക്ഷമോ പ്രകടമോ അല്ല. അതിനാല്‍ തനിക്ക് അജ്ഞാതമായ ഒന്നിനെക്കുറിച്ച് ആലോചിച്ച് അസ്വസ്ഥനാവുന്നതിലര്‍ഥമില്ല. അതിരുകളില്ലാത്ത ദൈവകാരുണ്യത്തില്‍ പ്രതീക്ഷയര്‍പ്പിച്ച് പണിയെടുക്കുകയാണ് വേണ്ടത്. നിരാശ വിശ്വാസികള്‍ക്ക് അന്യമത്രെ. 'നിങ്ങള്‍ ദൈവകാരുണ്യത്തെക്കുറിച്ച് നിരാശരാവരുത്''(39: 53).
അധ്വാനിക്കാനുള്ള ദൃഢനിശ്ചയത്തിനുശേഷം മാത്രമേ അല്ലാഹുവില്‍ ഭരമേല്‍പിക്കാവൂ എന്നാണ് മതമനുശാസിക്കുന്നത്. 'നീ ഒരു കാര്യത്തില്‍ ദൃഢനിശ്ചയമെടുത്തു കഴിഞ്ഞാല്‍ ദൈവത്തില്‍ ഭരമേല്‍പിക്കുക''(3:159).
ചുറുചുറുക്കോടെ ഓടിച്ചാടി നടക്കുന്ന ചെറുപ്പക്കാരന്‍ ആകസ്മികമായി വാതരോഗത്തിന്നടിപ്പെടുന്നു. പലവിധ ചികിത്സകളും നടത്തിനോക്കുന്നു. പക്ഷേ, ഒന്നും ഫലിക്കുന്നില്ല. അവസാനം അലോപ്പതിയും ആയുര്‍വേദവും, ഹോമിയോപ്പതിയുമുള്‍പ്പെടെ എല്ലാ വൈദ്യവിദ്യകളും രോഗം ഭേദമാവുകയില്ലെന്ന് വിധിയെഴുതുന്നു. അതിനാല്‍ താനിനി ഒരിക്കലും കട്ടിലില്‍ നിന്നിറങ്ങി നടക്കുകയില്ലെന്ന് അയാളറിയുന്നു. എങ്കില്‍ അയാള്‍ അത്യധികം അസ്വസ്ഥനാവുക സ്വാഭാവികമത്രെ. ഇത്തരമൊരു സാഹചര്യത്തില്‍ ആ രോഗിയെ ആശ്വസിപ്പിക്കാന്‍ ലോകത്തിലൊരു ഭൗതിക ദര്‍ശനത്തിനും പ്രത്യയശാസ്ത്രത്തിനും സാധ്യമല്ല. ഇരുപതാം നൂറ്റാണ്ടു കണ്ട വീര വിപ്‌ളവകാരി വി.ഐ. ലെനിന്‍ പോലും വാതരോഗത്തിനടിപ്പെട്ടപ്പോള്‍ അത്യധികം അസ്വസ്ഥനായി തന്റെ ആത്മമിത്രമായ സ്‌റാലിനോട് സയനൈഡ് ആവശ്യപ്പെട്ടത് അതിനാലാണല്ലോ. എന്നാല്‍ ദൈവ വിധിയില്‍ ദൃഢമായി വിശ്വസിക്കുന്നവര്‍ സകലതും സ്രഷ്ടാവില്‍ സമര്‍പ്പിച്ച് ആശ്വാസം കൊള്ളുന്നു. അവര്‍ മനഃസമാധാനത്തോടെ ആത്മഗതം ചെയ്യുന്നു: 'തനിക്ക് കൈയും കാലും കണ്ണും കാതും നാക്കും മൂക്കും ആയുസ്സും ആരോഗ്യവും ജീവനും ജീവിതവും നല്‍കിയത് ദൈവമാണ്. അവന്‍ കനിഞ്ഞേകിയ ആരോഗ്യം തല്‍ക്കാലം അവന്‍ തിരിച്ചെടുത്തിരിക്കുന്നു. ഇതൊരു പരീക്ഷണമാണ്. ദൈവ വിധി. പതര്‍ച്ച പറ്റിയാല്‍ പരാജയമായിരിക്കും ഫലം. ക്ഷമിച്ചാല്‍ വിജയവും. വാതവും രോഗവുമൊന്നുമില്ലാത്ത സ്വര്‍ഗം പ്രതിഫലമായി ലഭിക്കും.''
ഏകമകന്റെ മാതാപിതാക്കള്‍. കാണികളില്‍ ഏറെ കൌതുകമുണര്‍ത്തുന്ന നാലഞ്ചു വയസ്സ് പ്രായത്തില്‍ ആ കുഞ്ഞ് മരണമടയുന്നു. ഇത്തരം അനുഭവങ്ങള്‍ക്കിരയാകുന്ന വ്യക്തികളനുഭവിക്കുന്ന വ്യഥയും വേദനയും വിവരണാതീതമത്രെ. നിലവിലുള്ള കുടുംബസങ്കല്‍പം മുതലാളിത്ത വ്യവസ്ഥയുടെ സൃഷ്ടിയാണെന്നും സ്വകാര്യ ഉടമാവകാശമില്ലാതിരുന്ന സ്ത്രീ സ്വാതന്ത്യ്രത്തിന്റെ പ്രാകൃത കമ്യൂണിസ്‌റ് കാലഘട്ടത്തില്‍ കുട്ടികള്‍ അഛന്‍മാരെ തിരിച്ചറിഞ്ഞിരുന്നില്ലെന്നും വാദിച്ച കാറല്‍ മാര്‍ക്‌സ് പോലും തന്റെ മകന്റെ വിയോഗത്തില്‍ ഏറെ വിഹ്വലനാവുകയുണ്ടായി. 1855ല്‍ അദ്ദേഹത്തിന്റെ ഇഷ്ടപുത്രന്‍ എഡ്ഗാറിന് ഗുരുതരമായ രോഗം ബാധിച്ചു. എട്ടുവയസ്സുള്ള അതിസമര്‍ഥനായ ആ കുട്ടി മുഷ് എന്ന ഓമനപ്പേരിലാണ് അറിയപ്പെട്ടിരുന്നത്. ഏകമകന്റെ രോഗശയ്യക്കരികില്‍ നീണ്ട രാത്രികള്‍ ഉറക്കമൊഴിച്ച് കഴിച്ചുകൂട്ടിയ മാര്‍ക്‌സ് ആ ഘട്ടത്തില്‍ ഏംഗല്‍സിനെഴുതി: 'ഹൃദയം നീറുകയാണ്. തല പുകയുകയാണ്.'' പിന്നീട് മുഷ് മരണമടഞ്ഞപ്പോള്‍ അദ്ദേഹമെഴുതി: 'പാവം മുഷ് മരിച്ചു....എന്റെ ദുഃഖം എത്ര വലുതാണെന്നറിയാമല്ലോ. ഒട്ടേറെ കഷ്ടപ്പാടുകള്‍ അനുഭവിച്ചിട്ടുള്ളവനാണ് ഞാന്‍. പക്ഷേ, യഥാര്‍ഥ ദുഃഖമെന്താണെന്ന് ഇപ്പോഴാണെനിക്കു മനസ്സിലായത്.''
ഇത്തരം സാഹചര്യങ്ങളിലും വിധിയില്‍ വിശ്വസിക്കുന്നവര്‍ ആശ്വാസമനുഭവിക്കുന്നു. അവരുടെ മനസ്സ് മന്ത്രിക്കുന്നു: 'തനിക്ക് കുഞ്ഞിനെ കനിഞ്ഞേകിയത് കരുണാമയനായ ദൈവമാണ്. തന്റെ ഓമനമകനെ തിരിച്ചുവിളിച്ചതും അവന്‍ തന്നെ. എല്ലാം അവന്റെ അലംഘനീയമായ വിധി. തന്നേക്കാള്‍ സ്‌നേഹവും കരുണയും വാത്സല്യവുമുള്ള സ്രഷ്ടാവിന്റെ വശം കുഞ്ഞിനെ തിരിച്ചേല്‍പിക്കുകയാണ് താന്‍ ചെയ്തത്. അതിനാല്‍ അക്ഷമ കാണിക്കുന്നത് അസ്ഥാനത്താണ്. തീര്‍ത്തും അര്‍ഥശൂന്യവും. സഹനമവലംബിച്ചാല്‍ ലഭിക്കാനുള്ളത് സ്വര്‍ഗമാണ്. അനശ്വരമായ അതിന്റെ കവാടത്തില്‍ തന്റെ ഓമന മകന്‍ മന്ദസ്മിതനായി തന്നെ സ്വാഗതം ചെയ്യാന്‍ കാത്തുനിന്നേക്കാം.''
വിധി വിശ്വാസമില്ലാത്തവര്‍ വിപത്തുകള്‍ വരുമ്പോള്‍ അസഹ്യമായ അസ്വസ്ഥതക്കും അക്ഷമക്കും അടിപ്പെടുന്നു. വേവലാതിപ്പെടുകയും പരാതി പറഞ്ഞുകൊണ്ടിരിക്കുകയും ചെയ്യുന്നു. അവരുടെ ആശങ്കയും അരക്ഷിതബോധവുമകറ്റാന്‍ ആശ്വാസവചനങ്ങള്‍ പോലും അശക്തമായിരിക്കും. എന്നാല്‍ അല്ലാഹുവിന്റെ അലംഘനീയമായ വിധിയില്‍ വിശ്വാസമര്‍പ്പിക്കുന്നവര്‍ അനല്പമായ ആശ്വാസമനുഭവിക്കുന്നു. ക്ഷമിച്ചാലും അക്ഷമ കാണിച്ചാലും ഭൗതിക ഫലം ഒന്നുതന്നെ എന്ന് അവരറിയുന്നു. എന്നല്ല, അക്ഷമ അസ്വാസ്ഥ്യം അധികരിപ്പിക്കുകയാണ് ചെയ്യുക. ക്ഷമയും സഹനവും ശാന്തിയേകും. ദൈവസന്നിധിയിലോ ക്ഷമിക്കുന്നവര്‍ക്ക് മഹത്തായ പ്രതിഫലം. അക്ഷമ അപരാധവും. ഐഹിക ജീവിതം മുഴുക്കെ പരീക്ഷണമാണ്. അനുഗ്രഹങ്ങള്‍ നല്‍കിയും നിഷേധിച്ചും ദൈവമത് നിര്‍വഹിക്കുന്നു. ഫലമറിയുക മരണാനന്തരം മറുലോകത്താണ്. അനുഗ്രഹങ്ങള്‍ ലഭ്യമാകുന്ന അനുകൂലാവസ്ഥയില്‍ ആഹ്‌ളാദത്തില്‍ മിതത്വം പുലര്‍ത്തണമെന്ന് മതമാവശ്യപ്പെടുന്നു; അവ നിഷേധിക്കപ്പെടുമ്പോള്‍ ക്ഷമയും സഹനവും പാലിക്കണമെന്നും. വിജയം വാഗ്ദാനം ചെയ്യപ്പെട്ടത് ആവിധം പ്രവര്‍ത്തിക്കുന്നവര്‍ക്കാണ്.
'ഭയാശങ്കകള്‍, ക്ഷാമം, ജീവധനാദികളുടെ നഷ്ടം, വിളനാശം എന്നിവയിലൂടെ നാം നിങ്ങളെ നിശ്ചയമായും പരീക്ഷിക്കും. അപ്പോള്‍ ക്ഷമ അവലംബിക്കുകയും 'ഞങ്ങള്‍ ദൈവത്തിന്റേതാണല്ലോ, അവനിലേക്കാണല്ലോ ഞങ്ങള്‍ മടങ്ങേണ്ടതും എന്നു പറയുകയും ചെയ്യുന്നവരെ സന്തോഷ വാര്‍ത്ത അറിയിക്കുക. അവര്‍ക്ക് തങ്ങളുടെ നാഥനില്‍നിന്ന് വലിയ അനുഗ്രഹങ്ങള്‍ ലഭിക്കും. അവന്റെ കാരുണ്യം അവര്‍ക്ക് തണലേകും. അവര്‍ തന്നെയാകുന്നു സന്മാര്‍ഗം പ്രാപിച്ചവര്‍''(2: 155157).
ഇതംഗീകരിക്കുന്ന വിശ്വാസികള്‍ സ്വസ്ഥരും നിര്‍ഭയരുമായിരിക്കും. തങ്ങളുടെ വിധിയില്‍ തീര്‍ത്തും തൃപ്തരും. അത് തങ്ങള്‍ക്ക് അനുകൂലമോ പ്രതികൂലമോ എന്നത് അവരെ സംബന്ധിച്ചേടത്തോളം അപ്രസക്തമത്രെ. വ്യഥയും വേവലാതിയും വിതുമ്പലും വിഹ്വലതയും വിധിയിലൊരു വ്യത്യാസവും വരുത്തുകയില്ലല്ലോ. അവര്‍ നഷ്ടസന്ദര്‍ഭങ്ങളുടെയും സൌഭാഗ്യങ്ങളുടെയും പേരില്‍ വിലപിച്ച് കാലം കഴിക്കാതെ, ദിവ്യകാരുണ്യത്തില്‍ പ്രതീക്ഷയര്‍പ്പിച്ച് സ്വസ്ഥചിത്തരായി പുതിയ പ്രവര്‍ത്തനങ്ങളില്‍ വ്യാപൃതരാവുന്നു. അവര്‍ ഇന്നലെകളുടെ നഷ്ടങ്ങളോര്‍ത്ത് നെടുവീര്‍പ്പിടുകയല്ല; അവയെ വര്‍ത്തമാന ക്രിയകളിലും ഭാവികര്‍മങ്ങളിലും വളമാക്കി മാറ്റുകയാണ് ചെയ്യുക. അങ്ങനെ യഥാര്‍ഥ വിധിവിശ്വാസം കര്‍മപ്രേരകമായും മനഃശാന്തിയുടെ നിര്‍ഝരിയായും വര്‍ത്തിക്കുന്നു. വികലമായ വിധിവിശ്വാസമാണ് ആലസ്യത്തിലേക്കും കര്‍മശൂന്യതയിലേക്കും വഴിയൊരുക്കുക. ഇസ്‌ലാം ഒരിക്കലും അതംഗീകരിക്കുന്നില്ല.


\chapter{ക്ഷേത്രപ്രദക്ഷിണവും കഅ്ബാ ത്വവാഫും} 
 \section{ ഹിന്ദുക്കള്‍ ക്ഷേത്രത്തിനു ചുറ്റും പ്രദക്ഷിണം ചെയ്യുന്നപോലെത്തന്നെയല്ലേ മുസ്‌ലിംകള്‍ കഅ്ബക്കു ചുറ്റും കറങ്ങുന്നത്?}

 ഇബ്‌റാഹീം പ്രവാചകനാണ് വിശുദ്ധ കഅ്ബ പുനര്‍നിര്‍മിച്ചത്. ഹജ്ജിന് വിളംബരം ചെയ്തതും അദ്ദേഹം തന്നെ. അദ്ദേഹം പണിത കഅ്ബയില്‍ പ്രതിമകളോ പ്രതിഷ്ഠകളോ വിഗ്രഹങ്ങളോ ചിത്രങ്ങളോ ഒന്നുമുണ്ടായിരുന്നില്ല. എന്നല്ല, ഏകദൈവാരാധനയ്ക്കായാണ് ഇബ്‌റാഹീം നബിയും പുത്രന്‍ ഇസ്മാഈല്‍ പ്രവാചകനും കൂടി അത് പണിതത്. ഹജ്ജിന്റെ ഭാഗമായി അതിനു ചുറ്റും കറങ്ങുന്ന സമ്പ്രദായം അന്നുമുതല്‍ തന്നെ നിലനിന്നുപോന്നിരുന്നു. എന്നാല്‍ പില്‍ക്കാലത്ത് ജനം അന്ധവിശ്വാസങ്ങള്‍ക്കും അനാചാരങ്ങള്‍ക്കും അടിപ്പെടുകയും അതേ പ്രവാചകന്മാരുടെ പോലും ചിത്രങ്ങള്‍ അതില്‍ കോറിയിട്ട് അവയെ ആരാധിക്കുകയും ചെയ്തു. കഅ്ബയിലും അതിനു ചുറ്റും നിരവധി വിഗ്രഹങ്ങള്‍ പ്രതിഷ്ഠിച്ച് അവയെ ആരാധിക്കാന്‍ തുടങ്ങി. അങ്ങനെ ഏകദൈവാരാധനയുടെ പ്രകാശനമായി നടന്നുവന്നിരുന്ന കഅ്ബക്കു ചുറ്റുമുള്ള കറക്കം വിഗ്രഹാരാധനയായി പരിണമിച്ചു. ഈ ഘട്ടത്തിലാണ് പ്രവാചകനായ മുഹമ്മദ്‌നബി നിയോഗിതനായി വമ്പിച്ച വിശ്വാസവിപ്‌ളവത്തിലൂടെ അവരുടെ മനംമാറ്റി അവരെക്കൊണ്ടുതന്നെ കഅ്ബയിലെ വിഗ്രഹങ്ങള്‍ എടുത്തുമാറ്റി അതിനെ ശുദ്ധീകരിച്ചത്. അങ്ങനെ കഅ്ബക്കു ചുറ്റും കറങ്ങുന്ന ആരാധനാസമ്പ്രദായം ഇസ്‌ലാം തുടര്‍ന്നും നിലനിര്‍ത്തി. എന്നാല്‍ അത് വിഗ്രഹാരാധനയ്ക്കു പകരം ഏകദൈവാരാധനയ്ക്കാക്കി മാറ്റി. നഗ്‌നരായി നിര്‍വഹിച്ചിരുന്ന പ്രസ്തുത കര്‍മം മാന്യമായി വസ്ത്രം ധരിച്ചു മാത്രമേ നിര്‍വഹിക്കാവൂ എന്ന് നിഷ്‌കര്‍ഷിച്ചു. അപ്രകാരം അതിലെ എല്ലാ തിന്മകളും മ്‌ളേഛതകളും അവസാനിപ്പിച്ച് അതിനെ ശുദ്ധീകരിച്ചു.
മതങ്ങളുടെയെല്ലാം സ്രോതസ്സ് മൌലികമായി ഒന്നും അത് ദൈവത്തില്‍നിന്നുമായതിനാല്‍ ആരാധനകളിലും അനുഷ്ഠാനങ്ങളിലുമെല്ലാം സമാനത കണ്ടെത്തുക സ്വാഭാവികമാണ്. ഇന്ത്യയിലേക്ക് കുടിയേറിപ്പാര്‍ത്ത ആര്യന്മാര്‍ ഇബ്‌റാഹീം പ്രവാചകന്റെ അനുയായികളും പിന്‍മുറക്കാരുമായിരുന്നുവെന്ന ചില ചരിത്രഗവേഷകരുടെ നിഗമനം ഇവിടെ പ്രത്യേക പരിഗണന അര്‍ഹിക്കുന്നു. അതിനാല്‍ ഇബ്‌റാഹീം നബിയിലൂടെ നടപ്പാക്കപ്പെട്ട ആരാധനാകര്‍മത്തിന്റെ രൂപപരിണാമത്തിനിടയായ അനുഷ്ഠാനമായിരിക്കാം ഇവിടെ നിലനിന്നുവരുന്ന ക്ഷേത്രപ്രദക്ഷിണം. ഏതായാലും ദൈവത്തെ പ്രതിനിധീകരിക്കാന്‍ പ്രതിമകളും പ്രതിഷ്ഠകളും വിഗ്രഹങ്ങളും സ്ഥാപിക്കുന്നതിനെയും അവയെ ആരാധിക്കുന്നതിനെയും അതിന്റെ ഭാഗമായി അവ സ്ഥാപിച്ച ഭവനങ്ങളെ ചുറ്റുന്നതിനെയും ഇസ്‌ലാം അനുകൂലിക്കുന്നില്ലെന്നു മാത്രമല്ല; ശക്തമായി വിലക്കുകയും ചെയ്യുന്നു.
ആരാധനാനുഷ്ഠാനങ്ങളിലെ സമാനത മതങ്ങളുടെ സ്രോതസ്സിനെ സംബന്ധിച്ച സത്യസന്ധവും വസ്തുനിഷ്ഠവുമായ അന്വേഷണത്തിനും അതുവഴി മതപരമായ ഏകത കണ്ടെത്താനും കൈവരിക്കാനും കഴിഞ്ഞാല്‍ അത് അതിമഹത്തായ നേട്ടമായിരിക്കും. അത്തരമൊരന്വേഷണത്തിനും പഠനത്തിനും ഈ ചര്‍ച്ച പ്രചോദകമായെങ്കില്‍!
\chapter{ദൈവം കഅ്ബയിലോ? }
 \section{ മുസ്‌ലിംകള്‍ എന്തിനാണ് നമസ്‌കാരത്തില്‍ കഅ്ബയിലേക്ക് തിരിഞ്ഞുനില്‍ക്കുന്നത്? കഅ്ബയിലാണോ ദൈവം? അല്ലെങ്കില്‍ കഅ്ബ ദൈവത്തിന്റെ പ്രതീകമോ പ്രതിഷ്ഠയോ?}

 ഇസ്‌ലാമിന്റെ വീക്ഷണത്തില്‍ ദൈവം ഏതെങ്കിലും പ്രത്യേകസ്ഥലത്ത് പരിമിതനോ കുടിയിരുത്തപ്പെട്ടവനോ അല്ല. ദൈവത്തിന് പ്രതിമകളോ പ്രതിഷ്ഠകളോ ഇല്ല.
'കിഴക്കും പടിഞ്ഞാറും അല്ലാഹുവിന്റേതാകുന്നു. നിങ്ങള്‍ എവിടെ തിരിഞ്ഞാലും അവിടെയെല്ലാം അവന്റെ വദനമുണ്ട്. അല്ലാഹു അതിവിശാലനും സര്‍വജ്ഞനുമത്രെ.''(ഖുര്‍ആന്‍ 2: 115)
'ആകാശഭൂമികളുടെ സകല സംഗതികളും അല്ലാഹു അറിയുന്നുവെന്ന് നിങ്ങളറിയുന്നില്ലേ? ഒരിക്കലും മൂന്നുപേര്‍ തമ്മില്‍ രഹസ്യസംഭാഷണം നടക്കുന്നില്ല, അവരില്‍ നാലാമനായി അല്ലാഹു ഇല്ലാതെ. അല്ലെങ്കില്‍ അഞ്ചുപേരുടെ രഹസ്യസംഭാഷണംആറാമനായി അല്ലാഹു ഇല്ലാതെ നടക്കുന്നില്ല. രഹസ്യംപറയുന്നവര്‍ ഇതിലും കുറച്ചാവട്ടെ കൂടുതലാവട്ടെ, അവരെവിടെയായിരുന്നാലും അല്ലാഹു അവരോടൊപ്പമുണ്ടായിരിക്കും.'' (58:7)
'മനുഷ്യനെ സൃഷ്ടിച്ചത് നാമാകുന്നു. അവന്റെ മനസ്സിലുണരുന്ന തോന്നലുകള്‍ വരെ നാമറിയുന്നു. അവന്റെ കണ്ഠനാഡിയേക്കാള്‍ അവനോടടുത്തവനത്രെ നാം.''(50: 16)
ലോകമെങ്ങുമുള്ള മുഴുവന്‍ മനുഷ്യരെയും അഖില ജീവിത മേഖലകളിലും ഏകീകരിക്കുന്ന സമഗ്ര ജീവിതപദ്ധതിയാണ് ഇസ്‌ലാം. അതിന്റെ ആരാധനാക്രമം വിശ്വാസികളെ ഏകീകരിക്കുന്നതില്‍ അനല്‍പമായ പങ്കുവഹിക്കുന്നു. ഇതു സാധ്യമാവണമെങ്കില്‍ എല്ലാവരുടെയും ആരാധനാരീതി ഒരേവിധമാവേണ്ടതുണ്ടല്ലോ. അതിനാല്‍ നമസ്‌കാരത്തില്‍ വിശ്വമെങ്ങുമുള്ള വിശ്വാസികള്‍ക്ക് തിരിഞ്ഞുനില്‍ക്കാന്‍ ഒരിടം അനിവാര്യമത്രെ. അത് ദൈവത്തെ മാത്രം ആരാധിക്കാനായി ആദ്യമായി നിര്‍മിക്കപ്പെട്ട കഅ്ബയായി നിശ്ചയിക്കപ്പെടുകയാണുണ്ടായത്. അതുകൊണ്ട് ആ വിശുദ്ധ ദേവാലയം ലോകജനതയെ ഏകീകരിക്കുന്ന കേന്ദ്രബിന്ദുവാണ്. ദൈവം പ്രത്യേകമായി കുടിയിരുത്തപ്പെട്ട ഇടമോ ദൈവത്തിന്റെ പ്രതീകമോ പ്രതിഷ്ഠയോ ഒന്നുമല്ല. മറിച്ച്, അത് ഏകദൈവാരാധനയുടെ പ്രതീകമാണ്.
'നിസ്സംശയം, മനുഷ്യര്‍ക്കായി നിര്‍മിക്കപ്പെട്ട പ്രഥമ ദേവാലയം മക്കയില്‍ സ്ഥിതിചെയ്യുന്നതുതന്നെയാകുന്നു. അത് അനുഗൃഹീതവും ലോകര്‍ക്കാകമാനം മാര്‍ഗദര്‍ശക കേന്ദ്രവുമായിട്ടത്രെ നിര്‍മിക്കപ്പെട്ടിട്ടുള്ളത്.'' (ഖുര്‍ആന്‍ 3:96)
'ഈ മന്ദിരത്തെ നാം ജനങ്ങള്‍ക്ക് ഒരു കേന്ദ്രവും അഭയസ്ഥാനവുമായി നിശ്ചയിച്ചതും സ്മരിക്കുക''(2:125). 'വിശുദ്ധഗേഹമായ കഅ്ബാലയത്തെ അല്ലാഹു ജനങ്ങള്‍ക്ക് (സാമൂഹിക ജീവിതത്തിന്റെ) നിലനില്‍പിനുള്ള ആധാരമാക്കി നിശ്ചയിച്ചിരിക്കുന്നു''(5: 97).
അതിനാല്‍ വിശുദ്ധ ദേവാലയത്തെയല്ല ആരാധിക്കേണ്ടത്, അതിന്റെ നാഥനായ ദൈവത്തെ മാത്രമാണ്. 'അതിനാല്‍ നിങ്ങള്‍ ഈ മന്ദിരത്തിന്റെ നാഥനെ വണങ്ങുവിന്‍''(106:3).
ചില പ്രമുഖ ചരിത്രകാര•ാര്‍ പോലും തെറ്റിദ്ധരിക്കുകയും തെറ്റായി രേഖപ്പെടുത്തുകയും ചെയ്തതുപോലെ കഅ്ബ ഒരു കല്ലല്ല. കല്ലുകൊണ്ട് നിര്‍മിക്കപ്പെട്ട പന്ത്രണ്ടു മീറ്റര്‍ നീളവും പത്തുമീറ്റര്‍ വീതിയും പതിനഞ്ചുമീറ്റര്‍ ഉയരവുമുള്ള ഒരു മന്ദിരമാണ്. കഅ്ബ എന്ന പദം തന്നെ ഘനചതുരത്തെ(ക്യൂബ്)യാണ് പ്രതിനിധീകരിക്കുന്നത്. നിര്‍മാണചാതുരിയോ ശില്‍പഭംഗിയോ കലകളോ കൊത്തുപണികളോ ഒട്ടുമില്ലാത്ത ലാളിത്യത്തിന്റെ പ്രതീകമാണത്.
എന്നാല്‍ ലോകത്തിലെ ഏറ്റവും ശ്രദ്ധേയമായ ഭവനമാണത്. നൂറു കോടിയോളം മനുഷ്യര്‍ നിത്യവും നന്നെ ചുരുങ്ങിയത് അഞ്ചു നേരമെങ്കിലും അതിന്റെ നേരെ തിരിഞ്ഞുനില്‍ക്കുന്നു. നിരവധി നൂറ്റാണ്ടുകളിലൂടെ കടന്നുപോയ കോടാനുകോടി വിശ്വാസികളുടെ മുഖം അന്ത്യവിശ്രമത്തിനായി തിരിച്ചുവയ്ക്കപ്പെട്ടതും കഅ്ബയുടെ നേരെയാണ്. ജനവികാരങ്ങളുമായി ഈവിധം കെട്ടുപിണഞ്ഞു കിടക്കുന്ന ഒരു മന്ദിരവും ലോകത്ത് വേറെയില്ല. ദൈവത്തിന്റെ ഭവനമാണത്. അതുകൊണ്ടുതന്നെ മുഴുവന്‍ മനുഷ്യരുടേതുമാണ്. ഏകദൈവാരാധനയുടെ പ്രതീകവും എല്ലാ ഏകദൈവാരാധകരുടെയും പ്രാര്‍ഥനയുടെ ദിശയുമാണത്.

\section{'എന്തിനാണ് മുസ്‌ലിംകള്‍ കഅ്ബക്കു ചുറ്റും കറങ്ങുന്നത്? എന്താണ് അതിന്റെ പ്രയോജനം? ഒരര്‍ഥവുമില്ലാത്ത ആചാരമല്ലേ അത്?'}
ദൈവമാണ് തന്നെ എങ്ങനെ ആരാധിക്കണമെന്ന് തീരുമാനിക്കേണ്ടത്. പൂര്‍വപ്രവാചകനായ ഇബ്‌റാഹീം നബിയുടെ കാലം തൊട്ടേയുള്ള ദൈവനിശ്ചിതമായ ആരാധനാകര്‍മമാണത്. നമുക്കു തോന്നിയതുപോലെയല്ലല്ലോ നാം ദൈവത്തെ വണങ്ങേണ്ടത്. അങ്ങനെ ആയിരുന്നുവെങ്കില്‍ നമുക്ക് യുക്തമെന്ന് തോന്നുന്ന രീതികള്‍ ആവിഷ്‌കരിക്കാമായിരുന്നു. എന്നാല്‍ ആരാധനാകര്‍മങ്ങള്‍ ഏതൊക്കെയെന്നും എപ്പോള്‍, ഏതുവിധമെന്നും ദൈവം കണിശമായി നിര്‍ണയിച്ചിട്ടുണ്ട്. തന്റെ ദൂതന്മാരിലൂടെ സമൂഹത്തെ പഠിപ്പിച്ചിട്ടുമുണ്ട്. അവയില്‍ എന്തെങ്കിലും കൂട്ടിച്ചേര്‍ക്കാനോ വെട്ടിക്കുറയ്ക്കാനോ മാറ്റംവരുത്താനോ ആര്‍ക്കും അനുവാദമോ അവകാശമോ ഇല്ല. കഅ്ബക്കു ചുറ്റുമുള്ള കറക്കവും ഈവിധം നിര്‍ണിത രൂപമുള്ള, ദൈവനിര്‍ദിഷ്ടമായ ആരാധനാ കര്‍മമാണ്. എന്നാല്‍ ഇതിന്റെ പിന്നില്‍ മഹത്തായ യുക്തിയും അര്‍ഥവുമുണ്ടെന്ന് അല്‍പം ആലോചിച്ചാല്‍ മനസ്സിലാക്കാവുന്നതാണ്.
മനുഷ്യന്‍ പ്രപഞ്ചഘടനയോട് താദാത്മ്യം പ്രാപിക്കുന്നു; അതിലെ അത്ഭുതകരമായ സംവിധാനത്തോട് ഉള്‍ച്ചേരുന്ന അതിവിശിഷ്ടമായ ആരാധനാകര്‍മമാണത്. വിശുദ്ധ കഅ്ബയാണതിന്റെ കേന്ദ്രബിന്ദു. ജനം അതിനു ചുറ്റും കറങ്ങുന്നു. സൌരയൂഥത്തിലെ ഗ്രഹങ്ങള്‍ സൂര്യനു ചുറ്റുമെന്നപോലെ. പരമാണുവിലെ ഇലക്ട്രോണുകള്‍ ന്യൂക്‌ളിയസിനു ചുറ്റും കറങ്ങുന്നതുപോലെ. ഏഴുതവണ ചുറ്റിയാലേ ഒരു പ്രാവശ്യമായി പരിഗണിക്കുകയുള്ളൂ. അഥവാ ഏഴു പ്രാവശ്യം ചുറ്റുന്നതാണ് ഒരു ത്വവാഫ്. ഇവിടെ ഏഴ് എന്നത് പ്രപഞ്ചഘടനയെ പ്രതിനിധീകരിക്കുന്നു. ഭൂഖണ്ഡങ്ങള്‍ ഏഴാണല്ലോ. സമുദ്രവും ഏഴുതന്നെ. ആകാശവും ഏഴാണെന്ന് ഖുര്‍ആന്‍ പറയുന്നു.
വലതുവശത്തിന് ഏറെ പ്രാമുഖ്യം കല്‍പിച്ച ഇസ്‌ലാം ത്വവാഫില്‍ കഅ്ബ, തീര്‍ഥാടകന്റെ ഇടതുവശത്താവണമെന്ന് നിഷ്‌കര്‍ഷിച്ചിരിക്കുന്നു. ഇത് വളരെയേറെ ശ്രദ്ധേയമത്രെ. ഇതുവഴി കഅ്ബക്കു ചുറ്റുമുള്ള കറക്കം പ്രകൃതിവ്യവസ്ഥയുമായി പൂര്‍ണമായും പൊരുത്തപ്പെടുന്നു. സൌരയൂഥത്തിലെ ഗോളങ്ങള്‍ സൂര്യനെ ചുറ്റുന്നത്, അത് ഇടതുവശം വരുംവിധമാണ്. അഥവാ ത്വവാഫിലേതുപോലെ, ഘടികാരത്തിന്റെ സൂചി പിറകോട്ട് തിരിയുംവിധമാണ്. ഗ്രഹങ്ങള്‍ സ്വന്തം അക്ഷത്തില്‍ കറങ്ങുന്നതും ആവിധം തന്നെ. ധൂമകേതുക്കള്‍ സൂര്യനുചുറ്റും ചലിക്കുന്നതും അതേ ദിശയിലാണ്. അണ്ഡബീജസങ്കലനം നടക്കുംമുമ്പ് പുരുഷബീജങ്ങള്‍ അണ്ഡത്തിനു ചുറ്റും കറങ്ങുന്നതും ആന്റിക്‌ളോക്ക് വൈസിലാണ്. അങ്ങനെ വിശിഷ്ടമായ ഈ ആരാധനാകര്‍മം പ്രപഞ്ചവ്യവസ്ഥയോട് വിസ്മയകരമാംവിധം യോജിച്ചുവന്നിരിക്കുന്നു. പരമാണു മുതല്‍ ഗാലക്‌സി വരെയുള്ള പ്രവിശാലമായ ഈ പ്രപഞ്ചത്തിന്റെ ഭാഗമാണ് താനെന്നും അവയൊക്കെ സ്രഷ്ടാവായ ദൈവത്തിന് വഴങ്ങി, വണങ്ങുന്നപോലെ താനും അവനെ മാത്രം ആരാധിച്ചും അനുസരിച്ചും ജീവിക്കേണ്ടവനാണെന്നുമുള്ള ബോധമുണര്‍ത്തുന്ന ഈ അനുഷ്ഠാനം താനെങ്ങനെ ചെയ്യുമെന്നതിന്റെ പ്രതീകാത്മകമായ പ്രഖ്യാപനം കൂടിയാണ്.
കഅ്ബക്കു ചുറ്റുമുള്ള കറക്കത്തിന്റെ അതിമഹത്തരവും അത്യന്തം വിസ്മയകരവുമായ ഈ അര്‍ഥതലങ്ങള്‍ അറിയാന്‍ സാധിച്ചത് അടുത്തകാലത്ത് മാത്രമാണ്. ഇനിയും പിടികിട്ടാത്ത പല മാനങ്ങളും അതിനുണ്ടായേക്കാം. ദൈവനിര്‍ദിഷ്ടമായ ആരാധന അവന്റെ സൃഷ്ടികളായ മനുഷ്യര്‍ അതിന്റെ യുക്തിയും ന്യായവും മനസ്സിലായാലും ഇല്ലെങ്കിലും നിശ്ചിത രൂപത്തില്‍ നിര്‍വഹിക്കാന്‍ ബാധ്യസ്ഥരാണ്.
\chapter{ആറാം നൂറ്റാണ്ടിലെ മതം! }
  \section{പതിനാലു നൂറ്റാണ്ടിനകം ലോകം വിവരിക്കാനാവാത്ത വിധം മാറി. പുരോഗതിയുടെ പാരമ്യതയിലെത്തിയ ആധുനിക പരിഷ്‌കൃതയുഗത്തില്‍ ആറാം നൂറ്റാണ്ടിലെ മതവുമായി നടക്കുന്നത് വിഡ്ഢിത്തമല്ലേ?}
   
 കാലം മാറിയിട്ടുണ്ട്, ശരിയാണ്. ലോകത്തിന്റെ കോലവും മാറിയിരിക്കുന്നു. മനുഷ്യനിന്ന് വളരെയേറെ പുരോഗതി പ്രാപിച്ചിട്ടുണ്ട്. ശാസ്ത്രം വിവരണാതീതമാം വിധം വളര്‍ന്നു. സാങ്കേതികവിദ്യ സമൃദ്ധമായി. വിജ്ഞാനം വമ്പിച്ച വികാസം നേടി. ജീവിതസൗകര്യങ്ങള്‍ സീമാതീതമായി വര്‍ധിച്ചു. നാഗരികത നിര്‍ണായക നേട്ടങ്ങള്‍ കൈവരിച്ചു. ജീവിതനിലവാരം വളരെയേറെ മെച്ചപ്പെട്ടു. എന്നാല്‍ മനുഷ്യനില്‍ ഇവയെല്ലാം എന്തെങ്കിലും മൌലികമായ മാറ്റം വരുത്തിയിട്ടുണ്ടോ? വിചാരവികാരങ്ങളെയും ആചാരക്രമങ്ങളെയും ആരാധനാരീതികളെയും പെരുമാറ്റ സമ്പ്രദായങ്ങളെയും സ്വഭാവ സമീപനങ്ങളെയും അല്‍പമെങ്കിലും സ്വാധീനിച്ചിട്ടുണ്ടോ? ഇല്ലെന്നതല്ലേ സത്യം? സഹസ്രാബ്ദങ്ങള്‍ക്കപ്പുറം അന്ധവിശ്വാസങ്ങള്‍ സമൂഹത്തെ അടക്കിഭരിച്ചിരുന്നു. സമകാലീന സമൂഹത്തിലെ സ്ഥിതിയും അതുതന്നെ. അന്നത്തെപ്പോലെ ഇന്നും മനുഷ്യന്‍ അചേതന വസ്തുക്കളെ ആരാധിക്കുന്നു. ചതിയും വഞ്ചനയും കളവും കൊള്ളയും പഴയതുപോലെ തുടരുന്നു. മദ്യം മോന്തിക്കുടിക്കുന്നതില്‍ പോലും മാറ്റമില്ല. ലൈംഗിക അരാജകത്വത്തിന്റെ അവസ്ഥയും അതുതന്നെ. എന്തിനേറെ, ആറാം നൂറ്റാണ്ടിലെ ചില അറേബ്യന്‍ ഗോത്രങ്ങള്‍ ചെയ്തിരുന്നതുപോലെ പെണ്‍കുഞ്ഞുങ്ങളെ ആധുനിക മനുഷ്യനും ക്രൂരമായി കൊലപ്പെടുത്തുന്നു. അന്ന് ഒന്നും രണ്ടുമൊക്കെയായിരുന്നുവെങ്കില്‍ ഇന്ന് ലക്ഷങ്ങളും കോടികളുമാണെന്നു മാത്രം. വൈദ്യവിദ്യയിലെ വൈദഗ്ധ്യം അത് അനായാസകരമാക്കുകയും ചെയ്തിരിക്കുന്നു. നാം കിരാതമെന്ന് കുറ്റപ്പെടുത്തുന്ന കാലത്ത് സംഭവിച്ചിരുന്നപോലെ തന്നെ ഇന്നും മനുഷ്യന്‍ തന്റെ സഹജീവിയെ ക്രൂരമായി കൊലപ്പെടുത്തുന്നു. അന്ന് കൂര്‍ത്തുമൂര്‍ത്ത കല്ലായിരുന്നു കൊലയ്ക്ക് ഉപയോഗിച്ചിരുന്നതെങ്കില്‍ ഇന്ന് വന്‍ വിസ്‌ഫോടന ശേഷിയുള്ള ബോംബാണെന്നു മാത്രം. അതിനാല്‍ കൊല ഗണ്യമായി വര്‍ധിച്ചിരിക്കുന്നു. ചുരുക്കത്തില്‍, ലോകത്തുണ്ടായ മാറ്റമൊക്കെയും തീര്‍ത്തും ബാഹ്യമത്രെ. അകം അന്നും ഇന്നും ഒന്നുതന്നെ. മനുഷ്യന്റെ മനമൊട്ടും മാറിയിട്ടില്ല. അതിനാല്‍ മൌലികമായൊരു മാറ്റവും സംഭവിച്ചിട്ടില്ല. പിന്നിട്ട നൂറ്റാണ്ടുകളിലെ പുരുഷാന്തരങ്ങളില്‍ മനംമാറ്റവും അതുവഴി ജീവിത പരിവര്‍ത്തനവും സൃഷ്ടിച്ച ആദര്‍ശവിശ്വാസങ്ങള്‍ക്കും മൂല്യബോധത്തിനും മാത്രമേ ഇന്നും അതുണ്ടാക്കാന്‍ സാധിക്കുകയുള്ളൂ. ആറാം നൂറ്റാണ്ടിലെ മാനവമനസ്സിന് സമാധാനവും ജീവിതത്തിന് വിശുദ്ധിയും കുടുംബത്തിന് സൈ്വര്യവും സമൂഹത്തിന് സുരക്ഷയും രാഷ്ട്രത്തിന് ഭദ്രതയും നല്‍കിയ ദൈവികജീവിതവ്യവസ്ഥക്ക് ഇന്നും അതിനൊക്കെയുള്ള കരുത്തും പ്രാപ്തിയുമുണ്ട്. പ്രയോഗവല്‍ക്കരണത്തിനനുസൃതമായി അതിന്റെ സമകാലിക പ്രസക്തിയും പ്രാധാന്യവും സദ്ഫലങ്ങളും പ്രകടമാകും; പ്രകടമായിട്ടുണ്ട്; പ്രകടമാകുന്നുമുണ്ട്. വിശുദ്ധ ഖുര്‍ആനും പ്രവാചകചര്യയും സമര്‍പ്പിക്കുന്ന സമഗ്ര ജീവിതവ്യവസ്ഥയില്‍ ഒരംശം പോലും ആധുനിക ലോകത്തിന് അപ്രായോഗികമോ അനുചിതമോ ആയി ഇല്ലെന്നതാണ് വസ്തുത. എന്നല്ല; അതിന്റെ പ്രയോഗവല്‍ക്കരണത്തിലൂടെ മാത്രമേ മാനവരാശി ഇന്നനുഭവിക്കുന്ന പ്രശ്‌നങ്ങളും പ്രയാസങ്ങളും പൂര്‍ണമായും പരിഹരിക്കപ്പെടുകയുള്ളൂ.
\chapter{ എന്തുകൊണ്ടിപ്പോള്‍ ദൈവദൂതന്‍മാര്‍ വരുന്നില്ല? }
\section{  നബിതിരുമേനി ദൈവത്തിന്റെ അന്ത്യദൂതനായതെന്തുകൊണ്ട്? ദൈവത്തിന്റെ മാര്‍ഗദര്‍ശനം ഇന്നും ആവശ്യമല്ലേ? എങ്കില്‍ എന്തുകൊണ്ടിപ്പോള്‍ ദൈവദൂതന്മാരുണ്ടാകുന്നില്ല?}

 മനുഷ്യരുടെ മാര്‍ഗദര്‍ശനമാണല്ലോ ദൈവദൂതന്‍മാരുടെ നിയോഗലക്ഷ്യം. മുഹമ്മദ് നബിക്കു മുമ്പുള്ള പ്രവാചകന്മാരിലൂടെ അവതീര്‍ണമായ ദൈവികസന്ദേശം ചില പ്രത്യേക കാലക്കാര്‍ക്കും ദേശക്കാര്‍ക്കും മാത്രമുള്ളവയായിരുന്നു. ലോകവ്യാപകമായി ആ സന്ദേശങ്ങളുടെ പ്രചാരണവും അവയുടെ ഭദ്രമായ സംരക്ഷണവും സാധ്യമായിരുന്നില്ലെന്നതാവാം ഇതിനു കാരണം. ഏതായാലും അവരിലൂടെ ലഭ്യമായ ദൈവികസന്ദേശം മനുഷ്യ ഇടപെടലുകളില്‍നിന്ന് തീര്‍ത്തും മുക്തമായ നിലയില്‍ ലോകത്തിന്ന് എവിടെയുമില്ല. എന്നാല്‍ പതിനാലു നൂറ്റാണ്ടിനപ്പുറം മുഹമ്മദ് നബിതിരുമേനിയിലൂടെ അവതീര്‍ണമായ ഖുര്‍ആന്‍ മുഴുവന്‍ മനുഷ്യര്‍ക്കും മാര്‍ഗദര്‍ശനമേകാന്‍ പര്യാപ്തമത്രെ. അന്നത് ലോകമെങ്ങും എത്താന്‍ സൗകര്യപ്പെടുമാറ് മനുഷ്യനാഗരികത വളര്‍ന്നു വികസിച്ചിരുന്നു. പ്രവാചക നിയോഗാനന്തരം ഏറെക്കാലം കഴിയുംമുമ്പേ അന്നത്തെ അറിയപ്പെട്ടിരുന്ന നാടുകളിലെങ്ങും ഖുര്‍ആന്റെ സന്ദേശം ചെന്നെത്തുകയും പ്രചരിക്കുകയും ചെയ്തു. ഖുര്‍ആന്‍ മനുഷ്യരുടെ എല്ലാവിധ ഇടപെടലുകളില്‍നിന്നും കൂട്ടിച്ചേര്‍ക്കലുകളില്‍നിന്നും വെട്ടിച്ചുരുക്കലുകളില്‍നിന്നും മുക്തമായി തനതായ നിലയില്‍ ഇന്നും നിലനില്‍ക്കുന്നു. ലോകാവസാനം വരെ അത് അവ്വിധം സുരക്ഷിതമായി ശേഷിക്കും. ദൈവികമായ ഈ ഗ്രന്ഥത്തില്‍ ഒരക്ഷരം പോലും മാറ്റത്തിന് വിധേയമായിട്ടില്ല. അതിന്റെ സൂക്ഷ്മമായ സംരക്ഷണം ദൈവംതന്നെ ഏറ്റെടുക്കുകയും ചെയ്തിരിക്കുന്നു. അവന്‍ വാഗ്ദാനം ചെയ്യുന്നു: 'ഈ ഖുര്‍ആന്‍ നാം തന്നെ അവതരിപ്പിച്ചതാണ്. നാം തന്നെയാണതിന്റെ സംരക്ഷകനും''(15: 9).
മനുഷ്യരാശിക്കായി നല്‍കപ്പെട്ട ദൈവികമാര്‍ഗദര്‍ശനം വിശുദ്ധഖുര്‍ആനിലും അതിന്റെ വ്യാഖ്യാനവിശദീകരണമായ പ്രവാചകചര്യയിലും ഭദ്രമായും സൂക്ഷ്മമായും നിലനില്‍ക്കുന്നതിനാല്‍ പുതിയ മാര്‍ഗദര്‍ശനത്തിന്റെയോ പ്രവാചകന്റെയോ ആവശ്യമില്ല. അതുകൊണ്ടുതന്നെ കഴിഞ്ഞ പതിനാലുനൂറ്റാണ്ടില്‍ പുതിയ വേദഗ്രന്ഥമോ ദൈവദൂതനോ ഉണ്ടായിട്ടില്ല. ഇനി ഉണ്ടാവുകയുമില്ല. ഇക്കാര്യം ഖുര്‍ആന്‍തന്നെ ഊന്നിപ്പറഞ്ഞിട്ടുണ്ട്. 'ജനങ്ങളേ, മുഹമ്മദ് നബി നിങ്ങളിലുള്ള പുരുഷന്മാരിലാരുടെയും പിതാവല്ല. പ്രത്യുത, അദ്ദേഹം അല്ലാഹുവിന്റെ ദൂതനും പ്രവാചകന്മാരില്‍ അവസാനത്തെയാളുമാകുന്നു. അല്ലാഹു സര്‍വസംഗതികളും അറിയുന്നവനല്ലോ''(32: 40).

\chapter{പ്രവാചകന്‍ മതസ്ഥാപകനല്ല }
\section{}  യേശു ദൈവപുത്രനാണെന്ന് ക്രിസ്ത്യാനികളും മുഹമ്മദ് നബി ദൈവത്തിന്റെ അന്ത്യദൂതനാണെന്ന് മുസ്‌ലീങ്ങളും വാദിക്കുന്നു. ഇതെല്ലാം സ്വന്തം മതസ്ഥാപകരെ മഹത്വവല്‍ക്കരിക്കാനുള്ള കേവലം അവകാശവാദങ്ങളല്ലേ?

   
 ഇസ്‌ലാമിനെയും മുസ്‌ലിംകളെയും സംബന്ധിച്ച ഗുരുതരമായ തെറ്റുധാരണകളാണ് ഈ ചോദ്യത്തിനു കാരണം. മുഹമ്മദ് നബി നമ്മുടെയൊക്കെ പ്രവാചകനാണ്. ഏതെങ്കിലും ജാതിക്കാരുടെയോ സമുദായക്കാരുടെയോ മാത്രം നബിയല്ല. മുഴുവന്‍ ലോകത്തിനും സകല ജനത്തിനും വേണ്ടി നിയോഗിക്കപ്പെട്ട ദൈവദൂതനാണ്. അദ്ദേഹത്തെ സംബന്ധിച്ച് വിശുദ്ധഖുര്‍ആന്‍ പറയുന്നത്, 'ലോകര്‍ക്കാകെ അനുഗ്രഹമായിട്ടല്ലാതെ നിന്നെ നാം നിയോഗിച്ചിട്ടില്ല.' (21: 107) എന്നാണ്.
അപ്രകാരംതന്നെ എക്കാലത്തെയും ഏതു ദേശത്തെയും എല്ലാ നബിമാരെയും തങ്ങളുടെ സ്വന്തം പ്രവാചകന്മാരായി സ്വീകരിക്കാന്‍ മുസ്‌ലിംകള്‍ ബാധ്യസ്ഥരാണ്. അവര്‍ക്കിടയില്‍ ഒരുവിധ വിവേചനവും കല്‍പിക്കാവതല്ല. മുസ്‌ലിംകള്‍ ഇപ്രകാരം പ്രഖ്യാപിക്കാന്‍ ശാസിക്കപ്പെട്ടിരിക്കുന്നു: 'പറയുക: ഞങ്ങള്‍ അല്ലാഹുവില്‍ വിശ്വസിക്കുന്നു. ഇബ്‌റാഹീം, ഇസ്മാഈല്‍, ഇസ്ഹാഖ്, യഅ്ഖൂബ്, അദ്ദേഹത്തിന്റെ സന്തതികള്‍ എന്നിവര്‍ക്കവതരിപ്പിക്കപ്പെട്ടിരുന്നതിലും മോശെ, യേശു എന്നിവര്‍ക്കും ഇതര പ്രവാചകന്മാര്‍ക്കും അവരുടെ നാഥങ്കല്‍നിന്നവതരിപ്പിച്ചിട്ടുള്ള മാര്‍ഗദര്‍ശനങ്ങളിലും ഞങ്ങള്‍ വിശ്വസിക്കുന്നു. അവരിലാരോടും ഞങ്ങള്‍ വിവേചനം കല്‍പിക്കുന്നില്ല. ഞങ്ങള്‍ അല്ലാഹുവിന്റെ ആജ്ഞാനുവര്‍ത്തികളല്ലോ.''(ഖുര്‍ആന്‍ 3:84)
ഖുര്‍ആന്‍ പറയുന്നു: 'ദൈവദൂതന്‍ തന്റെ നാഥനില്‍നിന്ന് തനിക്ക് അവതരിച്ച മാര്‍ഗദര്‍ശനത്തില്‍ വിശ്വസിച്ചിരിക്കുന്നു. ഈ ദൂതനില്‍ വിശ്വസിക്കുന്നവരാരോ അവരും ആ മാര്‍ഗദര്‍ശനത്തെ വിശ്വസിച്ചംഗീകരിച്ചിരിക്കുന്നു. അവരെല്ലാവരും അല്ലാഹുവിലും അവന്റെ മലക്കുകളിലും വേദഗ്രന്ഥങ്ങളിലും ദൂതന്മാരിലും വിശ്വസിക്കുന്നു. അവരുടെ പ്രഖ്യാപനമിവ്വിധമത്രെ: ഞങ്ങള്‍ ദൈവദൂതന്മാരില്‍ ആരോടും വിവേചനം കാണിക്കുന്നില്ല. ഞങ്ങള്‍ വിധി ശ്രവിച്ചു. വിധേയത്വമംഗീകരിച്ചു. നാഥാ, ഞങ്ങള്‍ നിന്നോട് മാപ്പിരക്കുന്നവരാകുന്നു. ഞങ്ങള്‍ നിന്നിലേക്കുതന്നെ മടങ്ങുന്നവരാണല്ലോ''(2: 285).
പ്രവാചകന്മാരല്ല മതസ്ഥാപകരെന്നും അവര്‍ ദൈവികസന്ദേശം മനുഷ്യരാശിക്കെത്തിക്കുന്ന ദൈവദൂതന്മാരും പ്രബോധകരും മാത്രമാണെന്നും ഈ വിശുദ്ധവാക്യങ്ങള്‍ അസന്ദിഗ്ധമായി വ്യക്തമാക്കുന്നു. അതിനാല്‍ മുഹമ്മദ് നബിയല്ല ഇസ്‌ലാമിന്റെ സ്ഥാപകന്‍. ഇസ്‌ലാം അദ്ദേഹത്തിലൂടെ ആരംഭിച്ചതുമല്ല. ആദിമമനുഷ്യന്‍ മുതല്‍ മുഴുവന്‍ മനുഷ്യര്‍ക്കും ദൈവം നല്‍കിയ ജീവിത വ്യവസ്ഥയാണത്. ആ ജീവിതവ്യവസ്ഥ ജനങ്ങള്‍ക്കെത്തിച്ചുകൊടുക്കാനായി നിയോഗിതരായ സന്ദേശവാഹകരാണ് പ്രവാചകന്മാര്‍. അവര്‍ ദൈവത്തിന്റെ പുത്രന്മാരോ അവതാരങ്ങളോ അല്ല. മനുഷ്യരില്‍നിന്നു തന്നെ ദൈവത്താല്‍ നിയുക്തരായ സന്ദേശവാഹകര്‍ മാത്രമാണ്. ഭൂമിയില്‍ ജനവാസമാരംഭിച്ചതു മുതല്‍ ലോകത്തിന്റെ വിവിധ ഭാഗങ്ങളില്‍ വ്യത്യസ്ത കാലഘട്ടങ്ങളില്‍ ഇത്തരം അനേകായിരം ദൈവദൂതന്മാര്‍ നിയോഗിതരായിട്ടുണ്ട്. അവരിലെ അവസാനത്തെ കണ്ണിയാണ് മുഹമ്മദ് നബിതിരുമേനി.പ്രവാചകന്‍
\chapter{മതമേതായാലും മനുഷ്യന്‍ നന്നായാല്‍ പോരേ? }
 \section{ മതം ദൈവികമാണെങ്കില്‍ ലോകത്ത് വിവിധ മതങ്ങളുണ്ടായത് എന്തുകൊണ്ട്? വ്യത്യസ്ത ദേശക്കാര്‍ക്കും കാലക്കാര്‍ക്കും വെവ്വേറെ മതമാണോ ദൈവം നല്‍കിയത്? അങ്ങനെയാണെങ്കില്‍ തന്നെ വിവിധ മതങ്ങള്‍ക്കിടയില്‍ പരസ്പര ഭിന്നതയും വൈരുധ്യവും ഉണ്ടാവാന്‍ കാരണമെന്ത്?}
 

മാനവസമൂഹത്തിന് ദൈവം നല്‍കിയ ജീവിതവ്യവസ്ഥയാണ് മതം. മനുഷ്യന്‍ ആരാണെന്നും എവിടെനിന്നു വന്നുവെന്നും എങ്ങോട്ടു പോകുന്നുവെന്നും ജീവിതം എന്താണെന്നും ഏതു വിധമാവണമെന്നും മരണശേഷം എന്ത് എന്നും ഒക്കെയാണ് അത് മനുഷ്യന് പറഞ്ഞുകൊടുക്കുന്നത്. അങ്ങനെ ജനജീവിതത്തെ നേര്‍വഴിയിലൂടെ നയിച്ച് വിജയത്തിലെത്തിക്കുകയാണ് മതം ചെയ്യുന്നത്; ചെയ്യേണ്ടത്. ദൈവം തന്റെ സമഗ്രമായ ഈ ജീവിതവ്യവസ്ഥ കാലദേശഭേദമില്ലാതെ എല്ലാ ജനസമൂഹങ്ങള്‍ക്കും സമ്മാനിച്ചിട്ടുണ്ട്. തന്റെ സന്ദേശവാഹകരിലൂടെയാണത് അവതരിപ്പിച്ചത്. എല്ലാ ദൈവദൂതന്മാരും മനുഷ്യരാശിക്കു നല്‍കിയ ജീവിതാദര്‍ശം മൌലികമായി ഒന്നുതന്നെയാണ്. അല്ലാഹു അറിയിക്കുന്നു: 'എല്ലാ സമുദായങ്ങളിലേക്കും നാം ദൂതനന്മാരെ നിയോഗിച്ചിട്ടുണ്ട്. അവരിലൂടെ എല്ലാവരെയും അറിയിച്ചു: നിങ്ങള്‍ ദൈവത്തെ മാത്രം വണങ്ങി, വഴങ്ങി, വിധേയമായി ജീവിക്കുക. പരിധിലംഘിച്ച അതിക്രമകാരികളെ നിരാകരിക്കുക.''(ഖുര്‍ആന്‍ 16: 36)
അങ്ങനെ ഒരു ജനതയില്‍ ദൈവദൂതന്‍ നിയോഗിതനാകുന്നു. തന്റെ ജനതയെ ദൈവികസന്മാര്‍ഗത്തിലേക്ക് ക്ഷണിക്കുന്നു. അപ്പോള്‍ അവരിലൊരു വിഭാഗം അദ്ദേഹത്തെ അംഗീകരിക്കുകയും പിന്തുടരുകയും ചെയ്യുന്നു. അവശേഷിക്കുന്നവര്‍ പൂര്‍വികാചാരങ്ങളില്‍ ഉറച്ചുനില്‍ക്കുന്നു. ദൈവദൂതന്റെ വിയോഗാനന്തരം ഹ്രസ്വമോ ദീര്‍ഘമോ ആയ കാലം കഴിയുന്നതോടെ അദ്ദേഹത്തിന്റെ അനുയായികള്‍തന്നെ ദൈവിക സന്മാര്‍ഗത്തില്‍നിന്ന് വ്യതിചലിക്കുന്നു. പണ്ഡിത പുരോഹിതന്മാര്‍ അവരില്‍ അന്ധവിശ്വാസങ്ങളും അനാചാരങ്ങളും വളര്‍ത്തുന്നു. ആവിധം സമൂഹം സത്യശുദ്ധമായ ദൈവപാതയില്‍നിന്ന് വ്യതിചലിക്കുമ്പോള്‍ വീണ്ടും ദൈവദൂതന്‍ നിയോഗിതനാവുന്നു. സമൂഹത്തിലെ ഒരുവിഭാഗം അദ്ദേഹത്തെ പിന്തുടരുകയും മറ്റുള്ളവര്‍ പഴയ സമ്പ്രദായങ്ങളില്‍തന്നെ നിലകൊള്ളുകയും ചെയ്യുന്നു. അതോടെ അദ്ദേഹത്തെ അനുഗമിക്കുന്നവര്‍ ഒരു മതാനുയായികളായും അല്ലാത്തവര്‍ മറ്റൊരു മതക്കാരായും അറിയപ്പെടുന്നു. യഥാര്‍ഥത്തില്‍ എല്ലാ ദൈവദൂതന്മാരും ജനങ്ങളെ ക്ഷണിച്ചത് ഒരേ സ്രഷ്ടാവിലേക്കും അവന്റെ ജീവിതവ്യവസ്ഥയിലേക്കുമാണ്. ഇസ്രായേല്യര്‍ ദൈവികസന്മാര്‍ഗത്തില്‍നിന്ന് വ്യതിചലിച്ചപ്പോള്‍ അവരെ നേര്‍വഴിയിലേക്ക് നയിക്കാന്‍ മോശെ പ്രവാചകന്‍ നിയോഗിതനായി. പിന്നീട് അദ്ദേഹത്തിന്റെ അനുയായികള്‍ വഴിപിഴച്ചപ്പോള്‍ യേശു നിയോഗിതനായി. യേശുവിന്റെ ക്ഷണം നിരാകരിച്ച് തങ്ങളുടെ ദുരാചാരങ്ങളിലുറച്ചുനിന്നവര്‍ ജൂതന്മാരായി അറിയപ്പെട്ടു. യേശുവിനെ പിന്തുടര്‍ന്നുവന്നവര്‍ ക്രിസ്ത്യാനികളായിത്തീര്‍ന്നു. ഈവിധം സമൂഹം സത്യപാതയില്‍നിന്ന് വ്യതിചലിക്കുമ്പോള്‍ അവരെ സന്മാര്‍ഗത്തിലേക്ക് നയിക്കാന്‍ നിയോഗിതരാവുന്ന പുതിയ പ്രവാചകന്മാരെ അനുധാവനം ചെയ്യാതെ പൂര്‍വികാചാരങ്ങളെ പിന്തുടര്‍ന്നതിനാലാണ് ലോകത്ത് വിവിധ മതങ്ങളും അവയ്ക്കിടയില്‍ ഭിന്നതയും വൈരുധ്യങ്ങളും ഉണ്ടായത്.


 \section{'നദികള്‍ പലയിടങ്ങളിലും നിന്നാരംഭിച്ച് പല വഴികളിലൂടെ ഒഴുകി സമുദ്രത്തിലെത്തുന്നു. ആ വിധം മനുഷ്യന്‍ വ്യത്യസ്ത മാര്‍ഗമവലംബിച്ച് ദൈവത്തിലെത്തുന്നു. അതിനാല്‍ ശ്രീനാരായണഗുരു പറഞ്ഞപോലെ 'മതമേതായാലും മനുഷ്യന്‍ നന്നായാല്‍ മതി'യെന്നതല്ലേ ശരി?''}

ദൈവത്തിന്റെ പ്രീതിയും പ്രതിഫലവും നേടാനുള്ള പാതയാണല്ലോ മതം. തന്നിലേക്ക് വന്നുചേരാനുള്ള വഴിയേതെന്ന് നിശ്ചയിക്കേണ്ടത് സ്രഷ്ടാവായ ദൈവം തന്നെയാണ്. അതവന്‍ നിശ്ചയിക്കുകയും തന്റെ ദൂതന്മാരിലൂടെ മാനവസമൂഹത്തെ അറിയിക്കുകയും ചെയ്തിട്ടുണ്ട്.
പരസ്പരവിരുദ്ധമായ കാര്യങ്ങള്‍ ഒരേസമയം സത്യമാവുക സാധ്യമല്ലെന്നത് സുവിദിതമാണല്ലോ. ഗണിതശാസ്ത്രത്തില്‍ രണ്ടും രണ്ടും ചേര്‍ത്താല്‍ നാലാണെന്നത് ശരിയാണ്. മൂന്നാണെന്നതും അഞ്ചാണെന്നതും തെറ്റാണ്. ഓക്‌സിജനും ഹൈഡ്രജനും കൂടിച്ചേര്‍ന്നാല്‍ വെള്ളമുണ്ടാകുമെന്നത് സത്യമാണ്. മദ്യമുണ്ടാകുമെന്നത് അസത്യമാണ്. ഇവ്വിധം തന്നെ മതത്തില്‍ സത്യവും അസത്യവും ഏതെന്നത് മൌലിക പ്രധാനമാണ്. അദ്വൈതമാണ് സത്യമെന്നും ദ്വൈതം അസത്യമാണെന്നും ശ്രീശങ്കരാചാര്യര്‍ പറയുന്നു. ദ്വൈതമാണ് സത്യമെന്നും അദ്വൈതം അസത്യമാണെന്നും മാധ്വാചാര്യര്‍ സ്ഥാപിക്കുന്നു. ഇത് രണ്ടും അസത്യമാണെന്നും വിശിഷ്ടാദ്വൈതമാണ് സത്യമെന്നും രാമാനുജാചാര്യര്‍ അവകാശപ്പെടുന്നു. ദൈവാവതാര സങ്കല്പം സത്യമാണെന്ന് വിശ്വസിക്കുന്നവരും അസത്യമാണെന്ന് വിശ്വസിക്കുന്നവരും ഹൈന്ദവ പണ്ഡിതന്‍മാരിലുണ്ട്. പുനര്‍ജന്‍മസങ്കല്പത്തിന്റെ സ്ഥിതിയും ഇതുതന്നെ. ഹൈന്ദവ പണ്ഡിതന്‍മാര്‍ തന്നെ ഹിന്ദുമതത്തിലെ വിവിധ വിശ്വാസങ്ങള്‍ സത്യമാണെന്ന് വിശ്വസിക്കുന്നില്ലെന്ന് ഇത് തെളിയിക്കുന്നു. അതിനാല്‍ വിവിധ മതങ്ങളിലെ ഏകദൈവത്വം, ത്രിയേകത്വം, ബഹുദൈവത്വം, ദ്വൈതം, അദ്വൈതം, ഏകദൈവാരാധന, ബഹുദൈവാരാധന, വിഗ്രഹാരാധന, വിഗ്രഹാരാധന കൊടിയ പാപമാണെന്ന വിശ്വാസം, പരലോക വിശ്വാസം, പുനര്‍ജന്മവിശ്വാസം, മനുഷ്യരൊക്കെ പരിശുദ്ധരായാണ് ജനിക്കുന്നതെന്ന വീക്ഷണം, പാപികളായാണ് പിറക്കുന്നതെന്ന പ്രസ്താവം പോലുള്ള കാര്യങ്ങള്‍ ഒരേസമയം സത്യവും ശരിയും സ്വീകാര്യവുമാവുകയില്ല. ആവുമെന്ന് പറയുന്നത് രണ്ടും രണ്ടും നാലാണെന്നതും അഞ്ചാണെന്നതും മൂന്നാണെന്നതും സത്യവും ശരിയുമാണെന്നു പറയുന്നതുപോലെ പരമാബദ്ധമാണ്.
അതുകൊണ്ടുതന്നെ മനുഷ്യ ഇടപെടലുകളില്‍നിന്ന് മുക്തമായ, സത്യശുദ്ധമായ ദൈവിക സന്‍മാര്‍ഗം ഏതെന്ന് അന്വേഷിച്ച് കണ്ടെത്തി അത് പിന്തുടരുകയാണ് വേണ്ടത്. വിജയത്തിന്റെ വഴിയും അതുതന്നെ. താന്‍ വിശ്വസിക്കുന്ന കാര്യവും അതിനു നേരെ വിരുദ്ധമായതും ഒരേസമയം സത്യവും സ്വീകാര്യവുമാണെന്ന്, ബോധമുള്ള ആരും അംഗീകരിക്കുകയില്ല. കമ്യൂണിസവും മുതലാളിത്തവും ഒരുപോലെ ശരിയും നല്ലതുമാണെന്ന് അവയെക്കുറിച്ചറിയുന്ന അവരുടെ അനുയായികളാരും പറയുകയില്ലല്ലോ.
യഥാര്‍ഥത്തില്‍ വേണ്ടത് മനുഷ്യനെ നന്നാക്കുന്ന മതമാണ്. അഥവാ, ഐഹികജീവിതത്തില്‍ മനസ്സിന് സമാധാനവും വ്യക്തിജീവിതത്തില്‍ വിശുദ്ധിയും കുടുംബത്തിന് സൈ്വരവും സമൂഹത്തിന് സമാധാനവും രക്ഷയും രാഷ്ട്രത്തിന് ക്ഷേമവും ഭദ്രതയും ലോകത്ത് പ്രശാന്തിയും സര്‍വോപരി പരലോകത്ത് നരകത്തില്‍നിന്ന് മോചനവും സ്വര്‍ഗലബ്ധിയും ഉറപ്പുവരുത്തുന്ന മതം. മതമില്ലാതെ ഇത് സാധ്യവുമല്ല.


\section{ഏത് മതവും സ്വീകരിക്കാമെന്ന് ഖുര്‍ആനില്‍ തന്നെ ഉണ്ടല്ലോ. ഖുര്‍ആന്‍ 2: 62ല്‍ ഇങ്ങനെ കാണാം: 'ഉറപ്പായി അറിയുക: ഈ പ്രവാചകനില്‍ വിശ്വസിച്ചവരോ ജൂതന്മാരോ ക്രിസ്ത്യാനികളോ സാബികളോ ആരുമാവട്ടെ, അല്ലാഹുവിലും അന്ത്യദിനത്തിലും വിശ്വസിക്കുകയും സല്‍ക്കര്‍മങ്ങളനുഷ്ഠിക്കുകയും ചെയ്യുന്നവര്‍ക്ക് അവരുടെ രക്ഷിതാവിങ്കല്‍ പ്രതിഫലമുണ്ട്. അവര്‍ ഭയപ്പെടാനോ ദുഃഖിക്കാനോ ഇടയാവുന്നതല്ല'. ഇതിനെക്കുറിച്ച് എന്തു പറയുന്നു?''}

യഹൂദരുടെ വംശീയവാദത്തെ നിരാകരിക്കുന്ന വിശുദ്ധവാക്യമാണിത്. തങ്ങളുടെ വംശം മാത്രമാണ് ദൈവത്തിന് പ്രിയപ്പെട്ടവരും സ്വര്‍ഗാവകാശികളുമെന്നായിരുന്നു അവരുടെ അവകാശവാദം. തങ്ങള്‍ എങ്ങനെ ജീവിച്ചാലും രക്ഷപ്രാപിക്കുമെന്നും മറ്റുള്ളവരെല്ലാം നരകാവകാശികളായിരിക്കുമെന്നും അവര്‍ വിശ്വസിച്ചു. ഇതിനെ ഖണ്ഡിച്ചുകൊണ്ട് ഖുര്‍ആന്‍ പറഞ്ഞു: 'വംശമോ ജാതിയോ സമുദായമോ അല്ല വിജയനിദാനം; മറിച്ച് ദൈവത്തിലും പരലോകത്തിലുമുള്ള വിശ്വാസവും സല്‍ക്കര്‍മവുമാണ്.''
ദൈവത്തിലും മരണാനന്തര ജീവിതത്തിലും യഥാവിധി വിശ്വസിക്കലും സല്‍ക്കര്‍മങ്ങളനുഷ്ഠിക്കലും തന്നെയാണ് ഇസ്‌ലാം. ആര്‍ ആവിധം ജീവിക്കുന്നുവോ അവര്‍ മുസ്‌ലിംകളാണ്. അവരുടെ വംശമോ വര്‍ഗമോ ദേശമോ ജാതിയോ ഏതായിരുന്നാലും ശരി. ജൂതന്മാരും ക്രിസ്ത്യാനികളും സാബികളും ദൈവത്തിലും മരണാനന്തര ജീവിതത്തിലും വിശ്വസിച്ച് സല്‍ക്കര്‍മങ്ങളനുഷ്ഠിക്കുന്നതോടെ അവര്‍ അവരല്ലാതായിത്തീരുന്നു. വിശ്വാസങ്ങളിലും ആരാധനകളിലും ആചാരങ്ങളിലും അനുഷ്ഠാനങ്ങളിലും ജീവിതരീതികളിലും സമൂലമായ മാറ്റം സംഭവിക്കുന്നു. പിന്നെ, നിലവിലുള്ള പേരിലവരറിയപ്പെടാനും ഇടയില്ല. അതിനാല്‍ ഖുര്‍ആന്‍ ഇവിടെയും വിജയത്തിന്റെ നിദാനമായി ഊന്നിപ്പറഞ്ഞത് ഇസ്‌ലാമിന്റെ അടിസ്ഥാനങ്ങളായ സത്യവിശ്വാസവും സല്‍ക്കര്‍മങ്ങളും തന്നെയാണ്.
(വിശദീകരണത്തിന് ഡയലോഗ് സെന്റര്‍ പ്രസിദ്ധീകരിച്ച 'സര്‍വമതസത്യവാദം' എന്ന കൃതി കാണുക.
\chapter{ഇസ്‌ലാമും പരിണാമസിദ്ധാന്തവും }
\section{ ശാസ്ത്രവിരുദ്ധമായി ഒന്നും ഇസ്‌ലാമിലില്ലെന്നാണല്ലോ പറയപ്പെടുന്നത്. എങ്കില്‍ ഇസ്‌ലാം പരിണാമസിദ്ധാന്തത്തെ അംഗീകരിക്കുന്നുണ്ടോ?}

A ഖണ്ഡിതമായി തെളിയിക്കപ്പെട്ട ശാസ്ത്ര സത്യങ്ങള്‍ക്ക് വിരുദ്ധമായതൊന്നും ഇസ്‌ലാമിലില്ലെന്നത് തീര്‍ത്തും ശരിയാണ്. എന്നാല്‍, ശാസ്ത്രനിഗമനങ്ങള്‍ക്കോ ചരിത്രപരമായ അനുമാനങ്ങള്‍ക്കോ ഇതു ബാധകമല്ല. പരിണാമവാദം ശാസ്ത്രീയമായി തെളിയിക്കപ്പെട്ട ഒന്നല്ല. ചരിത്രപരമായ ഒരനുമാനം മാത്രമാണ്.
പരിണാമം രണ്ടിനമാണെന്ന് ജീവശാസ്ത്രജ്ഞര്‍ പറയുന്നു. സൂക്ഷ്മപരിണാമവും സ്ഥൂലപരിണാമവും. സ്പീഷ്യസിനകത്തു നടക്കുന്ന നിസ്സാര മാറ്റങ്ങളെപ്പറ്റിയാണ് സൂക്ഷ്മപരിണാമമെന്നു പറയാറുള്ളത്. മനുഷ്യരില്‍ നീണ്ടവരും കുറിയവരും വെളുത്തവരും കറുത്തവരും ആണും പെണ്ണും പ്രതിഭാശാലികളും മന്ദബുദ്ധികളുമെല്ലാമുണ്ടല്ലോ. ഒരേ കുടുംബത്തില്‍ ഒരേ മാതാപിതാക്കളുടെ മക്കളില്‍പോലും ഈ വൈവിധ്യം ദൃശ്യമാണ്. ഇത്തരം മാറ്റങ്ങളെക്കുറിക്കുന്ന സൂക്ഷ്മപരിണാമത്തെ ഇസ്‌ലാം എതിര്‍ക്കുന്നില്ല. എന്നാല്‍ ഒരു ജീവി മറ്റൊന്നായി മാറുന്ന അഥവാ ഒരു സ്പീഷ്യസ് മറ്റൊന്നായി മാറുന്നുവെന്ന് അവകാശപ്പെടുന്ന സ്ഥൂലപരിണാമത്തെ സംബന്ധിച്ചാണ് വ്യത്യസ്തമായ വീക്ഷണമുള്ളത്. അതിലൊട്ടും അസാംഗത്യവുമില്ല. കാരണം സ്ഥൂലപരിണാമത്തിന് ശാസ്ത്രത്തിലോ ചരിത്രത്തിലോ ഒരു തെളിവുമില്ല. കേരളത്തിലെ പരിണാമവാദികളുടെ മുന്നണിപ്പോരാളിയായ ഡോ. കുഞ്ഞുണ്ണി വര്‍മതന്നെ ഇക്കാര്യം തുറന്നുസമ്മതിച്ചിട്ടുണ്ട്. 'പരീക്ഷണാത്മകമായ തെളിവുകളില്ലെന്ന് പറഞ്ഞത് സ്ഥൂലപരിണാമത്തെക്കുറിച്ചു മാത്രമാണ്. അതേസമയം, പരീക്ഷണവാദത്തിലടങ്ങിയിട്ടുള്ള മിക്ക തത്ത്വങ്ങളെക്കുറിച്ചും സൂക്ഷ്മമായ പരിണാമമാറ്റങ്ങളെക്കുറിച്ചും പരീക്ഷണങ്ങള്‍ നടത്തി ആശിച്ച ഫലങ്ങള്‍ സമ്പാദിക്കുവാന്‍ ശാസ്ത്രജ്ഞ•ാര്‍ക്കു കഴിഞ്ഞിട്ടുണ്ട്''(ഉദ്ധരണം: സൃഷ്ടിവാദവും പരിണാമവാദികളും, പുറം 33).
പരിണാമവാദത്തിന്റെ ഉപജ്ഞാതാവായി വാഴ്ത്തപ്പെടുന്ന ചാള്‍സ് ഡാര്‍വിന്‍ പോലും അതിനെ അനിഷേധ്യമായ ഒരു സിദ്ധാന്തമായി തറപ്പിച്ചു പറഞ്ഞിട്ടില്ല. അദ്ദേഹം എഴുതുന്നു: 'വിശേഷരൂപങ്ങളുടെ വ്യതിരിക്തതയും അവ അനേകം പരിവര്‍ത്തനകണ്ണികളാല്‍ ചേര്‍ക്കപ്പെട്ടിട്ടില്ല എന്നതും വളരെ വ്യക്തമായ പ്രശ്‌നമാണ്...''
'എന്റെ സിദ്ധാന്തപ്രകാരം സിലൂറിയന്‍ ഘട്ടത്തിനു മുമ്പ് തീര്‍ച്ചയായും എവിടെയെങ്കിലും അടിഞ്ഞുകൂടിയിരിക്കാവുന്ന വിപുലമായ ഫോസില്‍പാളികളുടെ അഭാവം ഉള്‍ക്കൊള്ളുന്നതിലുള്ള പ്രയാസം വളരെ വലുതാണ്. ഇതു വിശദീകരിക്കാനാവാതെ തുടരും. ഞാനിവിടെ അവതരിപ്പിച്ച വീക്ഷണങ്ങളോടുള്ള, സാധുതയുള്ള എതിര്‍വാദമായി ന്യായമായും ഇതുന്നയിക്കപ്പെട്ടേക്കാം.''(ഒറിജിന്‍ ഓഫ് സ്പീഷ്യസ്, പേജ് 314. ഉദ്ധരണം, ഡാര്‍വിനിസം: പ്രതീക്ഷയും പ്രതിസന്ധിയും, പുറം 23).
'ഡാര്‍വിന്‍ കൃതിയിലെ ഒമ്പതാം അധ്യായത്തിന്റെ തലക്കെട്ട് 'ഭൂശാസ്ത്ര രേഖയുടെ അപൂര്‍ണതയെപ്പറ്റി' എന്നാണ്. ഫോസില്‍ ശൃംഖലയിലെ വിടവുകളെപ്പറ്റി ഡാര്‍വിന്‍ നല്‍കുന്ന വിശദീകരണങ്ങളാണ് അതിന്റെ ഉള്ളടക്കം. അദ്ദേഹം എഴുതി: 'വിശേഷരൂപങ്ങളുടെ വ്യതിരിക്തതയും അവ അനേകം പരിവര്‍ത്തനകണ്ണികളാല്‍ ചേര്‍ക്കപ്പെട്ടിട്ടില്ല എന്നതും വളരെ വ്യക്തമായ പ്രശ്‌നമാണ്'' (ഒറിജിന്‍ ഓഫ് സ്പീഷ്യസ്, പേജ് 291. ഉദ്ധരണം, ഡാര്‍വിനിസം: പ്രതീക്ഷയും പ്രതിസന്ധിയും, പുറം 30).
'ഒരേ ഗ്രൂപ്പിലെ വിവിധ സ്പീഷ്യസുകള്‍ പഴക്കമേറിയ പാലിയോ സോയിക് കല്‍പത്തിന്റെ ആദ്യഘട്ടമായ സിലൂറിയന്‍ പാളികളില്‍ പെട്ടെന്ന് പ്രത്യക്ഷപ്പെടുന്നതായി അക്കാലത്തെ ഉത്ഖനനങ്ങള്‍ തെളിയിച്ചിരുന്നു. സിലൂറിയന് തൊട്ടുമുമ്പുള്ള ക്രസ്‌റേഷ്യന്‍ പാളിയില്‍ ഇവയുടെ മുന്‍ഗാമികളെ കാണേണ്ടിയിരുന്നു. പക്ഷേ, ലഭ്യമായില്ല. ഇതു ഗുരുതരമായ പ്രശ്‌നമാണെന്ന് സമ്മതിച്ചുകൊണ്ട് ഡാര്‍വിന്‍ എഴുതുന്നു: 'ഈ വിസ്തൃതമായ പ്രാഗ്ഘട്ടങ്ങളുടെ രേഖകള്‍ എന്തുകൊണ്ടു കാണുന്നില്ലെന്ന ചോദ്യത്തിന് സംതൃപ്തമായ ഉത്തരം നല്‍കാന്‍ എനിക്കു കഴിയില്ല''(ഒറിജിന്‍ ഓഫ് സ്പീഷ്യസ്, പേജ്, 313. ഉദ്ധരണം, ഡാര്‍വിനിസം: പ്രതീക്ഷയും പ്രതിസന്ധിയും, പുറം 30).
ചാള്‍സ് ഡാര്‍വിനു ശേഷം പരിണാമവാദത്തിന് ഉപോദ്ബലകമായ തെളിവുകളൊന്നും ലഭിച്ചിട്ടില്ലെന്നത് പ്രത്യേകം പ്രസ്താവ്യമാണ്. ചിക്കാഗോ സര്‍വകലാശാലയിലെ ഭൂഗര്‍ഭ ശാസ്ത്രവകുപ്പിന്റെ ചെയര്‍മാന്‍ ഡേവിഡ് എം. റൂപ്പ് എഴുതുന്നു: 'തന്റെ സിദ്ധാന്തവും ഫോസില്‍ തെളിവുകളും തമ്മിലുള്ള പൊരുത്തക്കേടിന് സാമാന്യ പരിഹാരമായി ഫോസില്‍ രേഖ വളരെ അപൂര്‍വമാണെന്നാണ് ഡാര്‍വിന്‍ പറഞ്ഞത്. ഡാര്‍വിനുശേഷം നൂറ്റിരുപത് വര്‍ഷം കഴിഞ്ഞിരിക്കുന്നു. ഫോസില്‍ രേഖയെപ്പറ്റിയുള്ള വിജ്ഞാനം വളരെയേറെ വികസിച്ചിട്ടുണ്ട്. നമുക്കിപ്പോള്‍ രണ്ടു ലക്ഷത്തി അമ്പതിനായിരം ഫോസില്‍ സ്പീഷ്യസുകളുണ്ടെങ്കിലും സ്ഥിതി അത്രയൊന്നും മാറിയിട്ടില്ല. പരിണാമരേഖ ഇപ്പോഴും വിസ്മയിപ്പിക്കുംവിധം ഭംഗമുള്ളതാണ്. വിരോധാഭാസമെന്നോണം ഡാര്‍വിന്റെ കാലത്തുണ്ടായിരുന്നതിനേക്കാള്‍ കുറഞ്ഞ പരിണാമ ഉദാഹരണങ്ങളേ നമുക്കുള്ളൂ'' (ഇീിളഹശരെേ യലംേലലി ഉമൃംശിശാെ മിറ ജമഹീലമിീേഹീഴ്യ ആൗഹഹലേേശി എശലഹറ ങൗലൌാ ീള ചമൗേൃമഹ ഒശേെീൃ്യ: 50 ഖമി 1979, ജ: 22 ഉദ്ധരണം കയശറ പുറം 23).
ഡാര്‍വിനിസത്തില്‍ പരിണാമവാദികള്‍ക്കുതന്നെ വിശ്വാസം നഷ്ടപ്പെട്ടുകൊണ്ടിരിക്കുകയാണ്. ഓക്‌സ്‌ഫോര്‍ഡ് സര്‍വകലാശാല 1982ല്‍ പുറത്തിറക്കിയ, പ്രമുഖ ജന്തുശാസ്ത്രജ്ഞനും പരിണാമവാദിയുമായ ഹൊവാഡിന്റെ ഡാര്‍വിനെ സംബന്ധിച്ച ജീവചരിത്ര കൃതിയിലിങ്ങനെ കാണാം: 'ഡാര്‍വിന്റെ മരണശതാബ്ദിയോടെ വിജ്ഞാനത്തിനുള്ള ഡാര്‍വിന്റെ സംഭാവനയുടെ വിലയേയും നിലയേയും പറ്റി വ്യാപകമായ സംശയമനോഭാവവും അസ്വസ്ഥതയും ഉണ്ടായിവരുന്നു''(ഉദ്ധരണം: സൃഷ്ടിവാദവും പരിണാമവാദികളും, പുറം 54).
പ്രമുഖ ഫോസില്‍ ശാസ്ത്രജ്ഞരായ സ്‌റീഫന്‍ ഗൌള്‍ഡും നീല്‍സ് എല്‍െ്രെഡഡ്ജും ഡാര്‍വിനിസത്തില്‍ വിശ്വാസം നഷ്ടപ്പെട്ടതിനാല്‍ പുതിയ സിദ്ധാന്തം ആവിഷ്‌കരിക്കുകയുണ്ടായി. വിശ്വവിഖ്യാതരായ പരിണാമവാദികള്‍ക്കുപോലും തങ്ങള്‍ മുമ്പോട്ടുവച്ച സിദ്ധാന്തത്തിന്റെ ദൌര്‍ബല്യങ്ങളെക്കുറിച്ച് നന്നായറിയാം. സത്യസന്ധരായ ചിലരെങ്കിലും അത് തുറന്നുപറയാറുണ്ട്. പ്രശസ്ത പുരാജീവിശാസ്ത്രജ്ഞനും പരിണാമ വിശ്വാസിയുമായ ഡോ. കോളിന്‍ പാറ്റേഴ്‌സണ്‍ എഴുതിയ കത്തിലിങ്ങനെ കാണാം: 'പക്ഷികളുടെയെല്ലാം മുന്‍ഗാമിയായിരുന്നുവോ ആര്‍ക്കിയോപ്‌ടെറിക്‌സ്? ഒരുപക്ഷേ, ആയിരിക്കാം. ഒരുപക്ഷേ അല്ലായിരിക്കാം. ഈ ചോദ്യത്തിനു ഉത്തരം കണ്ടെത്താന്‍ യാതൊരു മാര്‍ഗവുമില്ല. ഒരു രൂപത്തില്‍നിന്നും മറ്റൊരു രൂപമുണ്ടായതെങ്ങനെയെന്നതിനെപ്പറ്റി കഥകള്‍ മെനയാന്‍ എളുപ്പമാണ്; പ്രകൃതിനിര്‍ധാരണം ഓരോ ഘട്ടത്തെയും എങ്ങനെയാണ് പിന്തുണച്ചതെന്നു പറയാനും. പക്ഷേ, അത്തരം കഥകള്‍ ശാസ്ത്രത്തിന്റെ ഭാഗമല്ല. കാരണം അവയെ പരീക്ഷണ വിധേയമാക്കാനാവില്ല'' (1979 ഏപ്രില്‍ 10ലെ പാറ്റേഴ്‌സന്റെ കത്ത് ് ടരീുല െകക ഠവല ഏൃലമ േഉലയമലേ ുു. 1415. ഉദ്ധരണം: കയശറ പുറം 68).
പ്രമുഖ സൃഷ്ടിവാദ എഴുത്തുകാരനായ സുന്റര്‍ലാന്റിന്റെ കത്തിനുള്ള മറുപടിക്കത്തായിരുന്നു കോളിന്‍ പാറ്റേഴ്‌സന്റേത്. അതിലദ്ദേഹം ഇത്രകൂടി കുറിച്ചിടുകയുണ്ടായി: 'പരിണാമപരമായ പരിവര്‍ത്തനങ്ങളുടെ നേരിട്ടുള്ള ഉദാഹരണങ്ങള്‍ എന്റെ പുസ്തകത്തില്‍ ഇല്ലാത്തതിനെക്കുറിച്ചുള്ള താങ്കളുടെ അഭിപ്രായങ്ങളോട് ഞാന്‍ പൂര്‍ണമായി യോജിക്കുന്നു. ജീവിച്ചിരിക്കുന്നതോ ഫോസില്‍ രൂപത്തിലുള്ളതോ ആയ അത്തരം ഏതെങ്കിലും ഒന്നിനെക്കുറിച്ച് എനിക്കറിയാമായിരുന്നെങ്കില്‍ തീര്‍ച്ചയായും ഞാനത് ഉള്‍പ്പെടുത്തുമായിരുന്നു. അത്തരം പരിവര്‍ത്തനങ്ങള്‍ ഒരു ആര്‍ട്ടിസ്‌റിനെക്കൊണ്ട് ഭാവനയില്‍ വരപ്പിച്ചുകൂടേയെന്ന് താങ്കള്‍ നിര്‍ദേശിക്കുന്നു. പക്ഷേ, അതേപ്പറ്റിയുള്ള വിവരങ്ങള്‍ അയാള്‍ക്ക് എവിടെനിന്ന് ലഭിക്കും? സത്യസന്ധമായി പറഞ്ഞാല്‍ എനിക്ക് നല്‍കാനാവില്ല.''(കയശറ പുറം 68)
ശാസ്ത്രത്തിന്റെ കണ്ടെത്തലുകളൊക്കെ ഡാര്‍വിനിസത്തെ ശക്തിപ്പെടുത്തിയിട്ടേയുള്ളൂവെന്ന് അടുത്തകാലം വരെ ശക്തമായി വാദിച്ചിരുന്ന ഡോ. എ.എന്‍. നമ്പൂതിരിപോലും മറിച്ചുപറയാന്‍ നിര്‍ബന്ധിതനാവുകയുണ്ടായി. അദ്ദേഹം എഴുതി: 'ശാസ്ത്രരംഗത്തെ പുതിയ മാറ്റങ്ങള്‍ ഉള്‍ക്കൊള്ളാനും ഊര്‍ജസ്വലമായി നിലനിര്‍ത്താനും ഡാര്‍വിനിസത്തിനു ഇതുവരെ കഴിഞ്ഞു. അടുത്തകാലത്താണ് ചിത്രം മാറിയത്. ഇപ്പോഴും ഡാര്‍വിനിസത്തിന്റെ പ്രഭയ്ക്ക് പൊതുവെ മങ്ങലേറ്റിട്ടില്ല. എങ്കിലും ജീവന്റെ ചരിത്രത്തിലെ ചില ഘട്ടങ്ങളിലെങ്കിലും പരിണാമത്തിന്റെ പ്രവര്‍ത്തനരീതി ഡാര്‍വിന്‍ സങ്കല്‍പത്തിന് അനുയോജ്യമല്ല എന്ന സൂചനകളുണ്ട്. ഡാര്‍വിനിസം നേരിടുന്ന ആദ്യത്തെ പ്രതിസന്ധി.'' (ഡാര്‍വിനിസം വഴിത്തിരിവില്‍, കലാകൌമുദി 1076, പേജ് 19).
ശാസ്ത്രസത്യങ്ങളുടെ നേരെ ഉന്നയിക്കപ്പെടുന്ന ചോദ്യങ്ങള്‍ക്ക് തൃപ്തികരമായ മറുപടികളുണ്ടായിരിക്കും. എന്നാല്‍ പരിണാമവാദത്തിനു നേരെ ധൈഷണിക തലത്തില്‍നിന്ന് ഉയര്‍ത്തപ്പെട്ട ചോദ്യങ്ങള്‍ക്കൊന്നും ഇതേവരെ മറുപടി ലഭിച്ചിട്ടില്ല. ഉദാഹരണത്തിന് ഒന്നുമാത്രമിവിടെ ഉദ്ധരിക്കാം.'ജീവികളില്‍ നിരന്തരമായി വ്യതിയാനങ്ങള്‍ ഉണ്ടായിക്കൊണ്ടിരിക്കുന്നു. അനുകൂല വ്യതിയാനങ്ങളെ പ്രകൃതിനിര്‍ധാരണം വഴി അതിജീവിപ്പിക്കുന്നതിനാല്‍ അവ പാരമ്പര്യമായി കൈമാറപ്പെടുന്നു. അനുകൂലഗുണങ്ങള്‍ സ്വരൂപിക്കപ്പെട്ട് പുതിയ ജീവിവര്‍ഗങ്ങളുണ്ടാവുന്നു'' എന്നാണ് ഡാര്‍വിനിസത്തിന്റെ വാദം. ഇതു ശരിയാണെങ്കില്‍ ഏകദേശം മുന്നൂറുകോടി വര്‍ഷങ്ങള്‍ക്കപ്പുറം ഉണ്ടായ (ഇന്നും നിലനില്‍ക്കുന്ന) ഏകകോശ ലളിതജീവികളായ അമീബകള്‍ എന്തുകൊണ്ടവശേഷിച്ചു? ഇന്നു കാണപ്പെടുന്ന അമീബകളുടെ മുന്‍തലമുറകളില്‍ നിരന്തരമായി വ്യതിയാനങ്ങള്‍ ഉണ്ടായില്ലേ? ഇല്ലെങ്കില്‍ എന്തുകൊണ്ട്? അവ മറ്റൊരു ജീവിവര്‍ഗമായി പരിണമിക്കാതെ പ്രതികൂല പരിതഃസ്ഥിതികളെ എങ്ങനെ അതിജീവിച്ചു? മുന്നൂറുകോടി വര്‍ഷങ്ങള്‍ക്കപ്പുറം നിലനിന്ന ഏകകോശങ്ങളുടെ ഒരു ശ്രേണി അതേപടി തുടരുകയും മറ്റൊരു ശ്രേണി അസംഖ്യം പുതിയ സ്പീഷ്യസുകളിലൂടെ സസ്തനികളിലെത്തുകയും ചെയ്തതെന്തുകൊണ്ട്? ഇത്തരം ചോദ്യങ്ങള്‍ ഡാര്‍വിനിസത്തിന്റെ സൈദ്ധാന്തിക ശേഷിയെ പരീക്ഷിക്കുന്നവയാണ്. ഈ വസ്തുത തന്നെ കുഴക്കുന്നതായി ഡാര്‍വിന്‍ തന്നെ സമ്മതിച്ചിട്ടുണ്ട്. 1860 മെയ് 22ന് ഡാര്‍വിന്‍ അദ്ദേഹത്തിന്റെ പിന്തുണക്കാരനും ഹാര്‍വാര്‍ഡ് സര്‍വകലാശാലയിലെ സസ്യശാസ്ത്രജ്ഞനുമായിരുന്ന അസാഗ്രേക്ക് എഴുതി: 'കത്തുകളില്‍നിന്നും അഭിപ്രായ പ്രകടനങ്ങളില്‍നിന്നും മനസ്സിലാകുന്നതിനനുസരിച്ച് എന്റെ കൃതിയുടെ ഏറ്റവും വലിയ പോരായ്മ എല്ലാ ജൈവരൂപങ്ങളും പുരോഗമിക്കുന്നുവെങ്കില്‍ പിന്നെ ലളിതമായ ജൈവരൂപങ്ങള്‍ എന്തിനു നിലനില്‍ക്കുന്നു എന്നു വിശദീകരിക്കാന്‍ കഴിയാതെ പോയതാണ്.''(ഫ്രാന്‍സിസ് ഡാര്‍വിന്‍ എഡിറ്റു ചെയ്ത ഠവല ഹശളല മിറ ഹലേേലൃ െീള ഇവമൃഹ െഉമൃംശി എന്ന കൃതിയില്‍നിന്ന്, എന്‍. എം. ഹുസൈന്‍, ഡാര്‍വിനിസം: പ്രതീക്ഷയും പ്രതിസന്ധിയും, പുറം 18,19).
പരിണാമവാദികള്‍ ആദ്യകാലത്ത് തങ്ങളുടെ വാദത്തിന് തെളിവായി എടുത്തുകാണിച്ചിരുന്ന നിയാണ്ടര്‍താല്‍ മനുഷ്യരുടെ ഫോസിലുകള്‍ കണരോഗം ബാധിച്ച സാധാരണ മനുഷ്യരുടെ അസ്ഥികള്‍ ചേര്‍ത്തുവെച്ചതാണെന്ന് പിന്നീട് തെളിയിക്കപ്പെടുകയുണ്ടായി. അതോടെ മൃഗഛായയുള്ള നിയാണ്ടര്‍താല്‍ മനുഷ്യന്റെ കഥയും കണ്ണിയും അറ്റുപോയി കഥാവശേഷമായി. പകരമൊന്നു വയ്ക്കാനിന്നോളം പരിണാമവാദികള്‍ക്കു സാധിച്ചിട്ടില്ല. ഇപ്രകാരംതന്നെയാണ് ഹല്‍ട്ട് മനുഷ്യന്റെ കഥയും. മനുഷ്യനും ആള്‍ക്കുരങ്ങനുമിടയിലെ പ്രസിദ്ധമായൊരു കണ്ണിയായാണ് പരിണാമവാദികള്‍ ഹല്‍ട്ട് മനുഷ്യനെ അവതരിപ്പിച്ചിരുന്നത്. എന്നാല്‍ ആള്‍ക്കുരങ്ങിന്റെ താടിയെല്ല് മനുഷ്യന്റെ തലയോട്ടില്‍ ഘടിപ്പിച്ച് കൃത്രിമമായി ഉണ്ടാക്കിയതാണ് ഈ അര്‍ധ മനുഷ്യ ഫോസിലെന്ന് പിന്നീട് തെളിയിക്കപ്പെടുകയുണ്ടായി. ഇതെല്ലാം പരിണാമവാദത്തെ കഥാവശേഷമാക്കുന്നതില്‍ അതിപ്രധാനമായ പങ്കാണ് വഹിച്ചുകൊണ്ടിരിക്കുന്നത്.
ചുരുക്കത്തില്‍, പരിണാമവാദത്തിന് എന്തെങ്കിലും ശാസ്ത്രീയ അടിത്തറയോ പ്രമാണത്തിന്റെ പിന്‍ബലമോ ഇല്ല. കേവലം വികല ഭാവനയും അനുമാനങ്ങളും നിഗമനങ്ങളും മാത്രമാണത്. അതിന്റെ വിശ്വാസ്യത അടിക്കടി നഷ്ടപ്പെട്ടുകൊണ്ടിരിക്കുകയാണ്. ഇത്തരമൊന്നിനെ ഇസ്‌ലാം അംഗീകരിക്കുന്നുണ്ടോ എന്ന പ്രശ്‌നം ഉദ്ഭവിക്കുന്നില്ല. പരിണാമ സങ്കല്പം ഒരു സിദ്ധാന്തമായി തെളിയിക്കപ്പെടുമ്പോഴേ അതിന്റെ നേരെയുള്ള ഇസ്‌ലാമിന്റെ സമീപനം വിശകലനം ചെയ്യുന്നതിലര്‍ഥമുള്ളൂ.


\chapter{ഖുര്‍ആന്‍ ദൈവികഗ്രന്ഥമോ? }
 \section{ഖുര്‍ആന്‍ ദൈവികഗ്രന്ഥമാണെന്നാണല്ലോ മു്‌സ്‌ലിംകള്‍ അവകാശപ്പെടാറുള്ളത്. അത് മുഹമ്മദിന്റെ രചനയല്ലെന്നും ദൈവികമാണെന്നും എങ്ങനെയാണ് മനസ്സിലാവുക? എന്താണതിന് തെളിവ്?}

A ഖുര്‍ആന്‍ ദൈവികമാണെന്നതിനു തെളിവ് ആ ഗ്രന്ഥം തന്നെയാണ്. മുഹമ്മദ് നബിയുടെയും അദ്ദേഹത്തിലൂടെ അവതീര്‍ണമായ ഖുര്‍ആന്റെയും വ്യക്തമായ ചിത്രവും ചരിത്രവും മനുഷ്യരാശിയുടെ മുമ്പിലുണ്ട്. നബിതിരുമേനിയുടെ ജീവിതത്തിന്റെ ഉള്ളും പുറവും രഹസ്യവും പരസ്യവുമായ മുഴുവന്‍ കാര്യങ്ങളും ഒന്നൊഴിയാതെ രേഖപ്പെടുത്തപ്പെട്ടിട്ടുണ്ട്. ആധുനികലോകത്തെ മഹാന്മാരുടെ ചരിത്രം പോലും ആ വിധം വിശദമായും സൂക്ഷ്മമായും കുറിക്കപ്പെട്ടിട്ടില്ലെന്നതാണ് വസ്തുത.
അജ്ഞതാന്ധകാരത്തില്‍ ആണ്ടുകിടന്നിരുന്ന ആറാം നൂറ്റാണ്ടിലെ അറേബ്യയിലാണല്ലോ മുഹമ്മദ് ജനിച്ചത്. മരുഭൂമിയുടെ മാറില്‍ തീര്‍ത്തും അനാഥനായാണ് അദ്ദേഹം വളര്‍ന്നുവന്നത്. ചെറുപ്പത്തില്‍തന്നെ ഇടയവൃത്തിയിലേര്‍പ്പെട്ട മുഹമ്മദിന് എഴുതാനോ വായിക്കാനോ അറിയുമായിരുന്നില്ല. പാഠശാലകളില്‍ പോവുകയോ മതചര്‍ച്ചകളില്‍ പങ്കെടുക്കുകയോ ചെയ്തിരുന്നില്ല. മക്ക സാഹിത്യകാരന്മാരുടെയും കവികളുടെയും പ്രസംഗകരുടെയും കേന്ദ്രമായിരുന്നെങ്കിലും നാല്‍പതു വയസ്സുവരെ അദ്ദേഹം ഒരൊറ്റ വരി കവിതയോ ഗദ്യമോ പദ്യമോ രചിച്ചിരുന്നില്ല. പ്രസംഗപാടവം പ്രകടിപ്പിച്ചിരുന്നില്ല. സര്‍ഗസിദ്ധിയുടെ അടയാളമൊന്നും അദ്ദേഹത്തില്‍ ദൃശ്യമായിരുന്നില്ല.
ആത്മീയതയോട് അതിതീവ്രമായ ആഭിമുഖ്യമുണ്ടായിരുന്ന മുഹമ്മദ് മക്കയിലെ മലിനമായ അന്തരീക്ഷത്തില്‍നിന്ന് മാറി ധ്യാനത്തിലും പ്രാര്‍ഥനയിലും വ്യാപൃതനായി. ഏകാന്തവാസം ഏറെ ഇഷ്ടപ്പെട്ടു. വിശുദ്ധ കഅ്ബയില്‍നിന്ന് മൂന്നു കിലോമീറ്റര്‍ വടക്കുള്ള മലമുകളിലെ ഹിറാഗുഹയില്‍ ഏകാന്തവാസമനുഷ്ഠിക്കവെ മുഹമ്മദിന് ആദ്യമായി ദിവ്യസന്ദേശം ലഭിച്ചു. തുടര്‍ന്നുള്ള ഇരുപത്തിമൂന്നു വര്‍ഷങ്ങളില്‍ വിവിധ സന്ദര്‍ഭങ്ങളിലായി ലഭിച്ച ദിവ്യബോധനങ്ങളുടെ സമാഹാരമാണ് വിശുദ്ധ ഖുര്‍ആന്‍. അത് സാധാരണ അര്‍ഥത്തിലുള്ള ഗദ്യമോ പദ്യമോ കവിതയോ അല്ല. തീര്‍ത്തും സവിശേഷമായ ശൈലിയാണ് ഖുര്‍ആന്റേത്. അതിനെ അനുകരിക്കാനോ അതിനോട് മത്സരിക്കാനോ കിടപിടിക്കാനോ ഇന്നേവരെ ആര്‍ക്കും സാധിച്ചിട്ടില്ല. ലോകാവസാനം വരെ ആര്‍ക്കും സാധിക്കുകയുമില്ല.
അനുയായികള്‍ ദൈവികമെന്ന് അവകാശപ്പെടുന്ന ഒന്നിലേറെ ഗ്രന്ഥങ്ങള്‍ ലോകത്തുണ്ട്. എന്നാല്‍ സ്വയം ദൈവികമെന്ന് പ്രഖ്യാപിക്കുന്ന ഒരൊറ്റ ഗ്രന്ഥമേ ലോകത്തുള്ളൂ. ഖുര്‍ആനാണത്. ഖുര്‍ആന്‍ ദൈവത്തില്‍നിന്ന് അവതീര്‍ണമായതാണെന്ന് അത് അനേകം തവണ ആവര്‍ത്തിച്ചാവര്‍ത്തിച്ചു പ്രഖ്യാപിക്കുന്നു. അതോടൊപ്പം അതിലാര്‍ക്കെങ്കിലും സംശയമുണ്ടെങ്കില്‍, 114 അധ്യായങ്ങളുള്ള ഖുര്‍ആനിലെ ഏതെങ്കിലും ഒരധ്യായത്തിന് സമാനമായ ഒരധ്യായമെങ്കിലും കൊണ്ടുവരാന്‍ അത് വെല്ലുവിളിക്കുന്നു. അതിന് ലോകത്തുള്ള ഏതു സാഹിത്യകാരന്റെയും പണ്ഡിതന്റെയും ബുദ്ധിജീവിയുടെയും സഹായം തേടാമെന്ന കാര്യം ഉണര്‍ത്തുകയും ചെയ്തു. അല്ലാഹു പറയുന്നു: 'നാം നമ്മുടെ ദാസന്ന് അവതരിപ്പിച്ചിട്ടുള്ള ഈ ഗ്രന്ഥത്തെക്കുറിച്ച്, അതു നമ്മില്‍ നിന്നുള്ളതു തന്നെയോ എന്നു നിങ്ങള്‍ സംശയിക്കുന്നുവെങ്കില്‍ അതുപോലുള്ള ഒരദ്ധ്യായമെങ്കിലും നിങ്ങള്‍ കൊണ്ടുവരിക. അതിന്ന് ഏകനായ അല്ലാഹുവിനെകൂടാതെ, സകല കൂട്ടാളികളുടെയും സഹായം തേടിക്കൊള്ളുക. നിങ്ങള്‍ സത്യവാന്മാരെങ്കില്‍ അതു ചെയ്തുകാണിക്കുക.''(ഖുര്‍ആന്‍ 2: 23)
പ്രവാചകകാലം തൊട്ടിന്നോളം നിരവധി നൂറ്റാണ്ടുകളിലെ ഇസ്‌ലാം വിമര്‍ശകരായ എണ്ണമറ്റ കവികളും സാഹിത്യകാരന്മാരും ഈ വെല്ലുവിളിയെ നേരിടാന്‍ ശ്രമിച്ചിട്ടുണ്ട്. രണ്ടിലൊരനുഭവമേ അവര്‍ക്കൊക്കെയും ഉണ്ടായിട്ടുള്ളൂ. മഹാഭൂരിപക്ഷവും പരാജയം സമ്മതിച്ച് ഖുര്‍ആന്റെ അനുയായികളായി മാറുകയായിരുന്നു. അവശേഷിക്കുന്നവര്‍ പരാജിതരായി പിന്മാറുകയും. നബിതിരുമേനിയുടെ കാലത്തെ പ്രമുഖ സാഹിത്യകാരന്മാരായിരുന്ന ലബീദും ഹസ്സാനും കഅ്ബുബ്‌നു സുഹൈറുമെല്ലാം ഖുര്‍ആന്റെ മുമ്പില്‍ നിരുപാധികം കീഴടങ്ങിയവരില്‍ പെടുന്നു.
യമനില്‍നിന്നെത്തിയ ത്വുഫൈലിനെ ഖുര്‍ആന്‍ കേള്‍ക്കുന്നതില്‍ നിന്ന് ഖുറൈശികള്‍ വിലക്കി. ഏതോ അന്തഃപ്രചോദനത്താല്‍ അതു കേള്‍ക്കാനിടയായ പ്രമുഖ കവിയും ഗായകനുമായ അദ്ദേഹം പറഞ്ഞതിങ്ങനെയാണ്: 'ദൈവമാണ! അവന്‍ സര്‍വശക്തനും സര്‍വജ്ഞനുമല്ലോ. ഞാനിപ്പോള്‍ ശ്രവിച്ചത് അറബി സാഹിത്യത്തിലെ അതുല്യമായ വാക്യങ്ങളത്രെ. നിസ്സംശയം, അവ അത്യുല്‍കൃഷ്ടം തന്നെ. മറ്റേതിനെക്കാളും പരിശുദ്ധവും. അവ എത്ര ആശയ സമ്പുഷ്ടം! അര്‍ഥപൂര്‍ണം! എന്തുമേല്‍ മനോഹരം! ഏറെ ആകര്‍ഷകവും! ഇതുപോലുള്ള ഒന്നും ഞാനിതുവരെ കേട്ടിട്ടില്ല. അല്ലാഹുവാണ! ഇത് മനുഷ്യവചനമല്ല. സ്വയംകൃതവുമല്ല. ദൈവികം തന്നെ, തീര്‍ച്ച. നിസ്സംശയം ദൈവികവാക്യങ്ങളാണിവ.''
മുഗീറയുടെ മകന്‍ വലീദ് ഇസ്‌ലാമിന്റെയും പ്രവാചകന്റെയും കടുത്ത എതിരാളിയായിരുന്നു. ഖുര്‍ആന്‍ ഓതിക്കേള്‍ക്കാനിടയായ അയാള്‍ തന്റെ അഭിപ്രായം ഇങ്ങനെ രേഖപ്പെടുത്തി: 'ഇതില്‍ എന്തെന്നില്ലാത്ത മാധുര്യമുണ്ട്. പുതുമയുണ്ട്. അത്യന്തം ഫലസമൃദ്ധമാണിത്. നിശ്ചയമായും ഇത് അത്യുന്നതി പ്രാപിക്കും. മറ്റൊന്നും ഇതിനെ കീഴ്‌പെടുത്തുകയില്ല. ഇതിനു താഴെയുള്ളതിനെ ഇത് തകര്‍ത്ത് തരിപ്പണമാക്കും. ഒരിക്കലും ഒരു മനുഷ്യനിങ്ങനെ പറയുക സാധ്യമല്ല.''
വിവരമറിഞ്ഞ പ്രവാചകന്റെ പ്രധാന പ്രതിയോഗി അബൂജഹ്ല്! വലീദിനെ സമീപിച്ച് ഖുര്‍ആനെ സംബന്ധിച്ച് മതിപ്പ് കുറയ്ക്കുന്ന എന്തെങ്കിലും പറയാനാവശ്യപ്പെട്ടു. നിസ്സഹായനായ വലീദ് ചോദിച്ചു: 'ഞാനെന്തു പറയട്ടെ; ഗാനം, പദ്യം, കവിത, ഗദ്യം തുടങ്ങി അറബി സാഹിത്യത്തിന്റെ ഏതു ശാഖയിലും എനിക്കു നിങ്ങളെക്കാളേറെ പരിജ്ഞാനമുണ്ട്. അല്ലാഹുവാണ! ഈ മനുഷ്യന്‍ പറയുന്ന കാര്യങ്ങള്‍ക്ക് അവയോടൊന്നും സാദൃശ്യമില്ല. അല്ലാഹു സാക്ഷി! ആ സംസാരത്തില്‍ അസാധാരണ മാധുര്യവും സവിശേഷ സൌന്ദര്യവുമുണ്ട്. അതിന്റെ ശാഖകള്‍ ഫലസമൃദ്ധവും തളിരുകള്‍ ശ്യാമസുന്ദരവുമാണ്. ഉറപ്പായും അത് മറ്റേതു വാക്യത്തേക്കാളും ഉല്‍കൃഷ്ടമാണ്. ഇതര വാക്യങ്ങള്‍ സര്‍വവും അതിനു താഴെയും.''
ഇത് അബൂജഹ്ലിനെ അത്യധികം അസ്വസ്ഥനാക്കി. അയാള്‍ പറഞ്ഞു: 'താങ്കള്‍ ആരാണെന്നറിയാമോ? അറബികളുടെ അത്യുന്നതനായ നേതാവാണ്. യുവസമൂഹത്തിന്റെ ആരാധ്യനാണ്; എന്നിട്ടും താങ്കള്‍ ഒരനാഥച്ചെക്കനെ പിന്‍പറ്റുകയോ? അവന്റെ ഭ്രാന്തന്‍ ജല്‍പനങ്ങളെ പാടിപ്പുകഴ്ത്തുകയോ? താങ്കളെപ്പോലുള്ള മഹാന്മാര്‍ക്കത് കുറച്ചിലാണ്. അതിനാല്‍ മുഹമ്മദിനെ പുഛിച്ചു തള്ളുക.''
അബൂജഹ്ലിന്റെ ലക്ഷ്യം പിഴച്ചില്ല. അഹന്തക്കടിപ്പെട്ട വലീദ് പറഞ്ഞു: 'മുഹമ്മദ് ഒരു ജാലവിദ്യക്കാരനാണ്. സഹോദരങ്ങളെ തമ്മില്‍ തല്ലിക്കുന്നു. ഭാര്യാഭര്‍തൃബന്ധം മുറിച്ചുകളയുന്നു. കുടുംബഭദ്രത തകര്‍ക്കുന്നു. നാട്ടില്‍ കുഴപ്പങ്ങള്‍ സൃഷ്ടിക്കുന്ന ഒരു ജാലവിദ്യക്കാരന്‍ മാത്രമാണ് മുഹമ്മദ്.''
എത്ര ശ്രമിച്ചിട്ടും വലീദിനെപ്പോലുള്ള പ്രഗത്ഭനായ സാഹിത്യകാരന് ഖുര്‍ആന്നെതിരെ ഒരക്ഷരം പറയാന്‍ സാധിച്ചില്ലെന്നത് ശ്രദ്ധേയമത്രെ.
നാല്‍പതു വയസ്സുവരെ നബിതിരുമേനി ജീവിതത്തിലൊരൊറ്റ കളവും പറഞ്ഞിട്ടില്ല. അതിനാല്‍ അദ്ദേഹം അല്‍അമീന്‍ (വിശ്വസ്തന്‍) എന്ന പേരിലാണറിയപ്പെട്ടിരുന്നത്. അത്തരമൊരു വ്യക്തി ദൈവത്തിന്റെ പേരില്‍ പെരുങ്കള്ളം പറയുമെന്ന് സങ്കല്‍പിക്കുക പോലും സാധ്യമല്ല. മാത്രമല്ല; അത്യുല്‍കൃഷ്ടമായ ഒരു ഗ്രന്ഥം സ്വയം രചിക്കുന്ന ആരെങ്കിലും അത് തന്റേതല്ലെന്നും തനിക്കതില്‍ ഒരു പങ്കുമില്ലെന്നും പറയുമെന്ന് പ്രതീക്ഷിക്കാവതല്ല. അഥവാ പ്രവാചകന്‍ ഖുര്‍ആന്‍ സ്വന്തം സൃഷ്ടിയാണെന്ന് അവകാശപ്പെട്ടിരുന്നെങ്കില്‍ അറേബ്യന്‍ ജനത അദ്ദേഹത്തെ അത്യധികം ആദരിക്കുമായിരുന്നു. എന്നാല്‍ അദ്ദേഹത്തിന് ലഭിച്ചത് കൊടിയ പീഡനങ്ങളാണല്ലോ.
ലോകത്ത് അസംഖ്യം ഗ്രന്ഥങ്ങള്‍ രചിക്കപ്പെട്ടിട്ടുണ്ട്. അവയില്‍ ഏറെ ശ്രദ്ധേയമായവ ചരിത്രത്തില്‍ ചില മാറ്റങ്ങള്‍ സൃഷ്ടിക്കുകയും വിപ്‌ളവങ്ങള്‍ക്ക് നിമിത്തമാവുകയും ചെയ്തിട്ടുണ്ട്. എന്നാല്‍ വിശുദ്ധ ഖുര്‍ആനെപ്പോലെ, ഒരു ജനതയുടെ ജീവിതത്തെ സമഗ്രമായി സ്വാധീനിക്കുകയും പരിവര്‍ത്തിപ്പിക്കുകയും ചെയ്ത മറ്റൊരു ഗ്രന്ഥവും ലോകത്തില്ല. വിശ്വാസം, ജീവിതവീക്ഷണം, ആരാധന, ആചാരാനുഷ്ഠാനങ്ങള്‍, വ്യക്തിജീവിതം, കുടുംബരംഗം, സാമൂഹിക മേഖല, സാമ്പത്തിക വ്യവസ്ഥ, സാംസ്‌കാരിക മണ്ഡലം, രാഷ്ട്രീയ ഘടന, ഭരണസമ്പ്രദായം, സ്വഭാവരീതി, പെരുമാറ്റക്രമം തുടങ്ങി വ്യക്തിയുടെയും കുടുംബത്തിന്റെയും സമൂഹത്തിന്റെയും രാഷ്ട്രത്തിന്റെയും ലോകത്തിന്റെയും എല്ലാ അവസ്ഥകളെയും വ്യവസ്ഥകളെയും വിശുദ്ധ ഖുര്‍ആന്‍ അടിമുടി മാറ്റിമറിക്കുകയുണ്ടായി. നിരക്ഷരനായ ഒരാള്‍ ഈ വിധം സമഗ്രമായ ഒരു മഹാവിപ്‌ളവം സൃഷ്ടിച്ച ഗ്രന്ഥം രചിക്കുമെന്ന് സങ്കല്‍പിക്കാനാവില്ല. ശത്രുക്കളെപ്പോലും വിസ്മയകരമായ വശ്യശക്തിയാല്‍ കീഴ്‌പെടുത്തി മിത്രമാക്കി മാറ്റി, അവരെ തീര്‍ത്തും പുതിയ മനുഷ്യരാക്കി പരിവര്‍ത്തിപ്പിച്ച ഗ്രന്ഥമാണ് ഖുര്‍ആന്‍. രണ്ടാം ഖലീഫ ഉമറുല്‍ ഫാറൂഖ് ഈ ഗണത്തിലെ ഏറെ ശ്രദ്ധേയനായ വ്യക്തിയത്രെ. ഇന്നും ഖുര്‍ആന്‍ ആഴത്തില്‍ പഠിക്കാന്‍ സന്നദ്ധരാവുന്നവര്‍ അനായാസം അതിന്റെ അനുയായികളായി മാറുന്നു.
ഖുര്‍ആന്‍ മാനവസമൂഹത്തിന്റെ മുമ്പില്‍ സമ്പൂര്‍ണമായൊരു ജീവിത വ്യവസ്ഥ സമര്‍പ്പിക്കുന്നു. മനുഷ്യ മനസ്സുകള്‍ക്ക് സമാധാനം സമ്മാനിക്കുകയും വ്യക്തിജീവിതത്തെ വിശുദ്ധവും കുടുംബവ്യവസ്ഥയെ സൈ്വരമുള്ളതും സമൂഹഘടനയെ ആരോഗ്യകരവും രാഷ്ട്രത്തെ ഭദ്രവും ലോകത്തെ പ്രശാന്തവുമാക്കുകയും ചെയ്യുന്ന സമഗ്രമായ പ്രത്യയശാസ്ത്രമാണത്. കാലാതീതവും ദേശാതീതവും നിത്യനൂതനവുമായ ഇത്തരമൊരു ജീവിതപദ്ധതി ഉള്‍ക്കൊള്ളുന്ന ഒരു ഗ്രന്ഥവും ലോകത്ത് വേറെയില്ല. ലോകത്തിലെ കോടിക്കണക്കിന് കൃതികളിലൊന്നുപോലും ഖുര്‍ആനിനെപ്പോലെ സമഗ്രമായ ഒരു ജീവിതക്രമം സമര്‍പ്പിക്കുന്നില്ല. നിരക്ഷരനായ ഒരാള്‍ക്ക് ഈ വിധമൊന്ന് രചിക്കാനാവുമെന്ന്, ബോധമുള്ള ആരും അവകാശപ്പെടുകയില്ല.
മനുഷ്യചിന്തയെ ജ്വലിപ്പിച്ച് വിചാരവികാരങ്ങളിലും വിശ്വാസവീക്ഷണങ്ങളിലും വമ്പിച്ച വിപ്‌ളവം സൃഷ്ടിച്ച വിശുദ്ധ ഖുര്‍ആന്‍ ചരിത്രത്തില്‍ തുല്യതയില്ലാത്ത, എക്കാലത്തും ഏതു നാട്ടുകാര്‍ക്കും മാതൃകായോഗ്യമായ സമൂഹത്തെ വാര്‍ത്തെടുത്ത് പുതിയൊരു സംസ്‌കാരത്തിനും നാഗരികതയ്ക്കും ജന്മം നല്‍കി. നൂറ്റിപ്പതിനാല് അധ്യായങ്ങളില്‍, ആറായിരത്തിലേറെ സൂക്തങ്ങളില്‍, എണ്‍പത്താറായിരത്തിലേറെ വാക്കുകളില്‍, മൂന്നു ലക്ഷത്തി ഇരുപത്തിനാലായിരത്തോളം അക്ഷരങ്ങളില്‍ വ്യാപിച്ചുകിടക്കുന്ന ഖുര്‍ആന്റെ പ്രധാന നിയോഗം മാനവസമൂഹത്തിന്റെ മാര്‍ഗദര്‍ശനമാണ്. മുപ്പതു ഭാഗമായും അഞ്ഞൂറ്റിനാല്‍പത് ഖണ്ഡികകളായും വിഭജിക്കപ്പെട്ട ഈ ഗ്രന്ഥത്തിന്റെ പ്രധാനപ്രമേയം മനുഷ്യനാണ്. എങ്കിലും അവന്റെ മാര്‍ഗസിദ്ധിക്ക് സഹായകമാംവിധം ചിന്തയെ ഉത്തേജിപ്പിക്കാനും കാര്യങ്ങള്‍ ബോധ്യപ്പെടുത്താനും വിജ്ഞാനം വികസിപ്പിക്കാനും ആവശ്യമായ ചരിത്രവും പ്രവചനങ്ങളും ശാസ്ത്രസൂചനകളുമെല്ലാം അതിലുണ്ട്. ഈ രംഗത്തെല്ലാം അക്കാലത്തെ ജനതയ്ക്ക് തീര്‍ത്തും അജ്ഞാതമായിരുന്ന കാര്യങ്ങള്‍ വിശുദ്ധ ഖുര്‍ആന്‍ അനാവരണം ചെയ്യുകയുണ്ടായി. ചിലതുമാത്രമിവിടെ ചേര്‍ക്കുന്നു.
1. അല്ലാഹു പറയുന്നു: 'സത്യനിഷേധികള്‍ ചിന്തിക്കുന്നില്ലേ? ഉപരിലോകങ്ങളും ഭൂമിയും ഒട്ടിച്ചേര്‍ന്നതായിരുന്നു. പിന്നീട് നാമവയെ വേര്‍പ്പെടുത്തി.''(21: 30)
ഈ സത്യം ശാസ്ത്രം കണ്ടെത്തിയത് ഖുര്‍ആന്‍ അവതീര്‍ണമായി അനേക നൂറ്റാണ്ടുകള്‍ പിന്നിട്ട ശേഷമാണെന്നത് സുവിദിതമത്രെ.
2. 'ജീവനുള്ളതിനെയെല്ലാം ജലത്തില്‍നിന്നാണ് നാം സൃഷ്ടിച്ചത്'' (21: 30). ഈ വസ്തുത ശാസ്ത്രജ്ഞന്മാര്‍ കണ്ടെത്തിയത് സമീപകാലത്തു മാത്രമാണ്.
3. 'അതിനുപുറമെ അവന്‍ ഉപരിലോകത്തിന്റെയും സംവിധാനം നിര്‍വഹിച്ചു. അത് ധൂളി(നെബുല)യായിരുന്നു'' (41: 11). ഈ സൃഷ്ടിരഹസ്യം ശാസ്ത്രം അനാവരണംചെയ്തത് അടുത്ത കാലത്താണ്.
4. 'സൂര്യന്‍ അതിന്റെ നിര്‍ണിത കേന്ദ്രത്തില്‍ സഞ്ചരിച്ചുകൊണ്ടിരിക്കുന്നു. അത് പ്രതാപശാലിയും സര്‍വജ്ഞനുമായ അല്ലാഹുവിന്റെ ക്രമീകരണമത്രെ.''(36: 38)
കോപ്പര്‍ നിക്കസിനെപ്പോലുള്ള പ്രമുഖരായ ശാസ്ത്രജ്ഞര്‍ പോലും സൂര്യന്‍ നിശ്ചലമാണെന്ന് വിശ്വസിക്കുന്നവരായിരുന്നു. അടുത്ത കാലം വരെയും സൂര്യന്‍ ചലിക്കുന്നുവെന്ന സത്യം അംഗീകരിക്കാന്‍ ഭൌതിക ശാസ്ത്രജ്ഞന്മാര്‍ സന്നദ്ധരായിരുന്നില്ല. എങ്കിലും അവസാനം ഖുര്‍ആന്റെ പ്രസ്താവം സത്യമാണെന്ന് സമ്മതിക്കാനവര്‍ നിര്‍ബന്ധിതരായി.
5. 'ഉപരിലോകത്തെ നാം സുരക്ഷിതമായ മേല്‍പുരയാക്കി. എന്നിട്ടും അവര്‍ നമ്മുടെ പ്രാപഞ്ചിക ദൃഷ്ടാന്തങ്ങള്‍ ശ്രദ്ധിക്കുന്നേയില്ല.''(21: 32)
അടുത്ത കാലം വരെയും ഖുര്‍ആന്റെ വിമര്‍ശകര്‍ ഈ വാക്യത്തിന്റെ പേരില്‍ പരിഹാസം ഉതിര്‍ക്കുക പതിവായിരുന്നു. എന്നാല്‍ ഏറെ മാരകമായ കോസ്മിക് രശ്മികളില്‍നിന്ന് ഭൂമിയെയും അതിലെ ജന്തുജാലങ്ങളെയും മനുഷ്യരെയും കാത്തുരക്ഷിക്കുന്ന ഓസോണ്‍ പാളികളെക്കുറിച്ച അറിവ് ഇന്ന് സാര്‍വത്രികമാണ്. അന്തരീക്ഷത്തിന്റെ ഈ മേല്‍പ്പുരയാണ് ഉല്‍ക്കകള്‍ ഭൂമിയില്‍ പതിച്ച് വിപത്തുകള്‍ വരുത്തുന്നത് തടയുന്നത്. കാലാവസ്ഥയെ നിയന്ത്രിക്കുന്നതിലും അതിന് അനല്‍പമായ പങ്കുണ്ട്. മലിനീകരണം കാരണം അതിന് പോറല്‍ പറ്റുമോ എന്ന ആശങ്ക പരിസ്ഥിതി ശാസ്ത്രജ്ഞന്മാര്‍ നിരന്തരം പ്രകടിപ്പിക്കാന്‍ തുടങ്ങിയിരിക്കുന്നു. ജീവിതമിവിടെ സാധ്യമാവണമെങ്കില്‍ ഖുര്‍ആന്‍ പറഞ്ഞ സുരക്ഷിതമായ മേല്‍പ്പുര അനിവാര്യമത്രെ. ശ്വസനത്തിനാവശ്യമായ വായുവിന്റെ മണ്ഡലത്തെ ഭദ്രമായി നിലനിര്‍ത്തുന്നതും ഈ മേല്‍പ്പുരതന്നെ.
6. 'നാം ഭൂമിയെ തൊട്ടിലും പര്‍വതങ്ങളെ ആണികളുമാക്കിയില്ലേ?'' (78:7). 'ഭൂമിയില്‍ നാം ഉറച്ച പര്‍വതങ്ങളുണ്ടാക്കിയിരിക്കുന്നു, ഭൂമി അവരെയുമായി തെന്നിപ്പോവാതിരിക്കാന്‍. ഭൂമിയില്‍ നാം വിശാലമായ വഴികളുണ്ടാക്കി, ജനം തങ്ങളുടെ മാര്‍ഗമറിയാന്‍''(21:31).
ഭൂമിയുടെ സന്തുലിതത്വത്തില്‍ പര്‍വതങ്ങള്‍ വഹിക്കുന്ന പങ്ക് അടുത്തകാലം വരെയും അജ്ഞാതമായിരുന്നു. എന്നാലിന്ന് ഭൂകമ്പങ്ങള്‍ തടയുന്നതിലും ഭൂഗോളത്തിന്റെ ആന്തരികവും ബാഹ്യവുമായ ഘടന സംരക്ഷിക്കുന്നതിലും അവയുടെ പങ്ക് പ്രമുഖ ഭൂഗര്‍ഭശാസ്ത്രജ്ഞന്മാരെല്ലാം ഊന്നിപ്പറഞ്ഞിട്ടുണ്ട്.
7. 'നിശ്ചയം, നാം ഉപരിലോകത്തെ വികസിപ്പിച്ചുകൊണ്ടേയിരിക്കുന്നു''(51: 47). പ്രപഞ്ചഘടനയെ സംബന്ധിച്ച പ്രാഥമിക ജ്ഞാനമുള്ളവരിലെല്ലാം ഒടുങ്ങാത്ത വിസ്മയം സൃഷ്ടിക്കാന്‍ ഖുര്‍ആന്റെ ഈ പ്രസ്താവം പര്യാപ്തമത്രെ.
8. ഹോളണ്ടുകാരനായ സ്വാമര്‍ഡാം എന്ന ജന്തുശാസ്ത്രജ്ഞന്‍ തേനീച്ചകളില്‍ കൂട് ഉണ്ടാക്കുകയും തേന്‍ ഉല്‍പാദിപ്പിക്കുകയും ചെയ്യുന്നത് പെണ്‍വര്‍ഗമാണെന്ന് തെളിയിച്ചത് 1876ല്‍ മാത്രമാണ്. എന്നാല്‍ ഈ രണ്ടും തേനീച്ചകളിലെ സ്ത്രീകളാണ് ചെയ്യുകയെന്ന് പതിനാലു നൂറ്റാണ്ട് മുമ്പു തന്നെ ഖുര്‍ആന്‍ അതിനെ പരാമര്‍ശിക്കുന്ന വാക്യങ്ങളിലെ സ്ത്രീലിംഗ പ്രയോഗത്തിലൂടെ സ്ഥാപിക്കുകയുണ്ടായി. (16: 68, 69)
9. ലോകത്തിലെ അറുനൂറു കോടി മനുഷ്യരുടെയും കൈവിരലടയാളം 600 കോടി രൂപത്തിലാണ്. സൃഷ്ടിയിലെ മഹാവിസ്മയങ്ങളിലൊന്നാണിത്. എന്നാല്‍ വിരല്‍ത്തുമ്പിലെ ഈ മഹാത്ഭുതം മനുഷ്യന്‍ തിരിച്ചറിഞ്ഞത് സമീപകാലത്താണ്. എങ്കിലും വിശുദ്ധ ഖുര്‍ആന്‍ പതിനാലു നൂറ്റാണ്ടുകള്‍ക്കപ്പുറം ഈ മഹാവിസ്മയത്തിലേക്ക് ശ്രദ്ധ ക്ഷണിക്കുകയുണ്ടായി. 'മനുഷ്യന്‍ വിചാരിക്കുന്നുവോ, നമുക്കവന്റെ എല്ലുകള്‍ ശേഖരിക്കാനാവില്ലെന്ന്? നാമവന്റെ വിരല്‍ക്കൊടികള്‍പോലും കൃത്യമായി നിര്‍മിക്കാന്‍ കഴിവുള്ളവനായിരിക്കെ എന്തുകൊണ്ടില്ല?''(75: 3,4)
10. സൂര്യന്‍ വിളക്കുപോലെ സ്വയം പ്രകാശിക്കുന്നതും ചന്ദ്രന്‍ സൂര്യകിരണം തട്ടി പ്രകാശം പ്രതിബിംബിക്കുന്നതുമാണെന്ന് ലോകം മനസ്സിലാക്കിയത് അടുത്ത കാലത്താണ്. ഖുര്‍ആന്‍ ഇക്കാര്യം അസന്ദിഗ്ധമായി സൂചിപ്പിച്ചിട്ടുണ്ട്. 'ഉപരിലോകത്ത് കോട്ടകളുണ്ടാക്കുകയും അതിലൊരു ദീപവും പ്രകാശിക്കുന്ന ചന്ദ്രനും സ്ഥാപിക്കുകയും ചെയ്തവനാരോ അവന്‍ മഹത്തായ അനുഗ്രഹമുടയവനത്രെ''(25:61).
ഈ ദീപം സൂര്യനാണെന്ന് ഖുര്‍ആന്‍ തന്നെ വ്യക്തമാക്കുന്നു: 'അവന്‍ പ്രകാശമായി ചന്ദ്രനെയും വിളക്കായി സൂര്യനെയും നിശ്ചയിച്ചു.''(71: 16)
11. മനുഷ്യജ•ത്തില്‍ പുരുഷബീജത്തിന് മാത്രമാണ് പങ്കെന്നായിരുന്നു പതിനെട്ടാം നൂറ്റാണ്ടുവരെ നിലനിന്നിരുന്ന ധാരണ. സ്ത്രീയുടെ ഗര്‍ഭാശയം കുഞ്ഞു വളരാനുള്ള ഇടം മാത്രമായാണ് ഗണിക്കപ്പെട്ടിരുന്നത്. സ്ത്രീയുടെ അണ്ഡത്തിന്റെ പങ്ക് തിരിച്ചറിഞ്ഞത് അതിനുശേഷം മാത്രമാണ്. ഖുര്‍ആന്‍ ജന്മത്തിലെ സ്ത്രീപുരുഷ പങ്കിനെ വ്യക്തമായി ഊന്നിപ്പറഞ്ഞിട്ടുണ്ട്: 'മനുഷ്യരേ, നിശ്ചയമായും നാം നിങ്ങളെ ഒരാണില്‍നിന്നും പെണ്ണില്‍നിന്നും സൃഷ്ടിച്ചിരിക്കുന്നു''(49:13). 'കൂടിച്ചേര്‍ന്നുണ്ടാവുന്ന ഒരു ബീജത്തില്‍നിന്ന് നാം നിശ്ചയമായും മനുഷ്യനെ സൃഷ്ടിച്ചിരിക്കുന്നു'' (76: 2).
12. കുഞ്ഞിന്റെ ലിംഗനിര്‍ണയം നിര്‍വഹിക്കുന്നത് പുരുഷബീജമാണെന്ന് വിശുദ്ധഖുര്‍ആന്‍ വ്യക്തമാക്കിയെങ്കിലും ശാസ്ത്രലോകമിത് തിരിച്ചറിഞ്ഞത് വളരെ വൈകിയാണ്. 'സ്രവിക്കപ്പെടുന്ന ഒരു ബീജത്തില്‍നിന്ന് ആണ്‍പെണ്‍ ഇണകളെ സൃഷ്ടിച്ചതും അവനാണ് ''(53: 45, 46).
ഒരു തുള്ളി ഇന്ദ്രിയത്തില്‍ അസംഖ്യം ബീജങ്ങളുണ്ടാവുമെങ്കിലും അവയിലൊന്നു മാത്രമാണ് ജനനത്തില്‍ പങ്കുചേരുന്നതെന്ന കാര്യവും ഖുര്‍ആനിവിടെ വ്യക്തമാക്കുന്നു. ജനിതകശാസ്ത്രം കണ്ടെത്തിയ നിരവധി വസ്തുതകള്‍ വിശുദ്ധ വേദഗ്രന്ഥത്തില്‍ രേഖപ്പെടുത്തപ്പെട്ടതായി കാണാം. അത് സവിസ്തരമായ പഠനമര്‍ഹിക്കുന്നതായതിനാല്‍ ഇവിടെ വിശദീകരിക്കുന്നില്ല.
13. 'രണ്ട് സമുദ്രങ്ങളെ കൂട്ടിച്ചേര്‍ത്തതും അവന്‍തന്നെ. ഒന്ന് രുചികരമായ തെളിനീര്‍. മറ്റേത് ചവര്‍പ്പുറ്റ ഉപ്പുനീരും. രണ്ടിനുമിടയില്‍ ഒരു മറയുണ്ട്. അവ കൂടിക്കലരുന്നതിനെ വിലക്കുന്ന ഒരു തടസ്സം''(25:53).
പതിനാറാം നൂറ്റാണ്ടില്‍ തുര്‍ക്കി അമീറുല്‍ ബഹ്‌റ് സയ്യിദ് അലി റഈസ് രചിച്ച മിര്‍ആത്തുല്‍ മമാലിക് എന്ന ഗ്രന്ഥത്തില്‍, പേര്‍ഷ്യന്‍ ഗള്‍ഫിന്റെ അടിത്തട്ടില്‍ ഇത്തരം ജലാശയങ്ങളുള്ളതായി രേഖപ്പെടുത്തിയിട്ടുണ്ട്. എങ്കിലും അത് കണ്ടെത്തിയതിനു തെളിവുണ്ടായിരുന്നില്ല. എന്നാല്‍ ഇപ്പോള്‍ ബഹ്‌റൈന്‍ തീരത്തുനിന്ന് മൂന്നര കിലോമീറ്റര്‍ ദൂരെ പേര്‍ഷ്യന്‍ ഗള്‍ഫില്‍ ഉമ്മുസുവാലിയില്‍ വമ്പിച്ച ശുദ്ധജലശേഖരം ഉപ്പുവെള്ളത്തില്‍ കലരാതെ കണ്ടെത്തിയിരിക്കുന്നു. അങ്ങനെ നൂറ്റാണ്ടുകള്‍ക്കപ്പുറം ഖുര്‍ആന്‍ അറിയിച്ച കാര്യം കണ്ടെത്താന്‍ മനുഷ്യസമൂഹത്തിന് സാധിക്കുകയുണ്ടായി.
14. നൂഹ്നബിയുടെ കപ്പല്‍ ജൂദി പര്‍വതത്തിലാണ് ചെന്നുതങ്ങിയതെന്ന് ഖുര്‍ആന്‍ പറയുന്നു (11:44). ഇവമൃഹല െആലൃഹശെേ നോഹയുടെ നഷ്ടപ്പെട്ട പേടകം (ഠവല ഘീേെ ടവശു ീള ചീമവ) എന്ന ഗ്രന്ഥത്തില്‍ 1883ല്‍ കിഴക്കന്‍ തുര്‍ക്കിയിലെ അറാറത്ത് പര്‍വതനിരകളിലെ ജൂദിമലയില്‍ 450 അടി നീളവും 150 അടി വീതിയും 50 അടി ഉയരവുമുള്ള കപ്പല്‍ കണ്ടെത്തിയതായി വ്യക്തമാക്കിയിട്ടുണ്ട്. പര്യവേക്ഷണവേളയില്‍ കണ്ടെടുക്കപ്പെട്ട ഈ കപ്പല്‍ നോഹാ പ്രവാചകന്റേതാണെന്ന് ഇന്ന് അംഗീകരിക്കപ്പെട്ടിരിക്കുന്നു.
നിരക്ഷരനായ ഒരാള്‍ക്കെന്നല്ല, ആറാം നൂറ്റാണ്ടില്‍ നിലവിലുണ്ടായിരുന്ന എല്ലാ വിജ്ഞാനങ്ങളും ആര്‍ജിച്ച മഹാപണ്ഡിതനുപോലും ഇത്തരം കാര്യങ്ങള്‍ കണ്ടെത്താന്‍ ആവില്ലെന്ന് സത്യസന്ധതയുടെ നേരിയ അംശമുള്ള ഏവരും അംഗീകരിക്കും. ദൈവികമെന്ന് സ്വയം അവകാശപ്പെട്ട ഒരു ഗ്രന്ഥം അന്നെന്നല്ല, തുടര്‍ന്നുള്ള നിരവധി നൂറ്റാണ്ടുകളിലും മുഴു ലോകത്തിനും അജ്ഞാതമായിരുന്ന ഇത്തരം കാര്യങ്ങള്‍ തുറന്നുപ്രഖ്യാപിക്കുവാന്‍ ധൈര്യംകാണിക്കുകയും ഒന്നൊഴിയാതെ അവയൊക്കെയും സത്യമാണെന്ന് സ്ഥാപിതമാകുകയും ചെയ്തതുതന്നെ ഖുര്‍ആന്‍ ദൈവികഗ്രന്ഥമാണെന്നതിന് അനിഷേധ്യമായ തെളിവാണ്. സ്വയം ദൈവികമെന്ന് അവകാശപ്പെടുന്ന ഒരു ഗ്രന്ഥം ശാസ്ത്രവസ്തുതകള്‍ അനാവരണം ചെയ്യാന്‍ ധൈര്യപ്പെട്ടുവെന്നതും പില്‍ക്കാലത്തെ മനുഷ്യധിഷണയുടെ കണ്ടെത്തലുകളിലൊന്നുപോലും അവയ്ക്ക് വിരുദ്ധമായില്ലെന്നതും ആലോചിക്കുന്ന ആരെയും വിസ്മയഭരിതരാക്കാതിരിക്കില്ല.
മനനം ചെയ്യുകവഴി ദര്‍ശനങ്ങള്‍ ഉരുത്തിരിഞ്ഞു വന്നേക്കാം. ശാസ്ത്രസത്യങ്ങള്‍ കണ്ടെത്താനും കഴിഞ്ഞേക്കാം. എന്നാല്‍ ചരിത്രവസ്തുതകള്‍ കേവല ചിന്തയിലൂടെ ഉരുത്തിരിച്ചെടുക്കുക സാധ്യമല്ല. നിരക്ഷരനായ നബിതിരുമേനിയിലൂടെ അവതീര്‍ണമായ ഖുര്‍ആന്‍ ഗതകാല സമൂഹങ്ങളുടെ ചരിത്രം വിശദമായി വിശകലനം ചെയ്യുന്നു. അവയിലൊന്നുപോലും വസ്തുനിഷ്ഠമല്ലെന്ന് സ്ഥാപിക്കാന്‍ ഇസ്‌ലാമിന്റെ വിമര്‍ശകര്‍ക്ക് ഇന്നോളം സാധിച്ചിട്ടില്ല. എന്നല്ല; അവയൊക്കെ തീര്‍ത്തും സത്യനിഷ്ഠമാണെന്ന് ലഭ്യമായ രേഖകളും പ്രമാണങ്ങളും തെളിയിക്കുകയും ചെയ്യുന്നു.
ദൈവിക ഗ്രന്ഥമെന്ന് സ്വയം പ്രഖ്യാപിക്കുന്ന ഖുര്‍ആന്‍ ഭാവിയെ സംബന്ധിച്ച് നിരവധി പ്രവചനങ്ങള്‍ നടത്താന്‍ ധൈര്യപ്പെട്ടുവെന്നതും അവയൊക്കെയും സത്യമായി പുലര്‍ന്നുവെന്നതും അതിന്റെ അമാനുഷികതക്ക് മതിയായ തെളിവാണ്.
ടോള്‍സ്‌റോയി, വിക്ടര്‍ യൂഗോ, മാക്‌സിം ഗോര്‍ക്കി, ഷേയ്ക്‌സ്പിയര്‍, ഗോയ്‌ഥെ, ഷെല്ലി, മില്‍ട്ടന്‍ തുടങ്ങി കാലം നിരവധി സാഹിത്യകാരന്മാരെ കാണുകയും, അവരുടെ രചനകളുമായി പരിചയപ്പെടുകയും ചെയ്തിട്ടുണ്ട്. എന്നാല്‍ നൂറ് വര്‍ഷം പിന്നിടുമ്പോഴേക്കും ലോകത്തിലെ ഏത് മഹദ്ഗ്രന്ഥത്തിലെയും പല പദങ്ങളും ശൈലികളും പ്രയോഗങ്ങളും കാലഹരണപ്പെടുകയും പ്രയോഗത്തിലില്ലാതാവുകയും ചെയ്യുന്നു.
യേശുവിന്റെ ഭാഷയായ അരാമിക്കില്‍ ലോകത്തെവിടെയും ഇന്ന് ബൈബിളില്ല. സുവിശേഷങ്ങള്‍ രചിക്കപ്പെട്ട ഭാഷയിലും ശൈലിയിലും അവ നിലനില്‍ക്കുന്നുമില്ല. ഉള്ളവ വിവര്‍ത്തനങ്ങളായതിനാല്‍ അവയുടെ ഭാഷയും ശൈലിയും നിരന്തരം മാറിക്കൊണ്ടിരിക്കുകയും ചെയ്യുന്നു. ഇന്ത്യയിലെ വേദഭാഷയും ഇന്ന് ജീവല്‍ഭാഷയോ പ്രയോഗത്തിലുള്ളതോ അല്ല.
എന്നാല്‍ പതിനാലു നൂറ്റാണ്ടു പിന്നിട്ടശേഷവും ഖുര്‍ആന്റെ ഭാഷയും ശൈലിയും പ്രയോഗങ്ങളും ഇന്നും അറബിയിലെ ഏറ്റം മികച്ചവയും അതുല്യവും അനനുകരണീയവുമായി നിലകൊള്ളുന്നു. അറബി ഭാഷ അറിയുന്ന ആരെയും അത് അത്യധികം ആകര്‍ഷിക്കുന്നു. ആര്‍ക്കും അതിന്റെ ആശയം അനായാസം മനസ്സിലാക്കാന്‍ സാധിക്കുകയും ചെയ്യുന്നു. ഇവ്വിധം നിത്യനൂതനമായ ഒരു ഗ്രന്ഥവും ലോകത്ത് എവിടെയും ഒരു ഭാഷയിലും കണ്ടെത്താനാവില്ല.
എന്നാല്‍ ഇത്തരം ഏതൊരു വിവരണത്തേക്കാളുമേറെ ഖുര്‍ആന്റെ ദൈവികത ബോധ്യമാവാന്‍ സഹായകമാവുക അതിന്റെ പഠനവും പാരായണവുമത്രെ.

\section{'ഖുര്‍ആന്‍ ദൈവികമെന്നതിന് മുഴുവന്‍ തെളിവുകളും ഖുര്‍ആനില്‍നിന്നുള്ളവയാണല്ലോ. ഇതെങ്ങനെയാണ് സ്വീകാര്യമാവുക?''}
സ്വര്‍ണവള സ്വര്‍ണനിര്‍മിതമാണെന്നതിനു തെളിവു ആ വളതന്നെയാണ്. മാവ് മാവാണെന്നതിനു തെളിവു ആ വൃക്ഷം തന്നെയാണല്ലോ. 'യുദ്ധവും സമാധാനവും' ടോള്‍സ്‌റോയിയുടേതാണെന്നതിന്നും 'വിശ്വചരിത്രാവലോകം' നെഹ്‌റുവിന്റെതാണെന്നതിന്നും പ്രസ്തുത ഗ്രന്ഥങ്ങളാണ് ഏറ്റവും പ്രബലവും സ്വീകാര്യവുമായ തെളിവ്. അവ്വിധം തന്നെ ഖുര്‍ആന്‍ ദൈവികമാണെന്നതിന്ന് ഏറ്റം ശക്തവും അനിഷേധ്യവുമായ തെളിവ് ആ ഗ്രന്ഥം തന്നെയാണ്.
\chapter{ഖുര്‍ആന്‍ ബൈബിളിന്റെ അനുകരണമോ? }
 \section{ മുഹമ്മദ് തന്റെ കാലത്തെ യഹൂദെ്രെകസ്തവ പണ്ഡിതന്മാരുമായി ബന്ധപ്പെട്ട് പഠിച്ച കാര്യങ്ങള്‍ സ്വന്തം ഭാഷയിലും ശൈലിയിലും അവതരിപ്പിക്കുകയാണുണ്ടായത്. പൂര്‍വസമൂഹങ്ങളെ സംബന്ധിച്ച ഖുര്‍ആന്റെയും ബൈബിളിന്റെയും വിവരണം ഒരേ വിധമാവാന്‍ കാരണം അതാണ് ഒന്നിലേറെ ഇംഗ്‌ളീഷ് പുസ്തകങ്ങളില്‍ ഈ വിധമുള്ള പരാമര്‍ശം വായിക്കാനിടയായി. ഇതിനെക്കുറിച്ച് എന്തു പറയുന്നു?}
ഇസ്‌ലാമിനോട് കൊടിയ ശത്രുത വച്ചുപുലര്‍ത്തുന്ന പാശ്ചാത്യന്‍ എഴുത്തുകാര്‍ വ്യാപകമായി പ്രചരിപ്പിച്ച തീര്‍ത്തും വ്യാജമായ ആരോപണമാണിത്. ഈ ആരോപണത്തിന് സത്യവുമായി വിദൂരബന്ധം പോലുമില്ലെന്ന് ഖുര്‍ആനും ബൈബിളും ഒരാവൃത്തി വായിക്കുന്ന ഏവര്‍ക്കും വളരെ വേഗം ബോധ്യമാകും.
മാനവരാശിക്ക് ദൈവിക ജീവിതവ്യവസ്ഥ സമര്‍പ്പിക്കാന്‍ നിയുക്തരായ സന്ദേശവാഹകരാണ് പ്രവാചകന്മാര്‍. അതിനാല്‍ അവരിലൂടെ സമര്‍പിതമായ ദൈവികസന്മാര്‍ഗത്തില്‍ ഏകത ദൃശ്യമാവുക സ്വാഭാവികമത്രെ. ദൈവദൂതന്മാരുടെ അധ്യാപനങ്ങളില്‍നിന്ന് അനുയായികള്‍ വ്യതിചലിച്ചില്ലായിരുന്നുവെങ്കില്‍ മതങ്ങള്‍ക്കിടയില്‍ വൈവിധ്യമോ വൈരുധ്യമോ ഉണ്ടാവുമായിരുന്നില്ല. എന്നല്ല; ദൈവദൂതന്മാരുടെ അടിക്കടിയുള്ള നിയോഗം സംഭവിച്ചതുതന്നെ മുന്‍ഗാമികളുടെ മാര്‍ഗത്തില്‍നിന്ന് അവരുടെ അനുയായികള്‍ വ്യതിചലിച്ചതിനാലാണ്.
മുഹമ്മദ് നബി നിയോഗിതനായ കാലത്ത് മോശയുടെയോ യേശുവിന്റെയോ സന്ദേശങ്ങളും അധ്യാപനങ്ങളും തനതായ സ്വഭാവത്തില്‍ നിലവിലുണ്ടായിരുന്നില്ല. ജൂതെ്രെകസ്തവ സമൂഹങ്ങള്‍ അവയില്‍ ഗുരുതരമായ കൃത്രിമങ്ങളും വെട്ടിച്ചുരുക്കലുകളും കൂട്ടിച്ചേര്‍ക്കലുകളും നടത്തിയിരുന്നു. അതിനാല്‍ ആ പ്രവാചകന്മാര്‍ പ്രബോധനംചെയ്ത കാര്യങ്ങളില്‍ ചെറിയ ഒരംശം മാത്രമാണ് ബൈബിളിലുണ്ടായിരുന്നത്. അവയുമായി മുഹമ്മദ് നബിയിലൂടെ അവതീര്‍ണമായ വിവരണങ്ങള്‍ ഒത്തുവരിക സ്വാഭാവികമാണ്. അങ്ങനെ സംഭവിച്ചിട്ടുമുണ്ട്. ബൈബിളിലും ഖുര്‍ആനിലും കാണപ്പെടുന്ന സാദൃശ്യം അതത്രെ.
മതത്തിന്റെ അടിസ്ഥാന സിദ്ധാന്തമായ ദൈവവിശ്വാസത്തില്‍തന്നെ ജൂതെ്രെകസ്തവ വീക്ഷണവും മുഹമ്മദ് നബിയുടെ പ്രബോധനവും തമ്മില്‍ പ്രകടമായ അന്തരവും വൈരുധ്യവും കാണാം. മുഹമ്മദ് നബി കണിശമായ ഏകദൈവസിദ്ധാന്തമാണ് സമൂഹസമക്ഷം സമര്‍പ്പിച്ചത്. എന്നാല്‍ ജൂതെ്രെകസ്തവ സമൂഹങ്ങള്‍ ഇന്നത്തെപ്പോലെ അന്നും വികലമായ ദൈവവിശ്വാസമാണ് വെച്ചുപുലര്‍ത്തിയിരുന്നത്. നബിതിരുമേനി ഇത് അംഗീകരിച്ച് അനുകരിച്ചില്ലെന്ന് മാത്രമല്ല, അദ്ദേഹത്തിലൂടെ അവതീര്‍ണമായ ഖുര്‍ആന്‍ അതിനെ നിശിതമായി എതിര്‍ക്കുക കൂടി ചെയ്തു. 'യഹൂദന്മാര്‍ പറയുന്നു: 'ഉസൈര്‍ ദൈവപുത്രനാകുന്നു.' െ്രെകസ്തവര്‍ പറയുന്നു: 'മിശിഹാ ദൈവപുത്രനാകുന്നു'. ഇതെല്ലാം അവര്‍ വായകൊണ്ട് പറയുന്ന നിരര്‍ഥകമായ ജല്‍പനങ്ങളത്രെ. അവര്‍, തങ്ങള്‍ക്കുമുമ്പ് സത്യനിഷേധത്തിലകപ്പെട്ടവരുടെ വാദത്തോട് സാദൃശ്യംവഹിക്കുന്നു'' (9:30).
'അല്ലാഹു പുത്രന്മാരെ സ്വീകരിച്ചിരിക്കുന്നുവെന്ന് വാദിക്കുന്നവരെ താക്കീത് ചെയ്യാനുമാണ് ഈ ഗ്രന്ഥം അവതീര്‍ണമായത്. അവര്‍ക്ക് ഇക്കാര്യത്തെക്കുറിച്ച് യാതൊരറിവുമില്ല. അവരുടെ പൂര്‍വികര്‍ക്കും ഉണ്ടായിരുന്നില്ല. അവരുടെ വായകളില്‍നിന്ന് വമിക്കുന്നത് ഗുരുതരമായ വാക്കുതന്നെ. വെറും കള്ളമാണവര്‍ പറയുന്നത്''(18: 4, 5).
'അല്ലാഹു മൂവരില്‍ ഒരുവനാകുന്നു എന്നു വാദിച്ചവര്‍ തീര്‍ച്ചയായും സത്യനിഷേധികളായിരിക്കുന്നു. ഏകദൈവമല്ലാതെ വേറെ ദൈവമേയില്ല. അവര്‍ തങ്ങളുടെ ഇത്തരം വാദങ്ങളില്‍നിന്ന് വിരമിച്ചില്ലെങ്കില്‍ അവരിലെ നിഷേധികളെ വേദനാനിരതമായ ശിക്ഷ ബാധിക്കുകതന്നെ ചെയ്യും. ഇനിയും അവര്‍ പശ്ചാത്തപിക്കുകയും ദൈവത്തോട് മാപ്പപേക്ഷിക്കുകയും ചെയ്യുന്നില്ലേ? അല്ലാഹു ഏറെ പൊറുക്കുന്നവനും മാപ്പ് നല്‍കുന്നവനുമല്ലോ''( 5: 73, 74).
ഇന്നത്തെ ജൂതെ്രെകസ്തവ സമൂഹത്തെപ്പോലെ അന്നത്തെ യഹൂദരും െ്രെകസ്തവരും യേശുവിന്റെ കുരിശുമരണത്തില്‍ വിശ്വസിക്കുന്നവരായിരുന്നു. എന്നാല്‍ ഖുര്‍ആന്‍ ഇതിനെ ശക്തമായി നിഷേധിക്കുന്നു. 'യഹൂദര്‍ പറഞ്ഞു: 'മസീഹ് ഈസബ്‌നു മര്‍യമിനെ ദൈവദൂതനെ ഞങ്ങള്‍ കൊന്നുകളഞ്ഞിരിക്കുന്നു.' സത്യത്തിലോ, അവരദ്ദേഹത്തെ വധിച്ചിട്ടില്ല, ക്രൂശിച്ചിട്ടുമില്ല. പിന്നെയോ, സംഭവം അവര്‍ക്ക് അവ്യക്തമാവുകയാണുണ്ടായത്. അദ്ദേഹത്തെ സംബന്ധിച്ച് ഭിന്നാഭിപ്രായമുള്ളവരും സംശയഗ്രസ്തര്‍ തന്നെ. അവര്‍ക്ക് അതിനെക്കുറിച്ചൊരറിവുമില്ല; ഊഹത്തെ പിന്‍പറ്റുന്നതല്ലാതെ. അവരദ്ദേഹത്തെ ഉറപ്പായും കൊന്നിട്ടില്ല''(4: 157).
ദൈവത്തെയും ദൈവദൂതന്മാരെയും സംബന്ധിച്ച് അബദ്ധജടിലമായ അനേകം പ്രസ്താവനകള്‍ ബൈബിളിലുണ്ട്. അവയൊന്നും ഖുര്‍ആനിലില്ലെന്നു മാത്രമല്ല; അവയുടെ സത്യസന്ധവും കൃത്യവും വസ്തുനിഷ്ഠവുമായ വിവരണം നല്‍കുന്നുമുണ്ട്. ഉദാഹരണത്തിന് ചിലതു മാത്രമിവിടെ പറയാം:
'വെയിലാറിയപ്പോള്‍ തോട്ടത്തിലൂടെ കര്‍ത്താവായ ദൈവം നടക്കുന്ന ശബ്ദം അവര്‍ കേട്ടു. ദൈവസന്നിധിയില്‍ നിന്നകന്ന് മനുഷ്യനും ഭാര്യയും തോട്ടത്തിലെ വൃക്ഷങ്ങള്‍ക്കിടയില്‍ പോയി ഒളിച്ചു'' (ഉല്‍പത്തി 3:8,9).
'അനന്തരം കര്‍ത്താവായ ദൈവം അരുള്‍ചെയ്തു: നോക്കുക, മനുഷ്യന്‍ നന്മതിന്മകള്‍ അറിഞ്ഞ് നമ്മില്‍ ഒരുവനെപ്പോലെ ആയിത്തീര്‍ന്നിരിക്കുന്നു. ഇനി ഇപ്പോള്‍ അവര്‍ കൈനീട്ടി ജീവവൃക്ഷത്തിന്റെ കനികൂടി പറിച്ച് തിന്ന് എന്നെന്നും ജീവിക്കാന്‍ ഇടവരരുത്''(ഉല്‍പത്തി 3:22).
ഇസ്ലാമിലെ ദൈവസങ്കല്‍പത്തിനും വിശ്വാസത്തിനും കടകവിരുദ്ധമാണ് ഈ പ്രസ്താവങ്ങള്‍. ദൈവദൂതന്മാരെ സംബന്ധിച്ച് ബൈബിളിലുള്ള പലതും അവിശ്വസനീയങ്ങളാണെന്നു മാത്രമല്ല; അവരെ അത്യന്തം അവഹേളിക്കുന്നവയും കൊടുംകുറ്റവാളികളായി ചിത്രീകരിക്കുന്നവയുമാണ്. നോഹയെക്കുറിച്ച് പറയുന്നു: 'നോഹ് വീഞ്ഞ് കുടിച്ച് ലഹരി ബാധിച്ച് നഗ്‌നനായി കൂടാരത്തില്‍ കിടന്നു. പിതാവിന്റെ നഗ്‌നത കണ്ടിട്ട് കാനാനിന്റെ പിതാവായ ഹാം വെളിയില്‍ ചെന്ന് മറ്റു രണ്ട് സഹോദരന്മാരോട് വിവരം പറഞ്ഞു. ശേമും യാഫെതും കൂടി ഒരു വസ്ത്രം എടുത്ത് ഇരുവരുടെയും തോളുകളിലായി ഇട്ട്, പിറകോട്ട് നടന്നുചെന്ന് പിതാവിന്റെ നഗ്‌നത മറച്ചു. മുഖം തിരിച്ചു നടന്നതിനാല്‍ അവര്‍ പിതാവിന്റെ നഗ്‌നത കണ്ടില്ല. മദ്യലഹരി വിട്ടുണര്‍ന്ന് തന്റെ ഇളയ പുത്രന്റെ പ്രവൃത്തിയെക്കുറിച്ച് അറിഞ്ഞപ്പോള്‍ നോഹ് പറഞ്ഞു: കാനാന്‍ ശപിക്കപ്പെട്ടവന്‍. അയാള്‍ സ്വന്തം സഹോദരന്മാര്‍ക്ക് അടിമകളില്‍ അടിമയായിരിക്കും. കര്‍ത്താവ് ശേമിനെ അനുഗ്രഹിക്കട്ടെ''(ഉല്‍പത്തി 9: 2126).
മദ്യപിച്ച് ലഹരിക്കടിപ്പെട്ട് നഗ്‌നനാവുകയും ഒരു കുറ്റവുമില്ലാതെ പേരക്കുട്ടിയെ ശപിക്കുകയും ചെയ്ത നോഹ് ഖുര്‍ആന്‍ പരിചയപ്പെടുത്തുന്ന പരമപരിശുദ്ധനായ നൂഹ് നബിയില്‍നിന്നെത്രയോ വ്യത്യസ്തനത്രെ.
പ്രവാചകനായ അബ്രഹാമിനെപ്പറ്റി ബൈബിള്‍ പറയുന്നു: 'ക്ഷാമം രൂക്ഷമായതിനാല്‍ അബ്രാം പ്രവസിക്കുന്നതിനായി ഈജിപ്തിലേക്ക് പുറപ്പെട്ടു. ഈജിപ്തില്‍ പ്രവേശിക്കുന്നതിനു മുമ്പ് അബ്രാം ഭാര്യ സാറായോട് പറഞ്ഞു: 'നീ സുന്ദരിയാണെന്ന് എനിക്കറിയാം. ഈജിപ്തുകാര്‍ നിന്നെ കാണുമ്പോള്‍ 'ഇവള്‍ ഇയാളുടെ ഭാര്യയാണ്' എന്നു പറഞ്ഞ് എന്നെ കൊല്ലുകയും നിന്നെ ജീവനോടെ വിടുകയും ചെയ്യും. നീ നിമിത്തം എനിക്കു നന്മ വരാന്‍ നീ എന്റെ സഹോദരിയാണെന്ന് പറയുക. നീ നിമിത്തം എന്റെ ജീവന്‍ രക്ഷിക്കുകയും ചെയ്യും.' അബ്രാം ഈജിപ്തില്‍ പ്രവേശിച്ചപ്പോള്‍ സ്ത്രീ അത്യന്തം സുന്ദരിയാണെന്ന് ഈജിപ്തുകാര്‍ കണ്ടു. അവളെ കണ്ട ഫറോവന്റെ പ്രഭുക്കന്മാര്‍ ഫറോവന്റെ മുമ്പില്‍ അവളെപറ്റി പ്രശംസിച്ചു സംസാരിച്ചു. സ്ത്രീയെ ഫറോവന്റെ അരമനയിലേക്ക് കൂട്ടിക്കൊണ്ടുപോയി. അവള്‍ നിമിത്തം ഫറോവന്‍ അബ്രാമിനോട് ദയാപൂര്‍വം പെരുമാറി. അയാള്‍ക്ക് ആടുമാടുകളെയും ഭൃത്യന്മാരെയും ഭൃത്യകളെയും പെണ്‍കഴുതകളെയും ഒട്ടകങ്ങളെയും നല്‍കി''(ഉല്‍പത്തി 12: 1016).
സ്വന്തം സഹധര്‍മിണിയെ ഭരണാധികാരിക്ക് വിട്ടുകൊടുത്ത് സമ്മാനം സ്വീകരിക്കുന്ന നീചരില്‍ നീചനായ ബൈബിളിലെ അബ്രാമും ഖുര്‍ആന്‍ പരിചയപ്പെടുത്തുന്ന ആദര്‍ശശാലിയും ത്യാഗസന്നദ്ധനും ധീരനും വിപ്‌ളവകാരിയുമായ ഇബ്‌റാഹീം നബിയും തമ്മില്‍ ഒരു താരതമ്യം പോലും സാധ്യമല്ല.
പ്രവാചകനായ ലോത്തിനെക്കുറിച്ച് ബൈബിള്‍ പറയുന്നു: 'സോവറില്‍ പാര്‍ക്കാന്‍ ലോത്ത് ഭയപ്പെട്ടു. അതുകൊണ്ട് അയാള്‍ രണ്ടു പുത്രിമാരെയും കൂട്ടി സോവര്‍ നഗരത്തില്‍നിന്ന് പോയി മലയില്‍ താമസിച്ചു. അവിടെ ഒരു ഗുഹയില്‍ അവര്‍ പുത്രിമാരോടൊത്ത് പാര്‍ത്തു. മൂത്ത പുത്രി ഇളയവളോട് പറഞ്ഞു: 'നമ്മുടെ പിതാവ് വൃദ്ധനായിരിക്കുന്നു. ഭൂമിയിലെ നടപ്പനുസരിച്ച് നമ്മോട് ഇണ ചേരാന്‍ ഭൂമിയില്‍ ഒരു മനുഷ്യനുമില്ല. വാ, നമുക്ക് പിതാവിനെ വീഞ്ഞ് കുടിപ്പിക്കാം. പിതാവിനോടൊപ്പം ശയിച്ച് പിതാവില്‍നിന്ന് സന്തതികളെ നേടാം!' അന്നു രാത്രി അവര്‍ പിതാവിനെ വീഞ്ഞു കുടിപ്പിച്ചു. മൂത്തപുത്രി അകത്തുചെന്ന് പിതാവിനോടൊപ്പം ശയിച്ചു. അവള്‍ എപ്പോള്‍ വന്നു ശയിച്ചെന്നോ എപ്പോള്‍ എഴുന്നേറ്റു പോയെന്നോ ഒന്നും അയാള്‍ അറിഞ്ഞില്ല. അടുത്ത ദിവസം മൂത്ത മകള്‍ ഇളയവളോടു പറഞ്ഞു: 'ഇന്നലെ ഞാന്‍ നമ്മുടെ പിതാവിനോടൊപ്പം ശയിച്ചു. ഇന്നു രാത്രിയും നമുക്ക് പിതാവിനെ വീഞ്ഞ് കുടിപ്പിക്കാം. അനന്തരം നീ അകത്തുപോയി പിതാവിനോടൊപ്പം ശയിച്ച് നമ്മുടെ പിതാവിലൂടെ നമുക്ക് സന്തതികളെ നേടിയെടുക്കാം.' അന്നു രാത്രിയും അവര്‍ പിതാവിനെ വീഞ്ഞ് കുടിപ്പിച്ചു. ഇളയപുത്രി എഴുന്നേറ്റു ചെന്ന് അയാളുടെ കൂടെ ശയിച്ചു. അവള്‍ എപ്പോള്‍ വന്നു ശയിച്ചുവെന്നോ എപ്പോള്‍ എഴുന്നേറ്റുപോയെന്നോ ഒന്നും അയാള്‍ അറിഞ്ഞില്ല. അങ്ങനെ ലോത്തിന്റെ രണ്ടു പുത്രിമാരും പിതാവിനാല്‍ ഗര്‍ഭവതികളായി. മൂത്തവള്‍ ഒരു പുത്രനെ പ്രസവിച്ചു. അവന് മോവാബ് എന്നു പേരിട്ടു. അയാളാണ് ഇന്നോളമുള്ള മോവാബിയരുടെ പിതാവ്. ഇളയവളും ഒരു പുത്രനെ പ്രസവിച്ചു. അവന്ന് ബെന്‍അമ്മീ എന്ന് പേരിട്ടു. അയാളാണ് ഇന്നോളമുള്ള അമ്മേനിയരുടെ പിതാവ്''(ഉല്‍പത്തി 19: 3038).
അത്യന്തം ഹീനവും നീചവും മ്‌ളേഛവുമായ വൃത്തിയിലേര്‍പ്പെട്ടതായി ബൈബിള്‍ പരിചയപ്പെടുത്തിയ ലോത്തും ജീവിതവിശുദ്ധിക്കും ലൈംഗിക സദാചാരത്തിനും ജീവിതകാലം മുഴുവന്‍ നിലകൊണ്ട പരിശുദ്ധിയുടെ പ്രതീകമായ ലൂത്വ് നബിയും ഒരിക്കലും സമമാവുകയില്ല. അരാജകവാദികളായ കാമവെറിയന്മാര്‍ മെനഞ്ഞുണ്ടാക്കിയ കള്ളക്കഥകള്‍ പരിശുദ്ധരായ പ്രവാചകന്മാരുടെമേല്‍ വച്ചുകെട്ടുകയായിരുന്നു ബൈബിള്‍. ഇത്തരം എല്ലാവിധ അപഭ്രംശങ്ങളില്‍നിന്നും തീര്‍ത്തും മോചിതമാണ് വിശുദ്ധ ഖുര്‍ആന്‍.
പ്രവാചകത്വത്തിന്റെ യഥാര്‍ഥ അവകാശിയായിരുന്ന ഏശാവില്‍നിന്ന് അപ്പവും പയര്‍പായസവും നല്‍കി അതു വാങ്ങുകയായിരുന്നു ബൈബിളിന്റെ ഭാഷയില്‍ യാക്കോബ്: 'ഒരിക്കല്‍ യാക്കോബ് പായസം ഉണ്ടാക്കിക്കൊണ്ടിരിക്കുമ്പോള്‍ ഏശാവ് വെളിയില്‍നിന്ന് വിശന്നുവലഞ്ഞു കയറിവന്നു. 'ആ ചെമന്ന പായസത്തില്‍ ഒരു ഭാഗം എനിക്ക് തരിക. ഞാന്‍ വിശന്നുവലയുന്നു' എന്ന് ഏശാവ് യാക്കോബിനോട് പറഞ്ഞു. 'ആദ്യം നിന്റെ ജന്മാവകാശം എനിക്കു വില്‍ക്കുക' എന്ന് യാക്കോബ് പറഞ്ഞു. ഏശാവ് മറുപടി പറഞ്ഞു: 'മരിക്കാറായിരിക്കുന്ന എനിക്ക് ജന്മാവകാശം കൊണ്ട് എന്തു പ്രയോജനം?' യാക്കോബ് പറഞ്ഞു: 'ആദ്യംതന്നെ എന്നോട് പ്രതിജ്ഞ ചെയ്യുക.' ഏശാവ് അപ്രകാരം പ്രതിജ്ഞ ചെയ്തു. ജന്മാവകാശം യാക്കോബിനു വിറ്റു. തുടര്‍ന്ന് യാക്കോബ് ഏശാവിന് അപ്പവും പയര്‍പായസവും കൊടുത്തു.''(ഉല്‍പത്തി 25: 2934).
സ്വന്തം ജ്യേഷ്ഠന്‍ വിശന്നുവലഞ്ഞപ്പോള്‍ അതിനെ ചൂഷണംചെയ്ത് അയാളുടെ ജന്മാവകാശം തട്ടിയെടുത്ത ക്രൂരനാണ് ബൈബിളിലെ യാക്കോബ്. എന്നാല്‍ ഖുര്‍ആനിലെ യഅ്ഖൂബ്‌നബി വിശുദ്ധനും ക്ഷമാശീലനും പരമ മര്യാദക്കാരനുമത്രെ. ബൈബിള്‍ വിവരണമനുസരിച്ച് യാക്കോബിന്റെ പിതാവ് ഇസ്ഹാഖ് കള്ളം പറഞ്ഞവനാണ്. 'ഇസ്ഹാഖ് ഗറാറില്‍ താമസിച്ചു. അവിടത്തെ നിവാസികള്‍ അയാളുടെ ഭാര്യയെപ്പറ്റി അന്വേഷിച്ചപ്പോള്‍ അവള്‍ എന്റെ സഹോദരിയാണ് എന്ന് അയാള്‍ പറഞ്ഞു. കാരണം, അവള്‍ എന്റെ ഭാര്യയാണ് എന്നു പറയാന്‍ ഇസ്ഹാഖ് ഭയപ്പെട്ടു. 'റിബെക്ക സുന്ദരിയാകയാല്‍ അവള്‍ക്കു വേണ്ടി സ്ഥലവാസികള്‍ എന്നെ കൊലപ്പെടുത്തിയേക്കും' എന്ന് അയാള്‍ ചിന്തിച്ചു''.(ഉല്‍പത്തി 26: 67).
തികഞ്ഞ വഞ്ചനയും ചതിയും ചെയ്താണ് യാക്കോബ് പിതാവിന്റെ അനുഗ്രഹവും പ്രാര്‍ഥനയും സമ്പാദിച്ചത്. ജ്യേഷ്ഠ സഹോദരന്‍ ഏശാവിന്റെ അവകാശം അന്യായമായി തട്ടിയെടുക്കുകയായിരുന്നു അയാള്‍ (ഉല്‍പത്തി 27: 138).
ബൈബിള്‍ വിവരണമനുസരിച്ച് യാക്കോബിന്റെ ഭാര്യാപിതാവ് ലാബാന്‍ കൊടിയ ചതിയനും ഭാര്യ റാഫേല്‍ വിഗ്രഹാരാധകയുമാണ് (ഉല്‍പത്തി 29: 2530, 31: 1723). പ്രവാചകനായ യാക്കോബിന്റെ പുത്രി വ്യഭിചരിക്കപ്പെട്ടതായും യഹൂദാ മകന്റെ ഭാര്യയെ വ്യഭിചരിച്ചതായും ബൈബിള്‍ പറയുന്നു (ഉല്‍പത്തി 38: 1330).
ദൈവദൂതനായ ദാവീദ് തന്റെ രാജ്യത്തെ പട്ടാളക്കാരനായ ഊറിയായുടെ ഭാര്യ ബത്‌ശേബയെ വ്യഭിചരിച്ചതായും അവളെ ഭാര്യയാക്കാനായി ഊറിയയെ യുദ്ധമുന്നണിയിലേക്കയച്ച് കൊല്ലിച്ചതായും ബൈബിള്‍ പറയുന്നു (ശാമുവേല്‍ 11: 116). പ്രവാചകനായ സോളമന്‍ ദൈവശാസന ധിക്കരിച്ച് വിലക്കപ്പെട്ടവരെ വിവാഹം കഴിച്ചതായും ബൈബിളില്‍ കാണാം (രാജാക്കന്മാര്‍ 11: 114).
പൂര്‍വപ്രവാചകന്മാരുടെ പില്‍ക്കാല ശിഷ്യന്മാര്‍ പ്രവാചകാധ്യാപനങ്ങളില്‍നിന്ന് വ്യതിചലിച്ച് പൈശാചിക ദുര്‍ബോധനങ്ങള്‍ക്ക് വശംവദരായപ്പോള്‍ സ്വന്തം അധര്‍മങ്ങളും സദാചാരരാഹിത്യങ്ങളും പ്രവാചകന്മാരിലും ആരോപിക്കുകയും അത് വേദപുസ്തകങ്ങളില്‍ കൂട്ടിച്ചേര്‍ക്കുകയുമായിരുന്നു. ഇങ്ങനെ കൈകടത്തപ്പെട്ട പൂര്‍വവേദങ്ങളുടെ അനുകരണമാണ് വിശുദ്ധ ഖുര്‍ആനെന്ന് അതിനെ സംബന്ധിച്ച നേരിയ അറിവെങ്കിലുമുള്ള ആരും ആരോപിക്കുകയില്ല. ദൈവദൂതന്മാരെപ്പറ്റി പ്രചരിപ്പിക്കപ്പെട്ട കള്ളക്കഥകള്‍ തിരുത്തി അവരുടെ യഥാര്‍ഥ അവസ്ഥ അനാവരണം ചെയ്യുകയാണ് ഖുര്‍ആന്‍. അതുകൊണ്ടുതന്നെ ബൈബിള്‍ പരിചയപ്പെടുത്തുന്ന പ്രവാചകന്മാരില്‍ പലരും ചതിയന്മാരും തെമ്മാടികളും കൊടും കുറ്റവാളികളും ക്രൂരന്മാരുമാണെങ്കില്‍, ഖുര്‍ആനിലവര്‍ എക്കാലത്തും ഏവര്‍ക്കും മാതൃകാ യോഗ്യമായ സദ്ഗുണങ്ങളുടെ ഉടമകളായ മഹദ് വ്യക്തികളാണ്; മനുഷ്യരാശിയുടെ മഹാന്മാരായ മാര്‍ഗദര്‍ശകരും.
\chapter{വേദവും ഗീതയും ദൈവികമോ? }
  \section{ഇന്ത്യയിലെ ഹൈന്ദവവേദങ്ങള്‍ ദൈവികമാണെന്ന് അംഗീകരിക്കുന്നുണ്ടോ?}
 വിദ്യ, വിജ്ഞാനം എന്നൊക്കെയാണ് വേദമെന്ന പദത്തിന്റെ അര്‍ഥം. അധ്യാത്മജ്ഞാനമെന്നാണ് അതിന്റെ വിവക്ഷ. വേദങ്ങള്‍ അപൗരുഷേയങ്ങളാണെന്ന് വിശ്വസിക്കപ്പെടുന്നു. അത് മനുഷ്യനിര്‍മിതമല്ലെന്നും ദൈവപ്രോക്തമാണെന്നും ചില വേദപണ്ഡിതന്മാര്‍ അവകാശപ്പെടാറുണ്ട്. എന്നാല്‍ വേദങ്ങള്‍ സ്വയം അത്തരമൊരവകാശവാദമുന്നയിക്കുന്നില്ല.
ചരിത്രത്തിലെ എല്ലാ ജനസമൂഹങ്ങളിലേക്കും ദൈവദൂതന്മാര്‍ നിയോഗിതരായിട്ടുണ്ടെന്നും ദൈവികസന്ദേശം നല്‍കപ്പെട്ടിട്ടുണ്ടെന്നും ഇസ്‌ലാം സിദ്ധാന്തിക്കുന്നു. അതിനാല്‍ വേദങ്ങള്‍ ആര്യന്മാര്‍ക്ക് അവതീര്‍ണമായ ദിവ്യസന്ദേശങ്ങളുടെ ഭാഗമാവാനുള്ള സാധ്യത നിരാകരിക്കാനോ നിഷേധിക്കാനോ ന്യായമില്ല. എന്നല്ല, അത്തരമൊരു സാധ്യത തീര്‍ച്ചയായുമുണ്ട്. എന്നാല്‍ ഇന്ന് നിലവിലുള്ള ഋക്ക്, യജുസ്സ്, സാമം, അഥര്‍വം എന്നീ നാലു വേദങ്ങള്‍ ദൈവികമാണെന്ന് കരുതാവതല്ല. അവ മനുഷ്യ ഇടപെടലുകള്‍ക്ക് നിരന്തരം വിധേയമായതിനാല്‍ മൗലികത നഷ്ടമാവുകയും അപൗരുഷേയങ്ങളല്ലാതാവുകയും ചെയ്തിട്ടുണ്ട്. ഇക്കാര്യം വേദവിശ്വാസികള്‍ തന്നെ അസന്ദിഗ്ധമായി വ്യക്തമാക്കിയിട്ടുണ്ട്. ഡോക്ടര്‍ രാധാകൃഷ്ണന്‍ എഴുതുന്നു:
''ആര്യന്മാര്‍ അവരുടെ ആദിമ വാസസ്ഥലത്തുനിന്ന് തങ്ങളുടെ ഏറ്റവും വിലപിടിച്ച സമ്പത്തെന്ന നിലയില്‍ ഇന്ത്യയിലേക്ക് കൊണ്ടുവന്ന ദിവ്യപ്രചോദനമുള്‍ക്കൊണ്ട ഗാനങ്ങള്‍ ശേഖരിച്ചു സംരക്ഷിക്കേണ്ടതിന്റെ ആവശ്യം പുതിയ രാജ്യത്ത് ഇതര ദേവന്മാരെ ആരാധിക്കുന്ന വളരെയധികം ജനങ്ങളുമായി അവര്‍ക്ക് ഇടപെടേണ്ടിവന്നപ്പോള്‍ നേരിട്ടതിനാല്‍ അവ സമാഹരിക്കപ്പെട്ടതായി പൊതുവെ വിശ്വസിച്ചുവരുന്നു...
''വളരെക്കാലത്തോളം അഥര്‍വവേദത്തിന് ഒരു വേദത്തിന്റെ അന്തസ്സ് കൈവന്നിരുന്നില്ല. എങ്കിലും നമ്മുടെ ഇപ്പോഴത്തെ ഉദ്ദേശ്യത്തെ സംബന്ധിച്ചേടത്തോളം ഋഗ്വേദം പോലെത്തന്നെ അഥര്‍വവേദവും സ്വതന്ത്രങ്ങളായ ഉള്ളടക്കങ്ങളുടെ ചരിത്രപരമായ സമാഹാരമാകുന്നു. ചിന്തയുടെ ഒരു പില്‍ക്കാലയുഗത്തില്‍ നിര്‍മിക്കപ്പെട്ട ഈ വേദത്തില്‍ വ്യാപിച്ചു നില്‍ക്കുന്നത് വിഭിന്നമായ ഒരു മനോഭാവമത്രെ. തങ്ങള്‍ പതുക്കെപ്പതുക്കെ കീഴടക്കിക്കൊണ്ടിരുന്ന രാജ്യത്ത് മുമ്പുണ്ടായിരുന്ന ജനങ്ങള്‍ ആരാധിച്ചിരുന്ന ദേവന്മാരുടെയും ഭൂതപ്രേതാദികളുടെയും നേര്‍ക്ക് വൈദികാര്യന്മാര്‍ കൈക്കൊണ്ട ഒത്തുതീര്‍പ്പ് മനോഭാവത്തിന്റെ ഫലങ്ങളെ അതു കാണിക്കുന്നു''(ഭാരതീയ ദര്‍ശനം, മാതൃഭൂമി പ്രിന്റിംഗ് ആന്റ് പബ്ലിഷിംഗ് കമ്പനി ലിമിറ്റഡ്, കോഴിക്കോട്. പുറം 44, 45).
വേദങ്ങളിലെ ദൈവസങ്കല്‍പത്തില്‍ പോലും ഗുരുതരമായ മാറ്റം സംഭവിച്ചതായി ഡോ. രാധാകൃഷ്ണന്‍ വിലയിരുത്തുന്നു.
1017 സൂക്തങ്ങളും 10472 ഋക്കുകളും 8 അഷ്ടകങ്ങളും 10 മണ്ഡലങ്ങളുമുള്ള ഋഗ്വേദത്തിലെ പത്താം മണ്ഡലത്തെക്കുറിച്ച് രാധാകൃഷ്ണന്‍ പറയുന്നു: ''പത്താം മണ്ഡലം പില്‍ക്കാലത്ത് കൂട്ടിച്ചേര്‍ത്തതാണെന്ന് തോന്നുന്നു. വേദസൂക്തങ്ങളുടെ വികാസത്തിന്റെ അവസാനഘട്ടത്തില്‍ പ്രചാരത്തിലിരുന്ന വീക്ഷണങ്ങളാണ് അതിലടങ്ങിയത് എന്ന കാര്യത്തില്‍ സംശയമില്ല. ആദ്യകാലത്തെ ഭക്തിനിര്‍ഭരമായ കവിതയുടെ ജന്മസിദ്ധമായ വര്‍ണം ഇതില്‍ ദാര്‍ശനികചിന്തയുടെ വിളര്‍പ്പ് കലര്‍ന്ന് രോഗബാധിതമായപോലെ കാണപ്പെടുന്നു. സൃഷ്ടിയുടെ ഉത്ഭവം മുതലായ വിഷയങ്ങളെക്കുറിച്ചുള്ള ഊഹാപോഹപ്രധാനങ്ങളായ സൂക്തങ്ങള്‍ ഇതില്‍ കാണാം. ഈ അമൂര്‍ത്തങ്ങളായ സിദ്ധാന്തചര്‍ച്ചകളോടൊപ്പം അഥര്‍വവേദകാലത്തിലേക്കു ചേര്‍ന്ന അന്ധവിശ്വാസപരങ്ങളായ ആകര്‍ഷണോച്ചാടന മന്ത്രങ്ങളും ഇതില്‍ കാണാനുണ്ട്. ഭാവപ്രധാനങ്ങളായ സൂക്തങ്ങളില്‍ സ്വയം അനാവരണം ചെയ്ത മനസ്സിന്റെ പ്രായമെത്തലിനെ സൂചിപ്പിക്കുന്നവയാണ് ഊഹാപോഹങ്ങളായ സൂക്തങ്ങളെങ്കില്‍, അന്ധവിശ്വാസപരമായ സൂക്തങ്ങള്‍ കാണിക്കുന്നത്, അപ്പോഴേക്ക് ഇന്ത്യയിലെ ആദിമനിവാസികളുടെ സിദ്ധാന്തങ്ങളും ആചാരങ്ങളുമായി വൈദികാര്യന്മാര്‍ പരിചയപ്പെട്ടുകഴിഞ്ഞിരുന്നു എന്നാണ്. പത്താം മണ്ഡലം പില്‍ക്കാലത്തുണ്ടായതാണ് എന്നതിന് ഇവ സാക്ഷ്യം വഹിക്കുന്നു''(Ibid ,പുറം:47)
ഋഗ്വേദത്തില്‍പോലും ഒരു ദൈവികഗ്രന്ഥത്തിന് ഒട്ടും യോജിക്കാത്തവിധമുള്ള ജാതീയത കടന്നുകൂടിയതായി കാണാം:
''ബ്രാഹ്മണോസ്യ മുഖമാസീദ്
ബാഹൂ രാജന്യഃ കൃതഃ
ഊരു തദസ്യ യദ്വൈശ്യഃ
പദ്ഭ്യാം ശൂദ്രോ അജായത'' (ഋഗ്വേദം 10: 90: 12)
(ആ പ്രജാപതിക്ക് ബ്രഹ്മണന്‍ മുഖമായി, മുഖത്തില്‍ നിന്നുല്‍പന്നമായി. ക്ഷത്രിയന്‍ ബാഹുക്കളില്‍ നിന്നുളവായി. ഊരുക്കളില്‍നിന്ന് വൈശ്യനുണ്ടായി. അവിടത്തെ പാദങ്ങളില്‍നിന്ന് ശൂദ്രന്‍ ജനിച്ചു.)
ഋഗ്വേദത്തിന്റെ രണ്ടാം ഭാഗത്തിന്റെ അവതാരികയില്‍, വേദങ്ങളിലെ മുഴുവന്‍ ഭാഗവും ഇന്ന് ലഭ്യമല്ലെന്നും അവയില്‍ പലതും കൂട്ടിച്ചേര്‍ക്കപ്പെട്ടിട്ടുണ്ടെന്നും അവ്യക്തവും ദുരൂഹവുമായ പലതും അവയിലുണ്ടെന്നും എന്‍.വി. കൃഷ്ണവാരിയര്‍ വ്യക്തമാക്കുന്നു.
''വേദങ്ങള്‍ പ്രമാദമുക്തങ്ങളോ എല്ലാം ഉള്‍ക്കൊള്ളുന്നതോ അല്ലെ''ന്ന് ഡോ. എസ്. രാധാകൃഷ്ണന്‍ തന്റെ Indian Religions(പേജ് 22)ലും തെളിയിച്ചു പറഞ്ഞിട്ടുണ്ട്. നരേന്ദ്രഭൂഷണ്‍ എഴുതുന്നു: ''ക്രമേണ പാഠാന്തരങ്ങളും ശാഖകളും നിലവില്‍വരികയും ലോപിക്കുകയും ചെയ്തു. ചിലര്‍ ബ്രാഹ്മണങ്ങളെയും വേദങ്ങളില്‍ ഉള്‍പ്പെടുത്തി. ഇങ്ങനെ സംഹിതയും ബ്രാഹ്മണവുമെല്ലാം കൂടിക്കുഴഞ്ഞപ്പോള്‍ വ്യാസന്‍ അവയെ പുനര്‍നിര്‍ണയം ചെയ്തിരിക്കാം. അങ്ങനെയാവാം വ്യാസന്‍ വേദം പകുത്തെന്ന കഥ നിലവില്‍വന്നത്... ഇവയില്‍നിന്ന് നമുക്ക് അനുമാനിക്കാവുന്നത് സംഹിതകളും ബ്രാഹ്മണങ്ങളും കൂടിക്കുഴഞ്ഞ് അസ്സല്‍ വേദമേത്, പ്രവചനമേത്, ശാഖയേത് എന്ന് തിരിച്ചറിയാനാവാതെ വന്ന ഒരു കാലഘട്ടമുണ്ടായിരുന്നുവെന്നാണ്''(വൈദിക സാഹിത്യ ചരിത്രം, പുറം 51, 53).
സത്യവ്രത പട്ടേലിന്റെ ഭാഷയില്‍, ''ഹിന്ദുമതത്തിന്റെ ആദ്യ സ്രോതസ്സ് വേദമാണ്. മുഴുവിഭാഗങ്ങളും അതിനോട് കൂറുപുലര്‍ത്തുകയും അതിന്റെ മൗലികത അംഗീകരിക്കുകയും ചെയ്യുന്നു. ചില വേദസൂക്തങ്ങള്‍ നഷ്ടപ്പെട്ടിട്ടുണ്ട്. എങ്കിലും ഇപ്പോള്‍ നിലവിലുള്ളവതന്നെ ധാരാളമാണ്'' (Hinduisam Religion and Way of Life.Page 5).
ഋഗ്വേദം പത്താം മണ്ഡലത്തിലെ പത്താം സൂക്തത്തില്‍ സ്വന്തം സഹോദരനെ ലൈംഗികവേഴ്ചക്കു ക്ഷണിക്കുന്ന സഹോദരിയുടെ ഭാഷയും ശൈലിയും ഒരു ദൈവിക ഗ്രന്ഥത്തിന് ഒട്ടും അനുയോജ്യമല്ല. അപ്രകാരംതന്നെ സൂര്യഗ്രഹണത്തെയും ചന്ദ്രഗ്രഹണത്തെയും സംബന്ധിച്ച ഋഗ്വേദത്തിലെയും അഥര്‍വവേദത്തിലെയും പ്രസ്താവങ്ങള്‍ അബദ്ധജടിലങ്ങളും അന്ധവിശ്വാസാധിഷ്ഠിതങ്ങളുമത്രെ. അതിനാല്‍ വേദങ്ങളില്‍ നീക്കംചെയ്യേണ്ട പലതുമുണ്ടെന്ന് പ്രമുഖ ഹൈന്ദവ പണ്ഡിതന്മാര്‍ തറപ്പിച്ചു പറഞ്ഞിട്ടുണ്ട്. ഗുരു നിത്യചൈതന്യയതി എഴുതുന്നു: ''വേദങ്ങള്‍ ഏറ്റവും നികൃഷ്ടമായിട്ടുള്ള ഹിംസക്ക് അനുമതി നല്‍കുന്നുവെന്ന് മാത്രമല്ല, അവയെല്ലാം ചെയ്യേണ്ടതുതന്നെയാണെന്ന് വിവരിക്കുന്ന ഒട്ടേറെ മന്ത്രങ്ങളുണ്ട്. നിഷ്പക്ഷമതികളായിട്ടുള്ള ആര്‍ക്കും അത് കണ്ടില്ലെന്ന് ഭാവിക്കാന്‍ കഴിയുന്നതല്ല. ഹിന്ദുക്കളുടെ ഏറ്റവും ഉല്‍കൃഷ്ടമായ ശ്രുതി വേദമാണെന്ന് പറയുന്ന അതേ നാവുകൊണ്ട് അതില്‍ ഏറ്റവും ഹീനമായ മന്ത്രങ്ങളുണ്ടെന്നു കൂടി പറയേണ്ടിവരുന്നത് ഏതൊരു പണ്ഡിതന്നും ദുഃഖകരമായ അനുഭവം തന്നെയാണ്. എന്നാല്‍, ചില പണ്ഡിതന്മാര്‍ ചരിത്രസത്യങ്ങളോട് നീതികാണിക്കാതെ, വേദങ്ങളിലെ നീക്കംചെയ്യപ്പെടേണ്ടതായ ഈ മന്ത്രങ്ങളെ സാധൂകരിക്കാനും ന്യായീകരിക്കാനും ശ്രമിക്കുന്നത് ഒരിക്കലും ക്ഷന്തവ്യമാണെന്ന് പറയാവതല്ല.''(ഗുരുകുലം മാസിക. ഉദ്ധരണം, ഭാരതീയ നവോത്ഥാനത്തിന്റെ രൂപരേഖ, പേജ് 212)
ചുരുക്കത്തില്‍, നിലവിലുള്ള വേദങ്ങള്‍ ദൈവികങ്ങളോ അപൗരുഷേയങ്ങളോ ആണെന്ന് വേദവിശ്വാസികള്‍ പോലും അവകാശപ്പെടുകയില്ല. അവയില്‍ പലതും കൂട്ടിച്ചേര്‍ക്കപ്പെട്ടിട്ടുണ്ട്, കൃത്രിമങ്ങള്‍ക്കിരയായിട്ടുണ്ട്, പലതും നഷ്ടപ്പെട്ടിട്ടുമുണ്ട്. എന്നാല്‍ ഇതിലൊട്ടും അത്ഭുതമില്ല. വേദങ്ങളുടെ ചരിത്രം ആര്‍ക്കുമറിയില്ലല്ലോ. അതിന്റെ കാലത്തെക്കുറിച്ചുപോലും ഗുരുതരമായ അഭിപ്രായവ്യത്യാസമാണുള്ളത്. ക്രിസ്തുവിനു മുമ്പ് 6000 വര്‍ഷമാണെന്നും 4500 ആണെന്നും 3000 ആണെന്നും 1500 ആണെന്നുമൊക്കെ വാദിക്കുന്നവരുണ്ട്. ഇതേക്കുറിച്ച് ഖണ്ഡിതമായി ആരും ഒന്നും പറയുന്നില്ല. ആര്യന്മാര്‍ ഇന്ത്യയിലെത്തിയ ശേഷമാണ് അവ ക്രോഡീകരിച്ചതെന്ന കാര്യത്തില്‍ പക്ഷാന്തരമില്ല. വേദാവതരണത്തിനും ക്രോഡീകരണത്തിനുമിടയിലെ സുദീര്‍ഘമായ കാലവ്യത്യാസം അവയിലെ ഗുരുതരമായ മാറ്റങ്ങള്‍ക്കും കൃത്രിമങ്ങള്‍ക്കും കാരണമാവുക സ്വാഭാവികമത്രെ. ഡോ. രാധാകൃഷ്ണന്‍ എഴുതുന്നു: ''ആര്യന്മാര്‍ കൂടെ കൊണ്ടുവന്ന ചില ധാരണകളും വിശ്വാസങ്ങളും അവര്‍ ഇന്ത്യയുടെ മണ്ണില്‍ തുടര്‍ന്ന് വികസിപ്പിച്ചു. ഈ സൂക്തങ്ങള്‍ നിര്‍മിച്ചതിനു ശേഷം വളരെ നീണ്ട ഇടക്കാലം കഴിഞ്ഞായിരുന്നു അവ സമാഹരിക്കപ്പെട്ടത്''(ഭാരതീയ ദര്‍ശനം, പുറം 46).
വേദങ്ങള്‍ പ്രകാശിതമായ പ്രാക്തന വൈദികഭാഷ തന്നെ ഇന്നില്ലെന്നും നിലവിലുള്ള സംസ്‌കൃതം അതിന്റെ രൂപാന്തരമാണെന്നും വേദകാലത്തിനുശേഷം വളരെക്കഴിഞ്ഞാണ് അതുണ്ടായതെന്നും നരേന്ദ്രഭൂഷണ്‍ ഉള്‍പ്പെടെയുള്ള പണ്ഡിതന്മാര്‍ വ്യക്തമാക്കുന്നു (വൈദിക സാഹിത്യ ചരിത്രം, പുറം 27, 39).
ഇത്തരത്തിലുള്ള എല്ലാ മനുഷ്യ ഇടപെടലുകളില്‍നിന്നും കൂട്ടിച്ചേര്‍ക്കലുകളില്‍നിന്നും വെട്ടിച്ചുരുക്കലുകളില്‍നിന്നും കൂട്ടിക്കലര്‍ത്തലുകളില്‍നിന്നും തീര്‍ത്തും മോചിതമായ, അവതരണം തൊട്ടിതേവരെയുള്ള ചരിത്രം മനുഷ്യരാശിയുടെ മുമ്പില്‍ തെളിഞ്ഞുനില്‍ക്കുന്ന ഗ്രന്ഥം ഖുര്‍ആന്‍ മാത്രമായതിനാലാണ് അതിനെ നിലവിലുള്ള ഏക ദൈവികഗ്രന്ഥമെന്ന് പറയുന്നത്. അത് പൂര്‍വിക ദൈവിക ഗ്രന്ഥങ്ങളെയെല്ലാം സത്യപ്പെടുത്തുകയും ചെയ്യുന്നു.
നിലവിലുള്ള വേദങ്ങള്‍ മനുഷ്യ ഇടപെടലുകള്‍ക്ക് വിധേയമായിട്ടുണ്ടെന്ന് ഡോ. രാധാകൃഷ്ണന്‍, എന്‍.വി. കൃഷ്ണവാരിയര്‍, നരേന്ദ്രഭൂഷണ്‍, സത്യവ്രത പട്ടേല്‍ പോലുള്ള വേദപണ്ഡിതന്മാര്‍ അസന്ദിഗ്ധമായി വ്യക്തമാക്കിയതാണ്.
മഹാഭാരതത്തില്‍ കുരുക്ഷേത്രയുദ്ധഭൂമിയില്‍ കൗരവരോട് യുദ്ധം ചെയ്യാന്‍ പാണ്ഡവന്മാര്‍ ചെന്നപ്പോള്‍ പാണ്ഡവവീരനായ അര്‍ജുനന് ബന്ധുജനങ്ങളെ കൊല്ലാന്‍ വലിയ ദുഃഖം തോന്നി. അര്‍ജുനന്റെ തേരാളിയായ കൃഷ്ണന്‍ അര്‍ജുനന്റെ അസ്ഥാനത്തുള്ള ആ ഹൃദയ ദൗര്‍ബല്യം മാറ്റാന്‍ പറഞ്ഞുകൊടുത്ത തത്വോപദേശമാണ് ഗീതയെന്ന് വിശ്വസിക്കപ്പെടുന്നു. വ്യാസമഹര്‍ഷിയാണ് ഗീതയുള്‍പ്പെടെയുള്ള മഹാഭാരതം നിര്‍മിച്ചത്.
ശ്രീകൃഷ്ണന്‍ ഭഗവാന്റെ അവതാരമാണെന്നും അതിനാല്‍ ഗീത ഭഗവദ്ഗീതയാണെന്നും പ്രചരിപ്പിക്കപ്പെടുന്നു. എന്നാല്‍ ഭഗവാന്‍ ഒരിക്കലും മനുഷ്യരൂപത്തില്‍ അവതരിക്കില്ലെന്നും അവതരിച്ചിട്ടില്ലെന്നുമാണ് പ്രമുഖരായ ഹൈന്ദവ പണ്ഡിതന്മാരും പരിഷ്‌കര്‍ത്താക്കളും വ്യക്തമാക്കുന്നത്. സ്വാമി ദയാനന്ദ സരസ്വതി ഇക്കാര്യം ചോദ്യോത്തരരൂപത്തില്‍ ഇങ്ങനെ അവതരിപ്പിക്കുന്നു:
''പ്രശ്‌നം: ഈശ്വരന്‍ അവതാരം സ്വീകരിക്കുന്നുണ്ടോ?
ഉത്തരം: ഇല്ല. എന്തെന്നാല്‍ 'അജ ഏക പാത്' (35-53), 'സപര്‍യ്യഗാതുക്രമകായം' (408) എന്നു തുടങ്ങിയ യജുര്‍വേദ വചനങ്ങളില്‍നിന്ന് പരമേശ്വരന്‍ ജന്മം കൊള്ളുന്നില്ലെന്ന് മനസ്സിലാകുന്നുണ്ട്.
പ്രശ്‌നം:
യദാ യദാ ഹി ധര്‍മസ്യ
ഗ്ലാനിര്‍ഭവതി ഭാരത
അഭ്യുത്ഥാന മധര്‍മസ്യ
തദാത്മാനം സൃജാമ്യഹം (ഭഗവത്ഗീത അ: 4 ശ്ലോകം 7)
ധര്‍മത്തിന് ലോപം വരുമ്പോഴെല്ലാം ഞാന്‍ ശരീരം ധരിക്കുന്നു എന്ന് ശ്രീകൃഷ്ണന്‍ പറയുന്നുണ്ടല്ലോ?
ഉത്തരം: വേദവിരുദ്ധമായതുകൊണ്ട് ഇത് പ്രമാണമാകുന്നില്ല. ധര്‍മാത്മാവും ധര്‍മത്തെ രക്ഷിക്കാന്‍ ഇഛിക്കുന്നവനുമായ ശ്രീകൃഷ്ണന്‍ ''ഞാന്‍ യുഗം തോറും ജന്മം കൈക്കൊണ്ടു ശിഷ്ട ജനങ്ങളെ പരിപാലിക്കുകയും ദുഷ്ടന്മാരെ സംഹരിക്കുകയും ചെയ്യും'' എന്നിങ്ങനെ ആഗ്രഹിച്ചിട്ടുണ്ടായിരിക്കാം. അങ്ങനെ ചെയ്യുന്നതില്‍ ദോഷമൊന്നുമില്ല. എന്തെന്നാല്‍ 'പരോപകാരായ സതാം വിഭൂതയഃ' സജ്ജനങ്ങളുടെ ശരീരം, മനസ്സ്, ധനം എന്നിവയെല്ലാം പരോപകാരത്തിനായിട്ടുള്ളതാണല്ലോ. എന്നാലും ശ്രീകൃഷ്ണന്‍ ഇതുകൊണ്ട് ഈശ്വരനാണെന്ന് വരുവാന്‍ തരമില്ല''(സത്യാര്‍ഥ പ്രകാശം, പുറം 304, 305).
ഈശ്വരന് അവതാരമുണ്ടാവില്ലെന്ന് ശ്രീവാഗ്ഭടാനന്ദ ഗുരുവും അസന്ദിഗ്ധമായി വ്യക്തമാക്കിയിട്ടുണ്ട് (വാഗ്ഭടാനന്ദന്റെ സമ്പൂര്‍ണ കൃതികള്‍, പുറം 357359, 751, 752. മാതൃഭൂമി പ്രിന്റിംഗ് ആന്റ് പബ്ലിഷിംഗ് കമ്പനി ലിമിറ്റഡ്, കോഴിക്കോട്).
ഗീതതന്നെ അവതാര സങ്കല്‍പത്തെ നിരാകരിക്കുന്നു:\\
അവജാനന്തി മാം മൂഢാ\\
മാനുഷീം തനു മാ ശ്രിതം\\
പരം ഭാവമജാനന്തോ\\
മമ ഭൂത മഹേശ്വരം\\
മോഘാശാ മോഘ കര്‍മ്മാണോ\\
മോഘജ്ഞാനാ വിചേതസഃ\\
രാക്ഷസി മാസുരി ചൈവ\\
പ്രകൃതീം മോഹിനീം ശ്രീതാഃ\\
(ഭൂതങ്ങളുടെ മഹേശ്വരനെന്ന പരമമായ എന്റെ ഭാവത്തെ അറിയാത്ത മൂഢന്മാര്‍ എന്നെ മാനുഷികമായ ശരീരത്തെ ആശ്രയിച്ചവനായി തെറ്റായി മനസ്സിലാക്കുന്നു. അങ്ങനെ എന്നെ ധരിക്കുന്നവരുടെ ആശകളും അവര്‍ ചെയ്യുന്ന കര്‍മങ്ങളും അവര്‍ക്കുള്ള ജ്ഞാനവും നിഷ്ഫലങ്ങളാണ്. അവര്‍ അവിവേകികളും മനസ്സിനെ മോഹിപ്പിക്കുന്ന രാക്ഷസ പ്രകൃതിയെയും അസുര പ്രകൃതിയെയും ആശ്രയിച്ചുള്ളവരാകുന്നു.'' അധ്യായം 9, രാജ വിദ്യാ രാജഗുഹ്യയോഗം, ശ്ലോകം 11, 12).
ഗീതയിലെ ഇത്തരം സൂക്തങ്ങള്‍ ദിവ്യവചനങ്ങളാവാനുള്ള സാധ്യത നിഷേധിക്കാനാവില്ല. എന്നാല്‍ ശ്രീകൃഷ്ണന്‍ ഭഗവാനായിരുന്നുവെന്ന സങ്കല്‍പം തീര്‍ത്തും തെറ്റാണ്. ഗീതയുടെ തന്നെ ഉപര്യുക്ത പ്രസ്താവത്തിന് കടകവിരുദ്ധവുമാണ്. ശ്രീകൃഷ്ണന്‍ ദൈവദൂതനാവാനുള്ള സാധ്യതയാണ് ഏറെയുള്ളത്. എന്നാല്‍ അദ്ദേഹത്തിന്റെ വ്യക്തവും സത്യസന്ധവുമായ ചരിത്രം ലഭ്യമല്ലാത്തതിനാല്‍ ഇക്കാര്യത്തില്‍ ഖണ്ഡിതമായൊരു നിഗമനത്തിലെത്തുക സാധ്യമല്ല.
ഗീതയുടെ വ്യക്തവും ഖണ്ഡിതവുമായ ചരിത്രം ലഭ്യമല്ല. അത് മഹാഭാരതയുദ്ധത്തോടനുബന്ധിച്ചാണോ രചിക്കപ്പെട്ടതെന്ന കാര്യത്തില്‍ പോലും വീക്ഷണ വ്യത്യാസമുണ്ട്. ഗീതയുടെ കാലവും വിവാദവിധേയമാണ്. ക്രിസ്തുവിനു മുമ്പ് അഞ്ചാം നൂറ്റാണ്ടിലാണെന്നും നാലാം നൂറ്റാണ്ടിലാണെന്നും മൂന്നാം നൂറ്റാണ്ടിലാണെന്നും ക്രിസ്തുവിനു ശേഷം രണ്ടാം നൂറ്റാണ്ടിലാണെന്നും നാലാം നൂറ്റാണ്ടിലാണെന്നുമെല്ലാം അഭിപ്രായാന്തരമുണ്ട്. ഈ അവ്യക്തതയെക്കുറിച്ച് ഡോ. രാധാകൃഷ്ണന്‍ എഴുതുന്നു: ''ഭഗവദ്ഗീത മഹാഭാരതത്തിന്റെ യഥാര്‍ഥ ഭാഗമാണെന്ന് നാം സ്വീകരിച്ചാല്‍ തന്നെയും അതില്‍ പല കാലഘട്ടങ്ങളിലെയും കൃതികള്‍ ഉള്‍ക്കൊള്ളിച്ചിട്ടുള്ളതുകൊണ്ട് നമുക്ക് ഗീതയുടെ കാലത്തെപ്പറ്റി ഉറപ്പ് പറയാനാവില്ല.'' (ഭാരതീയ ദര്‍ശനം, ഭാഗം 1, പുറം 481).
ഗീതാകര്‍ത്താവ് തന്റെ കാലത്തെ വിവിധ ചിന്താധാരകളെ സമന്വയിപ്പിക്കുകയും സമാഹരിക്കുകയുമാണുണ്ടായതെന്ന് ഡോ. രാധാകൃഷ്ണന്‍ തുടര്‍ന്നു പറയുന്നു.
ഏതായാലും ഗീത ദൈവികമാണെന്ന് അദ്ദേഹത്തിന് അഭിപ്രായമില്ല. ''ദര്‍ശനം, മതം, സദാചാരങ്ങള്‍ എന്നിവയുടെ ഉപദേശങ്ങളടങ്ങുന്ന ഗ്രന്ഥമാണത്. അതിനെ ശ്രുതിയെന്നോ അഥവാ ഈശ്വരീയമായ അരുളപ്പാടെന്നോ കരുതുന്നില്ല. സ്മൃതിയെന്ന് പറയാവുന്നതാണ്''(Ibid പുറം 477).
ചുരുക്കത്തില്‍ ഗീതയില്‍ ദൈവിക സന്ദേശങ്ങളുടെ അംശങ്ങളും ആശയങ്ങളും ഉണ്ടാവാമെങ്കിലും അതൊരു ദൈവിക ഗ്രന്ഥമല്ല. ഒരു ദൈവിക ഗ്രന്ഥമാണെന്ന് ഗീത സ്വയം അവകാശപ്പെടുന്നില്ലെന്നതും ശ്രദ്ധേയമത്രെ.
\chapter{പരലോക വിശ്വാസവും പുനര്‍ജന്മസങ്കല്‍പവും }
  \section{ഇസ്‌ലാമിലെ പരലോക വിശ്വാസവും ഹിന്ദുമതത്തിലെ പുനര്‍ജന്മ സങ്കല്‍പവും തമ്മില്‍ വല്ല വ്യത്യാസവുമുണ്ടോ? ഉണ്ടെങ്കില്‍ എന്താണ്? പുനര്‍ജന്മ സങ്കല്‍പത്തെ ഇസ്‌ലാം അംഗീകരിക്കുന്നുണ്ടോ? ഇല്ലെങ്കില്‍ എന്തുകൊണ്ട്?}
 മരണത്തോടെ മനുഷ്യജീവിതം അവസാനിക്കുന്നില്ല എന്ന ഒരൊറ്റ കാര്യത്തില്‍ മാത്രമേ പരലോക വിശ്വാസവും പുനര്‍ജന്മ സങ്കല്‍പവും ഒത്തുവരുന്നുള്ളൂ. ഇസ്‌ലാമിക വിശ്വാസമനുസരിച്ച് ഭൂമി കര്‍മവേദിയാണ്. വിചാരണയും വിധിയും മരണശേഷം മറുലോകത്താണ്. ഭൂമിയിലെ കര്‍മങ്ങളുടെ അടിസ്ഥാനത്തിലാണ് പരലോകത്തെ രക്ഷാശിക്ഷകളുണ്ടാവുക. പദാര്‍ഥനിഷ്ഠമായ പ്രപഞ്ച ഘടനയുടെയും ഇവിടത്തെ അവസ്ഥകളുടെയും മാനദണ്ഡങ്ങളുടെയും അടിസ്ഥാനത്തില്‍, പഞ്ചേന്ദ്രിയങ്ങളുടെ പരിധിയില്‍ നിന്നുകൊണ്ട് പരലോകം എങ്ങനെയായിരിക്കുമെന്ന് സങ്കല്‍പിക്കുക സാധ്യമല്ല. എന്നാല്‍ ദൈവനിര്‍ദേശമനുസരിച്ച് വിശുദ്ധ ജീവിതം നയിക്കുന്നവര്‍ക്ക് അവിടെ സര്‍വവിധ സൗഭാഗ്യങ്ങളും സുഖസൗകര്യങ്ങളുമുള്ള ശാശ്വത സ്വര്‍ഗം ലഭിക്കുമെന്ന് ഇസ്‌ലാം പഠിപ്പിക്കുന്നു. ദൈവധിക്കാരിയായി പാപപങ്കിലമായ ജീവിതം നയിക്കുന്ന അക്രമികള്‍ നരകാവകാശികളാകുമെന്ന് മുന്നറിയിപ്പ് നല്‍കുകയും ചെയ്യുന്നു. മരണശേഷമുള്ള മറുലോകത്തെ രക്ഷാശിക്ഷകള്‍ നിഷ്‌കൃഷ്ടമായ വിചാരണക്കും നീതിനിഷ്ഠമായ വിധിക്കും ശേഷമായിരിക്കും ലഭിക്കുക. അതേസമയം പരലോകത്തെ വിചാരണയും വിധിയുമൊക്കെ മനുഷ്യനിവിടെ നിര്‍വഹിക്കുന്നതുപോലെയായിരിക്കുമെന്ന് ധരിക്കുന്നത് പരമാബദ്ധമത്രെ. എന്നാല്‍ രക്ഷ ലഭിക്കുന്നവന്‍ ഭൂമിയിലെ തന്റെ ഏതു സല്‍ക്കര്‍മം കാരണമാണത് കിട്ടിയതെന്നും ശിക്ഷ അനുഭവിക്കുന്നവന്‍ ഏതു പാപത്തിന്റെ പേരിലാണതിന് വിധേയനായതെന്നും വ്യക്തമായറിയും. അപ്പോഴല്ലേ അത് യഥാര്‍ഥ കര്‍മഫലവും രക്ഷാശിക്ഷയുമാവുകയുള്ളൂ.
എന്നാല്‍ പുനര്‍ജന്മ സങ്കല്‍പമനുസരിച്ച് മരണശേഷമുള്ള ജനനവും ജീവിതവും ഇവിടെ ഭൂമിയില്‍ തന്നെയാണ്. ധര്‍മം പാലിക്കാത്തവര്‍ മരണാനന്തരം ശുനകനോ ശൂദ്രനോ ചണ്ഡാലനോ മറ്റോ ആയി വീണ്ടും ഇവിടെ ജനിക്കുമെന്നതാണ് പുനര്‍ജന്മ സിദ്ധാന്തം. അതേസമയം ധര്‍മനിഷ്ഠമായ ജീവിതം നയിച്ചാല്‍ ബ്രാഹ്മണനോ ക്ഷത്രിയനോ വൈശ്യനോ ആയി പുനര്‍ജനിക്കുമെന്നും അത് അവകാശപ്പെടുന്നു. ഇവ്വിഷയകമായി ചില കാര്യങ്ങള്‍ ഇവിടെ വിശദമായി തന്നെ ഗ്രഹിക്കേണ്ടതുണ്ട്:
1. കാരണം വ്യക്തമാക്കുകയും പ്രതിയെ ബോധ്യപ്പെടുത്തുകയും ചെയ്യാതെ നാം ശിക്ഷ നടപ്പാക്കിയാല്‍ അത് തീര്‍ത്തും അനീതിയായാണ് അനുഭവപ്പെടുക. നാം ആരെയെങ്കിലും അടിക്കുകയോ ജയിലിലടയ്ക്കുകയോ വധിക്കുകയോ ആണെങ്കില്‍ അത് ഏത് കുറ്റത്തിന്റെ പേരിലാണെന്ന് വ്യക്തമാക്കുകയും ബന്ധപ്പെട്ടവരെ ബോധ്യപ്പെടുത്തുകയും വേണം. ഇത് നീതിയുടെ പ്രാഥമിക താല്‍പര്യമത്രെ. അതോടൊപ്പം വിചാരണക്കും വിധിക്കും ശേഷമായിരിക്കണം ശിക്ഷ നടപ്പാക്കപ്പെടുന്നത്. എന്നാല്‍ ശൂദ്രനോ ചണ്ഡാലനോ ഒരിക്കലും ഇവ്വിധം വിചാരണക്ക് വിധേയമാക്കപ്പെട്ടിട്ടില്ല. കഴിഞ്ഞ ജന്മത്തില്‍ താന്‍ എന്തു തെറ്റ് ചെയ്തതിന്റെ പേരിലാണ് ഈ ജന്മത്തില്‍ ഇവ്വിധമായതെന്നറിയുകയുമില്ല. കഴിഞ്ഞ ജന്മത്തിലെ കര്‍മഫലമാണ് ഈ ജീവിതത്തിലെ സ്ഥിതി നിര്‍ണയിക്കുന്നതെങ്കില്‍ അത് ഏത് കര്‍മത്തിന്റെ ഫലമാണെന്നറിയണമല്ലോ? ബ്രാഹ്മണനും ക്ഷത്രിയനും തങ്ങളുടെ ഏതു പുണ്യകര്‍മമാണ് അവ്വിധമാകാന്‍ കാരണമെന്നും അറിയുന്നില്ല. അതുകൊണ്ടുതന്നെ പുനര്‍ജന്മ സങ്കല്‍പം കര്‍മഫല സിദ്ധാന്തത്തിനും തദടിസ്ഥാനത്തിലുള്ള രക്ഷാശിക്ഷാ വിശ്വാസത്തിനും കടകവിരുദ്ധമത്രെ.\\
2. പുനര്‍ജന്മ സങ്കല്‍പ പ്രകാരം പാപി രണ്ടാം ജന്മത്തില്‍ നീചനും അധമനുമായിരിക്കും. അത്തരക്കാരില്‍ നിന്ന് നികൃഷ്ടജീവിതവും അധമവൃത്തികളുമാണുണ്ടാവുക. സുകൃതങ്ങളുണ്ടാവുകയില്ല. അതിനാല്‍ മൂന്നാം ജന്മം കൂടുതല്‍ അധമവും പതിതവുമായിരിക്കും. നാലാം ജന്മം അതിലും ഹീനമായിരിക്കും. ഇങ്ങനെ പാപികള്‍ ഓരോ ജന്മത്തിലും കൂടുതല്‍ കൂടുതല്‍ പാപികളാവുകയല്ലാതെ അവര്‍ക്ക് ഉദ്ഗതി പ്രാപിക്കുക സാധ്യമേയല്ല. മാത്രമല്ല, പാപിയായ മനുഷ്യന് അടുത്ത ജന്മത്തില്‍ മൃഗമാകാം. മൃഗത്തിന് പിന്നെ മനുഷ്യനാകാന്‍ സാധ്യമാണോ? പുതുതായി പിറക്കുന്ന മനുഷ്യര്‍ കഴിഞ്ഞ ജന്മത്തില്‍ എന്തായിരുന്നു? എന്തു പുണ്യം ചെയ്തതിന്റെ പേരിലാണ് മനുഷ്യരായി ജന്മംകൊണ്ടത്? ഇത്തരം ചോദ്യങ്ങള്‍ക്കൊന്നും പുനര്‍ജന്മ സിദ്ധാന്തത്തില്‍ ഉത്തരമില്ല.\\
3. ഹൈന്ദവ വേദങ്ങളിലെവിടെയും പുനര്‍ജന്മ സിദ്ധാന്തമില്ല. ഉപനിഷത്തുകളിലേ അതിനെ സംബന്ധിച്ച പരാമര്‍ശവും വിശദീകരണവുമുള്ളൂ. ഇക്കാര്യം ഡോ. എസ്. രാധാകൃഷ്ണന്‍ ഇങ്ങനെ വ്യക്തമാക്കുന്നു: ''ബ്രാഹ്മണം അടുത്ത ലോകത്തിലേ ജനനമരണങ്ങള്‍ നടക്കുന്നതായി സ്വീകരിക്കുന്നുള്ളൂ. ഉപനിഷത്തുകളിലായപ്പോള്‍ ആ വിശ്വാസം ഈ ലോകത്താണ് പുനര്‍ജന്മമെന്ന വിശ്വാസപ്രമാണമായി മാറി''(ഭാരതീയ ദര്‍ശനം, വാള്യം ഒന്ന്, പുറം 224, മാതൃഭൂമി പബ്‌ളിഷിംഗ് കമ്പനി, കോഴിക്കോട് 1995).\\
അതിനാല്‍ പുനര്‍ജന്മ സങ്കല്‍പം വേദപ്രോക്തമല്ലെന്നത് സുവിദിതമത്രെ.
4. ഉപനിഷത്തിലാണ് പുനര്‍ജന്മത്തെപ്പറ്റി പറയുന്നത്. ജാതിവ്യവസ്ഥയെ ന്യായീകരിക്കാനും അതിന്റെ കടുത്ത വിവേചനങ്ങളെയും ക്രൂരതകളെയും നിലനിര്‍ത്താനുമായി ആവിഷ്‌കരിക്കപ്പെട്ടതാണ് അതെന്ന് ആര്‍ക്കും അനായാസം മനസ്സിലാക്കാവുന്നതാണ്.
പുനര്‍ജന്മത്തെപ്പറ്റി പറയുന്ന ഛാന്ദോഗ്യോപനിഷത്തിലിങ്ങനെ കാണാം:
തദ്യ ഇഹ രമണീയചരണാ അഭ്യാശോ ഹ യത്തേ രമണീയാം യോനിമാപദ്യേരന്‍, ബ്രാഹ്മണയോനിം വാ ക്ഷത്രിയയോനിം വാ വൈശ്യയോനിം വാ. അഥ യ ഇഹ കപൂയചരണാ അഭ്യാശോ ഹ യത്തേ കപൂയാം യോനിമാപദ്യേരന്‍, ശ്വയോനിം വാ സൂകരയോനിം ചണ്ഡാലയോനിം, വാ
(അധ്യായം 5, ഖണ്ഡം 10, ശ്ലോകം 7)
(അവരില്‍ സല്‍ക്കര്‍മങ്ങള്‍ ചെയ്യുന്നവരാരോ അവര്‍ വേഗം തന്നെ ഉത്തമ യോനികളില്‍ ചെന്നു ചേരുന്നു. ബ്രാഹ്മണയോനിയിലോ ക്ഷത്രിയയോനിയിലോ വൈശ്യയോനിയിലോ എത്തിച്ചേരുന്നു. അതുപോലെ അശുഭചാരന്മാരായവരാരോ അവര്‍ അശുഭ യോനിയെ പ്രാപിക്കുന്നു. പട്ടിയുടെ യോനിയെയോ പന്നിയുടെ യോനിയെയോ ചണ്ഡാലന്റെ യോനിയെയോ പ്രാപിക്കുന്നു.)
ഇത് ഉയര്‍ന്ന ജാതിക്കാരില്‍, തങ്ങള്‍ മേല്‍ജാതിയിലായത് കഴിഞ്ഞ ജന്മത്തില്‍ പുണ്യം ചെയ്തതിനാലാണെന്നും അതിനാല്‍ തങ്ങളിന്നനുഭവിക്കുന്ന പ്രത്യേകാവകാശങ്ങള്‍ക്കെല്ലാം ന്യായമായും അര്‍ഹരാണെന്നുമുള്ള ബോധം വളര്‍ത്തുന്നു. അതോടൊപ്പം താഴ്ന്ന ജാതിക്കാര്‍ അവ്വിധമായത് അവരുടെത്തന്നെ കര്‍മഫലമാണെന്ന ധാരണയും സൃഷ്ടിക്കുന്നു. അവരെ പട്ടികളെയും പന്നികളെയുമെന്നപോലെ നികൃഷ്ട ജീവികളായി കാണാന്‍ പ്രേരിപ്പിക്കുകയും ചെയ്യുന്നു. അങ്ങനെ ജാതിവ്യവസ്ഥയുടെ എല്ലാവിധ വൃത്തികേടുകളെയും അരക്കിട്ടുറപ്പിക്കുകയും ന്യായീകരിക്കുകയുമാണ് പുനര്‍ജന്മ സിദ്ധാന്തം ചെയ്യുന്നത്.
5. മരണശേഷം മനുഷ്യാത്മാക്കള്‍ മൃഗശരീരത്തില്‍ നിലകൊള്ളുന്നുവെന്ന പ്രാകൃത ഗോത്രവിശ്വാസത്തില്‍ നിന്നാണ് പുനര്‍ജന്മ സങ്കല്‍പം രൂപം കൊണ്ടത്. ഡോ. എസ്. രാധാകൃഷ്ണന്‍ എഴുതുന്നു: ''കര്‍മപദ്ധതിയുടെയും പുനര്‍ജന്മ സിദ്ധാന്തത്തിന്റെയും ആശയം ആര്യമസ്തിഷ്‌കത്തില്‍ രൂപംകൊണ്ടതാണെന്ന് ഉറപ്പിച്ചു പറയാം. എന്നാല്‍ ഇതിനുള്ള പ്രേരണ ലഭിച്ചത് മരണശേഷം ആത്മാക്കള്‍ മൃഗശരീരങ്ങളില്‍ കുടികൊള്ളുന്നു എന്ന് വിശ്വസിക്കുന്ന ആദി ഗോത്രക്കാരില്‍ നിന്നാകാമെന്ന സംഗതിയും നിഷേധിക്കേണ്ടതില്ല.''(ഭാരതീയ ദര്‍ശനം, വാള്യം 1, പുറം 113)
6. സെമിറ്റിക് മതങ്ങളെപ്പോലെത്തന്നെ ഹിന്ദുമതത്തിലെ വേദങ്ങളും പരലോകത്തെപ്പറ്റിയാണ് പറയുന്നത്. ഋഗ്വേദത്തിലും സാമവേദത്തിലും അഥര്‍വ വേദത്തിലും പരലോകത്തെ സംബന്ധിച്ച പരാമര്‍ശം കാണാം. ഉദാഹരണത്തിന്:
രായേ അഗ്‌നേ മഹേ ത്വാ ദാനായ സമിധീമഹി\\
ഈഡിഷ്വാ ഹി മഹേ വൃഷന്‍ ദ്യാവാ ഹോത്രായാ പൃഥിവീ\\
(സാമവേദം, ആഗ്‌നേയ കാണ്ഡം: 1103)
(എല്ലാ ആഗ്രഹങ്ങളും പൂര്‍ണമാക്കുന്ന ഭഗവാനേ, അങ്ങയില്‍നിന്നും ഏറ്റവും വലിയ ദാനം കിട്ടാന്‍ അങ്ങയെ ഞങ്ങള്‍ സന്തോഷിപ്പിക്കുന്നു. അറിവിന്നായും ഭൂമിയില്‍ സമാധാനത്തിനായും പരലോകത്ത് ശാന്തിക്കായും ഞങ്ങള്‍ അങ്ങയെ സ്തുതിക്കുന്നു.)
അഥര്‍വ വേദം, പരലോകത്തെപ്പറ്റി ചിന്തിച്ച് സല്‍ക്കര്‍മങ്ങളനുഷ്ഠിക്കാന്‍ ദമ്പതികളോടാവശ്യപ്പെടുന്നു\\.
അന്വാരഭേഥാമനുസംരഭേഥാമേതം ലോകം ശ്രദ്ദധാനാഃ സചന്തേ (6: 12: 3)\\
വേദങ്ങളിലെന്നപോലെത്തന്നെ ഉപനിഷത്തുകളിലും പുരാണങ്ങളിലുമെല്ലാം പരലോകത്തെയും സ്വര്‍ഗ നരകങ്ങളെയും സംബന്ധിച്ച ധാരാളം പരാമര്‍ശങ്ങളുണ്ട്.
ഈശാവാസ്യോപനിഷത്തില്‍ മരണാനന്തരമുള്ള നരകത്തെ സംബന്ധിച്ച് ഇങ്ങനെ കാണാം:\\
അസുര്യാ നാമ തേ ലോകാഃ അന്ധേന തമസാവൃതാഃ\\
താംസ്‌തേ പ്രേത്യാഭിഗച്ഛന്തി യേ കേ ചാത്മഹനോ ജനാഃ (മന്ത്രം: 3)\\
(സൂര്യനില്ലാത്ത ആ ലോകങ്ങള്‍ അജ്ഞതാന്ധകാരാവൃതങ്ങളാകുന്നു. ആത്മഘാതികള്‍, അഥവാ ഈശ്വര സ്മരണയില്ലാതെ വിഷയാസക്തരായി കഴിയുന്നവര്‍ വീണ്ടും ദുഃഖനിരതമായ ആ ലോകങ്ങളിലെത്തുന്നു).
സ്വര്‍ഗത്തെ സംബന്ധിച്ച് കഠോപനിഷത്ത് പറയുന്നു:\\
'സ്വര്‍ഗേ ലോകേ ന ഭയം കിഞ്ചനാസ്തി\\
ന തത്ര ത്വം ന ജരയാ ബിഭേതി\\
ഉഭേ തീര്‍ത്വാശനായാപിപാസേ\\
ശോകാതിഗോ മോദതേ സ്വര്‍ഗലോകേ' (1:1: 12)\\
(സ്വര്‍ഗലോകത്ത് അല്‍പം പോലും ഭയമില്ല. അവിടെ ആര്‍ക്കും മരണമില്ല. ജരാനരയാല്‍ ആര്‍ക്കും ഭയവുമില്ല. അവിടെ വസിക്കുന്നവര്‍ വിശപ്പിനും ദാഹത്തിനും ശോകാദികള്‍ക്കും അതീതന്മാരായി എല്ലാവിധ ആനന്ദവും അനുഭവിക്കുന്നു.)
മനുസ്മൃതിയിലിങ്ങനെ കാണാം:
നാമുത്ര ഹി സഹായാര്‍ഥം പിതാ മാതാ ച തിഷ്ഠത\\
ന പുത്രദാരാ ന ജ്ഞാതിര്‍ ധര്‍മസ്തിഷ്ഠതി കേവലഃ (4: 239)\\
(പരലോകത്ത് സഹായത്തിന് അച്ഛനമ്മമാരെ കിട്ടുകയില്ല. മക്കളോ ഭാര്യയോ ബന്ധുക്കളോ അവിടെ പ്രയോജനപ്പെടുകയില്ല. സഹായിക്കാന്‍ ധര്‍മം മാത്രം നിലകൊള്ളുന്നു.)\\
ഏകഃ പ്രജായതേ ജന്തുരേക ഏവ പ്രലീയതേ\\
ഏകോ(അ)നുഭുങ്ക്‌തേ സുകൃതമേക ഏവ ച ദുഷ്‌കൃതം (4: 240)\\
(ജീവി ജനിക്കുന്നതും മരിക്കുന്നതും തനിച്ചാണ്. സ്വര്‍ഗത്തില്‍ സുകൃതം അനുഭവിക്കുന്നതും നരകത്തില്‍ ദുഷ്‌കൃതം അനുഭവിക്കുന്നതും തനിച്ചുതന്നെ.)
ഭഗവദ് ഗീതയും സ്വര്‍ഗനരകങ്ങളുടെ കാര്യം ഊന്നിപ്പറയുന്നു. ഉദാഹരണത്തിന് ചിലതു മാത്രമിവിടെ ഉദ്ധരിക്കാം.
ഉത്സന്നകുലധര്‍മാണാം മനുഷ്യാണാം ജനാര്‍ദന!
നരകേ നിയതം വാസോ ഭവതീത്യനു ശുശ്രുമ (പ്രഥമാധ്യായം, ശ്ലോകം 44)
(ഹേ ജനാര്‍ദനാ, കുലധര്‍മങ്ങള്‍ നശിച്ചുപോയ മനുഷ്യരുടെ വാസം എന്നെന്നേക്കും നരകത്തിലാണെന്ന് ഞങ്ങള്‍ കേട്ടറിഞ്ഞിട്ടുണ്ട്.)
അഥ ചേത്ത്വമിമം ധര്‍മ്യം\\
സംഗ്രാമം ന കരിഷ്യസി\\
തതഃ സ്വധര്‍മം കീര്‍ത്തിം ച\\
ഹിത്വാ പാപമവാപ്‌സ്യസി (ദ്വിതീയാധ്യായം, ശ്ലോകം 33)\\
(എന്നാല്‍ അര്‍ജുനാ, നീ സ്വധര്‍മാനുസൃതമായ ഈ യുദ്ധം ചെയ്യാതിരിക്കുന്ന പക്ഷം സ്വര്‍ഗപ്രാപ്തിയുണ്ടാവുകയില്ലെന്നു തന്നെയല്ല സ്വധര്‍മത്തെയും യശസ്സിനെയും ഉപേക്ഷിക്കുന്നതുകൊണ്ട് നിനക്ക് പാപം നേരിടുന്നതുമാണ്.)\\
ഹതോ വാ പ്രാപ്‌സ്യസി സ്വര്‍ഗം\\
ജിത്വാ വാ ഭോക്ഷ്യസേ മഹീം,\\
തസ്മാദുത്തിഷ്ഠ കൌന്തേയ\\
യുദ്ധായ കൃതനിശ്ചയഃ (ദ്വിതീയാധ്യായം, ശ്ലോകം 37)\\
(അല്ലയോ കുന്തിപുത്രാ, ശത്രുക്കള്‍ നിന്നെ വധിച്ചാല്‍, നിനക്ക് സ്വര്‍ഗത്തില്‍ പോകാം. നീ ശത്രുക്കളെ ജയിക്കുന്നതായാല്‍ രാജ്യമനുഭവിക്കുകയും ചെയ്യാം. അതുകൊണ്ട് രണ്ടായാലും നീ യുദ്ധം ചെയ്യാനുറച്ചുകൊണ്ടു വേഗം എഴുന്നേല്‍ക്കുക.)
വേദകാലം മുതല്‍ക്കു തന്നെ പരലോക വിശ്വാസമുണ്ടായിരുന്നുവെന്ന് ഇതെല്ലാം വ്യക്തമാക്കുന്നു. ഡോ. രാധാകൃഷ്ണന്‍ എഴുതുന്നു: ''സ്വര്‍ഗം, നരകം എന്നിവയെപ്പറ്റിയുള്ള ചില അവ്യക്ത ധാരണകളില്‍ നിന്ന് ചിന്താശക്തിയുള്ള മനസ്സുകള്‍ക്ക് ഒഴിഞ്ഞുമാറാന്‍ കഴിഞ്ഞിരുന്നില്ല. പുനര്‍ജന്മസിദ്ധാന്തങ്ങള്‍ അപ്പോഴും വിദൂരത്തിലായിരുന്നു. മരണത്തോടെ എല്ലാം അവസാനിക്കുന്നില്ലെന്ന ഉറപ്പ് വൈദികാര്യന്മാര്‍ക്ക് ഉണ്ടായിരുന്നു.''(ഭാരതീയ ദര്‍ശനം, വാള്യം 1, പുറം 91)
രാഹുല്‍ സാംകൃത്യായന്‍ എഴുതുന്നു: ''വേദത്തിലെ ഋഷിമാര്‍ ഈ ലോകത്തിന് വിഭിന്നമായ മറ്റൊരു ലോകമുണ്ടെന്ന് വിശ്വസിച്ചിരുന്നു. അവിടേക്കാണ് മരണാനന്തരം സല്‍ക്കര്‍മികള്‍ പോകുക. അവരവിടെ ആനന്ദപൂര്‍വം ജീവിക്കുന്നതാണ്. താഴെയുള്ള പാതാളം അന്ധകാരമായ നരകലോകമാണ്. അവിടേക്കാണ് ദുഷ്‌കര്‍മികള്‍ പോകുന്നത്'' (വിശ്വദര്‍ശനങ്ങള്‍, പുറം 552).
അതിനാല്‍ പുനര്‍ജന്മ സിദ്ധാന്തം അനിസ്‌ലാമികമെന്ന പോലെത്തന്നെ അവൈദികവുമാണ്. അത് ജാതിവ്യവസ്ഥയുടെ ജീര്‍ണതകള്‍ സംരക്ഷിക്കാന്‍ നിര്‍മിക്കപ്പെട്ടതാണ്. പരലോകത്തെയും സ്വര്‍ഗനരകങ്ങളെയും സംബന്ധിച്ച വിശ്വാസമാണ് ഇസ്‌ലാമികമെന്നതുപോലെ തന്നെ ഹൈന്ദവവും. വേദപ്രോക്തവും ധര്‍മനിഷ്ഠവും അതുമാത്രമത്രെ.
(വിശദ പഠനത്തിന് ഡയലോഗ് സെന്റര്‍ കേരള പ്രസിദ്ധീകരിച്ച 'പുനര്‍ജന്മ സങ്കല്പവും പരലോകവിശ്വാസവും' എന്ന കൃതി കാണുക. വിതരണം: ഇസ്‌ലാമിക് പബ്ലിഷിംഗ് ഹൗസ്)
\chapter{ഭൂമിയിലെ വൈകല്യം സ്വര്‍ഗത്തിലുമുണ്ടാകുമോ? }
 \section{ ഭൂമിയിലെ അതേ അവസ്ഥയിലായിരിക്കുമോ മനുഷ്യരെല്ലാം പരലോകത്തും? വികലാംഗരും വിരൂപരുമെല്ലാം ആ വിധം തന്നെയാകുമോ?}
 ഭൗതിക പ്രപഞ്ചത്തിലെ പദാര്‍ഥനിഷ്ഠമായ പശ്ചാത്തലത്തില്‍ കാര്യങ്ങള്‍ ഗ്രഹിക്കാനാവശ്യമായ പഞ്ചേന്ദ്രിയങ്ങളും ബൗദ്ധിക നിലവാരവുമാണ് നമുക്ക് നല്‍കപ്പെട്ടിരിക്കുന്നത്. അതുപയോഗിച്ച് തീര്‍ത്തും വ്യത്യസ്തമായ പരലോകത്തെ അവസ്ഥ മനസ്സിലാക്കുക സാധ്യമല്ല. അതുകൊണ്ടുതന്നെ ദിവ്യബോധനങ്ങളിലൂടെ ലഭ്യമായ അറിവുമാത്രമേ ഇക്കാര്യത്തില്‍ അവലംബനീയമായുള്ളൂ. പരലോക ജീവിതത്തിന്റെ സൂക്ഷ്മാംശങ്ങളൊക്കെയും അതില്‍ വിശദീകരിക്കപ്പെട്ടിട്ടില്ല. സ്വര്‍ഗം സുഖസൗകര്യങ്ങളുടെ പാരമ്യതയും നരകം കൊടിയ ശിക്ഷയുടെ സങ്കേതവുമായിരിക്കുമെന്ന ധാരണ വളര്‍ത്താനാവശ്യമായ സൂചനകളും വിവരണങ്ങളുമാണ് വിശുദ്ധ ഖുര്‍ആനിലും പ്രവാചകവചനങ്ങളിലുമുള്ളത്. ഒരു കണ്ണും കാണാത്തതും ഒരു കാതും കേള്‍ക്കാത്തതും ഒരു മനസ്സും സങ്കല്‍പിക്കാത്തതുമായ സുഖാസ്വാദ്യതകളായിരിക്കും സ്വര്‍ഗത്തിലുണ്ടാവുകയെന്ന് അവ വ്യക്തമാക്കുന്നു. പരലോകത്തെ സ്ഥിതി ഭൂമിയില്‍ വച്ച് പൂര്‍ണമായും ഗ്രഹിക്കാനാവില്ലെന്നാണ് ഇത് തെളിയിക്കുന്നത്.
എന്നാല്‍ സ്വര്‍ഗാവകാശികളായ സുകര്‍മികള്‍ വാര്‍ധക്യത്തിന്റെ വിവശതയോ രോഗത്തിന്റെ പ്രയാസമോ വൈരൂപ്യത്തിന്റെ അലോസരമോ മറ്റെന്തെങ്കിലും വിഷമതകളോ ഒട്ടുമനുഭവിക്കുകയില്ല. നരകാവകാശികളായ ദുഷ്‌കര്‍മികള്‍ നേരെ മറിച്ചുമായിരിക്കും. സങ്കല്‍പിക്കാനാവാത്ത ദുരന്തങ്ങളും ദുരിതങ്ങളുമായിരിക്കും അവരെ ആവരണം ചെയ്യുക.
\section{ അങ്ങനെയാണെങ്കില്‍ അല്‍പകാലം കഴിയുമ്പോള്‍ സ്വര്‍ഗജീവിതത്തോടും മടുപ്പനുഭവപ്പെടുമല്ലോ?}
 ഇപ്പോഴുള്ള മാനസികാവസ്ഥയുടെയും വികാരവിചാരങ്ങളുടെയും അടിസ്ഥാനത്തിലാണ്, അല്‍പകാലം സ്വര്‍ഗീയ സുഖജീവിതം നയിക്കുമ്പോള്‍ മടുപ്പു തോന്നുമെന്ന് പറയുന്നത്. ഭൂമിയില്‍ നമുക്ക് സന്തോഷവും സന്താപവും സ്‌നേഹവും വെറുപ്പും പ്രത്യാശയും നിരാശയും അസൂയയും പകയും കാരുണ്യവും ക്രൂരതയും അതുപോലുള്ള വിവിധ വികാരങ്ങളും അനുഭവപ്പെടാറുണ്ട്. എന്നാല്‍ സ്വര്‍ഗത്തിലെ മാനസികാവസ്ഥ ഒരിക്കലും ഇതുപോലെയാകില്ല. അവിടെ വെറുപ്പോ വിദ്വേഷമോ അസൂയയോ നിരാശയോ മറ്റു വികല വികാരങ്ങളോ ഒരിക്കലും ഉണ്ടാവുകയില്ല. ഇക്കാര്യം ഖുര്‍ആനും പ്രവാചക വചനങ്ങളും അസന്ദിഗ്ധമായി വ്യക്തമാക്കിയിട്ടുണ്ട്. മനംമടുപ്പ് ഉള്‍പ്പെടെ എന്തെങ്കിലും അഹിതകരമായ അനുഭവമുണ്ടാവുന്ന ഇടമൊരിക്കലും സ്വര്‍ഗമാവുകയില്ല. ''ആഗ്രഹിക്കുന്നതെന്തും അവിടെ ഉണ്ടാകു''മെന്ന് (41: 31) പറഞ്ഞാല്‍ മടുപ്പില്ലാത്ത മാനസികാവസ്ഥയും അതില്‍ പെടുമല്ലോ.
 \section{ഇസ്‌ലാം വിഭാവന ചെയ്യുന്ന സ്വര്‍ഗത്തെ സംബന്ധിച്ച വിവരണം കേട്ടപ്പോഴെല്ലാം അത് സമ്പന്ന സമൂഹത്തിന്റെ സുഖസമൃദ്ധമായ ജീവിതത്തിന് സമാനമായാണനുഭവപ്പെട്ടത്. ഭൂമിയിലെ ജീവിതം പോലെത്തന്നെയാണോ സ്വര്‍ഗജീവിതവും?}
 അഭൗതികമായ ഏതിനെ കുറിച്ചും അറിവ് ലഭിക്കാനുള്ള ഏക മാധ്യമം ദിവ്യബോധനം മാത്രമത്രെ. അതിനാല്‍ ദൈവം, സ്വര്‍ഗം, നരകം, മാലാഖമാര്‍, പിശാചുക്കള്‍ എന്നിവയെക്കുറിച്ച് ദൈവദൂതന്മാരിലൂടെ ലഭിച്ച വിശദീകരണങ്ങളല്ലാതെ മറ്റൊന്നും ആര്‍ക്കും അറിയുകയില്ല. ഭൂമിയില്‍ ജീവിക്കുന്ന മനുഷ്യന്റെ പരിമിതിയില്‍നിന്നുകൊണ്ട് അവന് മനസ്സിലാക്കാന്‍ കഴിയുന്ന ഭാഷയിലും ശൈലിയിലുമാണ് ദൈവം അഭൗതിക കാര്യങ്ങളെ സംബന്ധിച്ച വിശദീകരണം നല്‍കിയത്. അതിനാല്‍ സ്വര്‍ഗത്തെക്കുറിച്ച് വിവരിക്കവെ, അല്ലലും അലട്ടുമൊട്ടുമില്ലാത്ത സംതൃപ്തവും ആഹ്ലാദഭരിതവുമായ ജീവിതമാണ് അവിടെ ഉണ്ടാവുകയെന്ന വസ്തുത വ്യക്തമാക്കിയ വിശുദ്ധ ഖുര്‍ആന്‍ ഇങ്ങനെ പറയുന്നു: ''അവിടെ നിങ്ങള്‍ ആശിക്കുന്നതെല്ലാം ലഭിക്കും. നിങ്ങള്‍ക്കു വേണമെന്ന് തോന്നുന്നതെല്ലാം നിങ്ങളുടേതാകും.'' (41: 31)
മനുഷ്യന്റെ സകല സങ്കല്‍പങ്ങള്‍ക്കും ഉപരിയായ സ്വര്‍ഗീയ സുഖത്തെ സംബന്ധിച്ച് പ്രവാചകന്‍ പറഞ്ഞത്, ഒരു കണ്ണും കാണാത്തതും ഒരു കാതും കേള്‍ക്കാത്തതും ഒരു മനസ്സും മനനം ചെയ്തിട്ടില്ലാത്തതുമെന്നാണ്. അതിനാല്‍ സ്വര്‍ഗജീവിതം ഏതു വിധമായിരിക്കുമെന്ന് ഇവിടെവച്ച് നമുക്ക് കണക്കുകൂട്ടുക സാധ്യമല്ല. എന്നാല്‍ പ്രയാസമൊട്ടുമില്ലാത്തതും മോഹങ്ങളൊക്കെയും പൂര്‍ത്തീകരിക്കപ്പെടുന്നതും സുഖവും സന്തോഷവും സംതൃപ്തിയും സമാധാനവും നിറഞ്ഞതുമായിരിക്കുമെന്നതില്‍ സംശയമില്ല.
\chapter{സുകര്‍മികളെല്ലാം സ്വര്‍ഗാവകാശികളാവേണ്ടതല്ലേ? }
 \section{ ഒരാള്‍ ദൈവത്തിലും മരണാനന്തര ജീവിതത്തിലുമൊന്നും വിശ്വസിക്കുന്നില്ല. അതേസമയം മദ്യപിക്കുകയോ വ്യഭിചരിക്കുകയോ ചെയ്യുന്നില്ല. ആരെയും ദ്രോഹിക്കുന്നില്ല. എല്ലാവര്‍ക്കും സാധ്യമാവുന്നത്ര ഉപകാരം ചെയ്ത് ജീവിക്കുന്നു. അയാള്‍ക്ക് സ്വര്‍ഗം ലഭിക്കുമോ?}
 ഏതൊരാള്‍ക്കും ഏതു കാര്യത്തിലും പരമാവധി ശ്രമിച്ച് ഫലപ്രാപ്തിയിലെത്തിയാല്‍ പോലും അയാള്‍ ഉദ്ദേശിച്ചതും ലക്ഷ്യം വച്ചതുമല്ലേ ലഭിക്കുകയുള്ളൂ. ദൈവത്തിലും മരണാനന്തര ജീവിതത്തിലും വിശ്വാസമില്ലാത്ത വ്യക്തി തിന്മയുപേക്ഷിക്കുന്നതും നന്മ പ്രവര്‍ത്തിക്കുന്നതും സ്വര്‍ഗം ലക്ഷ്യം വച്ചായിരിക്കില്ലെന്നതില്‍ സംശയമില്ല. അയാളങ്ങനെ ചെയ്യുന്നത് സമൂഹത്തില്‍ സല്‍പ്പേരും പ്രശസ്തിയും ലഭിക്കാനായിരിക്കാം. എങ്കില്‍ അതാണയാള്‍ക്ക് ലഭിക്കുക. അഥവാ മനസ്സംതൃപ്തിക്കും ആത്മനിര്‍വൃതിക്കും വേണ്ടിയാണെങ്കില്‍ അതാണുണ്ടാവുക. ദൈവത്തിന്റെ പ്രീതിയും പ്രതിഫലവും പ്രതീക്ഷിച്ചും ലക്ഷ്യം വച്ചും ജീവിക്കുന്നവര്‍ക്കേ അത് ലഭിക്കുകയുള്ളൂ. അതിനാല്‍ നേരത്തെ വ്യക്തമാക്കിയ പോലെ സ്വര്‍ഗമാഗ്രഹിച്ച് അതിന് നിശ്ചയിക്കപ്പെട്ട വഴിയിലൂടെ സഞ്ചരിക്കുന്നവരേ അവിടെ എത്തിച്ചേരുകയുള്ളൂ. എന്നാല്‍ ഇന്ന വ്യക്തി സ്വര്‍ഗത്തിലായിരിക്കും അല്ലെങ്കില്‍ നരകത്തിലായിരിക്കും എന്ന് നമുക്ക് തീരുമാനിക്കാനോ പറയാനോ സാധ്യമല്ല. അത് ദൈവനിശ്ചയമാണ്. അവനും അവന്‍ നിശ്ചയിച്ചു കൊടുക്കുന്ന ദൂതന്മാര്‍ക്കും മാത്രമേ അതറിയുകയുള്ളൂ.
 \chapter{അമുസ്‌ലിംകളെല്ലാം നരകത്തിലോ? }
 \section{ മുസ്‌ലിം സമുദായത്തില്‍ ജനിക്കുന്നവര്‍ക്ക് ദൈവത്തെയും പ്രവാചകനെയും വേദഗ്രന്ഥത്തെയും സ്വര്‍ഗനരകങ്ങളെയും സംബന്ധിച്ച അറിവ് സ്വാഭാവികമായും ലഭിക്കും. മറ്റുള്ളവര്‍ക്കത് കിട്ടുകയില്ല. അതിനാല്‍ ആ അറിവ് ലഭിക്കാത്തതിന്റെ പേരില്‍ അതനുസരിച്ച് ജീവിക്കാന്‍ സാധിക്കാത്തവരൊക്കെ നരകത്തിലായിരിക്കുമെന്നാണോ പറയുന്നത്?}
 മുസ്‌ലിം സമുദായത്തില്‍ ജനിക്കുകവഴി, ദൈവത്തെയും ദൈവിക ജീവിത വ്യവസ്ഥയെയും സംബന്ധിച്ച വ്യക്തമായ അറിവു ലഭിച്ച ശേഷം അതനുസരിച്ച് ജീവിക്കാത്തവന്‍ സത്യനിഷേധി (കാഫിര്‍) യാണ്. അവര്‍ക്ക് മരണശേഷം കൊടിയ ശിക്ഷയുണ്ടാകുമെന്ന് ഖുര്‍ആന്‍ മുന്നറിയിപ്പ് നല്‍കിയിട്ടുണ്ട്. മാത്രമല്ല, ദൈവത്തെയും ദൈവിക ജീവിതക്രമത്തെയും സംബന്ധിച്ച് അറിവുള്ളവരെല്ലാം മറ്റുള്ളവരെ അതറിയിച്ചുകൊടുക്കാന്‍ ബാധ്യസ്ഥരാണ്. ഈ ബാധ്യത നിര്‍വഹിച്ചില്ലെങ്കില്‍ അതിന്റെ പേരിലും പരലോകത്ത് അവര്‍ ശിക്ഷാര്‍ഹരായിരിക്കും.
എന്നാല്‍ ദൈവത്തെയും ദൈവിക മതത്തെയും സംബന്ധിച്ച് ഒട്ടും കേട്ടറിവു പോലുമില്ലാത്തവര്‍ ശിക്ഷിക്കപ്പെടുമെന്നോ നരകാവകാശികളാകുമെന്നോ ഇസ്‌ലാം പഠിപ്പിക്കുന്നില്ല. ഖുര്‍ആനോ പ്രവാചക ചര്യയോ അങ്ങനെ പറയുന്നുമില്ല. മറിച്ച്, ദിവ്യ സന്ദേശം വന്നെത്തിയിട്ടില്ലാത്തവര്‍ ശിക്ഷിക്കപ്പെടില്ലെന്നാണ് വേദഗ്രന്ഥം പഠിപ്പിക്കുന്നത്. പതിനേഴാം അധ്യായം പതിനഞ്ചാം വാക്യത്തിലിങ്ങനെ കാണാം: ''ആര്‍ സന്മാര്‍ഗം സ്വീകരിക്കുന്നുവോ, അതവന്റെ തന്നെ ഗുണത്തിനു വേണ്ടിയാകുന്നു. ആര്‍ ദുര്‍മാര്‍ഗിയാകുന്നുവോ, അതിന്റെ ദോഷവും അവനുതന്നെ. ഭാരം വഹിക്കുന്നവരാരുംതന്നെ ഇതരന്റെ ഭാരം വഹിക്കുകയില്ല. (സന്മാര്‍ഗം കാണിക്കാനായി) ദൈവദൂതന്‍ നിയോഗിതനാവുന്നതുവരെ നാമാരെയും ശിക്ഷിക്കാറുമില്ല.''
അതേസമയം ദൈവത്തെ സംബന്ധിച്ച് കേള്‍ക്കാത്തവരോ സാമാന്യധാരണയില്ലാത്തവരോ ഉണ്ടാവുകയില്ല. അവര്‍ ദൈവത്തെക്കുറിച്ച് കൂടുതലന്വേഷിക്കാനും ആ ദൈവം വല്ല ജീവിതമാര്‍ഗവും നിശ്ചയിച്ചുതന്നിട്ടുണ്ടോ എന്ന് പരിശോധിക്കാനും ബാധ്യസ്ഥരാണ്. അപ്രകാരം തന്നെ സ്വര്‍ഗമുണ്ടെന്നും നിശ്ചിത മാര്‍ഗത്തിലൂടെ നീങ്ങുന്നവര്‍ക്കേ അത് ലഭിക്കുകയുള്ളൂവെന്നുമുള്ള കാര്യം കേട്ടറിഞ്ഞവരൊക്കെയും അതേക്കുറിച്ച് പഠിക്കാന്‍ കടപ്പെട്ടവരാണ്. അതിന്റെ ലംഘനം ശിക്ഷാര്‍ഹമായ കുറ്റമാവുക സ്വാഭാവികമാണല്ലോ.
\chapter{പരലോകത്തും സംവരണമോ? }
  \section{മുസ്‌ലിംകള്‍ മാത്രമേ സ്വര്‍ഗത്തില്‍ പ്രവേശിക്കുകയുള്ളൂവെന്നല്ലേ ഇസ്‌ലാം പറയുന്നത്? ഇത് തീര്‍ത്തും സങ്കുചിത വീക്ഷണമല്ലേ? പരലോകത്തും സംവരണമോ?}
 ഒരാള്‍ പരീക്ഷ പാസാകണമെന്നാഗ്രഹിക്കുന്നില്ല. പരീക്ഷക്കു വന്ന ചോദ്യങ്ങള്‍ക്ക് ഉത്തരമെഴുതുന്നുമില്ല. എങ്കില്‍ മറ്റെന്തൊക്കെ എഴുതിയാലും പരീക്ഷയില്‍ വിജയിക്കുകയില്ല. വിജയിക്കുമെന്ന് ആരും പ്രതീക്ഷിക്കുകയുമില്ല. അപ്രകാരംതന്നെ രോഗം മാറണമെന്ന് ആഗ്രഹിക്കുന്നില്ല; രോഗശമനത്തിന് നിര്‍ദേശിക്കപ്പെട്ട മരുന്ന് കഴിക്കുന്നുമില്ല. എന്നാലും രോഗം മാറണമെന്ന് ആരും പറയുകയില്ലല്ലോ. ഇവ്വിധംതന്നെ സ്വര്‍ഗം ലക്ഷ്യമാക്കാതെ, സ്വര്‍ഗലബ്ധിക്കു നിശ്ചയിക്കപ്പെട്ട മാര്‍ഗമവലംബിക്കാതെ ജീവിക്കുന്നവര്‍ക്ക് സ്വര്‍ഗം ലഭിക്കുകയില്ല. അത്തരക്കാര്‍ക്കും സ്വര്‍ഗം നല്‍കണമെന്ന് നീതിബോധമുള്ളവരാരും അവകാശപ്പെടുകയുമില്ല.
സ്വര്‍ഗം സജ്ജനങ്ങള്‍ക്കുള്ള ദൈവത്തിന്റെ ദാനമാണ്. വേദഗ്രന്ഥത്തില്‍ ദൈവദൂതന്മാരിലൂടെയാണ് അല്ലാഹു അത് വാഗ്ദാനം ചെയ്തത്. അത് ലഭ്യമാകാന്‍ വ്യക്തമായ മാര്‍ഗം നിശ്ചയിച്ചിട്ടുമുണ്ട്. അതിനാല്‍ ആര്‍ ദൈവം, സ്വര്‍ഗം, ദൈവദൂതന്മാര്‍, വേദഗ്രന്ഥം തുടങ്ങിയവയില്‍ യഥാവിധി വിശ്വസിച്ച് സ്വര്‍ഗം ലക്ഷ്യം വച്ച് അതിനു നിശ്ചയിക്കപ്പെട്ട വഴിയിലൂടെ സഞ്ചരിക്കുന്നുവോ അവര്‍ക്ക് സ്വര്‍ഗം ലഭിക്കും. ഇക്കാര്യത്തിലാരോടും ദൈവം ഒട്ടും വിവേചനം കാണിക്കുകയില്ല. എന്നാല്‍ സ്വര്‍ഗത്തില്‍ വിശ്വസിക്കുകയോ അത് ലക്ഷ്യം വെക്കുകയോ അത് വാഗ്ദാനം ചെയ്ത ദൈവത്തെയും ആ അറിവു നല്‍കിയ ദൈവദൂതനെയും വേദഗ്രന്ഥത്തെയും അംഗീകരിക്കുകയോ ചെയ്യാതെ, അതിനു നിശ്ചയിക്കപ്പെട്ട മാര്‍ഗമവലംബിക്കാതെ ജീവിക്കുന്നവര്‍ക്ക് അത് ലഭിക്കുമെന്ന് പ്രതീക്ഷിക്കാവതല്ല. ലഭിക്കണമെന്ന് പറയുന്നതിലൊട്ടും അര്‍ഥവുമില്ല. അതിനാലിതില്‍ സങ്കുചിതത്വത്തിന്റെയോ സംവരണത്തിന്റെയോ പ്രശ്‌നമില്ല. നിഷ്‌കൃഷ്ടമായ നീതിയാണ് ദീക്ഷിക്കപ്പെടുക.
\chapter{പരലോകം പരമാര്‍ഥമോ? }
  \section{മനുഷ്യന്‍ മരണമടഞ്ഞാല്‍ ചിലര്‍ മണ്ണില്‍ മറവുചെയ്യുന്നു. ഏറെ വൈകാതെ മൃതശരീരം മണ്ണോടു ചേരുന്നു. വേറെ ചിലര്‍ ജഡം ചിതയില്‍ ദഹിപ്പിക്കുന്നു. അതോടെ അത് ചാരമായി മാറുന്നു. നദിയിലൊഴുക്കപ്പെടുന്ന ശവശരീരങ്ങളെ മത്സ്യം വെട്ടിവിഴുങ്ങുകയും ചെയ്യുന്നു. ഈ വിധം അപ്രത്യക്ഷരാവുന്ന മനുഷ്യരെ വീണ്ടും ഉയിര്‍ത്തെഴുന്നേല്‍പിക്കുമെന്നല്ലേ ഇസ്‌ലാം അവകാശപ്പെടുന്നത്? ഇത് വിശ്വസനീയമാണോ? മനുഷ്യബുദ്ധിക്ക് നിരക്കുന്നതാണോ?}
 എന്നും എവിടെയും മഹാഭൂരിപക്ഷം ജനങ്ങളും ദൈവവിശ്വാസികളായിരുന്നു. അറിയപ്പെടുന്ന മനുഷ്യചരിത്രത്തില്‍ ഈശ്വരവിശ്വാസമില്ലാത്ത സമൂഹങ്ങളും ജനതകളും വളരെ വിരളമത്രെ. അതുകൊണ്ടുതന്നെ ദൈവദൂതന്മാര്‍ക്ക് അഭിമുഖീകരിക്കേണ്ടിവന്നത് നിരീശ്വരവാദികളെയായിരുന്നില്ല. മറിച്ച്, മരണാനന്തര ജീവിതത്തില്‍ വിശ്വാസമില്ലാത്ത ദൈവവിശ്വാസികളെയായിരുന്നു. പ്രവാചകന്മാര്‍ക്കെതിരെ അണിനിരന്ന വിവിധ കാലഘട്ടങ്ങളിലെ ജനങ്ങള്‍ പരലോകത്തെ അവിശ്വസിക്കുന്നവരും തള്ളിപ്പറയുന്നവരുമായിരുന്നു. അവരുന്നയിച്ച വാദങ്ങളും ചോദ്യങ്ങളും വിശുദ്ധഖുര്‍ആന്‍ ഇങ്ങനെ ഉദ്ധരിക്കുന്നു:
''അവര്‍ ദൈവനാമത്തില്‍ ശക്തമായി ആണയിട്ടു പറയുന്നു: മരിച്ചുപോകുന്നവരെ അല്ലാഹു വീണ്ടും ജീവിപ്പിച്ച് എഴുന്നേല്‍പിക്കുകയില്ല.'' (അധ്യായം 16, വാക്യം 38).
'ജീവിതമെന്നാല്‍ നമ്മുടെ ഈ ഐഹികജീവിതം മാത്രമേയുള്ളൂ. മരണാനന്തരം നാമൊരിക്കലും പുനരുജ്ജീവിപ്പിക്കപ്പെടാന്‍ പോകുന്നില്ല'' (അധ്യായം 6, വാക്യം 29).
''നിങ്ങള്‍ മരിച്ച് മണ്ണും അസ്ഥിയുമായ ശേഷം വീണ്ടും ഉയിര്‍ത്തെഴുന്നേല്‍പിക്കപ്പെടുമെന്ന് ഇയാള്‍ നിങ്ങളോട് പറയുന്നോ? എന്നാലത് വളരെ വളരെ വിദൂരം തന്നെ. ഇയാള്‍ നിങ്ങളോട് വാഗ്ദാനം ചെയ്യുന്നത് വളരെ വിദൂരമത്രെ! നമ്മുടെ ഐഹികജീവിതമല്ലാതെ ജീവിതമില്ല. ഇവിടെ നാം മരിക്കുന്നു. ജീവിക്കുന്നു. നാമൊരിക്കലും ഉയിര്‍ത്തെഴുന്നേല്‍പിക്കപ്പെടുകയില്ല.'' (അധ്യായം 23, വാക്യം 36, 37).
''അവര്‍ അവരുടെ മുന്‍ഗാമികള്‍ പറഞ്ഞതുതന്നെ പറയുന്നു. അവര്‍ പറയുന്നു: ഞങ്ങള്‍ മരിച്ച് മണ്ണും എല്ലുമായി മാറിക്കഴിഞ്ഞാല്‍ പിന്നെയും ഉയിര്‍ത്തെഴുന്നേല്‍പിക്കപ്പെടുമെന്നോ? ഈ വാഗ്ദാനം ഞങ്ങള്‍ ഏറെ കേട്ടിട്ടുള്ളതാണ്. പണ്ട് ഞങ്ങളുടെ പൂര്‍വപിതാക്കളും കേട്ടുപോന്നിട്ടുണ്ട്. അതാവട്ടെ കേവലം കെട്ടുകഥകള്‍ മാത്രമാകുന്നു''(അധ്യായം 23, വാക്യം 82, 83).
''അവര്‍ ചോദിക്കുന്നു: ഞങ്ങള്‍ മരിച്ച് മണ്ണും അസ്ഥിപഞ്ജരവുമായി മാറിക്കഴിഞ്ഞാല്‍ വീണ്ടും ഉയിര്‍ത്തെഴുന്നേല്‍പിക്കപ്പെടുമെന്നോ? ഞങ്ങളുടെ പൂര്‍വപിതാക്കളും അങ്ങനെ ഉയിര്‍ത്തെഴുന്നേല്‍പിക്കപ്പെടുമോ?'' (അധ്യായം 37, വാക്യം 16, 17).
പുരാതന കാലം മുതല്‍ക്കുതന്നെ ദൈവവിശ്വാസികളായ പലരും പരലോകത്തില്‍ വിശ്വസിച്ചിരുന്നില്ലെന്ന് ഉപര്യുക്ത വാക്യങ്ങള്‍ വ്യക്തമാക്കുന്നു. ഇന്നത്തെ അവസ്ഥയും ഭിന്നമല്ല. അവിശ്വാസികള്‍ ഉയിര്‍ത്തെഴുന്നേല്‍പിനെ നിഷേധിച്ചും അവിശ്വസിച്ചും സംശയിച്ചും ഉന്നയിച്ച ചോദ്യങ്ങള്‍ക്ക് വിശുദ്ധ ഖുര്‍ആന്‍ വ്യക്തമായ മറുപടി നല്‍കുന്നു. അവര്‍ ചോദിക്കുന്നു: ''ഞങ്ങള്‍ കേവലം അസ്ഥികളും മണ്ണുമായിത്തീര്‍ന്നാല്‍ പിന്നെ വീണ്ടും പുതിയ സൃഷ്ടിയായി എഴുന്നേല്‍പിക്കപ്പെടുമെന്നോ?'' അവരോടു പറയുക: നിങ്ങള്‍ കല്ലോ തുരുമ്പോ ആയിക്കൊള്ളുക. അല്ലെങ്കില്‍ ജീവനുള്‍ക്കൊള്ളാന്‍ തീരെ അസാധ്യമായതെന്ന് നിങ്ങള്‍ക്ക് തോന്നുന്ന മറ്റെന്തെങ്കിലും കടുത്ത സൃഷ്ടിയായിക്കൊള്ളുക. എന്നാലും നിങ്ങള്‍ എഴുന്നേല്‍പിക്കപ്പെടും.'' അവര്‍ തീര്‍ച്ചയായും ചോദിക്കും: ''ആരാണ് ഞങ്ങളെ വീണ്ടും ജീവിതത്തിലേക്ക് മടക്കുക?'' പറയുക: ''ആദ്യതവണ സൃഷ്ടിച്ചവനാരോ അവന്‍ തന്നെ.'' അവര്‍ തലയിളക്കി പരിഹാസത്തോടെ ചോദിക്കും: ''ഓഹോ, അതെപ്പോഴാണുണ്ടാവുക?'' പറയുക: അദ്ഭുതമെന്ത്? അടുത്തുതന്നെ അത് സംഭവിച്ചേക്കാം. ദൈവം നിങ്ങളെ വിളിക്കുന്ന നാളില്‍ അതിനുത്തരമായി നിങ്ങളവനെ സ്തുതിച്ചുകൊണ്ട് പുറപ്പെട്ടുവരും. ''ഞങ്ങള്‍ അല്‍പനേരം മാത്രമേ ഈ അവസ്ഥയില്‍ കഴിഞ്ഞിട്ടുള്ളൂ'' എന്നായിരിക്കും അന്നേരം നിങ്ങളുടെ തോന്നല്‍'' (അധ്യായം 17, വാക്യം 49-52).
''ഞങ്ങള്‍ കേവലം അസ്ഥികളും മണ്ണുമായിക്കഴിഞ്ഞ ശേഷം പുതിയ സൃഷ്ടിയായി വീണ്ടും എഴുന്നേല്‍പിക്കപ്പെടുകയോ'' എന്നു ചോദിച്ചതിനുള്ള പ്രതിഫലമാണിത്. ഭൂലോകത്തെയും വാനലോകങ്ങളെയും സൃഷ്ടിച്ച അല്ലാഹുവിന് ഇവരെപ്പോലുള്ളവരെയും സൃഷ്ടിക്കുവാന്‍ തീര്‍ച്ചയായും കഴിവുണ്ടെന്ന് ഇവര്‍ക്ക് മനസ്സിലായിട്ടില്ലേ? അവരെ പുനരുജ്ജീവിപ്പിച്ച് ഒരുമിച്ചുകൂട്ടുന്നതിന് അവനൊരു സമയം നിര്‍ണയിച്ചുവച്ചിട്ടുണ്ട്. അതുണ്ടാവുമെന്നതില്‍ സംശയമില്ല. പക്ഷേ, അതിനെ ധിക്കരിക്കുക തന്നെ വേണമെന്ന് ധിക്കാരികള്‍ക്ക് ശാഠ്യമുണ്ട്''(അധ്യായം 17, വാക്യം 98, 99).
''മനുഷ്യരേ, മരണാനന്തര ജീവിതത്തെക്കുറിച്ച് വല്ല സംശയവുമുണ്ടെങ്കില്‍ നിങ്ങള്‍ മനസ്സിലാക്കുക: ആദിയില്‍ നിങ്ങളെ നാം സൃഷ്ടിച്ചത് മണ്ണില്‍നിന്നാണ്. പിന്നെ രേതസ്‌കണത്തില്‍നിന്ന്. പിന്നെ ഒട്ടിപ്പിടിക്കുന്നതില്‍നിന്ന്, പിന്നെ രൂപം പ്രാപിച്ചതും അല്ലാത്തതുമായ മാംസപിണ്ഡത്തില്‍നിന്ന്. ഈ വിവരണം നിങ്ങള്‍ക്ക് യാഥാര്‍ഥ്യം വ്യക്തമാകാനത്രെ. നാമുദ്ദേശിക്കുന്ന ബീജത്തെ ഒരു നിശ്ചിത അവധി വരെ ഗര്‍ഭാശയങ്ങളില്‍ നിവസിപ്പിക്കുന്നു. പിന്നെ നിങ്ങളെ ശിശുവായി പുറത്തുകൊണ്ടുവരുന്നു. പിന്നെ നിങ്ങള്‍ യൗവനം പ്രാപിക്കുന്നു. നിങ്ങളില്‍ ചിലര്‍ നേരത്തെ തന്നെ തിരിച്ചുവിളിക്കപ്പെടുന്നു. ചിലരാവട്ടെ എല്ലാം അറിഞ്ഞശേഷം ഒന്നും അറിയാത്തവരായിത്തീരാന്‍, മോശമായ പ്രായാധിക്യത്തിലേക്ക് നയിക്കപ്പെടുന്നു. ഭൂമി വരണ്ടുകിടക്കുന്നതായി നീ കാണുന്നു. പിന്നെ നാമതില്‍ മഴ വര്‍ഷിപ്പിച്ചാല്‍ പെട്ടെന്നത് തുടികൊള്ളുന്നു. പുഷ്പിണിയാവുന്നു. കൗതുകമാര്‍ന്ന സകലയിനം ചെടികളെയും മുളപ്പിച്ചുതുടങ്ങുന്നു. അല്ലാഹു തന്നെയാകുന്നു യാഥാര്‍ഥ്യം. അവന്‍ നിര്‍ജീവമായതിനെ ജീവിപ്പിക്കുന്നു. അവന്‍ സകലതിനും കഴിവുള്ളവനാണെന്നതിനാലാണിതൊക്കെയും ഉണ്ടാവുന്നത്. അതിനാല്‍ പുനരുത്ഥാനവേള വരികതന്നെ ചെയ്യും. അതില്‍ സംശയമേയില്ല. ഖബ്‌റിലുള്ളവരെയെല്ലാം അല്ലാഹു ഉയിര്‍ത്തെഴുന്നേല്‍പിക്കും, തീര്‍ച്ച''(അധ്യായം 22, വാക്യം 57).
''നിങ്ങള്‍ ലോകരെയെല്ലാം സൃഷ്ടിക്കുകയും പിന്നെ പുനരുജ്ജീവിപ്പിക്കുകയും ചെയ്യുകയെന്നത് അത് അവന് ഒരു ജീവിയെ സൃഷ്ടിക്കുകയും പുനരുജ്ജീവിപ്പിക്കുകയും ചെയ്യുന്നതുപോലെയേയുള്ളൂ.'' (അധ്യായം 31, വാക്യം 28)
''സത്യനിഷേധികള്‍ വലിയ വാദമായി ഉന്നയിച്ചു, മരണാനന്തരം തങ്ങളൊരിക്കലും പുനരുജ്ജീവിപ്പിക്കപ്പെടുകയില്ലെന്ന്. അവരോടു പറയുക: അല്ല; എന്റെ നാഥനാണ് സത്യം! നിങ്ങള്‍ തീര്‍ച്ചയായും പുനരുജ്ജീവിപ്പിക്കപ്പെടുകതന്നെ ചെയ്യും. അങ്ങനെ ചെയ്യല്‍ അല്ലാഹുവിന് തീര്‍ത്തും അനായാസകരമത്രെ.'' (അധ്യായം 64, വാക്യം 7).
''ആകാശഭൂമികളെ സൃഷ്ടിച്ചവനും അവയുടെ സൃഷ്ടിയാല്‍ ക്ഷീണിക്കാത്തവനുമായ അല്ലാഹു തീര്‍ച്ചയായും മരിച്ചവരെ ജീവിപ്പിക്കാന്‍ കഴിവുള്ളവനാണെന്ന കാര്യം ഈ ജനത്തിന് മനസ്സിലായിട്ടില്ലെന്നോ? എന്തുകൊണ്ടില്ല? നിശ്ചയം, അവന്‍ സകല സംഗതികള്‍ക്കും കഴിവുള്ളവനാകുന്നു'' (അധ്യായം 46, വാക്യം 33).
''മനുഷ്യന്‍ വിചാരിക്കുന്നുണ്ടോ, അവന്‍ വെറുതെ വിടപ്പെടുമെന്ന്? അവന്‍ വിസര്‍ജിക്കപ്പെട്ട നിസ്സാരമായ ശുക്ലകണമായിരുന്നില്ലേ? പിന്നീടവന്‍ ഒട്ടിപ്പിടിക്കുന്ന വസ്തുവായി. അനന്തരം അല്ലാഹു അവന്റെ ശരീരം സൃഷ്ടിച്ചു. അവയവങ്ങള്‍ സംവിധാനിച്ചു. എന്നിട്ടതില്‍നിന്ന് സ്ത്രീയുടെയും പുരുഷന്റെയും രണ്ടു വര്‍ഗങ്ങളുണ്ടാക്കി. അവന്‍ മരിച്ചവരെ പുനരുജ്ജീവിപ്പിക്കാന്‍ കഴിവുള്ളവനല്ലെന്നോ?''(അധ്യായം 75, വാക്യം 36-40).
''അല്ലാഹുവിന്റെ ദൃഷ്ടാന്തങ്ങളിലൊന്ന് ഇതാകുന്നു: എന്തെന്നാല്‍ ഭൂമിയെ ചൈതന്യമറ്റു കിടക്കുന്നതായി നീ കാണുന്നു. പിന്നെ നാമതില്‍ മഴ വര്‍ഷിപ്പിച്ചാലോ, പെട്ടെന്നതാ അത് ചലനം കൊള്ളുകയും വളര്‍ന്നു വികസിക്കുകയും ചെയ്യുന്നു. നിശ്ചയം, ഈ മൃതഭൂമിയെ സജീവമാക്കിയതാരോ, അവന്‍ മരിച്ചുപോയവരെയും ജീവിപ്പിക്കുന്നവനാകുന്നു. അവന്‍ സകല സംഗതികള്‍ക്കും കഴിവുറ്റവനല്ലോ'' (അധ്യായം 41, വാക്യം 39).
''മനുഷ്യന്‍ വിചാരിക്കുന്നുവോ, അവന്റെ അസ്ഥികളെ സംഘടിപ്പിക്കാന്‍ നമുക്കാവില്ലെന്ന്. എന്തുകൊണ്ടാവില്ല? നാമവന്റെ വിരല്‍തുമ്പുകള്‍ വരെ കൃത്യമായി നിര്‍മിക്കാന്‍ കഴിവുള്ളവനല്ലോ.'' (അധ്യായം 75, വാക്യം 3, 4).
ചുരുക്കത്തില്‍, ഇല്ലായ്മയില്‍നിന്ന് മനുഷ്യനെ സൃഷ്ടിച്ച ദൈവത്തിന് അവനെ പുനരുജ്ജീവിപ്പിക്കുക ഒട്ടും പ്രയാസകരമായ കാര്യമല്ല. അറുനൂറു കോടി മനുഷ്യര്‍ക്ക് അറുനൂറു കോടി കൈവിരലുകളും വ്യതിരിക്തമായ തലമുടിയും രക്തത്തുള്ളികളും ഗന്ധവുമൊക്കെ നല്‍കി സൃഷ്ടികര്‍മം നിര്‍വഹിച്ച ദൈവത്തിന് മനുഷ്യനെ ഉയിര്‍ത്തെഴുന്നേല്‍പിക്കുക എന്നത് തീര്‍ത്തും അനായാസകരമത്രെ. നിര്‍ജീവമായ ഭൂമി മഴ കിട്ടിയാല്‍ സജീവമാകുന്ന പോലെ മരിച്ചു മണ്ണായ മനുഷ്യരെ അല്ലാഹു വിളിക്കുമ്പോള്‍ പുനരുജ്ജീവിച്ച് ഓടിയെത്തുന്നുവെന്നത് തീരെ അയുക്തികമോ അവിശ്വസനീയമോ അല്ല. ശരീരത്തിന്റെ നേരിയ ഒരു അംശമുപയോഗിച്ച് ക്ലോണിംഗിലൂടെ പൂര്‍ണ മനുഷ്യനെ സൃഷ്ടിക്കാന്‍ ശ്രമം നടക്കുന്ന ലോകത്ത്, സര്‍വശക്തനായ ദൈവം മനുഷ്യനെ പുനഃസൃഷ്ടിക്കുമെന്ന സത്യം അംഗീകരിക്കാന്‍ വിസമ്മതിക്കുന്നതാണ് അയുക്തികവും വിസ്മയകരവുമായ കാര്യം.
\chapter{പരലോകമുണ്ടെന്നതിന് തെളിവുണ്ടോ? }
  \section{മരണശേഷം നാം വീണ്ടും പുനരുജ്ജീവിപ്പിക്കപ്പെടുന്ന പരലോകമുണ്ടെന്നതിന് വല്ല തെളിവുമുണ്ടോ? ഉണ്ടെന്ന വിശ്വാസം തീര്‍ത്തും അയുക്തികമല്ലേ?}
 അറിവ് ആര്‍ജിക്കാന്‍ നമുക്കുള്ളത് അഞ്ച് ഇന്ദ്രിയങ്ങളാണ്. ഇവിടെ ഭൗതിക വിദ്യതന്നെ വിവിധയിനമാണ്. അവയോരോന്നിന്റെയും വാതില്‍ തുറക്കാന്‍ വ്യത്യസ്ത താക്കോലുകള്‍ വേണം. ഓരോന്നിനും സവിശേഷമായ മാനദണ്ഡങ്ങള്‍ അനിവാര്യമാണ്. ഗണിതശാസ്ത്രം പഠിക്കാനുപയോഗിക്കുന്ന മാനദണ്ഡമുപയോഗിച്ച് ശരീരശാസ്ത്രം പഠിക്കുക സാധ്യമല്ല. ഗണിതശാസ്ത്രത്തിലെ തന്നെ വിവിധ വശങ്ങള്‍ക്ക് വ്യത്യസ്ത മാധ്യമങ്ങള്‍ വേണം. ഒരു വൃത്തത്തിന് മുന്നൂറ്ററുപത് ഡിഗ്രിയും ത്രികോണത്തിന് നൂറ്റിയെണ്‍പതു ഡിഗ്രിയുമാണെന്ന സങ്കല്‍പം സ്വീകരിക്കാതെ ക്ഷേത്രഗണിതം അഭ്യസിക്കാനാവില്ല. ബീജഗണിതത്തിന് ഭിന്നമായ മാനദണ്ഡം അനിവാര്യമത്രെ. ഗോളശാസ്ത്രം, ഭൂഗര്‍ഭശാസ്ത്രം, ഭൂമിശാസ്ത്രം, സസ്യശാസ്ത്രം, ജന്തുശാസ്ത്രം പോലുള്ളവയ്‌ക്കെല്ലാം പ്രത്യേകം പ്രത്യേകം മാധ്യമങ്ങളാവശ്യമാണ്.
ഭൗതിക വിദ്യയുടെ വിവിധ വശങ്ങള്‍ക്ക് വ്യത്യസ്ത മാനദണ്ഡങ്ങള്‍ വേണമെന്നിരിക്കെ, ആധ്യാത്മിക ജ്ഞാനം നേടാന്‍ തീര്‍ത്തും ഭിന്നമായ മാര്‍ഗം അനിവാര്യമാണ്. ദൈവം, പരലോകം, സ്വര്‍ഗം, നരകം, മാലാഖ, പിശാച് പോലുള്ള അഭൗതിക കാര്യങ്ങളെ സംബന്ധിച്ച അറിവ് ആര്‍ജിക്കാന്‍ മനുഷ്യന്റെ വശം സ്വന്തമായൊരു മാധ്യമവുമില്ല; ചരിത്രത്തിന്റെ വ്യത്യസ്ത ദശാസന്ധികളില്‍ വിവിധ ദേശങ്ങളില്‍ നിയോഗിതരായ ദൈവദൂതന്മാരിലൂടെ ലഭ്യമായ ദിവ്യസന്ദേശങ്ങളല്ലാതെ. ദിവ്യബോധനമാകുന്ന ആറാം സ്രോതസ്സിലൂടെ അറിവ് ലഭിക്കുന്ന അനുഗൃഹീതരായ പ്രവാചകന്മാര്‍ പറയുന്ന കാര്യങ്ങള്‍ പഞ്ചേന്ദ്രിയങ്ങള്‍ മാത്രമുള്ളവര്‍ നിഷേധിക്കുന്നതും നിരാകരിക്കുന്നതും കണ്ണുള്ളവര്‍ പറയുന്നതിനെ കുരുടന്മാര്‍ ചോദ്യം ചെയ്യുന്നതുപോലെയാണ്. ഇലയുടെ നിറം പച്ചയും കാക്കയുടെ നിറം കറുപ്പും പാലിന്റെ വര്‍ണം വെളുപ്പുമാണെന്ന് നാം പറയുമ്പോള്‍ തങ്ങളത് കണ്ടറിഞ്ഞിട്ടില്ലെന്ന കാരണത്താല്‍ അന്ധന്മാര്‍ അതംഗീകരിക്കുന്നില്ലെങ്കില്‍ വര്‍ണപ്രപഞ്ചം അവര്‍ക്ക് തീര്‍ത്തും അന്യവും അപ്രാപ്യവുമായിരിക്കും. അതുപോലെ തന്നെയാണ് ദിവ്യസന്ദേശങ്ങളെ അംഗീകരിക്കാത്തവരുടെയും അവസ്ഥ. അഭൗതിക ജ്ഞാനം അവര്‍ക്ക് അന്യവും എന്നും അജ്ഞാതവുമായിരിക്കും.
മരണശേഷമുള്ള മറുലോകത്തെ കുറിച്ചും അവിടത്തെ സ്വര്‍ഗനരകങ്ങളെ സംബന്ധിച്ചും മനുഷ്യരാശിക്ക് അറിവു നല്‍കിയത് ചരിത്രത്തിന്റെ വിവിധ ഘട്ടങ്ങളില്‍ വിവിധ നാടുകളില്‍ നിയോഗിതരായ ദൈവദൂതന്മാരാണ്. അവര്‍ എല്ലാവരും അതിന്റെ അനിവാര്യത ഊന്നിപ്പറഞ്ഞിട്ടുണ്ട്. അതോടൊപ്പം ഭൗതിക പ്രപഞ്ചത്തിന്റെ ഘടനയും അതിലെ മനുഷ്യന്റെ അവസ്ഥയും അത്തരമൊരു ലോകത്തിന്റെ അനിവാര്യതയിലേക്ക് വെളിച്ചം വീശുകയും ചെയ്യുന്നു.
നീതി നടക്കണമെന്നാഗ്രഹിക്കാത്തവര്‍ നന്നേ കുറവാണ്. അനീതി പ്രവര്‍ത്തിക്കുന്നവര്‍ പോലും നീതിക്കുവേണ്ടി വാദിക്കുന്നു. തങ്ങള്‍ നീതിയുടെ വക്താക്കളെന്ന് അവകാശപ്പെടുകയും ചെയ്യുന്നു. ഒരാള്‍ വധിക്കപ്പെട്ടാല്‍ അയാളുടെ ആശ്രിതര്‍ കൊലയാളിയെ ശിക്ഷിക്കണമെന്ന് കൊതിക്കുന്നു. നീതിപുലരാനുള്ള മനുഷ്യരാശിയുടെ അദമ്യമായ ആഗ്രഹമാണല്ലോ ലോകത്ത് നിയമവും നീതിന്യായ വ്യവസ്ഥയും കോടതികളും നിയമപാലകരുമെല്ലാമുണ്ടാകാന്‍ കാരണം. നീതി നടക്കണമെന്ന് പറയാത്ത ആരുമുണ്ടാവില്ലെന്നര്‍ഥം. എന്നാല്‍, പൂര്‍ണാര്‍ഥത്തില്‍ നീതി പുലരുന്ന നാടും സമൂഹവുമില്ല. ആരെത്ര വിചാരിച്ചാലും നിഷ്‌കൃഷ്ടമായ നീതി നടപ്പാക്കുക സാധ്യവുമല്ല. ഒരാളെ കൊന്നാല്‍ പരമാവധി സാധിക്കുക കൊലയാളിയെ കൊല്ലാനാണ്. എന്നാല്‍ അതുകൊണ്ട് വധിക്കപ്പെട്ടവന്റെ, നിത്യവൈധവ്യത്തിന്റെ നിതാന്ത വേദന അനുഭവിക്കുന്ന വിധവയ്ക്ക് ഭര്‍ത്താവിനെയോ അനാഥത്വത്തിന്റെ പ്രയാസം പേറുന്ന മക്കള്‍ക്ക് പിതാവിനെയോ ലഭിക്കുകയില്ല. ഘാതകനെ വധിക്കുന്നതുപോലും കൊലയ്ക്കുള്ള യഥാര്‍ഥ പരിഹാരമോ പ്രതിവിധിയോ അല്ലെന്നര്‍ഥം. അത് ഭൂമിയില്‍ സമാധാനം സ്ഥാപിക്കാനുള്ള ഉപാധി മാത്രമത്രെ.
ഇതുതന്നെ കൊലയാളി ശിക്ഷിക്കപ്പെട്ടാലാണ്. എന്നാല്‍ ഭൂമിയിലെ യഥാര്‍ഥ അവസ്ഥ എന്താണ്? എന്നും എങ്ങും കൊലയാളികള്‍ സ്വൈരവിഹാരം നടത്തുന്നു. കൊള്ളക്കാര്‍ രംഗം കൈയടക്കുന്നു. ചൂഷകന്‍മാര്‍ മാന്യത ചമയുന്നു. അഴിമതിക്കാര്‍ സസുഖം വാഴുന്നു. അക്രമികള്‍ ആധിപത്യം നടത്തുന്നു. വഞ്ചകന്മാര്‍ വിഘ്‌നമൊട്ടുമില്ലാതെ നാട്ടിലെങ്ങും വിഹരിക്കുന്നു. നീതി നടത്തേണ്ട ന്യായാധിപന്മാര്‍ അനീതിക്ക് കൂട്ടുനില്‍ക്കുന്നു. പണത്തിനും പദവിക്കുമായി പരാക്രമികളുടെ പക്ഷം ചേരുന്നു. ഭരണാധികാരികള്‍ പൊതുമുതല്‍ കൊള്ളയടിക്കുന്നു. പൊതുജനങ്ങളെവിടെയും പീഡിപ്പിക്കപ്പെടുന്നു. അടിക്കടി അനീതിക്കും അക്രമത്തിനും അടിപ്പെടുന്നു. നീതിക്കായുള്ള അവരുടെ അര്‍ഥനകളൊക്കെയും വ്യര്‍ഥമാവുന്നു. അതിനാല്‍ നീതി പലപ്പോഴും മരീചിക പോലെ അപ്രാപ്യമത്രെ. മഹാഭൂരിപക്ഷത്തിനും തീര്‍ത്തും അന്യവും.
നന്മ നടത്തുന്നവര്‍ ഇവിടെ കൊടിയ കഷ്ടനഷ്ടങ്ങള്‍ക്കിരയാവുന്നു. ആരെയും അല്‍പവും അലോസരപ്പെടുത്താത്തവര്‍ അക്രമിക്കപ്പെടുന്നു. നിസ്വാര്‍ഥരായി നിലകൊള്ളുന്നവര്‍ നിരന്തരം മര്‍ദിക്കപ്പെടുന്നു. എന്നാല്‍ വിദ്രോഹവൃത്തികളില്‍ വ്യാപൃതരാവുന്നവരോ, വിപത്തേതുമേല്‍ക്കാതെ സസുഖം വാഴുന്നു.
ഒരാളെ കൊന്നാല്‍ പകരം കൊലയാളിയെ വധിക്കാന്‍ സാധിച്ചേക്കാം. പക്ഷേ, പത്തും നൂറും ആയിരവും പതിനായിരവും പേരെ വധിച്ചവരെ ശിക്ഷിക്കാന്‍ ആര്‍ക്കും സാധ്യമല്ല. അതിനാല്‍ നിഷ്‌കൃഷ്ടമായ നീതി നടത്താന്‍ ആര്‍ക്കുമിവിടെ സാധ്യമല്ല. കൊല്ലപ്പെടുന്ന നിരപരാധരും അവരുടെ ആശ്രിതരുമനുഭവിക്കുന്ന പ്രയാസങ്ങള്‍ അപാരമത്രെ. അവയ്ക്ക് പരിഹാരമുണ്ടാക്കാന്‍ ഏവരും അശക്തരും. അതുകൊണ്ടുതന്നെ മരണത്തോടെ മനുഷ്യജീവിതം ഒടുങ്ങുമെങ്കില്‍ ലോകഘടന തീര്‍ത്തും അനീതിപരമാണ്. അര്‍ഥശൂന്യവും അബദ്ധപൂര്‍ണവുമാണ്; വ്യവസ്ഥാപിതമായ പ്രപഞ്ചഘടനയോട് തീരെ പൊരുത്തപ്പെടാത്തതും. അറുനൂറു കോടി മനുഷ്യര്‍ക്ക് അറുനൂറു കോടി മുഖവും കൈവിരലും ഗന്ധവും വ്യതിരിക്തമായ തലമുടിയും രക്തത്തുള്ളികളുമെല്ലാം നല്കപ്പെട്ട് വ്യവസ്ഥാപിതമായും ആസൂത്രിതമായും യുക്തിനിഷ്ഠമായും സൃഷ്ടിക്കപ്പെട്ട മനുഷ്യരുടെ അന്ത്യം അപ്പാടെ അനീതിപരമാവുക അസംഭവ്യമത്രെ. അതിനാല്‍ യഥാര്‍ഥവും പൂര്‍ണവുമായ നീതി പുലരുക തന്നെ വേണം. ഭൂമിയിലത് അസംഭവ്യമായതിനാല്‍ പരലോകം അനിവാര്യമാണ്. എല്ലാവരും തങ്ങളുടെ കര്‍മഫലം അവിടെവെച്ച് അനുഭവിക്കുംനന്മ ചെയ്തവന്‍ രക്ഷയും തിന്മ ചെയ്തവന്‍ ശിക്ഷയും.
മരണശേഷം നീതി പുലരുന്ന മറുലോകമില്ലെങ്കില്‍ നീതിബോധമുള്ളവര്‍ നിത്യനിരാശരായിരിക്കും. സദാ അസ്വസ്ഥരും അസംതൃപ്തരുമായിരിക്കും. അതോടൊപ്പം എല്ലാ മനുഷ്യരും പ്രകൃത്യാ നീതി കൊതിക്കുന്നു. അതിനാല്‍ അത് പുലരുന്ന ഒരു പരലോകം നീതിയുടെ തേട്ടമത്രെ; മാനവ മനസ്സിന്റെ മോഹസാക്ഷാല്‍ക്കാരവും. വളരെ വ്യവസ്ഥാപിതമായി സൃഷ്ടിക്കപ്പെട്ട മനുഷ്യന്റെ ജീവിതം നീതിരഹിതമായി മരണത്തോടെ ഒടുങ്ങുമെന്ന് ധരിക്കുന്നതിനേക്കാള്‍ എന്തുകൊണ്ടും യുക്തിനിഷ്ഠവും ബുദ്ധിപൂര്‍വകവുമാണ് നീതി പുലരുന്ന പരലോകത്തെ സംബന്ധിച്ച വിശ്വാസം.
മനുഷ്യജീവിതത്തിലൂടെ കടന്നുപോകുന്ന സംഭവങ്ങളെല്ലാം അവനില്‍ കൃത്യമായും കണിശമായും രേഖപ്പെടുത്തുന്നുണ്ട്. എന്നാല്‍ സമര്‍ഥനായ ഡോക്ടര്‍ക്ക് ശസ്ത്രക്രിയയിലൂടെ അവ കണ്ടെത്താനാവില്ല. അതിവിദഗ്ധമായ ഉപകരണങ്ങള്‍ക്കുപോലും അവ പകര്‍ത്തുക സാധ്യമല്ല. അതോടൊപ്പം അവയൊക്കെ മനുഷ്യന് ഓര്‍ക്കാന്‍ കഴിയുകയും ചെയ്യുന്നു. അപ്രകാരം തന്നെ നാം തൊടുന്നേടത്തെല്ലാം നമ്മുടെ വിരലടയാളങ്ങള്‍ പതിയുന്നു. സഞ്ചരിക്കുന്നേടത്തൊക്കെ നമ്മുടെ ശരീരത്തിന്റെ ഗന്ധം വ്യാപരിക്കുന്നു. പറയുന്ന വാക്കുകള്‍ അന്തരീക്ഷത്തില്‍ ലയിച്ചു ചേരുകയും ചെയ്യുന്നു. അങ്ങനെ നമ്മുടെ വാക്കുകളും പ്രവൃത്തികളും ചലനങ്ങളും വിചാരവികാരങ്ങളുമെല്ലാം നമുക്കനുകൂലമോ പ്രതികൂലമോ ആയി സാക്ഷ്യം വഹിക്കാന്‍ സാധിക്കുമാറ് രേഖപ്പെടുത്തപ്പെടുന്നു. നമ്മില്‍ നിന്ന് കൊഴിഞ്ഞുപോകുന്ന മുടിയും ഇറ്റിവീഴുന്ന രക്തവും സ്രവിക്കുന്ന ഇന്ദ്രിയത്തുള്ളിയുമൊക്കെ നമുക്ക് അനുകൂലമോ പ്രതികൂലമോ ആയ സാക്ഷ്യമായിത്തീരുന്നു.
സര്‍വോപരി, ഇല്ലായ്മയില്‍നിന്ന് പ്രപഞ്ചത്തെയും മനുഷ്യനുള്‍പ്പെടെ അതിലുള്ള സര്‍വതിനെയും വളരെ ആസൂത്രിതമായും വ്യവസ്ഥാപിതമായും കൃത്യമായും കണിശമായും സൃഷ്ടിച്ച ദൈവത്തിന് മനുഷ്യന്റെ പുനഃസൃഷ്ടി ഒട്ടും പ്രയാസകരമോ അസാധ്യമോ അല്ല. അതുകൊണ്ടു തന്നെ മരണശേഷം മറുലോകത്ത് മനുഷ്യരെല്ലാം ഉയിര്‍ത്തെഴുന്നേല്‍പിക്കപ്പെടുമെന്നും ഭൂമിയിലെ കര്‍മങ്ങളുടെ ഫലം അവിടെവച്ച് അനുഭവിക്കേണ്ടി വരുമെന്നും ദൈവം തന്റെ ദൂതന്‍മാരിലൂടെ അറിയിച്ചതുപോലെ സംഭവിക്കുകതന്നെ ചെയ്യും. അതൊട്ടും അവിശ്വസനീയമോ അയുക്തികമോ അല്ല. മറിച്ച്, അത്യദ്ഭുതകരമായ അവസ്ഥയില്‍ സൃഷ്ടിക്കപ്പെട്ട മനുഷ്യന്റെ ജീവിതം അനീതിക്കിരയായി, അര്‍ഥരഹിതമായി എന്നെന്നേക്കുമായി അവസാനിക്കുമെന്ന് പറയുന്നതാണ് അയുക്തികവും അവിശ്വസനീയവും!
ഒരു കാര്യവും കൂടി സൂചിപ്പിക്കാനാഗ്രഹിക്കുന്നു. നാം ഇപ്പോഴുള്ള അവസ്ഥയില്‍ നിന്നുകൊണ്ട് മനസ്സിലാക്കുന്നതാണ് യഥാര്‍ഥ വസ്തുതയെന്ന് ധരിക്കുന്നത് ശരിയല്ല. ഉപകരണമേതുമില്ലാതെ നിഴലിനെ നോക്കുമ്പോള്‍ അത് തീര്‍ത്തും നിശ്ചലമാണെന്നാണ് നമുക്കു തോന്നുക. എന്നാല്‍ നിഴലിന് സദാ നേരിയ ചലനമുണ്ടല്ലോ. മരീചിക വെള്ളമാണെന്ന് കരുതാറുണ്ട്. അടുത്തെത്തുമ്പോഴാണ് സത്യം ബോധ്യമാവുക. നിദ്രാ വേളയിലെ സ്വപ്നത്തില്‍ കാണുന്ന കാര്യങ്ങള്‍ യാഥാര്‍ഥ്യമാണെന്നാണ് നമുക്ക് അപ്പോള്‍ തോന്നുക. ഉണരുന്നതോടെ മറിച്ച് അനുഭവപ്പെടുന്നു. അതിനാല്‍ പഞ്ചേന്ദ്രിയങ്ങള്‍ മാത്രമുള്ള നമുക്ക് മനസ്സിലാക്കാനാവാത്ത പലതും ദിവ്യബോധനമാകുന്ന അറിവിന്റെ ആറാം സ്രോതസ്സ് തുറന്നുകിട്ടുന്ന ദൈവദൂതന്‍മാര്‍ക്ക് ഗ്രഹിക്കാന്‍ കഴിയും. സ്വപ്നം കാണുന്നവന്‍ അപ്പോള്‍ പറയുന്ന കാര്യങ്ങള്‍ക്കപ്പുറമാണ് വസ്തുതയെന്ന്, ഉണര്‍ന്നിരിക്കുന്നവന്‍ അറിയുന്ന പോലെ ഇന്ദ്രിയ ബന്ധിതനായ മനുഷ്യന്‍ കാണുന്നതിനപ്പുറമാണ് സത്യമെന്ന് ദൈവദൂതന്‍മാരറിയുന്നു. അവരത് സമൂഹത്തെ അറിയിക്കുകയും ചെയ്യുന്നു. അഭൗതികജ്ഞാനത്തിന്റെ പിഴക്കാത്ത ഏകാവലംബം അതു മാത്രമത്രെ.
\chapter{അദ്വൈതത്തെ അംഗീകരിക്കാത്തതെന്തുകൊണ്ട്? }
  \section{''അദ്വൈതസിദ്ധാന്തത്തെ ഇസ്‌ലാം അംഗീകരിക്കുന്നുണ്ടോ? ഇല്ലെങ്കില്‍ എന്തുകൊണ്ട്?''}
 ശ്രീശങ്കരാചാര്യരാണ് ഒരു സിദ്ധാന്തമെന്ന നിലയില്‍ അദ്വൈതത്തെ പരിചയപ്പെടുത്തിയത്. ഏഴാം നൂറ്റാണ്ടിന്റെ അന്ത്യത്തിലും എട്ടാം നൂറ്റാണ്ടിന്റെ ആദ്യത്തിലുമായാണ് അദ്ദേഹം ജീവിച്ചതെന്നാണ് പൊതു ധാരണ. എന്നാല്‍ എട്ടാം നൂറ്റാണ്ടിന്റെ അന്ത്യത്തിലാണെന്നും ഒമ്പതാം നൂറ്റാണ്ടിന്റെ ആദ്യപാദത്തിലാണെന്നുമൊക്കെ വിശ്വസിക്കുന്നവരും വിരളമല്ല. ജന്മനാട് കേരളത്തിലെ കാലടിയാണെന്ന് കരുതപ്പെടുന്നു. ശങ്കരാചാര്യര്‍ തന്റെ ഗുരുവായ ഗോവിന്ദന്റെ ആചാര്യനായ ഗൗഢപാദനില്‍നിന്നാണ് അദ്വൈതസിദ്ധാന്തം സ്വീകരിച്ചത്. എങ്കിലും ബാദരായണന്റെ ബ്രഹ്മസൂത്രത്തിന്റെ വ്യാഖ്യാനമെന്ന നിലയിലാണ് അദ്ദേഹം അദ്വൈതം ആവിഷ്‌കരിച്ചത്. ഉപനിഷദ്വാക്യങ്ങളുടെ സമന്വയരൂപമാണ് ബ്രഹ്മസൂത്രം.
കേരളത്തിലെ അദ്വൈതത്തിന്റെ ഏറ്റം കരുത്തനായ വക്താവായിരുന്ന വിദ്യാവാചാസ്പതി വി. പനോളി എഴുതുന്നു: ''അദ്വൈത ദര്‍ശനത്തെ യാഥാര്‍ഥ്യമാക്കാന്‍ പോന്നതാണ് ആചാര്യവീക്ഷണം. ബ്രഹ്മം നിര്‍ഗുണമാണെന്നും ജീവാത്മാവും പരമാത്മാവും ഏകമാണെന്നും തെളിയിക്കാന്‍ ശ്രീശങ്കര ഭഗവദ്പാദര്‍ക്കല്ലാതെ മറ്റൊരാചാര്യനും സാധിച്ചിട്ടില്ല.''(ശ്രീശങ്കരദര്‍ശനം, മാതൃഭൂമി പ്രിന്റിംഗ് ആന്റ് പബ്ലിഷിംഗ് കമ്പനി ലിമിറ്റഡ്, 1998, പുറം 61)
ബുദ്ധമതത്തെ നേരിടാനാണ് ശ്രീശങ്കരാചാര്യര്‍ അദ്വൈതസിദ്ധാന്തം ആവിഷ്‌കരിച്ചത്. പനോളി എഴുതുന്നു: ''തരംതാണുപോയ ബുദ്ധമതത്തിന്റെ എതിര്‍പ്പുകളെ ശ്രീശങ്കരന് അഭിമുഖീകരിക്കേണ്ടിവന്നു. പ്രസ്തുത സാഹചര്യത്തില്‍ ബുദ്ധമതത്തിന്റെ ഭീഷണിയെ ചെറുക്കുവാന്‍ ശക്തവും നൂതനവുമായ ഒരു തത്വശാസ്ത്രം രൂപപ്പെടുത്തേണ്ടത് അത്യാവശ്യമായിരുന്നു. ശ്രീശങ്കരന്റെ അദ്വൈതമതം ധീരവും സമര്‍ഥവുമായ വിധത്തില്‍ ആ വെല്ലുവിളിയെ നേരിടുകയും ചെയ്തു.''(Ibid പുറം 138)
ബുദ്ധമതത്തെ നേരിടാന്‍ ശങ്കരാചാര്യര്‍ അവരുടെ ആശയങ്ങള്‍തന്നെ കടം കൊള്ളുകയായിരുന്നു. വി. പനോളി എഴുതുന്നു: ''അവയില്‍നിന്ന് സ്വീകാര്യമായതൊക്കെ ആചാര്യന്‍ സ്വീകരിക്കുകയും സ്വവീക്ഷണത്തില്‍ വിലയിപ്പിക്കുകയും ചെയ്തു. പ്രച്ഛന്നബുദ്ധന്‍ എന്ന അപരനാമം പോലും ശ്രീശങ്കരാചാര്യര്‍ക്ക് ഉണ്ടായിരുന്നു.''(Ibid പുറം 68)
ശങ്കരാചാര്യര്‍ അദ്വൈതത്തെ അവതരിപ്പിച്ചതിങ്ങനെയാണ്: ''ബ്രഹ്മസത്യം, ജഗന്മിഥ്യാ, ജീവോ ബ്രഹ്മൈവ നാ പരഃ''
(ബ്രഹ്മം സത്യം; ജഗത്ത് അയഥാര്‍ഥം; ജീവന്‍ ബ്രഹ്മമല്ലാതെ മറ്റൊന്നുമല്ല.)
വി. പനോളി എഴുതുന്നു: ''ആചാര്യസ്വാമികള്‍ ആയിരത്താണ്ടുകള്‍ക്കപ്പുറം ഈ ലോകത്തോടായി പ്രവചിച്ചു: ഈ ദൃശ്യപ്രപഞ്ചം എന്തില്‍നിന്നുദ്ഭവിച്ചുവോ അത് ബ്രഹ്മമാകുന്നു. അതാകട്ടെ ഏകവും നിത്യവുമാണ്. അങ്കരന്‍, അമരന്‍, പരമാത്മാവ് തുടങ്ങിയിട്ടുള്ള പേരുകളില്‍ ബ്രഹ്മം അറിയപ്പെടുന്നു. അതറിഞ്ഞു കഴിഞ്ഞാലാകട്ടെ ലോകത്തിലെ സര്‍വവും അറിയാന്‍ കഴിയും. വേദങ്ങള്‍ക്കും ബ്രഹ്മമെന്ന് പേരുണ്ട്. പരമാത്മാവും ജീവാത്മാവും ഒന്നുതന്നെ. കേവലം അവിദ്യകൊണ്ടാണ് അങ്ങനെയല്ലെന്ന് തോന്നുന്നത്. ഈ അവിദ്യ ഒന്നുമാത്രമാണ് ദുഃഖകാരണം. ആത്മജ്ഞാനോദയത്തില്‍ അവിദ്യ നീങ്ങും. കര്‍തൃത്വവും ഭോക്തൃത്വവും ജീവാത്മാവിന്റെ ധര്‍മങ്ങളാണ്. ജ്ഞാനികള്‍ക്ക് കര്‍മമില്ല; ശരീരബന്ധമില്ല. ബ്രഹ്മോപാസകര്‍ ബ്രഹ്മമായി ഭവിക്കുന്നു.''(Ibid, പുറം 113)
അദ്ദേഹം തന്നെ എഴുതുന്നു: ''എട്ടുകാലിയില്‍നിന്ന് നൂലെന്ന പോലെ, ശരീരത്തില്‍ നിന്ന് കേശരോമാദികളെന്ന പോലെ നിത്യവസ്തുവായ ബ്രഹ്മത്തില്‍നിന്ന് ഉദ്ഭൂതമായതാണ് ഈ ജഗത്ത് എന്ന് മുണ്ഡകോപനിഷത്തില്‍ പ്രസ്താവിച്ചതായി കാണാം.''(Ibid, പുറം 48)
പാരമാര്‍ഥികമായി ബ്രഹ്മം മാത്രമേയുള്ളൂവെന്നും ജഗത്ത് മായയാണെന്നും അവിദ്യയാണ് അത് യാഥാര്‍ഥ്യമായി അനുഭവപ്പെടാന്‍ കാരണമെന്നും സമര്‍ഥിക്കാനായി അദ്വൈതവാദികള്‍ പലവിധ ഉദാഹരണങ്ങളും സമര്‍പ്പിക്കാറുണ്ട്. ചിലതിവിടെ ചേര്‍ക്കുന്നു:\\
1. സ്വര്‍ണമാല എടുത്ത് എന്താണെന്ന് ചോദിച്ചാല്‍ ഏവരും മാല എന്നു പറയും. അത് ഉരുക്കി വളയാക്കിയാല്‍ വളയെന്നും പാദസരമാക്കിയാല്‍ ആ പേരിലും അറിയപ്പെടും. എന്നാല്‍ യഥാര്‍ഥത്തില്‍ അതൊന്നുമല്ല അവ; സ്വര്‍ണമെന്ന അവസ്ഥക്കു മാത്രമാണ് മാറ്റമില്ലാത്തത്. അവ്വിധം ഇക്കാണപ്പെടുന്ന ജഗത്ത് ബ്രഹ്മത്തിന്റെ വിവര്‍ത്തിത രൂപമാണ്. പ്രകാശിത രൂപം കണ്ട് അവ യഥാര്‍ഥമാണെന്ന് ധരിക്കുന്നത് അവിദ്യകാരണമാണ്. ആത്മജ്ഞാനം നേടുന്നതോടെ അവയെല്ലാം വ്യവഹാരികമാണെന്നും പാരമാര്‍ഥികമായി ബ്രഹ്മമേയുള്ളൂവെന്നും ബോധ്യമാവുന്നു. വളയും മാലയും പാദസരവും സ്വര്‍ണമാണെന്നറിയുന്നപോലെ.\\
2. സ്വപ്നത്തിലെ അനുഭവവും ഉണര്‍ച്ചയിലെ അനുഭവവും തമ്മില്‍ അന്തരമില്ല. എന്നാല്‍ സ്വപ്നത്തിലെ അനുഭവം യാഥാര്‍ഥ്യമല്ല. ഇവ്വിധംതന്നെ അവിദ്യയുള്ളപ്പോള്‍ യാഥാര്‍ഥ്യമായി തോന്നുന്നത് വിദ്യ നേടുന്നതോടെ മായയാണെന്ന് ബോധ്യമാകും. ഉറക്കത്തിലെ സ്വപ്നം ഉണര്‍ച്ചയില്‍ അയഥാര്‍ഥമായി അനുഭവപ്പെടുന്നപോലെ.\\
3. മങ്ങിയ വെളിച്ചത്തില്‍ കയര്‍ കണ്ടാല്‍ പാമ്പാണെന്ന് കരുതും. വെളിച്ചം വരുന്നതോടെ യാഥാര്‍ഥ്യം തിരിച്ചറിയും. അവ്വിധം അവിദ്യ കാരണമാണ് ജഗത് യാഥാര്‍ഥ്യമായിതോന്നുന്നത്. ബ്രഹ്മജ്ഞാനം ലഭിക്കുന്നതോടെ അത് മായയാണെന്നും ബ്രഹ്മം മാത്രമേ പാരമാര്‍ഥികമായുള്ളൂവെന്നും ബോധ്യമാകും.\\
അതിനാല്‍ ജീവാത്മാവിന് വ്യാവഹാരിക തലത്തിലേ അസ്തിത്വമുള്ളൂ. പാരമാര്‍ഥിക തലത്തിലത് ബ്രഹ്മം തന്നെയാണ്. അഥവാ ജീവാത്മാവും പരമാത്മാവും ഒന്നുതന്നെ. വിദ്യ നേടുന്നതോടെ ഇത് തിരിച്ചറിയും.
ഈ അദ്വൈത സിദ്ധാന്തത്തെ ഇസ്‌ലാം അംഗീകരിക്കുന്നുണ്ടോ എന്ന് ആലോചിക്കുന്നതിനുമുമ്പ് ഇത് ദൈവികമോ വൈദികമോ മതപരമോ ആണോ എന്ന് പരിശോധിക്കേണ്ടതുണ്ട്. ബ്രഹ്മസൂത്രത്തെ അവലംബിച്ചിട്ടാണല്ലോ ശങ്കരാചാര്യര്‍ ഈ സിദ്ധാന്തം ആവിഷ്‌കരിച്ചത്.
ആര്‍ക്കും എവ്വിധവും വ്യാഖ്യാനിക്കാന്‍ പാകത്തില്‍ അവ്യക്തവും ദുര്‍ഗ്രഹവുമാണ് ബ്രഹ്മസൂത്രം. ഡോ. രാധാകൃഷ്ണന്‍ എഴുതുന്നു: ''വാക്കുകള്‍ കഴിയാവുന്നത്ര കുറക്കാനുള്ള തിരക്കില്‍ സൂത്രങ്ങള്‍ ദുര്‍ഗ്രാഹ്യമാവുകയും വ്യാഖ്യാനം അനിവാര്യമാവുകയും ചെയ്തു.'' (ബ്രഹ്മസൂത്ര, പുറം23)
ഹാജിമി നകമുറ എഴുതുന്നു: ''വ്യാഖ്യാനമില്ലാതെ ബ്രഹ്മസൂത്രത്തിലെ ഒരു വാക്യം പോലും മനസ്സിലാക്കാന്‍ പ്രയാസമാണ്. അതിലെ ഒരു വാക്യത്തില്‍ വിഷയമോ വിഷയിയോ ഇല്ലാതിരിക്കുന്നത് സാധാരണമാണ്. ചിലപ്പോള്‍ ഏറ്റവും പ്രധാനപ്പെട്ട പദം തന്നെ വിട്ടുപോയിരിക്കും. അതില്ലാതെ വാചകം ഗ്രഹിക്കാനാവില്ല. രണ്ടോ ഏറിയാല്‍ പത്തോ വാക്കുകള്‍ അടങ്ങിയതാണ് ഒരു സൂത്രം. ദീര്‍ഘമായ സൂത്രം അപൂര്‍വമാണ്. ചിലപ്പോള്‍ ഒരു സൂത്രത്തില്‍ ഒരൊറ്റ വാക്കേ കാണൂ'' (History of Early Vedantha Philoso-phy, p. 440. ഉദ്ധരണം: ബ്രഹ്മസൂത്രം ദൈ്വതമോ അദ്വൈതമോ? പുറം 31)
വ്യാഖ്യാതാവിന്റെ ഭാവനയ്ക്കും സങ്കല്‍പത്തിനും വമ്പിച്ച പ്രാധാന്യവും പ്രാമുഖ്യവും കൈവരാനിത് കാരണമായി. ബ്രഹ്മസൂത്രത്തെ അവലംബിച്ച് പരസ്പര വിരുദ്ധമായ സിദ്ധാന്തങ്ങള്‍ രൂപംകൊള്ളാനുള്ള കാരണവും ഇതുതന്നെ. ശങ്കരാചാര്യരുടെ അദ്വൈതവും രാമാനുജാചാര്യരുടെ വിശിഷ്ടാദൈ്വതവും മാധ്വാചാര്യരുടെ ദൈ്വതവുമെല്ലാം ബ്രഹ്മസൂത്രത്തെ ആധാരമാക്കിയുള്ളവയാണല്ലോ. മൂന്നിനെയും ഒരേപോലെ പഠനവിധേയമാക്കിയ പണ്ഢിതന്മാര്‍ ബ്രഹ്മസൂത്രം അദ്വൈതത്തേക്കാള്‍ അടുത്തു നില്‍ക്കുന്നത് ദൈ്വതത്തോടാണെന്ന് അസന്ദിഗ്ധമായി വ്യക്തമാക്കിയിട്ടുണ്ട്. അദ്വൈതത്തിന്റെ യഥാര്‍ഥ സ്രോതസ്സ് അപഭ്രംശം വന്ന ബൗദ്ധ ദര്‍ശനമാണെന്നും അവരഭിപ്രായപ്പെട്ടിട്ടുണ്ട്.
ഇന്ത്യന്‍ ദര്‍ശനങ്ങളുടെ ചരിത്രം സമഗ്രമായി സമര്‍പ്പിച്ച പ്രമുഖ പണ്ഢിതനായ ദാസ്ഗുപ്ത എഴുതുന്നു: ''സൂത്രങ്ങളുടെ അടിസ്ഥാനത്തില്‍ വിധി കല്‍പിക്കുകയാണെങ്കില്‍ ശങ്കരന്റെ ദാര്‍ശനിക സിദ്ധാന്തത്തെ ബ്രഹ്മസൂത്രം പിന്തുണക്കുന്നതായി എനിക്കു തോന്നുന്നില്ല. ശങ്കരന്‍ തന്നെയും ദൈ്വതസ്വഭാവത്തില്‍ വ്യാഖ്യാനിച്ച ചില സൂത്രങ്ങള്‍ അതിലുണ്ടു താനും.'' (A History of Indian Philosophy, Vol. II, P. 2)
അദ്ദേഹം തന്നെ തുടരുന്നു: ''തന്റെ ദര്‍ശനം ഉപനിഷത്തുകള്‍ പഠിപ്പിക്കുന്നതാണെന്ന് തെളിയിക്കുന്നതിലായിരുന്നു ശങ്കരന് താല്‍പര്യം. പക്ഷേ, ആസ്തികവാദപരവും ദൈ്വതവീക്ഷണങ്ങളോടുകൂടിയതുമായ ധാരാളം വാക്യങ്ങള്‍ ഉപനിഷത്തുകളിലുണ്ട്. ഭാഷാപരമായ കൗശലം എത്രയുണ്ടായിരുന്നാലും ശങ്കരന്റെ സിദ്ധാന്തത്തെ പിന്തുണക്കുന്ന അര്‍ഥം ഈ വാചകങ്ങളില്‍ നിന്ന് കണ്ടെടുക്കാനാവില്ല.''(Ibid, P.2)
''ബ്രഹ്മസൂത്രത്തിന് ആറിലധികം വൈഷ്ണവ വ്യാഖ്യാനങ്ങളുണ്ട്. അവ ശങ്കരന്റെ വ്യാഖ്യാനത്തില്‍നിന്നു ഭിന്നമാണെന്നു മാത്രമല്ല, പരസ്പരം വ്യത്യസ്തങ്ങളാണ്. ദൈ്വത വീക്ഷണങ്ങള്‍ക്ക് അവ നല്‍കുന്ന ഊന്നല്‍ പലവിധത്തിലാണ്. അതെന്തൊക്കെയാവട്ടെ, ശങ്കരന്റെ വ്യാഖ്യാനത്തേക്കാള്‍ സൂത്രങ്ങളോട് കൂടുതല്‍ അടുത്തത് ബ്രഹ്മസൂത്രങ്ങളുടെ ദൈ്വത വ്യാഖ്യാനങ്ങളാണെന്ന് ഞാന്‍ വിശ്വസിക്കുന്നു''(Ibid, Vol. I, P. 420, 421).
ശങ്കരാചാര്യരുടെ അനുയായിയെന്ന് സ്വയം സമ്മതിക്കുന്ന പ്രമുഖ അദ്വൈതവാദി ഡോ. രാധാകൃഷ്ണന്‍ എഴുതുന്നു: ''അവിദ്യകൊണ്ടാണ് ലോകം യാഥാര്‍ഥ്യമായി തോന്നുന്നതെന്ന കാഴ്ചപ്പാടല്ല ബ്രഹ്മസൂത്രകാരനുണ്ടായിരുന്നതെന്ന് വ്യക്തമാണ്. പ്രപഞ്ചസൃഷ്ടി ഗൗരവകരമായ പ്രശ്‌നമായാണ് സൂത്രകാരന്‍ കണ്ടിട്ടുള്ളത്. സാംഖ്യന്മാരുടെ പ്രകൃതിയല്ല പ്രപഞ്ചത്തിനു കാരണമെന്നും അതീവ ബുദ്ധിയായ ബ്രഹ്മമാണ് അതിന്റെ സൃഷ്ടിക്കു പിന്നിലെന്നും സൂത്രകാരന്‍ വാദിക്കുന്നു. ലോകം മായയാണെന്ന യാതൊരു സൂചനയും ഇവിടെയില്ല''(Brahmastura, P. 252).
ജീവാത്മാവും പരമാത്മാവും ഒന്നാണെന്ന ശങ്കരദര്‍ശനത്തെയും രാധാകൃഷ്ണന്‍ നിരാകരിക്കുന്നു: 'ജീവാത്മാവ് കര്‍മനിരതമാണ്. അത് ആനന്ദത്തിലാണ്. ഗുണവും ദോഷവും ആര്‍ജിക്കുകയും സുഖവും ദുഃഖവും അനുഭവിക്കുകയും ചെയ്യുന്നു. പരമാത്മാവിന് വ്യത്യസ്തമായ പ്രകൃതമാണുള്ളത്. എല്ലാ ദുരിതങ്ങളില്‍നിന്നും അത് സ്വതന്ത്രമാണ്. രണ്ടും തികച്ചും ഭിന്നമായതിനാല്‍ ജീവാത്മാവിന്റെ അനുഭവങ്ങള്‍ പരമാത്മാവിനെ ബാധിക്കുന്നില്ല. ജീവാത്മാവ് സുഖവും ദുഃഖവും അനുഭവിക്കുന്നു. കാരണം അത് കര്‍മവിധേയമാണ്. എന്നാല്‍ പരമാത്മാവ് അതിനതീതവുമാണ്'' (Ibid, 273).
''സര്‍ഗശക്തിയായ ബ്രഹ്മം ജീവാത്മാവില്‍നിന്ന് വ്യത്യസ്തമാണ്. ജീവാത്മാവിന് സ്വയംഭൂവാകാനോ സ്വയം നശിക്കാനോ കഴിയില്ല. പ്രയോജനകരമല്ലാത്തതും ഇഷ്ടകരമല്ലാത്തതുമായ കര്‍മങ്ങളൊന്നും ബ്രഹ്മം ചെയ്യുന്നില്ല. ഉപകാരങ്ങള്‍ ചെയ്യേണ്ട ആവശ്യമോ ഉപദ്രവങ്ങള്‍ ചെയ്യാതിരിക്കേണ്ട ആവശ്യമോ അതിനില്ല. ബ്രഹ്മത്തിന് അറിയാത്തതോ ബ്രഹ്മത്തിന് പങ്കില്ലാത്തതോ ആയി യാതൊന്നുമില്ല. ജീവാത്മാവിന് വ്യത്യസ്തമായ സ്വഭാവമാണുള്ളത്. മേല്‍സൂചിപ്പിച്ച പരിമിതികളെല്ലാം അതിനുണ്ട്'' (Ibid, 355).
'ജീവാത്മാവും പരമാത്മാവും ഒന്നുതന്നെയാണെന്ന് നാം കരുതുന്നില്ല. എന്തുകൊണ്ടെന്നാല്‍ അവയ്ക്കിടയിലെ വ്യത്യാസങ്ങള്‍ വ്യക്തമായി പ്രസ്താവിക്കപ്പെട്ടിട്ടുണ്ട്.''(Ibid, 417).
ജീവാത്മാവും പരമാത്മാവും ഒന്നാണെന്ന ശങ്കരാചാര്യരുടെ സിദ്ധാന്തത്തെ ബ്രഹ്മസൂത്രം പിന്തുണക്കുന്നില്ലെന്ന വസ്തുത ഡോ. രാധാകൃഷ്ണനെപ്പോലെ നിരവധി ദാര്‍ശനികന്മാര്‍ വ്യക്തമാക്കിയിട്ടുണ്ട്. ശങ്കരാദൈ്വതത്തിന്റെ ശക്തനായ അനുകൂലിയായിരുന്നിട്ടുകൂടി ഡോ. രാധാകൃഷ്ണന് തന്റെ വിയോജിപ്പ് വ്യക്തമാക്കേണ്ടിവന്നത് അതിനാലാണ്. ഇപ്രകാരംതന്നെ ശങ്കരാചാര്യര്‍ നിരന്തരം ആവര്‍ത്തിക്കുന്ന 'മായ' എന്ന പദംതന്നെ ബ്രഹ്മസൂത്രത്തില്‍ ഒരിടത്തുമാത്രമേ പ്രയോഗിച്ചിട്ടുള്ളൂ (ബ്രഹ്മസൂത്രം 3:2:3). സ്വപ്നലോകം അതിന്റെ മൗലിക സ്വഭാവത്തില്‍തന്നെ വെറും തോന്നലാണെന്നും ഉണര്‍ച്ചാ അവസ്ഥയിലെ പ്രത്യേകതകള്‍ അതിനില്ലെന്നും വ്യക്തമാക്കുകയാണ് സൂത്രകാരനിവിടെ. അഥവാ, ഒരിടത്ത് 'മായ' എന്ന പദമുപയോഗിച്ചതുതന്നെ ശങ്കരാചാര്യരുടെ സിദ്ധാന്തത്തിന് കടകവിരുദ്ധമായ ആശയം പ്രകാശിപ്പിക്കാനാണ്. സ്വപ്നത്തിലെ അനുഭവവും ഉണര്‍ച്ചയിലേതും ഒരേപോലെയാണെന്ന വാദമാണല്ലോ ശങ്കരദര്‍ശനം സമര്‍പ്പിക്കുന്നത്.
യാഥാര്‍ഥ്യത്തെ പരമാര്‍ഥികമെന്നും വ്യാവഹാരികമെന്നും ശങ്കരന്‍ രണ്ടായി തിരിക്കുകയും വ്യാവഹാരികം മായയാണെന്ന് വാദിക്കുകയും ചെയ്യുന്നു. എന്നാല്‍ ഇത്തരമൊരു വിഭജനം തന്നെ ബ്രഹ്മസൂത്രത്തിന് അന്യമാണ്. ഡോ. നകമുറ എഴുതുന്നു: ''ഇത് വ്യാവഹാരികാര്‍ഥത്തിലുള്ള വിശദീകരണമാണ്. പരമാര്‍ഥിക തലത്തിലുള്ളതല്ല. പരമാര്‍ഥിക തലത്തില്‍ ഇവ രണ്ടും ഒന്നുതന്നെയാണ്. ശങ്കരനില്‍ നിന്നുണ്ടാവുന്ന ഇത്തരം ചിന്ത ബ്രഹ്മസൂത്രത്തിന്റെ സത്തയോട് നിരക്കുന്നതല്ല''(History of Early Vedantha Philosophy, P. 460. ഉദ്ധരണം: ബ്രഹ്മസൂത്രം ദദൈ്വതമോ അദ്വൈതമോ, പുറം 44).
ശങ്കരാചാര്യരുടെ അദ്വൈത സിദ്ധാന്തം ആദ്യാവസാനം വേദങ്ങള്‍ക്കും ഉപനിഷദ്വാക്യങ്ങളുടെ സമന്വയ രൂപമായ ബാദരായണന്റെ ബ്രഹ്മസൂത്രത്തിനും കടകവിരുദ്ധമാണെന്ന് ഇതെല്ലാം അസന്ദിഗ്ധമായി വ്യക്തമാക്കുന്നു. അദ്ദേഹത്തിന്റെ യഥാര്‍ഥ അവലംബമായ ഗൗഢപാദന്‍ ഇക്കാര്യം ബുദ്ധമതക്കാരോട് തുറന്നുസമ്മതിച്ചിരുന്നു. ഡോ. രാധാകൃഷ്ണന്‍ എഴുതുന്നു: ''തന്റെ വീക്ഷണം യാതൊരു ധര്‍മശാസ്ത്ര ഗ്രന്ഥങ്ങളെയും ഈശ്വരപ്രോക്തങ്ങളെയും ദിവ്യാനുഭൂതികളെയും ആധാരമാക്കുന്നില്ല എന്ന അടിസ്ഥാനത്തില്‍ അദ്ദേഹം ബുദ്ധമതക്കാരോട് നിവേദനം ചെയ്യുകയുണ്ടായി.''(ഭാരതീയ ദര്‍ശനം, വാള്യം രണ്ട്, പുറം 444)
ശങ്കരാചാര്യരുടെ അദ്വൈതം വേദവിരുദ്ധമെന്നപോലെത്തന്നെ യുക്തിവിരുദ്ധവുമാണ്. ഏറെ പ്രസക്തങ്ങളായ ഒട്ടേറെ ചോദ്യങ്ങളുടെ മുമ്പില്‍ ഉത്തരം നല്‍കാനാവാതെ പകച്ചുനില്‍ക്കാന്‍ അദ്വൈതവാദികള്‍ നിര്‍ബന്ധിതരാവുന്നതും അതുകൊണ്ടുതന്നെ. ചിലതുമാത്രമിവിടെ ചേര്‍ക്കുന്നു:
1. ലോകം മായയാണെന്ന് പറയുന്ന ശങ്കരാചാര്യരും അദ്ദേഹത്തിന്റെ ശരീരവും ശാരീരികാവയവങ്ങളും മനസ്സും ബുദ്ധിയും മായയല്ലേ? പ്രപഞ്ചം മായയാണെന്ന് കണ്ടെത്തിയ മനസ്സുതന്നെ മായയായതിനാല്‍ മായയുടെ കണ്ടെത്തലും മായയല്ലേ? അദ്വൈതസിദ്ധാന്തം മനസ്സിന്റെ കണ്ടെത്തലായതിനാല്‍ അദ്വൈതവും മായയല്ലേ? അപ്പോള്‍ ലോകം മായയാണെന്നതുള്‍പ്പെടെ അദ്വൈത സിദ്ധാന്തം മായയുടെ ദര്‍ശനമെന്ന നിലയില്‍ തീര്‍ത്തും മായയല്ലേ?
2. ബാദരായണനും അദ്ദേഹത്തിന്റെ ബ്രഹ്മസൂത്രവും മായയല്ലേ? മായയായ ബ്രഹ്മസൂത്രത്തിന് മായയായ ശങ്കരാചാര്യര്‍ നല്‍കിയ വ്യാഖ്യാനവും മായയല്ലേ?
3. യഥാര്‍ഥമായ ബ്രഹ്മവും അയഥാര്‍ഥമായ ജഗത്തും തമ്മിലുള്ള ബന്ധമെന്ത്? ജഗത് ബ്രഹ്മത്തില്‍നിന്ന് അന്യമല്ലെങ്കില്‍, ബ്രഹ്മത്തില്‍ അധിഷ്ഠിതമെങ്കില്‍ പരിവര്‍ത്തന വിധേയമായ ജഗത്തും പരിവര്‍ത്തന വിധേയമല്ലാത്ത ബ്രഹ്മവും എങ്ങനെ ഒന്നാകും? മായയും പരമാര്‍ഥവും താദാത്മ്യം പ്രാപിക്കുമോ?
4. യഥാര്‍ഥമായ ബ്രഹ്മം മായയായ പ്രപഞ്ചത്തെ സ്വീകരിക്കുമോ? ബ്രഹ്മം ജഗത്തിന് കാരണമാണോ? ആണെങ്കില്‍ കാരണമായ ബ്രഹ്മം അനന്തവും സ്വയംഭൂവുമായതിനാല്‍ കാര്യമായ ജഗത്തും അനന്തവും സ്വയംഭൂവുമാകേണ്ടതല്ലേ?
5. ജഗത്ത് ശങ്കരാചാര്യര്‍ പറഞ്ഞതു പോലെ നിഴലാണെങ്കില്‍ ന്യൂനതയില്ലാത്ത ബ്രഹ്മത്തിന്റെ നിഴലില്‍ ന്യൂനത ഉണ്ടാകുമോ? പാപമില്ലാത്ത ബ്രഹ്മത്തിന്റെ നിഴലില്‍ പാപമുണ്ടാകുമോ? നിഴലില്‍ പാപവും പുണ്യവും ഉണ്ടാവുന്നതെങ്ങനെ?
6. മായയ്ക്ക് അസ്തിത്വമുണ്ടോ? ഇല്ലെങ്കില്‍ ഇക്കാണുന്ന ജഗത്തോ? ജഗത്തിനെ ജനിപ്പിക്കാന്‍ കഴിയുമാറ് ശക്തമാണല്ലോ മായ. അസ്തിത്വമുണ്ടെങ്കില്‍ ബ്രഹ്മത്തോടൊപ്പം അസ്തിത്വമുള്ള മറ്റൊന്നുമുണ്ടോ? എങ്കില്‍ അത് ദൈ്വതമാവില്ലേ?
7. ജഗത്തിന് കാരണം ബ്രഹ്മമോ അതോ അവിദ്യയോ? അവിദ്യ ആരുടെ സൃഷ്ടിയാണ്? ബ്രഹ്മമാണോ അവിദ്യക്കു കാരണം? അവിദ്യ തന്നെ മായയല്ലേ? അപ്പോള്‍ മായയായ അവിദ്യയാണ് പ്രപഞ്ചം മായയാണെന്ന് തോന്നാതിരിക്കാന്‍ കാരണമെന്ന സിദ്ധാന്തവും മായയല്ലേ?
8. അവിദ്യ പരമാര്‍ഥിക സത്യമാണെന്നാണ് വാദമെങ്കില്‍ പരമാര്‍ഥികമായി ബ്രഹ്മവുമുണ്ട്. അവിദ്യയുമുണ്ട്. അത് ദൈ്വതത്തെ അംഗീകരിക്കലാവില്ലേ? രണ്ട് പരമാര്‍ഥിക സത്യമുണ്ടെന്ന് സമ്മതിക്കലല്ലേ? അതിനാല്‍ അവിദ്യ പരമാര്‍ഥിക സത്യമാണെന്നു പറഞ്ഞാലും വ്യാവഹാരികമാണെന്നു വാദിച്ചാലും പരമാബദ്ധമായിത്തീരുകയില്ലേ?
9. സ്വപ്നത്തിലെ അനുഭവവും ഉണര്‍ച്ചയിലെ അനുഭവവും ഒരുപോലെയാണെന്ന ശങ്കരാചാര്യരുടെ വാദം തീര്‍ത്തും തെറ്റല്ലേ? സ്വപ്നം യഥാര്‍ഥമാണെന്ന തോന്നല്‍ കേവലം തോന്നലാണെന്നതല്ലേ സത്യം? ഒരാള്‍ വീണ് കാലു മുറിഞ്ഞതായി സ്വപ്നംകണ്ടാല്‍ കാല്‍ മുറിഞ്ഞതായി തോന്നും. പക്ഷേ, അത് കേവലം തോന്നലാണ്. ഉണര്‍ച്ചയില്‍ കാല്‍ മുറിഞ്ഞാല്‍ അത് തോന്നലില്‍ നില്‍ക്കുമോ?
10. ഉണര്‍ച്ചയിലെ അനുഭവങ്ങള്‍ക്ക് കാര്യകാരണബന്ധമുണ്ടാവും. മുറിവുണ്ടായാലേ രക്തമൊലിക്കുകയുള്ളൂ. എന്നാല്‍ സ്വപ്നത്തില്‍ ഇത്തരം കാര്യകാരണ ബന്ധമൊന്നുമില്ലാതെ സംഭവങ്ങള്‍ നടന്നതായി അനുഭവപ്പെടുന്നു. സ്വപ്നവും ഉണര്‍ച്ചയിലെ അനുഭവവും ഒരുപോലെയല്ലെന്ന് ഇത് അസന്ദിഗ്ധമായി വ്യക്തമാക്കുന്നില്ലേ?
11. സ്വപ്നം സ്വയം അയഥാര്‍ഥമോ മായയോ അല്ല. സ്വപ്നം കാണണമെങ്കില്‍ കാണുന്ന വ്യക്തി ഉണ്ടാവണം; അയാള്‍ ഉറങ്ങണം; സ്വപ്നത്തില്‍ കാണുന്നവയ്ക്ക് ഏതോ അര്‍ഥത്തില്‍ യാഥാര്‍ഥ്യലോകത്ത് അസ്തിത്വമുണ്ടാകണം. അതിനാല്‍ സ്വപ്നം മായയാണെന്ന വാദമല്ലേ മായ? ഉറക്കം, ഉണര്‍ച്ച എന്നീ യാഥാര്‍ഥ്യങ്ങളില്‍ ഉറക്കമെന്നത് സംഭവിച്ചാലല്ലേ സ്വപ്നം ഉണ്ടാകുകയുള്ളൂ? ജീവിതത്തില്‍ ഒന്നും കണ്ടിട്ടില്ലാത്ത അന്ധന്‍ സ്വപ്നം കാണുമോ? അപ്പോള്‍ സ്വപ്നം കാണുന്നവന്നും കണ്ണ് എന്ന യാഥാര്‍ഥ്യം ഉണ്ടാവേണ്ടതില്ലേ?
12. ഇരുട്ടില്‍ കയറ് കാണുമ്പോള്‍ പാമ്പാണെന്ന് ധരിക്കുന്നത് ഇരുട്ട്, പാമ്പ്, കയറ് എന്നിവ യാഥാര്‍ഥ്യമായതുകൊണ്ടല്ലേ? വെളിച്ചമുണ്ടാകുമ്പോള്‍ കയറാണെന്ന് തിരിച്ചറിയുന്നത് കയറിന് അസ്തിത്വമുള്ളതുകൊണ്ടല്ലേ? യഥാര്‍ഥ ലോകത്ത് കയറും പാമ്പും ഇരുട്ടുമൊന്നുമില്ലെങ്കില്‍ പാമ്പാണെന്ന തോന്നല്‍തന്നെ ഉണ്ടാവുകയില്ലല്ലോ. അതിനാല്‍ ശങ്കരാചാര്യര്‍ ലോകം മിഥ്യയാണെന്നും അവിദ്യയാണ് അത് യാഥാര്‍ഥ്യമാണെന്ന് തോന്നാന്‍ കാരണമെന്നും വരുത്താനായി പറഞ്ഞ ഉദാഹരണം പോലും പ്രപഞ്ചം മായയല്ല, യാഥാര്‍ഥ്യമാണെന്നല്ലേ തെളിയിക്കുന്നത്?
13. പരമാര്‍ഥികാര്‍ഥത്തില്‍ ജീവാത്മാവും ബ്രഹ്മവും അല്ലെങ്കില്‍ പരമാത്മാവും ഒന്നുതന്നെയാണെന്നും അതിനാല്‍ ജീവാത്മാവ് മിഥ്യയാണെന്നും വാദിക്കുന്ന ശങ്കരാചാര്യര്‍ ജീവാത്മാവ് ബ്രഹ്മജ്ഞാനം നേടുകവഴി പരമാത്മാവ് അല്ലെങ്കില്‍ ബ്രഹ്മമാണെന്ന് തിരിച്ചറിയുമെന്ന് സിദ്ധാന്തിക്കുന്നു. വ്യാവഹാരിക തലത്തില്‍ മാത്രം അസ്തിത്വമുള്ള, പരമാര്‍ഥിക തലത്തില്‍ നിലനില്‍പില്ലാത്ത ജീവാത്മാവ് എങ്ങനെയാണ് പരമാര്‍ഥിക തലത്തില്‍ പരമാത്മാവായി മാറുക? സൂര്യനും വെള്ളത്തിലുള്ള അതിന്റെ പ്രതിബിംബവും പോലെയാണ് പരമാത്മാവും ജീവാത്മാവുമെന്ന് ഗൗഢപാദനും ശങ്കരാചാര്യരും ഉദാഹരിക്കുന്നു. പ്രതിബിംബം സൂര്യനോ സൂര്യന്റെ ഭാഗമോ അല്ല. യഥാര്‍ഥ അസ്തിത്വം പോലുമില്ലാത്ത അതെങ്ങനെയാണ് സൂര്യനായി മാറുക? പരമാര്‍ഥികമായി ജീവാത്മാവ് മിഥ്യയാണെങ്കില്‍ മായയായ ജീവാത്മാവ് എങ്ങനെയാണ് ബ്രഹ്മജ്ഞാനം നേടുക? ജീവാത്മാവ് വ്യാവഹാരികമോ പരമാര്‍ഥികമോ? വ്യാവഹാരികമാണെങ്കില്‍ അഥവാ മിഥ്യയാണെങ്കില്‍ പിന്നെ എങ്ങനെയാണ് അസ്തിത്വമില്ലാത്ത ഒന്ന് ബ്രഹ്മജ്ഞാനം നേടുക? പരമാര്‍ഥികമാണെങ്കില്‍ പരമാത്മാവിനോടൊപ്പം ജീവാത്മാവും അസ്തിത്വമുള്ളവയായി ഉണ്ടാവുകയില്ലേ? ഇത് ദൈ്വതമല്ലേ?
14. ബ്രഹ്മജ്ഞാനം വ്യാവഹാരികമോ പരമാര്‍ഥികമോ? വ്യാവഹാരികമാണെങ്കില്‍ ഇല്ലാത്ത ഒന്നിനെ എങ്ങനെയാണ് സ്വായത്തമാക്കുക? പരമാര്‍ഥികമാണെങ്കില്‍ ബ്രഹ്മത്തോടൊപ്പം അസ്തിത്വമുള്ള ബ്രഹ്മജ്ഞാനമെന്ന മറ്റൊന്നുകൂടി ഉണ്ടാവില്ലേ?
15. വ്യാവഹാരികാര്‍ഥത്തില്‍ മാത്രം നിലനില്‍ക്കുന്നതും മായയും സഗുണവുമായ സ്വര്‍ണം, എട്ടുകാലി, ശരീരം പോലുള്ളവയെ നിര്‍ഗുണവും പാരമാര്‍ഥികവുമായ ബ്രഹ്മത്തോട് ഉപമിച്ചത് ശരിയാണോ? വ്യാവഹാരികാര്‍ഥത്തില്‍ മാത്രം നിലനില്പുള്ള, അഥവാ മായയായ എട്ടുകാലിയുടെ മായയായ നൂലിനെയും മായയായ ശരീരത്തിലെ മായയായ മുടിയെയും എങ്ങനെ പാരമാര്‍ഥികമായ ബ്രഹ്മത്തോടുപമിക്കും?
16. അദ്വൈതമെന്നത് സമൂഹത്തിന്ന് സ്വീകരിക്കാവുന്ന പ്രായോഗിക സിദ്ധാന്തമല്ല. ഇക്കാര്യം വിദ്യാവാചാസ്പതി വി. പനോളി തന്നെ പലതവണ വ്യക്തമാക്കിയിട്ടുണ്ട്:
''വിവേകവിചാരവൈരാഗ്യശാലികളായ ആളുകള്‍ ലോകത്ത് പത്തു ലക്ഷത്തിലൊരാള്‍ പോലുമില്ലെന്നതാണ് യാഥാര്‍ഥ്യം. ജ്ഞാനോദയത്താല്‍ അനുഗൃഹീതരും പ്രാപഞ്ചികവസ്തുക്കളുടെ ക്ഷണികത്വം ഗ്രഹിച്ചവരുമായ പുണ്യാത്മാക്കള്‍ മാത്രമേ മോക്ഷാര്‍ഥികളായി തീരുകയുള്ളൂ. അപ്രകാരമുള്ളവരുടെ എണ്ണം എത്രയോ പരിമിതമായിരിക്കും.''(ശ്രീ ശങ്കരദര്‍ശനം, പുറം 136)
''അനേകം കോടിയില്‍ ഒന്നോ രണ്ടോ പേര്‍ യഥാര്‍ഥത്തില്‍ മുക്തിമാത്രം ലക്ഷ്യമാക്കിയവരായിട്ടുണ്ടാവാം.''(Ibid പുറം 14)
കോടികളില്‍ ഒന്നോ രണ്ടോ പേര്‍ക്കു മാത്രം സ്വീകരിക്കാവുന്ന ഒരു സിദ്ധാന്തത്തിനുവേണ്ടി ജനകോടികള്‍ നിലകൊള്ളുന്നത് നിരര്‍ഥകവും വിഡ്ഢിത്തവുമല്ലോ.
17. അദ്വൈതസിദ്ധാന്തം അവതരിപ്പിച്ചുകൊണ്ട് ശ്രീശങ്കരാചാര്യര്‍ പറഞ്ഞു: ''കര്‍മണാം ചാ വിദ്വ ദ്വിഷയഃ'' (കര്‍മങ്ങള്‍ അജ്ഞാനികള്‍ക്കുള്ളതാണ്.)
''ഭാരതീയരെ കര്‍മവിമുഖരാക്കിത്തീര്‍ത്തു ശ്രീ ശങ്കരാചാര്യര്‍'' എന്ന് ആക്ഷേപസ്വരത്തില്‍ ആചാര്യരെ അല്പം വിമര്‍ശിക്കുന്നുണ്ട് അരവിന്ദന്‍, 'ഈശോപനിഷത്ത്' എന്ന തന്റെ കൃതിയില്‍ (ഉദ്ധരണം Ibid, പുറം 35)
ഇതിനെ വിമര്‍ശിച്ചുകൊണ്ട് വി. പനോളി എഴുതുന്നു: 'വാളിന്റെ വായ്ത്തലയിലൂടെ നടന്നുനീങ്ങുക എന്നതുപോലെ ദുഷ്‌കരമാണ് യഥാര്‍ഥത്തില്‍ ആദ്ധ്യാത്മചര്യ. കോടികളില്‍ ഒരാള്‍ക്കെങ്കിലും സംസാരസാഗരത്തിന്റെ മറുകരപറ്റാന്‍ കഴിഞ്ഞെങ്കില്‍ അതുതന്നെ മഹാഭാഗ്യം. അപ്പോള്‍ ഭാരതീയരെ കര്‍മവിമുഖരാക്കിത്തീര്‍ത്തു ആചാര്യപാദര്‍ എന്നു പറയുന്നത് നിരര്‍ഥകമാണ്.''(Ibid, പുറം 35)
കോടികളില്‍ ഒരാള്‍ക്കുപോലും അദ്വൈതം അംഗീകരിച്ച് ജീവിക്കാനാവില്ലെന്നല്ലേ ഇത് വ്യക്തമാക്കുന്നത്? അതോടൊപ്പം സമൂഹം അദ്വൈതം അംഗീകരിച്ച് ജീവിച്ചാല്‍ എല്ലാവരും കര്‍മവിമുഖരാകുമെന്നും നാടും സമൂഹവും നശിക്കുമെന്നുമുള്ള വിമര്‍ശനം ശരിയാണെന്ന് സമ്മതിക്കുകയല്ലേ പനോളി പോലും ചെയ്യുന്നത്?
18. ശങ്കരാചാര്യര്‍ പറയുന്നു: ''ആത്മജ്ഞാനോദയത്തില്‍ അവിദ്യ നീങ്ങും. കര്‍തൃത്വവും ഭോക്തൃത്വവും ജീവാത്മാവിന്റെ ധര്‍മങ്ങളാണ്. ജ്ഞാനികള്‍ക്ക് കര്‍മ്മമില്ല. ശരീരബന്ധമില്ല. ബ്രഹ്മോപാസകര്‍ ബ്രഹ്മമായി ഭവിക്കുന്നു.'' (ഉദ്ധരണം Ibid പുറം 113). ഇവ്വിധം അദ്വൈതാനുഭവം നേടിയ ആരെങ്കിലും ഇന്ന് ലോകത്തുണ്ടോ? അഥവാ, ആഹാരം കഴിക്കാതെയും മറ്റു ശാരീരികാവശ്യങ്ങള്‍ നിര്‍വഹിക്കാതെയും(അതൊക്കെയും മായയെ അവലംബിക്കലാണല്ലോ.) ജീവിക്കുന്ന ആരെങ്കിലും ലോകത്തുണ്ടോ? ഉണ്ടെങ്കില്‍ ആര്‍? എവിടെ? ശ്രീ ശങ്കരാചാര്യര്‍ക്കുപോലും കര്‍മമില്ലാതെയും ശരീരബന്ധമില്ലാതെയും ജീവിക്കാന്‍ സാധിച്ചിട്ടില്ലെന്നതല്ലേ സത്യം? സ്വയം ആചരിക്കാന്‍ കഴിയാത്ത സിദ്ധാന്തമല്ലേ അദ്ദേഹം മുന്നോട്ടു വെച്ചത്?
19. അദ്വൈതം മനുഷ്യര്‍ക്ക് അവലംബിക്കാവുന്ന ഒരു ജീവിത രീതി സമര്‍പ്പിക്കുന്നില്ലെന്നിരിക്കെ എന്താണ് ഈ സിദ്ധാന്തത്തിന്റെ പ്രയോജനം? കോടികളിലൊരാള്‍ക്കുപോലും ആചരിക്കാന്‍ കഴിയാത്ത സിദ്ധാന്തം എന്തിന്? ആര്‍ക്കുവേണ്ടി?
ഇത്തരം നിരവധി പ്രശ്‌നങ്ങള്‍ക്ക് അദ്വൈതത്തില്‍ ഉത്തരമില്ല. യുക്തിയുടെ വരുതിയില്‍ എല്ലാം വരണമെന്ന് വാദമുള്ളതിനാലല്ല ഇത്തരം പ്രശ്‌നങ്ങളുന്നയിക്കുന്നത്. മറിച്ച്, എല്ലാം യുക്തിയുടെ വെളിച്ചത്തില്‍ വിശകലനം ചെയ്യുന്ന സമീപനം അദ്വൈത വാദികള്‍ സ്വീകരിച്ചതിനാലാണ്. മാത്രമല്ല, അദ്വൈത സിദ്ധാന്തം യുക്തിക്ക് അതീതമാണെന്നതല്ല പ്രശ്‌നം. അയുക്തികവും യുക്തിവിരുദ്ധവുമാണെന്നതാണ്. അതുകൊണ്ടുതന്നെ അദ്വൈതം ദാര്‍ശനിക പ്രശ്‌നങ്ങള്‍ പരിഹരിക്കുകയല്ല; കൂടുതല്‍ സങ്കീര്‍ണമാക്കുകയാണ് ചെയ്യുന്നത്. അരവിന്ദ് ഘോഷ് ഇക്കാര്യം ഇങ്ങനെ വ്യക്തമാക്കുന്നു: ''മിഥ്യ എന്ന അര്‍ഥത്തിലുള്ള മായാ സിദ്ധാന്തവും പ്രപഞ്ചം അയഥാര്‍ഥമാണെന്ന വാദവും ദാര്‍ശനിക പ്രശ്‌നങ്ങള്‍ പരിഹരിക്കുകയല്ല; സൃഷ്ടിക്കുകയാണ് ചെയ്യുന്നത്. അസ്തിത്വപ്രശ്‌നം അത് യഥാര്‍ഥമായി പരിഹരിക്കുന്നില്ല, മറിച്ച് അതിനെ അപരിഹാര്യമാക്കി വിടുകയാണ്''((Sri. Aurobindo, The Life Divine, 1949, Quoted by Thomas O' Neil, P, 201, ഉദ്ധരണം: ബ്രഹ്മസൂത്രംദൈ്വതമോ അദ്വൈതമോ, പുറം: 63)
അദ്വൈത സിദ്ധാന്തത്തെ ഒരു ജീവിതാദര്‍ശമെന്ന നിലയില്‍ ഉള്‍ക്കൊള്ളാന്‍ അതിന്റെ ഉപജ്ഞാതാവായ ശങ്കരാചാര്യര്‍ക്കു പോലും സാധിച്ചില്ലെന്നതാണ് വസ്തുത. ക്ഷേത്രങ്ങളോ തീര്‍ഥങ്ങളോ ആവശ്യമില്ലെന്ന് ശക്തമായി വാദിച്ചിരുന്ന അദ്ദേഹം തന്നെ ക്ഷേത്രങ്ങളില്‍ പോയിരുന്നു. അതിന്റെ പേരില്‍ അദ്ദേഹത്തിന് ആസന്നമരണനായിരിക്കേ അനുതപിക്കേണ്ടിവരികയുമുണ്ടായി. ഡോ. രാധാകൃഷ്ണന്‍ എഴുതുന്നു: ''ശങ്കരന്‍ മൃത്യുശയ്യയില്‍ കിടക്കുന്ന ഘട്ടത്തില്‍, താന്‍ പലപ്പോഴും ക്ഷേത്രങ്ങളില്‍ പോവാറുണ്ടായിരുന്നതിനെക്കുറിച്ച് ക്ഷമായാചനം ചെയ്തതായി പറയപ്പെടുന്നു. എന്തെന്നാല്‍ അത് ഈശ്വരന്റെ സര്‍വവ്യാപിത്വത്തെ നിഷേധിക്കലായി തോന്നിയിരുന്നു'' (ഭാരതീയ ദര്‍ശനം, വാള്യം രണ്ട്, പുറം 647).
അദ്വൈതത്തോട് നീതി പുലര്‍ത്താത്ത കാര്യങ്ങള്‍ ശ്രീശങ്കരാചാര്യര്‍ തന്നെ പറഞ്ഞിട്ടുണ്ടെന്നത് സുസമ്മതമത്രെ. വി. പനോളി എഴുതുന്നു: ''ശ്രീമദ് ഭാരതീ തീര്‍ഥസ്വാമികള്‍ (ശൃംഗേരി മഠം ശ്രീശങ്കരാചാര്യര്‍) കോഴിക്കോട്ടുവന്നപ്പോള്‍ (1985ല്‍) ഞാന്‍ അദ്ദേഹത്തെ സന്ദര്‍ശിക്കുകയുണ്ടായി. ഉഛൃംഖലമായ പാണ്ഡിത്യത്തിന്റെ ഉടമയെന്നതിലേറെ ശങ്കര ദര്‍ശനത്തിന്റെ ആധികാരിക വക്താവുംകൂടിയാണല്ലോ സ്വാമികള്‍. ആ നിലക്ക് ആചാര്യ സ്വാമികളുടെ കൃതികളെക്കുറിച്ച് അദ്ദേഹത്തിനുള്ള അഭിപ്രായമായിരിക്കും കൂടുതല്‍ സ്വീകാര്യമായിരിക്കുക. ''വിശ്വേത്തര ശങ്കരഭാഷ്യത്തില്‍ നിശ്ശങ്കം രേഖപ്പെടുത്തിയിട്ടുള്ള ആശയങ്ങളുടെ ഔന്നത്യം പുലര്‍ത്താത്തതും ചിലേടത്ത് ആ ആശയങ്ങള്‍ക്ക് അല്പമെങ്കിലും മങ്ങലേല്പിക്കുന്നതുമായ ചിന്താശകലങ്ങള്‍ 'വിവേക ചൂഡാമണി' തുടങ്ങിയ കൃതികളില്‍ കാണുന്നുണ്ടല്ലോ'' എന്ന സംശയം ഈ ലേഖകന്‍ ഉന്നയിച്ചപ്പോള്‍ മഹാനായ സ്വാമികള്‍ പ്രതികരിച്ചത് എങ്ങനെയെന്നോ?
''കാളിദാസകൃതികളായ രഘുവംശത്തിലും മറ്റും വ്യാകരണനിയമം അനുവദിക്കാത്ത പല പദപ്രയോഗങ്ങളും നാം കാണുന്നില്ലേ? ഏതോ ചില സാഹചര്യങ്ങള്‍ക്കു വിധേയനായിട്ട് മഹാകവി അങ്ങനെയൊക്കെ പദപ്രയോഗം ചെയ്തു എന്നു വിചാരിപ്പാനല്ലേ ഉപപത്തിയുള്ളൂ? ആചാര്യപാദരുടെ വിവേകചൂഡാമണി തുടങ്ങിയ കൃതികളിലും മറ്റു ചിലേടത്തും ദര്‍ശിക്കാവുന്ന ആശയങ്ങളുടെ വ്യതിയാനത്തിന്റെ സ്വഭാവവും ഏറെക്കുറെ മേല്‍പറഞ്ഞ രീതിയില്‍ രൂപം പൂണ്ടതാണെന്നു കരുതിയാല്‍ മതി.''(ശ്രീശങ്കര ദര്‍ശനം, പുറം 44, 45)
ശ്രീശങ്കരാചാര്യര്‍ക്കുപോലും എപ്പോഴും അദ്വൈതസിദ്ധാന്തത്തെ മുറുകെ പിടിക്കാന്‍ സാധിച്ചിരുന്നില്ല എന്നല്ലേ ഇതു വ്യക്തമാക്കുന്നത്? ശങ്കരാചാര്യര്‍ക്ക് ബ്രഹ്മജ്ഞാനം ലഭിച്ചിരുന്നില്ല എന്നും ഇത് തെളിയിക്കുന്നു. ലഭിച്ചിരുന്നുവെങ്കില്‍ ആശയവ്യതിയാനം സംഭവിക്കുമായിരുന്നില്ലല്ലോ.
വേദവിരുദ്ധവും യുക്തിവിരുദ്ധവുമായ അദ്വൈതസിദ്ധാന്തത്തെ ഇസ്‌ലാം അംഗീകരിക്കുന്നില്ല. വേദങ്ങളും ഉപനിഷദ് വാക്യങ്ങളുടെ സമന്വയ രൂപമായ ബ്രഹ്മസൂത്രവും പഠിപ്പിക്കുന്നപോലെ ബ്രഹ്മം അല്ലെങ്കില്‍ ഈശ്വരന്‍ അഥവാ അല്ലാഹുവാണ് പ്രപഞ്ചത്തെയും അതിലുള്ളവയെയും സൃഷ്ടിച്ചതെന്നും ഈ ലോകം മിഥ്യയോ പരമാര്‍ഥിക യാഥാര്‍ഥ്യമോ അല്ലെന്നും ദൈവത്തിന്റെ സൃഷ്ടിയായ ക്ഷണിക യാഥാര്‍ഥ്യമാണെന്നും അത് സിദ്ധാന്തിക്കുന്നു. സ്രഷ്ടാവ് നിര്‍ഗുണനല്ല; സഗുണനാണ്. എന്നാല്‍ സ്രഷ്ടാവും സൃഷ്ടിയും ഒരിക്കലും ഒരു കാര്യത്തിലും സമമോ സദൃശമോ അല്ല. സ്രഷ്ടാവിന്റെ സവിശേഷതകളൊന്നും സൃഷ്ടികള്‍ക്കില്ല. സ്രഷ്ടാവിന് തുല്യമായി ഒന്നുമില്ലെന്നര്‍ഥം. വേദവും ബ്രഹ്മസൂത്രവും സിദ്ധാന്തിക്കുന്നതും ഇതുതന്നെ.
സര്‍വശക്തനും സര്‍വജ്ഞനും സര്‍വനിയന്താവുമായ അല്ലാഹുവിനെയാണ് ഖുര്‍ആന്‍ പരിചയപ്പെടുത്തുന്നത്. ബ്രഹ്മസൂത്രം പരിചയപ്പെടുത്തുന്ന ബ്രഹ്മവും അതുതന്നെ.
'ജന്മാദി അസ്യയഥാ' (ബ്രഹ്മസൂത്രം 1:1:2) എന്നതിന്റെ വ്യാഖ്യാനത്തില്‍ ഡോ. രാധാകൃഷ്ണന്‍ എഴുതുന്നു: ''പ്രകൃതിപരമായ വ്യാഖ്യാനത്തിന്റെ അടിസ്ഥാനത്തിലുള്ള ദൈവശാസ്ത്രമാണ് ഈ സൂത്രത്തിലുള്ളത്. അനുഭവവേദ്യമായ നിരീക്ഷണങ്ങളില്‍നിന്നും ആത്യന്തിക യാഥാര്‍ഥ്യത്തെക്കുറിച്ചുള്ള സിദ്ധാന്തം നാം രൂപീകരിക്കുന്നു. അടുത്ത സൂത്രം വൈദിക സ്രോതസ്സുകളിലേക്ക് നമ്മുടെ ശ്രദ്ധ തിരിക്കുന്നു. ലോകത്തിന്റെ സ്വഭാവത്തെപ്പറ്റി ചിന്തിക്കുമ്പോള്‍ ഏക പരമോന്നത യാഥാര്‍ഥ്യത്തെപ്പറ്റിയുള്ള നിഗമനത്തിലാണ് നാമെത്തുന്നത്. ലോകം അസ്തിത്വത്തിനും നിലനില്‍പിനും വിലയനത്തിനും അവലംബിക്കുന്ന സ്വയം പര്യാപ്തനായ സൃഷ്ടിവൈഭവനില്‍ നാമെത്തുന്നു''(Brahmastura, P. 226).
അപ്പോള്‍ ശങ്കരാചാര്യര്‍ തന്റെ യഥാര്‍ഥ അവലംബമെന്ന് അവകാശപ്പെടുന്ന ബ്രഹ്മസൂത്രം 'ബ്രഹ്മമാണ് ലോകത്തിന്റെ സൃഷ്ടിക്കും നിലനില്‍പിനും നാശത്തിനും നിമിത്ത'മെന്ന് (1:1:2) കൃത്യമായി വ്യക്തമാക്കുന്നു. ബ്രഹ്മത്തിന്റെ അഭീഷ്ടമാണ് സൃഷ്ടിയുടെ കാരണമെന്ന് സിദ്ധാന്തിക്കുന്നു. ഇസ്‌ലാമും ഇതുതന്നെയാണ് പറയുന്നത്. അല്ലാഹുവാണ് പ്രപഞ്ചത്തിന്റെ സ്രഷ്ടാവും സംരക്ഷകനും നാഥനും നിയന്താവുമെന്നും അവന്റെ ഇഛയും (ഇറാദത്ത്, അല്ലെങ്കില്‍ മശീഅത്ത്) തീരുമാന(ഖദാഅ്)വുമാണ് സൃഷ്ടിക്ക് കാരണമെന്നും അത് വ്യക്തമാക്കുന്നു. ഇതിനു കടകവിരുദ്ധമായ അദ്വൈത സിദ്ധാന്തത്തെ നിരാകരിക്കുകയും ചെയ്യുന്നു.

\chapter{കഅ്ബയിലെ കറുത്ത കല്ലും ശിലാപൂജയും} 
  \section{വിഗ്രഹാരാധനയെ ശക്തമായെതിര്‍ക്കുന്ന മതമാണല്ലോ ഇസ്‌ലാം. എന്നിട്ടും കഅ്ബയില്‍ ഒരു കറുത്ത കല്ല് പ്രതിഷ്ഠിച്ചത് എന്തിനാണ്? മറ്റെല്ലാ ബിംബങ്ങളെയും എടുത്തുമാറ്റിയപ്പോള്‍ അതിനെ മാത്രം നിലനിര്‍ത്തിയത് എന്തിന്? ശിലാപൂജ ഇസ്‌ലാമിലും ഉണ്ടെന്നല്ലേ ഇത് വ്യക്തമാക്കുന്നത്? }   
 ഈ ചോദ്യം സമൂഹത്തില്‍ നിലനില്‍ക്കുന്ന വ്യാപകമായ തെറ്റുധാരണയിലേക്കാണ് വിരല്‍ ചൂണ്ടുന്നത്. അതിനാല്‍ ചില കാര്യങ്ങളിവിടെ വ്യക്തമാക്കാനാഗ്രഹിക്കുന്നു:
1. കഅ്ബയിലെ കറുത്ത കല്ല്(ഹജറുല്‍ അസ്‌വദ്) ചരിത്രത്തിലൊരിക്കലും ആരാലും ആരാധിക്കപ്പെട്ടിട്ടില്ല. വിശുദ്ധ കഅ്ബയിലും പരിസരത്തും മുന്നൂറ്റി അറുപതിലേറെ വിഗ്രഹങ്ങള്‍ പൂജിക്കപ്പെട്ടപ്പോഴും ആരും അതിനെ പൂജിച്ചിരുന്നില്ല. പ്രവാചകനിയോഗത്തിനു മുമ്പുള്ള വിഗ്രഹാരാധനയുടെ കാലത്തും ഹജറുല്‍ അസ്‌വദ് ആരാധ്യവസ്തുവായിരുന്നില്ലെന്നതാണ് വസ്തുത.\\
2. ലോകത്തിലെ ഏത് പൂജാവസ്തുവും അതിന്റെ സ്വന്തം പേരിലറിയപ്പെടാറില്ല. ഏതിന്റെ പ്രതിഷ്ഠയാണോ അതിന്റെ പേരിലാണത് അറിയപ്പെടുക. ബ്രഹ്മാവ്, വിഷ്ണു, ശിവന്‍, കൃഷ്ണന്‍, ഗുരുവായൂരപ്പന്‍ പോലുള്ളവരുടെ വിഗ്രഹങ്ങള്‍ അവരുടെ പേരിലാണ് വിളിക്കപ്പെടുക. അല്ലാതെ കല്ലിന്‍ കഷണം, മരക്കുറ്റി, ഓട്ടിന്‍കഷണം, മണ്‍കൂന എന്നിങ്ങനെ, എന്തുകൊണ്ടാണോ അവ നിര്‍മിക്കപ്പെട്ടത് അവയുടെ പേരിലല്ല. എന്നാല്‍ കഅ്ബയിലെ കറുത്ത കല്ല് അതേ പേരിലാണ് അറിയപ്പെടുന്നത്. കറുത്ത കല്ല് എന്നതിന്റെ അറബിപദമാണ് 'ഹജറുല്‍ അസ്വദ്' എന്നത്. അത് പ്രതിഷ്ഠയോ പൂജാവസ്തുവോ വിഗ്രഹമോ അല്ലെന്നതിന് ഈ നാമം തന്നെ മതിയായ തെളിവാണ്.\\
3. നൂറും അഞ്ഞൂറും ആയിരവും മീറ്റര്‍ ഓട്ടമത്സരം നടക്കുമ്പോള്‍ ഓട്ടം ആരംഭിക്കുന്നേടത്ത് ഒരടയാളമുണ്ടാകുമല്ലോ. അവ്വിധം വിശുദ്ധ കഅ്ബക്കു ചുറ്റും പ്രയാണം നടത്തുമ്പോള്‍ അതാരംഭിക്കാനുള്ള അടയാളമാണ് ഹജറുല്‍ അസ്‌വദ്. അതിനപ്പുറം അതിന് പ്രത്യേക പുണ്യമോ ദൈവികതയോ കല്‍പിക്കാന്‍ പാടില്ലെന്ന് ഇസ്‌ലാം പഠിപ്പിക്കുന്നു. പ്രവാചകന്റെ അടുത്ത അനുയായിയും രണ്ടാം ഖലീഫയുമായ ഉമറുല്‍ ഫാറൂഖ് ഇക്കാര്യം അസന്ദിഗ്ധമായി വ്യക്തമാക്കിയിട്ടുണ്ട്. ആ കല്ലിന് പ്രത്യേക പുണ്യം ആരും കല്‍പിക്കാതിരിക്കാനായി അദ്ദേഹം പറഞ്ഞു: ''നീ കേവലം ഒരു കല്ലാണ്. നബി തിരുമേനി നിന്നെ ചുംബിച്ചില്ലായിരുന്നുവെങ്കില്‍ നിന്നെ ഞാനൊരിക്കലും മുത്തുമായിരുന്നില്ല.''\\
''കഅ്ബക്കു ചുറ്റുമുള്ള പ്രയാണത്തിന് പ്രാരംഭം കുറിക്കാന്‍ അടയാളമായി കല്ലുതന്നെ വേണമെന്നുണ്ടോ?''                                                                                                           ഇതിന്റെ കാരണം തീര്‍ത്തും ചരിത്രപരമാണ്. അല്ലാഹുവിനെ ആരാധിക്കാനായി ആദ്യമായി നിര്‍മിക്കപ്പെട്ട ഭവനമാണ് കഅ്ബ. ദൈവം നിശ്ചയിച്ച സ്ഥലത്ത് ഇബ്‌റാഹീം നബിയും മകന്‍ ഇസ്മാഈല്‍ നബിയും കൂടിയാണത് നിര്‍മിച്ചത്. ആ വിശുദ്ധ മന്ദിരത്തിന്റെ ഭാഗമെന്ന് തീര്‍ച്ചയുള്ള കല്ലാണ് ഹജറുല്‍ അസ്‌വദ്. അതിനാല്‍ പ്രവാചകന്‍മാര്‍ പണിത ദേവാലയത്തിന്റെ ഭാഗമെന്ന ചരിത്രപരമായ പ്രാധാന്യമാണ് ആ കറുത്ത കല്ലിനുള്ളത്. 
 \section{ദിവ്യത്വം കല്‍പിക്കപ്പെടുന്ന ആരാധ്യവസ്തുവല്ലെങ്കില്‍ എന്തിനാണ് അതിനെ ചുംബിക്കുന്നത്?  }  
 മനുഷ്യര്‍ പതിവായി ചുംബിക്കാറുള്ളത് ആരാധ്യവസ്തുക്കളെയല്ലല്ലോ; സ്‌നേഹിക്കപ്പെടുന്നവയെയാണല്ലോ. തന്റെ പൂര്‍വികരായ ഇബ്‌റാഹീം നബിയും ഇസ്മാഈല്‍ നബിയും പണിത ദൈവിക ഭവനത്തിന്റെ ഭാഗമെന്ന നിലയില്‍ മുഹമ്മദ് നബി അതിനെ സ്‌നേഹിക്കുകയും ആദരിക്കുകയും ചെയ്തു. തന്റെ വിടവാങ്ങല്‍ തീര്‍ഥാടനത്തില്‍ പ്രവാചകന്‍ അതിനെ ചുംബിച്ചു. അതിനാല്‍ എക്കാലവും എല്ലാ തീര്‍ഥാടകരും അതിനെ ചുംബിച്ചുവരുന്നു. പിന്നാലെ വരുന്ന വിശ്വാസികള്‍ തന്റെ മാതൃക പിന്തുടരുമെന്ന് പ്രവാചകനറിയാമായിരുന്നു. എന്നിട്ടും പ്രവാചകനതു ചെയ്തു. പിന്‍ഗാമികളുടെ ചുണ്ടുകള്‍ തന്റെ ചുംബനത്തിന്റെ ഓര്‍മകളുമായി കറുത്ത കല്ലിന്‍മേല്‍ പതിയുമെന്ന പ്രതീക്ഷയോടെത്തന്നെ. അതിനാല്‍ ഉപകാരമോ ഉപദ്രവമോ ചെയ്യാനാവാത്ത വെറുമൊരു കല്ല് മാത്രമാണതെന്ന് അദ്ദേഹം പ്രത്യേകം ഊന്നിപ്പറഞ്ഞു.
ഹജറുല്‍ അസ്‌വദിനുള്ള ചുംബനം പ്രവാചകനോടുള്ള സ്‌നേഹപ്രകടനം മാത്രമത്രെ. അതില്‍ ആരാധനാവികാരമില്ല. ഉണ്ടാകാവതുമല്ല. കാലത്തിനപ്പുറത്തേക്ക് തന്റെ മുഴുവന്‍ അനുയായികള്‍ക്കുമായി നബിതിരുമേനി അര്‍പ്പിച്ച പ്രതീകാത്മകമായ പരിരംഭണത്തില്‍ പങ്കുചേരുകയാണ് അതിനെ ഉമ്മവയ്ക്കുന്നവരൊക്കെയും. നിരവധി നൂറ്റാണ്ടുകളിലെ അനേകം തലമുറകളിലെ കോടാനുകോടി വിശ്വാസികളുടെ അധരസ്പര്‍ശമേറ്റിടത്ത് സ്വന്തം ചുണ്ടുകള്‍ വയ്ക്കുന്നതിലൂടെ വിശ്വാസി തന്നെ പൂര്‍വികരുമായി ബന്ധിപ്പിക്കുകയാണ് ചെയ്യുന്നത്. നൂറ്റാണ്ടുകളിലൂടെ അവിരാമം തുടര്‍ന്നുവരുന്ന വിശ്വാസി സമൂഹത്തിന്റെ മഹാപ്രവാഹത്തിലെ ഒരു കണിക മാത്രമാണ് താനെന്ന ബോധം അത് തീര്‍ഥാടകനിലുണര്‍ത്തുന്നു. അങ്ങനെ നിരവധി നൂറ്റാണ്ടുകളിലൂടെ പരന്നുകിടക്കുന്ന പൂര്‍വികരുമായി തന്നെ വൈകാരികമായി ബന്ധിക്കുന്നു. അത് മാനവതയുടെ ഏകതയെക്കുറിച്ച അവബോധം വളര്‍ത്തുകയും അവാച്യമായ അനുഭൂതി നല്‍കുകയും ചെയ്യുന്നു. അതിനപ്പുറം ആ കറുത്ത കല്ലിന് ദൈവികത കല്‍പിക്കുകയോ, അതിനെ പ്രതിഷ്ഠയായി ഗണിക്കുകയോ, ആരാധനാ വികാരത്തോടെ തൊടുകയോ ചുംബിക്കുകയോ വീക്ഷിക്കുകയോ ചെയ്യുന്നത് അക്ഷന്തവ്യമായ അപരാധമാണ്. അത് ഇസ്‌ലാം കണിശമായി വിലക്കിയ ബഹുദൈവാരാധനയുടെ ഭാഗമത്രെ. 

വിഗ്രഹാരാധനയെ എന്തിനെതിര്‍ക്കുന്നു? 
  മുസ്‌ലിംകള്‍ ധരിക്കുന്നതുപോലെ ഞങ്ങള്‍ ബഹുദൈവവിശ്വാസികളോ ബഹുദൈവാരാധകരോ അല്ല. ദൈവം ഏകനാണെന്ന് വിശ്വസിച്ച് ദൈവത്തെ ആരാധിക്കുന്നവരാണ്. വിഗ്രഹങ്ങള്‍ പ്രതിഷ്ഠിക്കുന്നത് ദൈവത്തെ ഓര്‍ക്കാനും ദൈവത്തില്‍ ശ്രദ്ധ കേന്ദ്രീകരിക്കാനുമാണ്. എന്നിട്ടും നിങ്ങളെന്തിനാണ് വിഗ്രഹാരാധനയെ കുറ്റപ്പെടുത്തുന്നത്?
 ഇസ്‌ലാം ദൈവത്തിന്റെ ഏകത്വം ഊന്നിപ്പറയുന്നു. അവനെ മാത്രമേ വിളിച്ചു പ്രാര്‍ഥിക്കുകയും ആരാധിക്കുകയും ചെയ്യാവൂ എന്ന് കണിശമായി കല്‍പിക്കുന്നു. ദൈവത്തിന്റെ സത്തയിലും ഗുണവിശേഷങ്ങളിലും അധികാരാവകാശങ്ങളിലും അവനു പങ്കുകാരെ കല്‍പിക്കുന്നത് അക്ഷന്തവ്യമായ അപരാധമായി കണക്കാക്കുകയും ചെയ്യുന്നു. അതുകൊണ്ടുതന്നെ നിരവധി കാരണങ്ങളാല്‍ വിഗ്രഹാരാധനയോട് വിയോജിക്കേണ്ടിവരുന്നു.
1. ദൈവം അരൂപിയും അദൃശ്യനുമാണെന്ന് ഹിന്ദുമതമുള്‍പ്പെടെ ലോകത്തിലെ എല്ലാ മതങ്ങളും ഉദ്‌ഘോഷിക്കുന്നു.
കേനോപനിഷത്തിലിങ്ങനെ കാണാം:\\
'യന്മനസാ ന മനുതേ യേനാഹുര്‍മനോമതം\\
തദേവ ബ്രഹ്മത്വം വിദ്ധി നേദം യദിദമുപാസതേ'' (1: 6)\\
(മനസ്സിന് അറിവാന്‍ കഴിയാത്തതും എന്നാല്‍ മനസ്സിന് അറിവാനുള്ള കഴിവിനെ നല്‍കുന്നതുമായതിനെ ബ്രഹ്മമെന്ന് അറിയുക. ഇതാണ് ബ്രഹ്മം എന്നു വിചാരിച്ച് ഉപാസിക്കുന്നതൊന്നും ബ്രഹ്മമല്ല.)\\
'യച്ചക്ഷുഷാ ന പശ്യതി യേന ചക്ഷൂംഷി പശ്യതി\\
തദേവ ബ്രഹ്മ ത്വം വിദ്ധി നേദം യദിദമുപാസതേ'' (1:7)\\
(കണ്ണുകൊണ്ട് കാണ്മാന്‍ കഴിയാത്തതും കണ്ണുകൊണ്ട് വിഷയങ്ങളെ കാണുന്നതിന് ഹേതുഭൂതമായിട്ടുള്ളതുമേതോ അത് ബ്രഹ്മമെന്നറിഞ്ഞാലും. ഇതാണതെന്ന നിലയില്‍ ഉപാസിക്കുന്നതൊന്നും ബ്രഹ്മമല്ല.)\\
'യത് ശ്രോത്രേണ ന ശൃണോതി യേന ശ്രോത്രമിദം ശ്രുതം\\
തദേവ ബ്രഹ്മ ത്വം വിദ്ധി നേദം യദിദമുപാസതേ''(1: 8)\\
(ചെവികൊണ്ട് കേള്‍ക്കാന്‍ കഴിയാത്തതും എന്നാല്‍ ചെവിക്ക് കേള്‍ക്കാനുള്ള കഴിവ് നല്‍കുന്നതുമായതേതോ അതാണ് ബ്രഹ്മമെന്നറിഞ്ഞാലും. അതാണിതെന്ന നിലയില്‍ ഉപാസിക്കുന്നതൊന്നും ബ്രഹ്മമല്ല.)\\
ദൈവം നിരാകാരനാണെന്ന് ഹൈന്ദവദര്‍ശനം അസന്ദിഗ്ധമായി വ്യക്തമാക്കുന്നു. ഉപനിഷത്ത് ദൈവത്തെ വിശേഷിപ്പിച്ചത് 'നിര്‍ഗത ആകാരാത് സ നിരാകാരഃ' യാതൊരുവന് ആകൃതിയൊന്നുമില്ലയോ അങ്ങനെയുള്ളവന്‍ എന്നാണ് (സ്വാമി ദയാനന്ദ സരസ്വതി, സത്യാര്‍ഥപ്രകാശം, ആര്യപ്രാദേശിക പ്രതിനിധി സഭ, പഞ്ചാബ്, പുറം :38)
സ്വാമി ദയാനന്ദ സരസ്വതി എഴുതുന്നു: ''ഈശ്വരന്‍ നിരാകാരനാകുന്നു. എന്തെന്നാല്‍ സാകാരനാണെങ്കില്‍ വ്യാപകനായിരിക്കുന്നതല്ല. വ്യാപകനല്ലെങ്കില്‍ സര്‍വജ്ഞത്വം മുതലായ ഗുണങ്ങള്‍ ഈശ്വരനില്‍ ഘടിക്കുകയില്ല. എന്തെന്നാല്‍ പരിഛിന്നമായ വസ്തുവിലുള്ള ഗുണകര്‍മസ്വഭാവങ്ങളും പരിഛിന്നങ്ങളായിരിക്കുമല്ലോ. എന്നു മാത്രമല്ല, ഈശ്വരന്‍ സാകാരനാണെങ്കില്‍ ശീതോഷ്ണങ്ങള്‍, രോഗങ്ങള്‍, ദോഷങ്ങള്‍, ഛേദനം, ഭേദനം മുതലായ ക്രിയകളൊന്നുമേല്‍ക്കാത്തവനായിരിപ്പാന്‍ കഴിയുകയില്ല. അതിനാല്‍ ഈശ്വരന്‍ നിരാകാരനാണെന്നുതന്നെയാണ് തീരുമാനിക്കപ്പെടുന്നത്.''(സത്യാര്‍ഥപ്രകാശം, പുറം 288)
ഖുര്‍ആന്‍ പറയുന്നു: ''അല്ലാഹു സകല വസ്തുക്കളെയും സൃഷ്ടിച്ചിരിക്കുന്നു. അവന്‍ സകല സംഗതികളും അറിയുകയും ചെയ്യുന്നു. അവനാകുന്നു നിങ്ങളുടെ നാഥനായ ദൈവം. അവനല്ലാതെ ഒരു ആരാധ്യനുമില്ല. സകല വസ്തുക്കളുടെയും സ്രഷ്ടാവാണവന്‍. അതിനാല്‍ നിങ്ങള്‍ അവനെ അനുസരിച്ചു ജീവിക്കുക. അവന്‍ എല്ലാ കാര്യങ്ങളുടെയും ഉത്തരവാദിത്വമേറ്റവനാകുന്നു. കണ്ണുകള്‍ക്കവനെ കാണാനാവില്ല. അവനോ, കണ്ണുകളെ കണ്ടുകൊണ്ടിരിക്കുന്നു. അവന്‍ സൂക്ഷ്മദൃക്കും അഭിജ്ഞനുമല്ലോ''(6:101-103).
അരൂപിയും അദൃശ്യനുമായ ദൈവത്തിന് രൂപം സങ്കല്‍പിക്കുന്നത് കൃത്രിമമാണ്. മിഥ്യയെ സത്യവുമായി കലര്‍ത്തലാണ്. അത് ദൈവത്തെ സംബന്ധിച്ച് വളരെ തെറ്റായ സങ്കല്‍പം വിശ്വാസികളില്‍ വളര്‍ത്തുന്നു. അതുകൊണ്ടുതന്നെ അത് ദൈവത്തോടു ചെയ്യുന്ന കടുത്ത അനീതിയാണ്.
ഇതു സംബന്ധമായി ബ്രഹ്മാനന്ദ സ്വാമി ശിവയോഗി എഴുതുന്നു: 'നമ്മുടെ പിതാവിന്റെയോ ഗുരുവിന്റെയോ മറ്റോ ഛായയെടുത്തുവെച്ച് അവരുടെ അഭാവത്തില്‍ അവരെ കണ്ട് സന്തോഷിക്കുന്നു; വന്ദിക്കുന്നു. അപ്രകാരം ക്ഷേത്രത്തില്‍ ദൈവത്തിന്റെ പ്രതിമവെച്ച് പൂജിച്ച് സന്തോഷിക്കുന്നതാകുന്നു. അങ്ങനെ ചെയ്തിട്ടില്ലെങ്കില്‍ സ്വല്പബുദ്ധികള്‍ക്ക് ദൈവം ഉണ്ടെന്ന് മനസ്സിലാകുന്നത് എങ്ങനെയാണ്? എന്ന് കാര്‍മന്‍മാര്‍ വാദിക്കുന്നു. ഇതും അസംബന്ധം തന്നെ. പിതാവിനെയും ഗുരുവിനെയും മറ്റും കണ്ടുംകൊണ്ടാണ് ഫോട്ടോ എടുക്കുന്നത്. ആ ഛായയില്‍ അവരുടെ ആകൃതിവടിവും ഉണ്ട്. ദൈവത്തിന് ആകൃതിയില്ല. പിന്നെ എങ്ങനെ ഛായയെടുത്തു? ആര്‍ എടുത്തു? ബിംബപ്പണിക്കാരും മറ്റും കല്ലുകളിലും വല്ലതിലും കൊത്തുന്നതും വരക്കുന്നതുമാണോ ദൈവത്തിന്റെ ഛായ? ഇങ്ങനെ നാനാ വിഗ്രഹരൂപത്തിനാല്‍ ദൈവസ്വരൂപത്തെയും ആരാധനാക്രമത്തെയും തെറ്റിദ്ധരിപ്പിക്കുന്നത് മഹാ അനര്‍ഥമാകുന്നു.'' (വിഗ്രഹാരാധനാഖണ്ഡനം, പ്രസാധകര്‍: നിര്‍മലാനന്ദയോഗി, പ്രസിഡന്റ് ബ്രഹ്മാനന്ദ ശിവയോഗി സിദ്ധാശ്രമം, ആലത്തൂര്‍; പുറം 28, 29)\\
2. നമ്മില്‍ ശ്രദ്ധ കേന്ദ്രീകരിക്കാനോ നമ്മെ ഓര്‍ക്കാനോ വേണ്ടി കുരങ്ങന്റെയോ നായയുടെയോ ചിത്രമോ പ്രതിമയോ സ്ഥാപിക്കുന്നത് നാമാരും ഇഷ്ടപ്പെടുകയില്ലല്ലോ. കാരണം, കുരങ്ങനും നായയുമൊക്കെ നമ്മേക്കാള്‍ നിസ്സാരവും താഴെയുമാണെന്ന് നാം ധരിക്കുന്നു. അവ്വിധം തന്നെ ഈ പ്രപഞ്ചത്തിലുള്ള എല്ലാ വസ്തുക്കളും ദൈവത്തിന്റെ സൃഷ്ടികളും അവനോട് താരതമ്യം ചെയ്യാനാവാത്ത വിധം നന്നേ നിസ്സാരവുമാണ്. അവയിലേതെങ്കിലും ഒന്നിനെ, ദൈവത്തെ ഓര്‍ക്കാനും അവനില്‍ ശ്രദ്ധ കേന്ദ്രീകരിക്കാനുമായി സ്ഥാപിക്കുന്നത് ദൈവത്തോടു ചെയ്യുന്ന കടുത്ത അനീതിയാണ്; അതുകൊണ്ടുതന്നെ അക്ഷന്തവ്യമായ അപരാധവും.\\
3. ദൈവത്തെ ആരാധിക്കേണ്ടത് എങ്ങനെയാണെന്ന് പറയേണ്ടത് ദൈവമാണ്. വിഗ്രഹമോ പ്രതിമകളോ പ്രതിഷ്ഠകളോ സ്ഥാപിച്ചാണ് തന്നെ ആരാധിക്കേണ്ടതെന്ന് ദൈവം പറഞ്ഞിട്ടില്ല. അത്തരമൊരു ആരാധനാരീതി പഠിപ്പിച്ചിട്ടുമില്ല. എന്നല്ല, വിഗ്രഹങ്ങളോ പ്രതിഷ്ഠകളോ സ്ഥാപിച്ച് ആരാധന നടത്തരുതെന്ന് കണിശമായി കല്‍പിക്കുകയും ചെയ്തിട്ടുണ്ട്.\\
4. സാക്ഷാല്‍ ദൈവത്തെയല്ലാതെ വിളിച്ചു പ്രാര്‍ഥിക്കുന്നതും ആരാധിക്കുന്നതും അരുതാത്തതാണെന്ന് ഇസ്‌ലാമിനെപ്പോലെ ഹിന്ദുമതവും പഠിപ്പിക്കുന്നുണ്ട്. വിഗ്രഹാരാധന ബഹുദൈവാരാധനയല്ലെങ്കില്‍ അന്യാരാധനയെ വിമര്‍ശിക്കേണ്ടതില്ലല്ലോ. ഛന്ദോഗ്യോപനിഷത്തില്‍ ഇങ്ങനെ കാണാം: ''ഓമിത്യേതദക്ഷരമുദ്ഗീഥമുപാസീത'' (ഓംകാരം യാതൊരുവന്റെ നാമധേയമാണോ, യാതൊരുവന്‍ ഒരിക്കലും നശിക്കയില്ലയോ അവനെയാണ് ഉപാസിക്കേണ്ടത്; മറ്റൊരുവനെയല്ല.)
വിഖ്യാതമായ ഹൈന്ദവസ്‌തോത്രത്തിലിങ്ങനെ കാണാം:\\
''ത്വമേകം വരണ്യം ത്വമേകം ശരണ്യം\\
ത്വമേകം ജഗത്കാരണം വിശ്വരൂപം''\\
(നിന്നെ മാത്രം ഞങ്ങള്‍ ആരാധിക്കുന്നു. നിന്നോടുമാത്രം ഞങ്ങള്‍ ശരണം തേടുന്നു. ലോകോല്‍പത്തിക്കു കാരണം നീ തന്നെ. നീ വിശ്വരൂപം.)\\
ശ്വേതാശ്വതരോപനിഷത്തിലിങ്ങനെ കാണാം:\\
'തമീശ്വരാണാം പരമം മഹേശ്വരം\\
തം ദേവതാനാം പരമം ച ദൈവതം\\
പതിം പതീനാം പരമം പരസ്താത്\\
വിദാമ ദേവം ഭുവനേശമീഡ്യം'' (67)\\
(അത് എല്ലാ ഈശ്വരന്മാരുടെയും മഹാനായ ശാസകനും സര്‍വ ദേവന്മാര്‍ക്കും പരമാരാധ്യനും സമസ്ത പതികളുടെയും പതിയും ധസംരക്ഷകന്‍പ സമസ്ത ബ്രഹ്മാണ്ഡനായകനും ആകുന്നു. സ്തുത്യര്‍ഹനും പ്രകാശ സ്വരൂപനുമായ അത് സര്‍വത്തിനും അതീതമാണെന്ന് നാം മനസ്സിലാക്കുന്നു.)
ഗീതയിലിങ്ങനെ കാണാം:\\
''യാന്തി ദേവപ്രതാ ദേവാന് പിതൃന് യാന്തി പിതൃ വ്രതാഃ\\
ഭൂതാനി യാന്തി ഭൂതേ ജ്യായാന്തി മദ്യാജിനോ ള ഹമാം.''\\
(ദേവന്മാരെ ഉപാസിക്കുന്നവര്‍ ദേവന്മാരെ പ്രാപിക്കുന്നു. പിതൃക്കളെ ആരാധിക്കുന്നവര്‍ പിതൃക്കളെ അണയുന്നു. ഭൂതങ്ങളെ ഉപാസിക്കുന്നവര്‍ ഭൂതങ്ങളിലെത്തുന്നു. എന്നാല്‍ എന്നെ ഉപാസിക്കുന്നവര്‍ എന്നെ പ്രാപിക്കുന്നു.) (ഉദ്ധരണം: വിദ്യവാചാസ്പതി വി. പനോളി, 'ശ്രീശങ്കരദര്‍ശനം', പുറം 91)
സ്വാമി ദയാനന്ദ സരസ്വതി എഴുതുന്നു: ''സ്തുതി, പ്രാര്‍ഥന, ഉപാസന എന്നിവയെല്ലാം ശ്രേഷ്ഠനായിട്ടുള്ളവനു മാത്രമേ ചെയ്യപ്പെടാറുള്ളൂ. ഗുണം, കര്‍മം, സ്വഭാവം, സത്യവ്യവഹാരം എന്നീ വിഷയങ്ങളില്‍ മറ്റെല്ലാവരെക്കാളും ഉല്‍ക്കര്‍ഷം ആര്‍ക്കുണ്ടോ അവനെയാണ് ശ്രേഷ്ഠന്‍ എന്നു പറയേണ്ടത്. ആര്‍ ശ്രേഷ്ഠന്മാരില്‍ വച്ച് ശ്രേഷ്ഠനാകുന്നുവോ അവനെ പരമേശ്വരന്‍ എന്നു പറയുന്നു. മറ്റൊരുവനേയുമല്ല. അവന് തുല്യനായിട്ട് ഒരുവന്‍ ഇതിനു മുമ്പ് ഉണ്ടായിട്ടില്ല. ഇപ്പോള്‍ ഇല്ല. ഇനിമേല്‍ ഉണ്ടാവുകയുമില്ല. അങ്ങനെയിരിക്കെ അവനെക്കാള്‍ ഉല്‍കൃഷ്ടനായി ഒരുവന്‍ ഉണ്ടാകുന്നത് എങ്ങനെ? സത്യം, ന്യായം, ദയ, സര്‍വകര്‍മസാമര്‍ഥ്യം, സര്‍വജ്ഞത്വം മുതലായ അസംഖ്യം ഗുണങ്ങള്‍ അവന്നുള്ളതുപോലെ മറ്റൊരു ജഡപദാര്‍ഥത്തിനോ ജീവാത്മാവിനോ ഇല്ല. സത്യസ്വരൂപമായി വിളങ്ങുന്ന വസ്തുവിന്റെ ഗുണകര്‍മ സ്വഭാവങ്ങളും സത്യങ്ങളായിത്തന്നെയിരിക്കും. അതിനാല്‍ മനുഷ്യരെല്ലാം പരമേശ്വരനെയാണ് സ്തുതിക്കുകയും പ്രാര്‍ഥിക്കുകയും ഉപാസിക്കുകയും ചെയ്യേണ്ടത്. മറ്റൊരുവനെയുമല്ല. ബ്രഹ്മാവ്, വിഷ്ണു, ശിവന്‍ എന്ന പേരോടു കൂടിയ പുരാതനന്മാരായ വിദ്വാന്മാരും ദൈത്യന്മാര്‍, ദാനവന്മാര്‍ തുടങ്ങിയ നികൃഷ്ട മനുഷ്യരും മറ്റുള്ള സാധാരണ മനുഷ്യരും പരമേശ്വരനില്‍ പൂര്‍ണവിശ്വാസത്തോടുകൂടി അദ്ദേഹത്തെത്തന്നെയാണ് സ്തുതിക്കുകയും പ്രാര്‍ഥിക്കുകയും ഉപാസിക്കുകയും ചെയ്തിരുന്നത്. വേറൊരുവനെയല്ല. അതുകൊണ്ട് നമ്മളെല്ലാവരും അപ്രകാരം ചെയ്യുന്നതാണുചിതമായിട്ടുള്ളത്''(സത്യാര്‍ഥപ്രകാശം, പുറം 12,13).
ബ്രഹ്മാവ്, വിഷ്ണു, ശിവന്‍ എന്നിവരെപ്പോലും സ്തുതിക്കുകയോ പ്രാര്‍ഥിക്കുകയോ ഉപാസിക്കുകയോ ചെയ്യരുതെന്നാണല്ലോ ഇതിന്റെയര്‍ഥം. ഇക്കാര്യം ശ്രീ വാഗ്ഭടാനന്ദ ഗുരുവും അസന്ദിഗ്ധമായി വ്യക്തമാക്കിയിട്ടുണ്ട്. (വാഗ്ഭടാനന്ദന്റെ സമ്പൂര്‍ണ കൃതികള്‍. ഒന്നാം പതിപ്പ്, മാതൃഭൂമി പ്രിന്റിംഗ് ആന്റ് പബ്‌ളിഷിംഗ് കമ്പനി ലിമിറ്റഡ്. പുറം 357-359).
ബ്രഹ്മാനന്ദ സ്വാമി ശിവയോഗി വസിഷ്ഠനില്‍നിന്ന് ഉദ്ധരിക്കുന്നു:\\
''നട്ടകല്ലൈത്തൈവമെന്റു നാലു പുഷ്പം ചാര്‍ത്തുരിര്‍\\
ചുറ്റിവന്തുമൊണന്റു ചൊല്ലുമന്തിരമെതക്കടോ\\
നട്ടുകല്ലു പേയുമോ നാതതുള്ളിരുക്കയില്‍\\
ചുട്ടചട്ടിചട്ടുകം കറിച്ചുവൈയറിയുമോ''\\
(ഒരു കല്ലിനെ പ്രതിഷ്ഠിക്കുന്നു. അതില്‍ ഈശ്വര ഭാവനയോടു കൂടി പൂജിക്കുന്നു. അതിനെ പ്രദക്ഷിണം വെക്കുന്നു. മന്ത്രം ജപിക്കുന്നു. ആ പ്രതിഷ്ഠിച്ച കല്ലു കേള്‍ക്കുമോ? കറിവെക്കുന്ന ചട്ടിയോ കറിയിളക്കുന്ന ചട്ടുകമോ, കറിയുടെ രസത്തെ അറിയുമോ?) (വിഗ്രഹാരാധനാ ഖണ്ഡനം, പുറം 7)\\
5. സാധാരണക്കാര്‍ക്ക് ആരാധിക്കാന്‍ വിഗ്രഹം വേണമെന്ന വാദത്തെ ബ്രഹ്മാനന്ദസ്വാമി ശിവയോഗി ഇങ്ങനെ ഖണ്ഡിക്കുന്നു: ''കുട്ടികള്‍ക്ക് ചെറിയ കുപ്പായം വേണം. വലിയ കുപ്പായം പറ്റുകയില്ല. അപ്രകാരം അല്പബുദ്ധികള്‍ക്ക് വിഗ്രഹാരാധന വേണം. ബ്രഹ്മധ്യാനത്തിന് കഴിയുകയില്ല'' എന്ന് വാദിക്കുന്നു. ഇത് ''കുട്ടികള്‍ക്ക് കാണ്മാന്‍ ഒരു ചെറിയ സൂര്യന്‍ വേണം. വലിയ സൂര്യന്റെ വെളിച്ചം നോക്കിയാല്‍ കാണുകയില്ല എന്നു പറയുന്നപോലെ അസംബന്ധമാകുന്നു.''(വിഗ്രഹാരാധനാഖണ്ഡനം)\\
6. വിഗ്രഹാരാധകരായ മഹാഭൂരിപക്ഷവും വിഗ്രഹങ്ങള്‍ക്ക് പ്രത്യേകമായ പുണ്യവും ദിവ്യത്വവും കല്‍പിക്കുന്നവരാണ്. ആരുടെ പ്രതിഷ്ഠയാണോ, അവരുടെ ചൈതന്യം അതില്‍ ആവാഹിക്കപ്പെട്ടതായാണ് വിഗ്രഹാരാധകരുടെ വിശ്വാസം. മറിച്ച്, ശ്രദ്ധ കേന്ദ്രീകരിക്കാന്‍ മാത്രമായിരുന്നെങ്കില്‍ മുന്നില്‍ എന്തെങ്കിലുമൊന്ന് വച്ചാല്‍ മതിയല്ലോ. എന്നു മാത്രമല്ല, നിലവിലുള്ള വിഗ്രഹം മാറ്റി മറ്റു വല്ലതും വച്ചാല്‍ എതിര്‍ക്കപ്പെടുകയുമില്ല. എന്നാല്‍ തങ്ങളാരാധിക്കുന്ന വിഗ്രഹത്തെ മാറ്റുന്നത് വിശ്വാസികള്‍ക്ക് സങ്കല്‍പിക്കുക പോലും സാധ്യമല്ല. അതിനാല്‍ വിഗ്രഹം ദൈവത്തെ ഓര്‍ക്കാനും ശ്രദ്ധ കേന്ദ്രീകരിക്കാനും മാത്രമാണെന്ന വാദം വസ്തുനിഷ്ഠമല്ല. അത് ബഹുദൈവാരാധന തന്നെയാണ്. അതിനാലാണ് ഛന്ദോഗ്യോപനിഷത്തിനും സ്വാമി ദയാനന്ദ സരസ്വതിക്കും വാഗ്ഭടാനന്ദഗുരുവിനുമെല്ലാം അന്യദൈവാരാധനയെ എതിര്‍ക്കേണ്ടിവന്നത്.\\
7. ദൈവം അദ്വിതീയനും അതുല്യനും അസദൃശനുമാണെന്ന് എല്ലാ മതങ്ങളും പറയുന്നു. ദൈവത്തെ സംബന്ധിച്ച് ഇന്നതുപോലെ എന്നുപോലും പറയുക സാധ്യമല്ല. അതോടൊപ്പം മനുഷ്യന്റെ കണ്ണും മനസ്സും എവിടെയാണോ കേന്ദ്രീകരിക്കുന്നത് അതിന്റെ പ്രതിരൂപമാണ് മനസ്സില്‍ പതിയുക. വിഗ്രഹാരാധകന്റെ മനസ്സില്‍ വിഗ്രഹത്തിന്റെ പ്രതിബിംബമാണ് സ്ഥാനം പിടിക്കുക. അതിനാല്‍ ആരാധകന്റെ മനസ്സില്‍ ദൈവത്തിനു പകരം വിഗ്രഹമാണ് സ്ഥാനം പിടിക്കുക. അങ്ങനെ ദൈവത്തിന്റെ സ്ഥാനം വിഗ്രഹം കൈയടക്കുന്നു.\\
8. ആരാധ്യനോടുള്ള ആത്യന്തികമായ ആദരവിനാല്‍ ആരാധകന്‍ പരമമായ വിനയത്തോടെ നിര്‍വഹിക്കുന്ന കര്‍മമാണല്ലോ ആരാധന. സൃഷ്ടികളില്‍ ശ്രേഷ്ഠനായ മനുഷ്യന്‍ അത് തന്നെപ്പോലുള്ളവര്‍ക്കോ തന്നേക്കാള്‍ താഴെയുള്ളവയ്‌ക്കോ അര്‍പ്പിക്കുന്നത് തന്റെ മഹിതമായ പദവിക്കും മാന്യതക്കും നിരക്കുന്നതല്ല. അതുകൊണ്ടുതന്നെ ദൈവേതരര്‍ക്കുള്ള ആരാധന സ്വന്തത്തോടുള്ള അതിക്രമം കൂടിയായാണ് ഇസ്‌ലാം കാണുന്നത്. സര്‍വോപരി, സ്വാമി ദയാനന്ദ സരസ്വതി വ്യക്തമാക്കിയപോലെ പരമമായ ആദരവ് അര്‍പ്പിക്കേണ്ടത് അതര്‍ഹിക്കുന്നവന്നാണ്. ശ്രേഷ്ഠതയിലും മഹത്വത്തിലും ജ്ഞാനത്തിലും ശക്തിയിലും പൂര്‍ണതയുള്ളവന്‍ ദൈവം മാത്രമാണ്. അതിനാല്‍ അവന്‍ മാത്രമേ ആരാധന അര്‍ഹിക്കുന്നുള്ളൂ. അതവനുമാത്രം അവകാശപ്പെട്ടതുമാണ്. ദൈവത്തിനു മാത്രം അവകാശപ്പെട്ടത് മറ്റുള്ളവര്‍ക്ക് നല്‍കല്‍ ദൈവഹിതത്തിനെതിരും അവയെ ദൈവത്തിന്റെ സ്ഥാനത്ത് അവരോധിക്കലുമാണ്. വിഗ്രഹാരാധന കൊടിയ പാപമാകാനുള്ള കാരണവും അതുതന്നെ.\\
\chapter{ദൈവാവതാരം സത്യമോ മിഥ്യയോ? }
  \section{അവതാരസങ്കല്‍പത്തെ ഇസ്‌ലാം അംഗീകരിക്കുന്നില്ലെന്ന് താങ്കള്‍ പറഞ്ഞുവല്ലോ. ദൈവം വിവിധ രൂപേണ അവതരിക്കുമെന്ന വിശ്വാസം ഇസ്‌ലാമിന്ന് വിരുദ്ധമാണോ?'}
 ഇസ്‌ലാമിക വീക്ഷണത്തില്‍ ദൈവം ഏകനാണ്, അനാദിയാണ്, അനന്തനാണ്, അദൃശ്യനാണ്, അരൂപിയാണ്, അതുല്യനാണ്, അസദൃശനാണ്, അവിഭാജ്യനാണ്, ജനിമൃതികള്‍ക്കതീതനാണ്. എന്നാല്‍ ദൈവാവതാരമെന്ന് അവകാശപ്പെടുന്നവയ്‌ക്കെല്ലാം ജനനമുണ്ട്, മരണമുണ്ട്, രൂപമുണ്ട്, സ്ഥലകാലവുമായി ബന്ധമുണ്ട്, പരിധികളും പരിമിതികളുമുണ്ട്. ദൈവം ഇതിനെല്ലാം അതീതനത്രെ. അതിനാല്‍ ഇസ്‌ലാം അവതാര സങ്കല്‍പത്തെ അനുകൂലിക്കുന്നില്ലെന്നു മാത്രമല്ല, ശക്തമായെതിര്‍ക്കുകയും ചെയ്യുന്നു. എന്നാല്‍ ഇസ്‌ലാം മാത്രമല്ല അതിനെ എതിര്‍ക്കുന്നത്. ഹൈന്ദവവേദദര്‍ശനവും അവതാരസങ്കല്‍പത്തിനെതിരാണ്. അത് ഹിന്ദുമത വിശ്വാസത്തിന്റെ ഭാഗമേയല്ല, പില്‍ക്കാലത്ത് കൂട്ടിച്ചേര്‍ക്കപ്പെട്ടതാണ്. 
ദൈവം മനുഷ്യശരീര രൂപമണിയുന്നവനാണെന്ന വിശ്വാസത്തെ ഭഗവത്ഗീത ശക്തമായെതിര്‍ക്കുന്നു:\\
''അവജാനന്തി മാം മൂഢാഃ മാനുഷീം തനുമാശ്രിതം\\
പരം ഭാവമജാനന്തോ മമ ഭൂതമഹേശ്വരം.\\
മോഘാശാ മോഘകര്‍മാണോ മോഘജ്ഞാനാ വിചേതസഃ\\
രാക്ഷസീമാസൂരീം ചൈവ പ്രകൃതിം മോഹിനീം ശ്രിതാഃ''\\
(ഭൂതങ്ങളുടെ മഹേശ്വരനെന്ന പരമമായ എന്റെ ഭാവത്തെ അറിയാത്ത മൂഢന്മാര്‍ എന്നെ മാനുഷികമായ ശരീരത്തെ ആശ്രയിച്ചവനായി നിന്ദിക്കുന്നു. അങ്ങനെ എന്നെ ധരിക്കുന്നവരുടെ ആശകളും അവര്‍ ചെയ്യുന്ന കര്‍മങ്ങളും അവര്‍ക്കുള്ള ജ്ഞാനവും നിഷ്ഫലങ്ങളാണ്. അവര്‍ അവിവേകികളും മനസ്സിനെ മോഹിപ്പിക്കുന്ന രാക്ഷസപ്രകൃതിയെയും അസുരപ്രകൃതിയെയും ആശ്രയിച്ചുള്ളവരാകുന്നു.) (അധ്യായം 9, രാജവിദ്യാരാജ ഗുഹ്യയോഗം, ശ്ലോകം: 11, 12)

നമ്മുടെ രാജ്യത്ത് ദൈവാവതാരങ്ങളായി ഗണിച്ച് ആരാധിക്കപ്പെടുന്ന ബ്രഹ്മാവ്, വിഷ്ണു, ശിവന്‍, ശ്രീരാമന്‍, ശ്രീകൃഷ്ണന്‍ എന്നിവരെക്കുറിച്ച് പ്രമുഖ വേദപണ്ഢിതനായ വാഗ്ഭടാനന്ദ ഗുരു എഴുതുന്നു:\\
''സൃഷ്ടിസ്ഥിത്യന്തകരണീം ബ്രഹ്മവിഷ്ണു ശിവാത്മികാം\\
സ സംജ്ഞാം യാതി ഭഗവാനേക ഏവ മഹേശ്വരഃ '' (വിഷ്ണുപുരാണം 1-2-66).\\
അര്‍ഥം: സൃഷ്ടി സ്ഥിതി സംഹാരങ്ങള്‍ നടത്തുന്നതുകൊണ്ട് ഏകനായ ഈശ്വരനെ ബ്രഹ്മാവെന്നും വിഷ്ണുവെന്നും ശിവന്‍ എന്നും വിളിക്കുന്നു. ഇതുകൊണ്ട് പൗരാണികന്മാരുടെ ഇടയിലും ഏകദൈവവിശ്വാസമാണുള്ളതെന്ന് അനായാസേന ഗ്രഹിക്കാവുന്നതാണ്.

സത്യം അങ്ങനെയാണെങ്കിലും പൗരാണികന്മാര്‍ തുടങ്ങി ഇതുവരെയുള്ള ബഹുജനങ്ങള്‍ ബ്രഹ്മാദികളെ വെവ്വേറെ ഈശ്വരന്മാരായി വിശ്വസിച്ചുവരുന്നുണ്ടല്ലോ! ഇതിനു പരമാണുഗ്രായമായിട്ടെങ്കിലും വല്ല കാരണവും ഉണ്ടോ, ഉണ്ടെങ്കില്‍ അത് എന്തായിരിക്കും എന്നു ചോദിക്കുന്നു. എങ്കില്‍ ഞാന്‍ കുറഞ്ഞൊരു സമാധാനം പറയാം. പണ്ട് ബ്രഹ്മാവ്, വിഷ്ണു, ശിവന്‍ എന്നും മറ്റും ചില മനുഷ്യര്‍ ഉണ്ടായിരുന്നു. അവരെല്ലാം മതവാദികളും ബുദ്ധമതപ്രചാരണത്തോടുകൂടി ഉടഞ്ഞുതകര്‍ന്നുപോയ പൗരോഹിത്യ പ്രഭാവക്കോട്ടയുടെ പുനരുദ്ധാരണത്തിന്നായി കൈയുയര്‍ത്തി ചാടിപ്പുറപ്പെട്ട വീരവീരന്മാരും ആയിരുന്നു. ഇവരുടെ ഇടയില്‍ ബ്രഹ്മാവ് കുറെ നല്ല മനസ്ഥിതിയുള്ള ഒരാളായിരുന്നു എന്നു വിശ്വസിക്കാം. അദ്ദേഹം തന്റെ മതം പ്രചരിപ്പിക്കുന്നതിന് ആയുധം ഉപയോഗിച്ചിരുന്നില്ല. അതായിരുന്നു അദ്ദേഹത്തിനുള്ള വൈശിഷ്ട്യം. വിഷ്ണുവും ശിവനും തങ്ങളുടെ മതപ്രചാരണത്തിന്നായി കരവാള്‍, ഗദ, ശൂലം മുതലായവ നിഷ്‌കരുണം ഉപയോഗിച്ചു. അതിന്റെ ഫലമായിട്ട് അവരുടെ മതം ഇന്ത്യയില്‍ പ്രബലപ്പെട്ടു. വിഷ്ണു, വികുണ്ഠ എന്ന സ്ത്രീയുടെ മകനും വൈകുണ്ഠനെന്ന് പ്രസിദ്ധനുമാകുന്നു. അദ്ദേഹം ആര്യവര്‍ഗത്തില്‍ പെട്ട ഒരുവനായിരുന്നു എന്നു വിശ്വസിക്കുന്നതിന് വിരോധമില്ല. ഭിക്ഷാശിയായി സഞ്ചരിച്ചതുകൊണ്ടും സര്‍പ്പവശീകരണത്തില്‍ സാരസ്യം കാട്ടിയിരുന്നതുകൊണ്ടും 'നങ്ങ' എന്ന കുറവസ്ത്രീയുടെ പിതാവായിരുന്നതുകൊണ്ടും മദ്യപാനത്തില്‍ ആസക്തനായിരുന്നു എന്നറിയപ്പെട്ടതുകൊണ്ടും മറ്റും 'ശിവന്‍' ഒരു കുറവനായിരുന്നു എന്നൂഹിക്കാന്‍ വഴിയുണ്ട്. ഏതായാലും ശിവന്റെ നൈവേദ്യത്തെ ഇന്നും ബ്രാഹ്മണജാതിക്കാര്‍ ഭക്ഷിക്കാത്തതുകൊണ്ട് അദ്ദേഹം താഴ്ത്തപ്പെട്ട ഒരു ജാതിയില്‍ ഉള്‍പ്പെട്ട ആളായിരിക്കണമെന്നത് തീര്‍ച്ച തന്നെ. 'ബ്രഹ്മാവിന്റെ കഴുത്തറുത്ത പാതകം നിമിത്തമാണ് ബ്രാഹ്മണ ജാതിക്കാര്‍ ശിവനൈവേദ്യം ഭക്ഷിക്കാത്തത'' എന്നു പറയുന്നതു സ്വീകാര്യമല്ല. അങ്ങനെയാണെങ്കില്‍ അനേകായിരം ജനങ്ങളുടെ ഗളനാളം മുറിച്ച് ജനാര്‍ദനന്‍ എന്ന പേര്‍ സമ്പാദിച്ചതിന് പുറമെ മാതൃഹത്യാ മഹാപാതകവും കൂടി ചെയ്ത വിഷ്ണുവിന്റെ നൈവേദ്യം മേല്‍ജാതിക്കാര്‍ ആഹരിക്കുന്നതെങ്ങനെ? ഈ വിഷ്ണുവും ശിവനും പരമദ്രോഹികളായിരുന്നു എന്നതിനുപുറമെ അന്യോന്യമത്സരികളുമായിരുന്നു. ശിവന്‍ ബ്രഹ്മമതത്തെ നിഷ്പ്രയാസം ഉടച്ചിരുന്നു. ബ്രഹ്മാവിന്റെ ശിരസ്സറ്റുപോയി എന്ന ഐതിഹ്യത്തിന്റെ മൂലം അതാണ്. ബ്രഹ്മമതം തകര്‍ന്നുപോകയാല്‍ ബ്രഹ്മാവിനെ കുടിവെപ്പാനും ആരാധിപ്പാനും അധികം പേരുണ്ടായിരുന്നില്ല. തന്മൂലം അദ്ദേഹം അനാരാധ്യകോടിയില്‍ ഗണിക്കപ്പെട്ടു. വൈഷ്ണവന്മാരും ശൈവന്മാരും കൂടുതല്‍ ഉണ്ടാകയാല്‍ വിഷ്ണുവിന്റെയും ശിവന്റെയും മരണശേഷം അവരെ സാദരം കുടിവയ്ക്കുന്നതിനും ആരാധിക്കുന്നതിനും അത്യധികം സൗകര്യമുണ്ടായി. ഇന്ത്യയില്‍ വിഷ്ണുക്ഷേത്രങ്ങളും ശിവക്ഷേത്രങ്ങളും ധാരാളം ആവിര്‍ഭവിച്ചതിനു കാരണം ഇതുതന്നെ. ആയുധങ്ങള്‍ ഉപയോഗിച്ചു അനവരതം സാധുദ്രോഹം ചെയ്തിരുന്നതുകൊണ്ടും അന്യോന്യം കലഹങ്ങള്‍ നടത്തിയിരുന്നതുകൊണ്ടും മറ്റു പല കാരണങ്ങള്‍കൊണ്ടും ശിവന്റെയും വിഷ്ണുവിന്റെയും മനസ്ഥിതി ലേശം പോലും ഉല്‍കൃഷ്ടമായിരുന്നില്ലെന്ന് ധരിക്കുവാന്‍ ബുദ്ധിയുള്ള ഏതു മനുഷ്യന്നും എളുപ്പമാകുന്നു. ശിവന്റെ ദുഃസ്ഥിതിയെ വെളിവായി ഗ്രഹിക്കണമെങ്കില്‍ അദ്ദേഹം ഏര്‍പ്പെടുത്തിയ ശാക്തേയത്തിലുള്ള വാമതന്ത്രം ഒന്നു വായിച്ചുനോക്കിയാല്‍ മതി. മനുഷ്യരെ ഉടലോടെ നരകത്തില്‍ തള്ളിവിടുന്ന നഗ്‌നങ്ങളും ബീഭത്സങ്ങളുമായ ദുരാചാരങ്ങളാണ് അതില്‍ പ്രതിപാദിക്കപ്പെടുന്നത്. വൈഷ്ണവ മതത്തെ പ്രബലമായി പ്രചരിപ്പിച്ചവരെ വിഷ്ണുവിന്റെ അവതാര പുരുഷന്മാരാക്കിയും ശൈവമതത്തിന് പ്രാബല്യം വരുത്തിയവരെ ശിവന്റെ പുത്രന്മാരാക്കിയും വര്‍ണിക്കുകയാണ് പ്രായേണ ചെയ്തുകാണുന്നത്. ശ്രീരാമനും ശ്രീകൃഷ്ണനും മറ്റും അവതാര പുരുഷന്മാരായതും ഗണപതിയും സുബ്രഹ്മണ്യനും ശിവസൂതന്മാരായതും ഈ വഴിയില്‍ കൂടിയാകുന്നു.'' (വാഗ്ഭടാനന്ദന്റെ സമ്പൂര്‍ണ കൃതികള്‍, മാതൃഭൂമി പ്രിന്റിംഗ് ആന്റ് പബ്ലിഷിംഗ് കമ്പനി ലിമിറ്റഡ്, കോഴിക്കോട്. 1998 പുറം 751-752)\\

വാഗ്ഭടാനന്ദ ഗുരു തന്നെ ഇക്കാര്യം സംഭാഷണ രൂപേണ ഇങ്ങനെ വ്യക്തമാക്കുന്നു:\\
രാമന്‍: സ്‌നേഹിത, നിങ്ങള്‍ ഈശ്വരവിശ്വാസമില്ലാതെ ജീവിക്കുന്നതില്‍ ഞാന്‍ വ്യസനിക്കുന്നു.\\
കൃഷ്ണന്‍: ഞാന്‍ ഈശ്വരവിശ്വാസിയല്ലെന്ന് നിങ്ങളോടാരാണ് പറഞ്ഞത്?\\
രാമന്‍: ആരെങ്കിലും പറഞ്ഞുതരേണമോ? ശിവന്‍, വിഷ്ണു, ബ്രഹ്മാവ് തുടങ്ങിയ സകാരമൂര്‍ത്തികളെ നിഷേധിച്ചു പ്രസംഗങ്ങള്‍ നടത്തുന്ന നിങ്ങള്‍ ഒരു ഭയങ്കരനായ നിരീശ്വരനല്ലേ?\\
കൃഷ്ണന്‍: സ്‌നേഹിത, ആകാരഭേദങ്ങളോടുകൂടിയ ശിവനും വിഷ്ണുവും മറ്റും ഈശ്വരസ്ഥാനത്തില്‍ അഭിഷിക്തരാകുന്നതിന് അര്‍ഹന്മാരല്ലെന്നും ശിവാദികളെ സൃഷ്ടിച്ച സര്‍വശക്തനാണ് ഈശ്വരനെന്നും വിശ്വസിക്കുന്ന ഞാനല്ലയോ ഈശ്വരവിശ്വാസി? നശ്വരങ്ങളും ഈശ്വരസൃഷ്ടങ്ങളുമായ സാകാരവസ്തുക്കളെ ഈശ്വരന്മാരെന്ന് വിശ്വസിച്ചും തദനുസാരം ആരാധിച്ചും വരുന്ന നിങ്ങളല്ലെയോ ധൂര്‍ത്തന്മാരായ നിരീശ്വരന്മാര്‍?\\
രാമന്‍: ആശ്ചര്യം! ആശ്ചര്യം! സാകാരമൂര്‍ത്തികളായ ശിവാദികള്‍ നശ്വരന്മാരാണോ? എന്തൊരധികപ്രസംഗമാണിത്? എനിക്കിതൊന്നും കേള്‍ക്കേണ്ട.\\
കൃഷ്ണന്‍: സ്‌നേഹിത, ആകാരമുള്ള എല്ലാ പദാര്‍ഥങ്ങളും നശ്വരങ്ങളാണ്. ഇതിന്റെ തെളിവിന്ന് ഒരു ഘടത്തെ പരിശോധിച്ചാല്‍ മതി. ഘടം ആകാരമുള്ളതാകുന്നു, അത് നശിക്കുന്നതുമാകുന്നു. ഈ നിയമം പരമാണു കാരണവാദികള്‍ കൂടി സമ്മതിക്കാതിരിക്കില്ല. എന്തുകൊണ്ടെന്നാല്‍ ഈ നിയമം എവിടെയും ബാധകമാകുന്നു. എന്നാല്‍ സാകാരമായ ആ ഘടത്തിന് ഒരു കര്‍ത്താവു കൂടി വേണം. അല്ലാതെ ഘടം ഉണ്ടാവുകയില്ല. ആ കര്‍ത്താവാണ് കുശവന്‍. അതുപോലെ ശിവാദികള്‍ സാകാരന്മാരാണെങ്കില്‍ അവര്‍ നശിക്കുന്നവര്‍തന്നെ, സംശയമില്ല. അവരുടെ ഉല്‍പത്തിക്ക് ഒരു കര്‍ത്താവുണ്ടായിരിക്കണമെന്ന് അനുമാനിക്കണം. ആ കര്‍ത്താവത്രെ ഈശ്വരന്‍. 'വിശ്വസ്യ കര്‍ത്താ' എന്നും മറ്റും വേദങ്ങള്‍ ശ്ലാഘിക്കുന്നത് ആ ഈശ്വരനെയാകുന്നു. താദൃശനായ ഈശ്വരനെ വിശ്വസിക്കുകയും ജനിച്ചു മരിച്ച ശിവാദികളെ ഈശ്വരനെന്ന് വിശ്വസിക്കാതിരിക്കുകയും ചെയ്യുക. അതാണ് ഉചിതം. കഥയില്ലാതെ വല്ലതും വിശ്വസിക്കുകയും ആ മൂഢവിശ്വാസത്തെ ആക്ഷേപിക്കുന്ന ദയാശാലികളെ അധികപ്രസംഗികളെന്ന് അധിക്ഷേപിക്കുകയും ചെയ്യുന്നത് സുജനതക്ക് യോജിച്ചതല്ല.\\
രാമന്‍: കഷ്ടം! കഷ്ടം! ശിവാദികള്‍ ജനിച്ചു മരിച്ചവരെന്നോ! എന്തൊരു ധിക്കാരമാണിത്? ഇങ്ങനെ ആരെങ്കിലും പറയുമോ?\\
കൃഷ്ണന്‍: സ്‌നേഹിതാ, ശിവാദികള്‍ ജനിച്ചവരെങ്കില്‍ മരിക്കുകയില്ലേ? മരിച്ചിട്ടില്ലെങ്കില്‍ നാം കാണേണ്ടതുമാണ്; അവര്‍ സാകാരന്മാരാണല്ലോ! അവരെവിടെ? എന്തുകൊണ്ട് കാണുന്നില്ല?\\
രാമന്‍: ശിവ! ശിവ! അവരെ നാം കാണുകയോ? പാപികളായ നമുക്ക് അത്ര ഭാഗ്യമുണ്ടോ? പുണ്യവാളന്മാരായ സജ്ജനങ്ങള്‍ക്കു മാത്രമേ അവരെക്കാണാന്‍ കഴിയുകയുള്ളൂ!\\
കൃഷ്ണന്‍: രാവണന്‍ തുടങ്ങിയ രാക്ഷസന്മാരേക്കാള്‍ നാം പാപികളാണെന്നോ?\\
രാമന്‍: അതല്ല, രാവണാദികള്‍ നരഹത്യയും ബ്രഹ്മഹത്യയും ദേവന്മാരെ അടിച്ചോടിക്കുക എന്ന പാപകൃത്യവും ചെയ്ത ഘോരപാപികളല്ലേ? അത്ര പാപം നാം ചെയ്യുന്നില്ല. ചെയ്യുകയുമില്ല! രാവണാദികളേക്കാള്‍ നാം സുകൃതം ചെയ്തവരാണെന്ന്\\ പറയേണ്ടതുണ്ടോ?\\
കൃഷ്ണന്‍: രാവണനും മറ്റും ശിവാദികളെ പലപ്പോഴും കാണുകയുണ്ടായല്ലോ. എന്നു മാത്രമല്ല, ബാണാസുരന്റെ കോട്ടയില്‍ കുറച്ചുകാലം ശിവന്‍ കാവല്‍ക്കാരനായിരുന്നില്ലേ? ആദ്യത്തെ ഭാര്യ മരിച്ചതിനു ശേഷം പാര്‍വതിയെന്ന മലയിപ്പെണ്ണിനെ അദ്ദേഹം കല്യാണം കഴിച്ചില്ലേ? ഇങ്ങനെ മഹാപാപികളായ രാക്ഷസന്മാര്‍ക്കു കൂടി ശിവനെ കാണാന്‍ മാത്രമല്ല, ഭൃത്യനാക്കുവാന്‍ കൂടി സാധിച്ചിട്ടുണ്ടെങ്കില്‍ ഒരു മലയിക്കുറത്തിപ്പെണ്ണിന് അദ്ദേഹത്തെ ഭര്‍ത്താവാക്കി വരിക്കുവാന്‍ കഴിഞ്ഞിട്ടുണ്ടെങ്കില്‍ അദ്ദേഹത്തെ നമുക്ക് പാപശക്തിയാല്‍ കാണാന്‍ കഴിയില്ലെന്ന് പറയുന്നത് കഥകേടല്ലേ?\\
രാമന്‍: ശിവന്‍ രണ്ടാമതും കല്യാണം കഴിച്ചത് മലയത്തിപ്പെണ്ണിനെയോ? എന്താ പറയുന്നത്? അനാവശ്യമായി അസംബന്ധം പറയുന്നത് കേള്‍ക്കുമ്പോള്‍ എനിക്ക് ദേഷ്യം വളരുന്നു. വേണ്ട, വേണ്ടാസനം പറയണ്ട!\\
കൃഷ്ണന്‍: വേണ്ടാസനം പറയുകയല്ല. വേണ്ട ആസനം പറയുകയാണ്. പാര്‍വതിയെ ശിവന്‍ കല്യാണം കഴിച്ചതായി നിങ്ങള്‍ കേട്ടിട്ടുണ്ടോ?\\
രാമന്‍: ഉവ്വ്, കേട്ടിട്ടുണ്ട്.\\
കൃഷ്ണന്‍: പാര്‍വതിയെന്നു വച്ചാല്‍ ആരാണെന്നാണ് നിങ്ങള്‍ മനസ്സിലാക്കിയിട്ടുള്ളത്? പര്‍വതം മലയാണെങ്കില്‍ പാര്‍വതി മലയിയല്ലേ? സംസ്‌കൃതഭാഷയില്‍ ഭ്രമിക്കാതെ അര്‍ഥമൊന്നു ചിന്തിച്ചുനോക്കുക! മാതംഗന്‍ എന്ന മലക്കുറവന്റെ മകളായ പാര്‍വതിയെന്ന മലയിയെക്കുറിച്ച് നിങ്ങള്‍ മറ്റൊരു പ്രകാരം ധരിച്ചിട്ടുണ്ടെങ്കില്‍ ഈ നൂറ്റാണ്ടില്‍ ചിലവാകാത്ത ആ ധാരണാഭേദവും കൊണ്ട് മറയത്തുപോകുന്നതാണ് നല്ലത്! കല്യാണം കഴിച്ച് മക്കളോടുകൂടി അങ്ങനെയങ്ങനെ ജീവിച്ചും മരിച്ചും മണ്ണടിഞ്ഞും പോയ ശിവാദികളെക്കുറിച്ചുള്ള വിശ്വാസം അകലെ തള്ളി ശാശ്വതനും നിരാകാരനുമായ ഈശ്വരനെ വിശ്വസിച്ചു നല്ല മനുഷ്യനായി ജീവിക്കാന്‍ നോക്കൂ! ഞങ്ങളോട് ഉപദേശിപ്പാന്‍ വരുന്നത് പിന്നെയാവട്ടെ''(വാഗ്ഭടാനന്ദന്റെ സമ്പൂര്‍ണ കൃതികള്‍. പുറം :357-359)\\

ശ്രീസ്വാമി ദയാനന്ദസരസ്വതി ഈ വിഷയം ചോദ്യോത്തര രൂപത്തില്‍ ഇങ്ങനെ അവതരിപ്പിക്കുന്നു:
''പ്രശ്‌നം: ഈശ്വരന്‍ അവതാരം സ്വീകരിക്കുന്നുണ്ടോ ഇല്ലയോ?\\
ഉത്തരം: ഇല്ല, എന്തെന്നാല്‍ 'അജ ഏകപാത്' (35-53), 'സ പര്യഗാത് ശുക്രമകായം' (40-8) എന്നു തുടങ്ങിയ യജുര്‍വേദ വചനങ്ങളില്‍നിന്ന് പരമേശ്വരന്‍ ജന്മം കൊള്ളുന്നില്ലെന്ന് മനസ്സിലാകുന്നുണ്ട്.\\
പ്രശ്‌നം: യദാ യദാ ഹി ധര്‍മസ്യ ഗ്‌ളാനിര്‍ഭവതി ഭാരത!\\
അഭ്യുത്ഥാനമധര്‍മസ്യ തദാത്മാനം സൃജാമ്യഹം.\\
(ഭഗവത്ഗീത, അ. 4, ശ്ലോകം 7)\\

ധര്‍മത്തിന് ലോപം വരുമ്പോഴെല്ലാം ഞാന്‍ ശരീരം ധരിക്കുന്നു എന്ന് ശ്രീകൃഷ്ണന്‍ പറയുന്നുണ്ടല്ലോ?\\
ഉത്തരം: വേദവിരുദ്ധമായതുകൊണ്ട് ഇത് പ്രമാണമാകുന്നില്ല. ധര്‍മാത്മാവും ധര്‍മത്തെ രക്ഷിക്കാന്‍ ഇഛിക്കുന്നവനുമായ ശ്രീകൃഷ്ണന്‍ 'ഞാന്‍ യുഗംതോറും ജന്മം കൈക്കൊണ്ടു ശിഷ്ടജനങ്ങളെ പരിപാലിക്കുകയും ദുഷ്ടന്മാരെ സംഹരിക്കുകയും ചെയ്യും' എന്നിങ്ങനെ ആഗ്രഹിച്ചിട്ടുണ്ടായിരിക്കാം. അങ്ങനെ ചെയ്യുന്നതില്‍ ദോഷമൊന്നുമില്ല. എന്തെന്നാല്‍ 'പരോപകാരായ സതാം വിഭൂതയഃ' സജ്ജനങ്ങളുടെ ശരീരം, മനസ്സ്, ധനം എന്നിവയെല്ലാം പരോപകാരത്തിനായിട്ടുള്ളതാണല്ലോ. എന്നാലും ശ്രീകൃഷ്ണന്‍ ഇതുകൊണ്ട് ഈശ്വരനാണെന്ന് വരുവാന്‍ തരമില്ല.\\
പ്രശ്‌നം: ഇങ്ങനെയാണെങ്കില്‍, ലോകത്തില്‍ ഈശ്വരന്റെ ഇരുപത്തിനാല് അവതാരങ്ങളുണ്ടെന്നു പറയുന്നതിനെ എങ്ങനെയാണ് കൈക്കൊള്ളുന്നത്?\\
ഉത്തരം: വേദത്തിന്റെ അര്‍ഥം അറിഞ്ഞുകൂടാത്തതുകൊണ്ടും, സാമ്പ്രദായികന്മാരുടെ വഞ്ചനകള്‍ക്ക് വശംവദന്മാരാകുന്നതുകൊണ്ടും അവരവര്‍ക്ക് അറിവില്ലാത്തതുകൊണ്ടും അജ്ഞാതമാകുന്ന വലയില്‍ കുടുങ്ങിപ്പോകുന്നതുകൊണ്ടുമാണ് ഇങ്ങനെ പ്രമാണ ശൂന്യങ്ങളായ വാക്കുകള്‍ പറയുകയും അവയെ വിലവയ്ക്കുകയും ചെയ്യുന്നത്.\\
പ്രശ്‌നം: ഈശ്വരന്‍ അവതാരം ചെയ്യുന്നില്ലെങ്കില്‍ കംസന്‍, രാവണന്‍ മുതലായ ദുഷ്ടന്മാരെ സംഹരിപ്പാന്‍ എങ്ങനെ സാധിക്കും?\\
ഉത്തരം: ഒന്നാമത്, ജനിച്ചിട്ടുള്ളവരെല്ലാം നിശ്ചയമായും മരിക്കാതിരിക്കുകയില്ല. എന്നു മാത്രമല്ല, ശരീരം കൈക്കൊണ്ട് അവതരിക്കാതെ തന്നെ ജഗത്തിന്റെ സൃഷ്ടിസ്ഥിതിസംഹാരങ്ങള്‍ നടത്തുന്ന ഈശ്വരന്റെ മുമ്പില്‍ കംസന്‍, രാവണന്‍ മുതലായവര്‍ കേവലം ഒരു കൃമിയോട് സാദൃശ്യപ്പെടുത്തത്തക്കവര്‍ പോലുമല്ല. അദ്ദേഹം സര്‍വവ്യാപിയായതുകൊണ്ട് കംസരാവണാദികളുടെ ശരീരങ്ങളിലും കുടിനിറഞ്ഞിരിക്കുന്നവനാണ്. ഇഛിക്കുന്ന ക്ഷണത്തില്‍തന്നെ അവരുടെ മാറ് പിളര്‍ന്ന് അവരെ കൊന്നുകളയാന്‍ അദ്ദേഹത്തിനു കഴിയും. എണ്ണമറ്റ ഗുണകര്‍മ സ്വഭാവങ്ങളോടുകൂടിയ പരമേശ്വരന്‍ ഒരു ക്ഷുദ്രജന്തുവിനെ കൊല്ലുവാനായി ജനനമരണങ്ങളുള്ളവനായിത്തീരുന്നുവെന്നു പറയുന്ന മനുഷ്യനെ വിഡ്ഢിയോടല്ലാതെ മറ്റാരോടാണ് ഉപമിക്കാന്‍ സാധിക്കുക?.... ഇതിനു പുറമെ പരമേശ്വരന്‍ ജന്മം സ്വീകരിക്കുന്നുണ്ടെന്നുള്ളത് യുക്തികൊണ്ടും കിട്ടുന്നില്ല. അതിരില്ലാത്തതായ ഈ ആകാശം ഗര്‍ഭത്തില്‍ അടങ്ങിയെന്നോ കരതലത്തില്‍ ഒതുങ്ങിയെന്നോ ആരെങ്കിലും പറയുന്നതായാല്‍ അതു നേരായി വരുവാന്‍ ഒരിക്കലും നിവൃത്തിയില്ല. എന്തെന്നാല്‍ ആകാശം അതിരറ്റതും എങ്ങും നിറഞ്ഞതുമാകുന്നു. അതിനാല്‍ ആകാശം അകത്തേക്ക് പ്രവേശിക്കുകയും പുറത്തേക്ക് നിര്‍ഗമിക്കുകയും ചെയ്യുന്നില്ല. അതുപോലെത്തന്നെ പരമാത്മാവ് അനന്തനും സര്‍വവ്യാപിയുമായതുകൊണ്ട് അദ്ദേഹത്തിന് പ്രവേശ നിര്‍ഗമങ്ങള്‍ ഉണ്ടെന്ന് സിദ്ധിക്കുന്ന കാര്യം ഒരിക്കലും ശക്യമല്ല. ഒരു വസ്തു എവിടെ ഇല്ലയോ അവിടെ മാത്രമേ ഗതാഗതങ്ങള്‍ സംഭവിക്കാന്‍ തരമുള്ളൂ. പരമേശ്വരന്‍ എങ്ങുനിന്നോ വന്നുവെന്നു പറയുവാന്‍, ഗര്‍ഭത്തില്‍ വ്യാപിച്ചിരുന്നില്ലെന്നോ? പരമേശ്വരന്‍ പുറത്ത് ഉണ്ടായിരുന്നില്ലേ? ഉണ്ടായിരുന്നുവെങ്കില്‍ പിന്നെ എങ്ങനെയാണ് നിര്‍ഗമിച്ചു എന്നു പറയുന്നത്? അറിവില്ലാത്തവരല്ലാതെ മറ്റാരാണ് ഈശ്വരനെപ്പറ്റി ഇങ്ങനെ പറയുകയും വിചാരിക്കുകയും ചെയ്യുക? അതുകൊണ്ട് പരമേശ്വരനു ഗതാഗതങ്ങളും ജനനമരണങ്ങളും ഉണ്ടെന്ന് സിദ്ധിക്കുവാന്‍ ഒരിക്കലും നിവൃത്തിയില്ല. അതിനാല്‍ യേശു മുതലായവരും ഈശ്വരന്റെ അവതാരമല്ലെന്നു മനസ്സിലാക്കിക്കൊള്ളുക. എന്തെന്നാല്‍, രാഗദ്വേഷങ്ങള്‍, ക്ഷുത്പിപാസകള്‍, ഭയശോകങ്ങള്‍, സുഖദുഃഖങ്ങള്‍, ജനനമരണങ്ങള്‍ മുതലായ ഗുണങ്ങളോടു കൂടിയവരായിരുന്നതുകൊണ്ട് അവര്‍ മനുഷ്യര്‍ തന്നെയായിരുന്നു''(സത്യാര്‍ഥ പ്രകാശം, പുറം 304-306).

അതിനാല്‍ ദൈവത്തിന് അവതാരമില്ലെന്നും ഉണ്ടാവുകയില്ലെന്നും ഇസ്‌ലാമിനെപ്പോലെ ഹിന്ദുദര്‍ശനവും അംഗീകരിക്കുന്നു. പ്രപഞ്ചത്തിലുള്ളതെല്ലാം ദൈവത്തിന്റെ സൃഷ്ടികളാണ്. യേശുവും മോശെയും മുഹമ്മദും ബ്രഹ്മാവും വിഷ്ണുവും ശിവനും രാമനും കൃഷ്ണനുമൊന്നും ഇതില്‍നിന്നൊഴിവല്ല. അവരൊന്നും ഈശ്വരന്മാരോ ഈശ്വരാവതാരങ്ങളോ അല്ല. മറിച്ച് ഈശ്വരസൃഷ്ടങ്ങളാണ്. അതിനാല്‍, അവരെയൊന്നും വിളിച്ചു പ്രാര്‍ഥിക്കുകയോ ആരാധിക്കുകയോ അരുത്. അവര്‍ക്ക് നിവേദ്യങ്ങളോ നേര്‍ച്ചവഴിപാടുകളോ അര്‍പ്പിക്കരുത്. അവരുടെയൊക്കെ സ്രഷ്ടാവായ ദൈവത്തെ മാത്രമേ ആരാധിക്കാവൂ. മറിച്ചു സംഭവിക്കുന്നതെല്ലാം അജ്ഞതകൊണ്ടോ അന്ധവിശ്വാസം നിമിത്തമോ ആണ്. അങ്ങനെ ചെയ്യുന്നത് അക്ഷന്തവ്യമായ അപരാധമത്രെ.
\chapter{ഇസ്‌ലാമും അഭൗതിക ജ്ഞാനവും }
  \section{''ലോകത്ത് ആര്‍ക്കും തന്റെ ഭാവിയെ സംബന്ധിച്ച് ഒന്നും അറിയുകയില്ലെന്ന് താങ്കള്‍ ആമുഖ ഭാഷണത്തില്‍ പറഞ്ഞു. എന്നാല്‍ ജനനസമയത്തെ നക്ഷത്രഘടനയുടെ അടിസ്ഥാനത്തില്‍ ജാതകഫലം പ്രവചിക്കാറുണ്ടല്ലോ. പ്രമുഖ ജ്യോതിഷിമാരുടെ പ്രവചനങ്ങള്‍ പലപ്പോഴും യാഥാര്‍ഥ്യമാവാറുണ്ട്. ഇതിനെക്കുറിച്ച് എന്തു പറയുന്നു?''}
 നമ്മുടെ ജനനത്തീയതിയും രാവും രാശിയും നക്ഷത്രവുമൊക്കെ അറിയുകയാണെങ്കില്‍ ഭാവിയിലെന്തു സംഭവിക്കുമെന്ന് മനസ്സിലാക്കാമെന്ന ധാരണ സമൂഹത്തില്‍ വളരെ വ്യാപകമായി നിലനില്‍ക്കുന്നുണ്ട്. നമ്മുടെ നാട്ടില്‍ ദിനംപ്രതി പതിനായിരക്കണക്കിനും ലക്ഷക്കണക്കിനും പ്രതികള്‍ വിറ്റഴിക്കപ്പെടുന്ന പത്രങ്ങള്‍ പലതും 'ഈ ആഴ്ച നമുക്കെന്ത് സംഭവിക്കുമെന്ന്' പതിവായി നമ്മോട് പറഞ്ഞുതരാറുണ്ടല്ലോ. ഓരോ വാരാന്ത്യത്തിലും വരുംവാരത്തില്‍ വന്നുഭവിച്ചേക്കാവുന്ന സംഭവങ്ങളെ സംബന്ധിച്ച വിവരം ലഭിക്കുകയാണെങ്കില്‍ അത് അതിമഹത്തായ നേട്ടം തന്നെ. അനാവശ്യമായ ആശങ്കകളൊഴിവാക്കാനും പ്രാപിക്കാനാവാത്ത പ്രതീക്ഷകള്‍ വച്ചുപുലര്‍ത്താതിരിക്കാനും കരുതലോടെ കാര്യങ്ങള്‍ ചിട്ടപ്പെടുത്താനും കഴിയുമല്ലോ. എന്നാല്‍ നക്ഷത്രഫലത്തിന്റെ അടിസ്ഥാനത്തില്‍ ഭാവി അറിയാനാര്‍ക്കും സാധ്യമല്ലെന്നതാണ് സത്യം. നമ്മുടെ രാജ്യത്തെ മുന്‍പ്രധാനമന്ത്രിയായ ഇന്ദിരാഗാന്ധിക്ക് ജനനത്തീയതിയും രാവും രാശിയും നക്ഷത്രവുമൊക്കെ അറിയാമായിരുന്നു. ജ്യോത്സ്യന്മാരെ സ്ഥിരമായി കാണാറുമുണ്ടായിരുന്നു. എന്നിട്ടും സ്വന്തം അംഗരക്ഷകന്റെ വെടിയേറ്റ് മരിക്കുമെന്ന് അവരെ ആരും അറിയിച്ചില്ലല്ലോ. രാജീവ്ഗാന്ധിയുടെ അവസ്ഥയും വ്യത്യസ്തമായിരുന്നില്ല. തമിഴ്‌നാട്ടില്‍വച്ച് ബോംബ് സ്‌ഫോടനത്തില്‍ മരണമടയുമെന്ന് പ്രവചിക്കാന്‍ ഒരാള്‍ക്കും സാധിച്ചില്ല. ആരും അങ്ങനെയൊരു കാര്യം അദ്ദേഹത്തോടു പറഞ്ഞതുമില്ല. ലാത്തൂരിലെ പതിനായിരക്കണക്കിന് ആളുകള്‍ക്ക് ജനനത്തീയതിയും മറ്റു വിശദാംശങ്ങളുമറിയാമായിരുന്നു. ജാതകം തയ്യാറാക്കുന്ന ജ്യോതിഷിമാര്‍ പോലും അവരിലുണ്ടായിരുന്നു. എന്നിട്ടും ഭൂകമ്പത്തെപ്പറ്റി ആര്‍ക്കും മുന്നറിവുണ്ടായില്ല എന്നതാണല്ലോ വസ്തുത. ജാതകം നോക്കി ഭാവി മനസ്സിലാക്കാമായിരുന്നെങ്കില്‍ ഇങ്ങനെയൊന്നും സംഭവിക്കുമായിരുന്നില്ല.
പി.ജെ.എസ്. ഗിയാനി, പി.കെ. ചക്രവര്‍ത്തി, ജഗജിത് ഉപ്പല്‍, രാമന്‍തക്കാര്‍, മാലതി സിര്‍സിക്കാര്‍ തുടങ്ങിയ പ്രശസ്ത ജ്യോത്സ്യന്മാരുടെ പ്രമാദമായ പല പ്രവചനങ്ങളും പുലര്‍ന്നിട്ടില്ലെന്നതിന് അനുഭവം സാക്ഷിയാണ്. ഒമ്പതാം ലോക്‌സഭാ തെരഞ്ഞെടുപ്പില്‍ രാജീവ്ഗാന്ധി അധികാരത്തില്‍ വരുമെന്ന് ഉപ്പല്‍ പ്രവചിച്ചിരുന്നുവെങ്കിലും അത് പുലര്‍ന്നില്ല. ഗള്‍ഫ് യുദ്ധത്തിന്റെ ഒരു ദിവസം മുമ്പ് അടുത്തൊന്നും അതുണ്ടാവില്ലെന്ന് പ്രവചിച്ച പ്രശസ്ത ജ്യോത്സ്യനാണ് ബജന്‍ഭറുവാല. പക്ഷേ, പാളിപ്പോയ ഇത്തരം പ്രവചനങ്ങളെപ്പറ്റി പിന്നീട് പലരും വിസ്മരിക്കാറാണ് പതിവ്. അതേസമയം, വല്ലപ്പോഴും ഒത്തുവരികയോ വ്യാഖ്യാനിച്ച് ഒപ്പിക്കുകയോ ചെയ്യുന്ന പ്രവചനങ്ങളെ സംബന്ധിച്ച് വ്യാപകമായ പ്രചാരണം നടത്തുകയും ചെയ്യുന്നു. ഭോപ്പാലിലെ വാതകദുരന്തത്തെക്കുറിച്ചോ ബംഗ്‌ളാദേശിലെ ലക്ഷങ്ങളുടെ മരണത്തിനിടയാക്കിയ കൊടുങ്കാറ്റിനെ സംബന്ധിച്ചോ ദുരന്തം വാരിവിതറിയ ഇറാനിലെ ഭൂകമ്പത്തെപ്പറ്റിയോ മുന്നറിവ് നല്‍കാന്‍ ഒരു ജ്യോത്സ്യന്നും നക്ഷത്രഫലക്കാരന്നും കണക്കുനോട്ടക്കാരന്നും പ്രവചനക്കാരന്നും സാധിച്ചിട്ടില്ലെന്നതാണ് സത്യം.
അജ്ഞത, അന്ധവിശ്വാസം, ഭാവി അറിയാനുള്ള അദമ്യമായ അഭിവാഞ്ഛ ഇതൊക്കെ ഒത്തുചേരുമ്പോള്‍ ആളുകള്‍ ഭാവി പറഞ്ഞുതരുന്നവരെ പ്രതീക്ഷാപൂര്‍വം സമീപിക്കുന്നു. വിദ്യാസമ്പന്നരും ബുദ്ധിജീവികളും ശാസ്ത്രജ്ഞരും സാംസ്‌കാരിക നായകന്മാരും ഭരണാധികാരികളുമൊന്നും ഇതില്‍നിന്നൊഴിവല്ല. അതിനാലവര്‍ ചിലരില്‍ അഭൗതികമായ അറിവുകള്‍ ആരോപിക്കുന്നു. ദിവ്യമായ കഴിവുകള്‍ കണ്ടെത്തുന്നു. ഭാവി പ്രവചിക്കാന്‍ പ്രാപ്തരെന്ന് അവരെ പരിചയപ്പെടുത്തുന്നു. എന്നാല്‍ പ്രപഞ്ചസ്രഷ്ടാവായ ദൈവത്തിനല്ലാതെ ആര്‍ക്കും അഭൗതിക കാര്യങ്ങളറിയുകയില്ല. ഭാവിയില്‍ എന്തുനടക്കുമെന്ന് കണ്ടെത്താനാവില്ല. ഇസ്‌ലാം ഇക്കാര്യം അസന്ദിഗ്ധമായി വ്യക്തമാക്കിയിട്ടുണ്ട്. അകലത്തുള്ള അത്തിവൃക്ഷത്തില്‍ പഴമില്ലെന്ന് മനസ്സിലാക്കാന്‍ സാധിക്കാതിരുന്നതിനാല്‍ യേശു ശിഷ്യരെ അതിന്റെ അടുത്തേക്ക് കൂട്ടിക്കൊണ്ടുപോയി. തന്റെ പ്രിയപത്‌നി ആഇശക്കെതിരെ അപവാദാരോപണമുണ്ടായപ്പോള്‍ അത് അപ്പാടെ കള്ളമാണെന്ന് കണ്ടെത്താന്‍ മുഹമ്മദ് നബിതിരുമേനിക്ക് കഴിഞ്ഞില്ല. പിന്നീട്, ദിവ്യവെളിപാടുകളുണ്ടായപ്പോഴാണ് വസ്തുത വ്യക്തമായത്. ശ്രീരാമന്‍, കാട്ടില്‍ വച്ച് ജനിച്ചുവളര്‍ന്ന സ്വന്തം സന്താനങ്ങളായ ലവകുശന്മാര്‍ തന്റെ മുമ്പില്‍ വന്ന് രാമചരിതം ചൊല്ലിയപ്പോള്‍ അവരെ തിരിച്ചറിഞ്ഞില്ല. പ്രിയപത്‌നി ലങ്കയിലായിരിക്കെ അദ്ദേഹത്തിന് അമാനുഷിക മാര്‍ഗത്തിലൂടെ അവരെ കണ്ടെത്താന്‍ കഴിഞ്ഞില്ല. പ്രവാചകന്മാര്‍ക്കും പുണ്യപുരുഷന്മാര്‍ക്കുമൊന്നും തന്നെ അഭൗതിക കാര്യങ്ങളറിയില്ലെന്നും ഭാവിയിലെന്തു സംഭവിക്കുമെന്ന് മുന്‍കൂട്ടി കണ്ടെത്താനാവില്ലെന്നും ഈ വസ്തുതകള്‍ സംശയാതീതമായി വ്യക്തമാക്കുന്നു.
കാര്യകാരണ ബന്ധങ്ങള്‍ക്കതീതമായി അഭൗതിക കാര്യങ്ങള്‍ ദൈവത്തിനല്ലാതെ മറ്റാര്‍ക്കും അറിയില്ലെന്നത് ഇസ്‌ലാമിലെ ഈശ്വരവിശ്വാസത്തിന്റെ അടിസ്ഥാനമത്രെ. ഖുര്‍ആന്‍ ഇക്കാര്യം ഊന്നിപ്പറഞ്ഞിട്ടുണ്ട്:
''പറയുക: അല്ലാഹു അല്ലാതെ ആകാശഭൂമികളിലാരുംതന്നെ അഭൗതിക കാര്യങ്ങളറിയുന്നില്ല.''(27: 65)
''നബിയേ, പറയുക: എന്റെ വശം ദൈവത്തിന്റെ ഖജനാവുകളുണ്ടെന്ന് ഞാന്‍ നിങ്ങളോട് പറയുന്നില്ല. ഞാന്‍ അഭൗതിക കാര്യങ്ങള്‍ അറിയുന്നുമില്ല.''(6: 50)
നമ്മുടെ കാലത്തും ലോകത്തുമെന്നപോലെ അഭൗതിക മാര്‍ഗേണ പുണ്യവാളന്മാര്‍ക്ക് ദിവ്യജ്ഞാനമുണ്ടെന്ന മൂഢധാരണ ചരിത്രത്തില്‍ പലപ്പോഴും നിലനിന്നിരുന്നു. അതുകൊണ്ടുതന്നെ എല്ലാ ദൈവദൂതന്മാരും ആ ധാരണ തിരുത്താനും അത്തരം അന്ധവിശ്വാസങ്ങള്‍ക്ക് അറുതിവരുത്താനും ശ്രമിച്ചിരുന്നു. അതിനാല്‍ ദൈവദൂതനായ താനുള്‍പ്പെടെ ആര്‍ക്കും അഭൗതികജ്ഞാനമില്ലെന്ന് അവരെല്ലാം ഊന്നിപ്പറഞ്ഞിരുന്നു. ദൈവദൂതനായ നൂഹ് തന്റെ ജനതയോട് പറഞ്ഞു: ''അല്ലാഹുവിന്റെ ഖജനാവുകള്‍ എന്റെ കൈവശമുണ്ടെന്ന് ഞാന്‍ നിങ്ങളോടവകാശപ്പെടുന്നില്ല. എനിക്ക് അഭൗതികജ്ഞാനമുണ്ടെന്നും ഞാന്‍ പറയുന്നില്ല.''(11: 31)
അതിനാല്‍ കാര്യകാരണ ബന്ധങ്ങള്‍ക്കതീതമായ അഭൗതികജ്ഞാനം ദൈവദൂതന്മാര്‍ക്കോ പുണ്യപുരുഷന്മാര്‍ക്കോ ഭഗവാന്മാരെന്നവകാശപ്പെടുന്നവര്‍ക്കോ ജ്യോത്സ്യന്മാര്‍ക്കോ കണക്കുനോട്ടക്കാര്‍ക്കോ മറ്റാര്‍ക്കെങ്കിലുമോ ഇല്ല. ഉണ്ടെന്ന് അവകാശപ്പെടുന്നത് തീര്‍ത്തും വ്യാജമാണ്. നക്ഷത്രഫലവും ജ്യോത്സ്യപ്രവചനവുമെല്ലാം തെറ്റും തികഞ്ഞ അന്ധവിശ്വാസവുമത്രെ.
നക്ഷത്രഫലം വ്യാജവും കൊടിയ തട്ടിപ്പുമാണെന്ന് സ്വാമി ദയാനന്ദസരസ്വതി അസന്ദിഗ്ധമായി വ്യക്തമാക്കിയിട്ടുണ്ട്. ചോദ്യോത്തര രൂപത്തില്‍ ഇക്കാര്യം ഇങ്ങനെ അവതരിപ്പിക്കുന്നു:\\
പ്രശ്‌നം: രാജാക്കന്മാരും പ്രജകളും അടങ്ങിയ ഈ ലോകത്തില്‍ ചിലര്‍ സുഖമുള്ളവരായും ചിലര്‍ ദുഃഖമനുഭവിക്കുന്നതായും കാണപ്പെടുന്നു. അതെല്ലാം ഗ്രഹസ്ഥിതിയുടെ ഫലമല്ലേ?\\
ഉത്തരം: അതെല്ലാം പുണ്യപാപങ്ങളുടെ ഫലമാണ്.\\
പ്രശ്‌നം: ജ്യോതിശ്ശാസ്ത്രം കേവലം കളവാണോ?\\
ഉത്തരം: അല്ല. ആ ശാസ്ത്രത്തിലന്തര്‍ഭവിച്ച അങ്കഗണിതം, രേഖാഗണിതം, ബീജഗണിതം മുതലായ ഗണിതഭാഗങ്ങളെല്ലാം ശരിയായിട്ടുള്ളതാണ്. ഫലഭാഗം മുഴുവന്‍ ശുദ്ധവ്യാജമാണ്.\\
പ്രശ്‌നം: ജന്മപത്രം (ജാതകം) തീരെ നിഷ്ഫലമായിട്ടുള്ളതാണോ?\\
ഉത്തരം: അതെ, അതിനു ജന്മപത്രമെന്നല്ല, ശോകപത്രം എന്നാണ് പേര്‍ പറയേണ്ടത്. എന്തെന്നാല്‍ സന്താനം ഉണ്ടാകുമ്പോള്‍ സകല ജനങ്ങള്‍ക്കും ആനന്ദം ഉണ്ടാകാറുണ്ട്. എന്നാല്‍ ആ ആനന്ദം സന്താനത്തിന്റെ ജാതകം എഴുതി ഗ്രഹങ്ങളുടെ ഫലം പറഞ്ഞു കേള്‍ക്കുന്നതുവരെ മാത്രമേ നിലനില്‍ക്കുന്നുള്ളൂ. ജാതകം എഴുതിക്കേണ്ടതാണെന്ന് ജ്യോത്സ്യന്‍ പറയുമ്പോള്‍ കുട്ടിയുടെ അച്ഛനമ്മമാര്‍ ജ്യോത്സ്യനോട് ''വളരെ വിശേഷപ്പെട്ട ജാതകമായിരിക്കണം എഴുതുന്നത'' എന്നു പറയുന്നു. പറഞ്ഞേല്‍പിച്ചത് ധനികനാണെങ്കില്‍ ചുകന്ന നിറത്തിലും മഞ്ഞനിറത്തിലുമുള്ള വരകള്‍ കൊണ്ടും ചിത്രങ്ങള്‍കൊണ്ടും മോടിപിടിപ്പിച്ചും ദരിദ്രനാണെങ്കില്‍ സാധാരണ സമ്പ്രദായത്തിലും ഒരു ജാതകം എഴുതിയുണ്ടാക്കി ജ്യോത്സ്യന്‍ അത് വായിച്ചു കേള്‍പ്പിക്കാന്‍ വരുന്നു. അപ്പോള്‍ കുട്ടിയുടെ മാതാപിതാക്കന്മാര്‍ ജ്യോത്സ്യനു മുമ്പിലിരുന്നു ചോദിക്കുന്നു: ''ഇവന്റെ ജാതകം നല്ലതുതന്നെയല്ലേ?'' ജ്യോത്സ്യന്‍ പറയുന്നു: ''ഉള്ളതു മുഴുവന്‍ പറഞ്ഞു കേള്‍പ്പിച്ചേക്കാം. ഇവന്റെ ജന്മഗ്രഹങ്ങളും മിത്രഗ്രഹങ്ങളും ശുഭങ്ങളാകുന്നു. അതിന്റെ ഫലമായി ഇവന്‍ വലിയ ധനികനും കീര്‍ത്തിമാനും ആയിത്തീരും. ഏതു സഭയില്‍ കടന്നിരുന്നാലും ഇവന്റെ തേജസ്സ് മറ്റുള്ളവരുടേതിനേക്കാള്‍ മികച്ചുനില്‍ക്കും. ശരീരത്തിനു നല്ല ആരോഗ്യമുള്ളവനും രാജാക്കന്മാരാല്‍ കൂടി ബഹുമാനിക്കപ്പെടുന്നവനും ആയിത്തീരും.'' ഈ മാതിരി വാക്കുകള്‍ കേട്ട് കുട്ടിയുടെ അച്ഛനമ്മമാര്‍ പറയും: ''നല്ലത്. നല്ലതു ജ്യോത്സ്യരേ; അങ്ങ് വളരെ നല്ല ഒരാളാണ്.'' എന്നാല്‍ ഇതുകൊണ്ടൊന്നും തന്റെ കാര്യം സാധിക്കുകയില്ലെന്ന് ജ്യോത്സ്യന്നറിയാം. അതുകൊണ്ട് ജ്യോത്സ്യന്‍ പിന്നെയും പറയുന്നു: ''ഈ പറഞ്ഞ ഗ്രഹങ്ങളെല്ലാം വളരെ ശോഭനങ്ങള്‍ തന്നെയാണ്. എന്നാല്‍ ഈ ശുഭഗ്രഹങ്ങള്‍ വേറെ ചില ക്രൂരഗ്രഹങ്ങളോടു കൂടിച്ചേര്‍ന്നാണിരിക്കുന്നത്. അതുനിമിത്തം ഈ കുട്ടിക്ക് എട്ടാമത്തെ വയസ്സില്‍ മൃത്യുയോഗം ഉണ്ട്.'' ഇതു കേള്‍ക്കുമ്പോഴേക്കും മാതാപിതാക്കള്‍ പുത്രജനനംകൊണ്ടുണ്ടായ ആനന്ദമെല്ലാം പരിത്യജിച്ച് സന്താപസമുദ്രത്തില്‍ മുങ്ങി ജ്യോത്സ്യനോട് പറയുന്നു: ''ജ്യോത്സ്യരേ, ഇനി ഞങ്ങള്‍ എന്താണ് ചെയ്യേണ്ടത്?'' ജ്യോത്സ്യന്‍ പറയുന്നു: ''അതിനു വല്ല പരിഹാരവും ചെയ്യണം.'' പരിഹാരം എന്താണെന്ന് ഗൃഹസ്ഥന്‍ ചോദിക്കുമ്പോള്‍ ജ്യോത്സ്യന്‍ വീണ്ടും പറഞ്ഞു തുടങ്ങുന്നു: ''ദാനങ്ങള്‍ കൊടുക്കണം. ഗ്രഹശാന്തിക്കുള്ള മന്ത്രങ്ങള്‍ ജപിക്കണം. ദിവസംതോറും ബ്രാഹ്മണരെ കാല്‍കഴുകിച്ച് ഊട്ടണം. ഇങ്ങനെയെല്ലാം ചെയ്യുന്നതായാല്‍ ഗ്രഹപ്പിഴകളെല്ലാം തീരുമെന്നാണ് അനുമാനിക്കേണ്ടത്.'' ജ്യോത്സ്യന്‍ ഇവിടെ അനുമാനശബ്ദം പ്രയോഗിക്കുന്നത് വളരെ മുന്‍കരുതലോടുകൂടിയാണ്. ഒരുപക്ഷേ, കുട്ടി മരിച്ചുപോകുന്നതായാല്‍ അയാള്‍ പറയും: ''ഞങ്ങള്‍ എന്തു ചെയ്യും? പരമേശ്വരനു മേലെ ആരും ഇല്ല. ഞങ്ങള്‍ വളരെ പ്രയത്‌നം ചെയ്തുനോക്കി. നിങ്ങള്‍ പലതും ചെയ്യിച്ചു. പക്ഷേ, അവന്റെ കര്‍മഫലം അങ്ങനെയാണ്.'' കുട്ടി ജീവിക്കുകയാണ് ചെയ്തതെങ്കില്‍ അപ്പോഴും അയാള്‍ക്ക് ഇങ്ങനെ പറയാം: ''നോക്കുക, ഞങ്ങളുടെ മന്ത്രങ്ങളുടെയും ഞങ്ങള്‍ ഉപാസിക്കുന്ന ദേവതമാരുടെയും ബ്രാഹ്മണരുടെയും ശക്തി എത്ര വലുതാണ്. നിങ്ങളുടെ കുട്ടിയെ രക്ഷിച്ചുതന്നില്ലേ?'' വാസ്തവത്തില്‍ ജപം കൊണ്ടു ഫലമൊന്നുമുണ്ടായില്ലെങ്കില്‍ ആ ധൂര്‍ത്തന്മാരുടെ കൈയില്‍നിന്ന് അവര്‍ക്ക് കൊടുത്തതിന്റെ രണ്ടോ മൂന്നോ ഇരട്ടി പണം മടക്കിവാങ്ങേണ്ടതാണ്. കുട്ടി ജീവിച്ചിരിക്കുന്നതായാലും അങ്ങനെ മടക്കിമേടിക്കുക തന്നെയാണ് വേണ്ടത്. എന്തെന്നാല്‍, ''അത് അവന്റെ കര്‍മഫലമാണ്. ഈശ്വരന്റെ നിയമത്തെ ലംഘിക്കുവാന്‍ ആര്‍ക്കും ശക്തിയില്ല'' എന്നു ജ്യോതിഷി പറയുന്നപോലെ അവന്റെ കര്‍മഫലവും ഈശ്വരന്റെ നിയമവുമാണ് അവനെ രക്ഷിച്ചത്; നിങ്ങളുടെ പ്രവൃത്തിയല്ല എന്ന് ഗൃഹസ്ഥന് അയാളോടും പറയാവുന്നതാണല്ലോ. ദാനങ്ങളും മറ്റു പുണ്യകര്‍മങ്ങളും അനുഷ്ഠിപ്പിച്ചു പ്രതിഗ്രഹം വാങ്ങിയിട്ടുള്ള ഗുരു മുതലായവരോടും ജ്യോത്സ്യനോട് പറഞ്ഞതുപോലെ തന്നെ പറയേണ്ടതാണ്.(സത്യാര്‍ഥപ്രകാശം, ആര്യപ്രാദേശിക പ്രതിനിധി സഭ, പഞ്ചാബ്. പുറം: 4547)
ജ്യോതിഷത്തിന്റെ നിരര്‍ഥകത വ്യക്തമാക്കാനായി സ്വാമി വിവേകാനന്ദന്‍ ഒരു കഥ ഉദ്ധരിക്കുന്നു: 'ഒരു ജ്യോതിഷി ഒരു രാജാവിന്റെ അടുക്കല്‍ ചെന്ന് അദ്ദേഹം ആറു മാസത്തിനുള്ളില്‍ മരിക്കുമെന്ന് പ്രവചിച്ചു. അതുകേട്ടു ഭയന്ന രാജാവ് അപ്പോള്‍തന്നെ മരിക്കുമെന്ന നിലയിലായി. അപ്പോള്‍ മന്ത്രി അവിടെയെത്തി ''ജ്യോതിഷികള്‍ പൊതുവെ വിഡ്ഢികളാണെന്നും അവര്‍ പറയുന്നത് വിശ്വസിക്കേണ്ടതില്ലെ''ന്നും പറഞ്ഞ് രാജാവിനെ ആശ്വസിപ്പിക്കാന്‍ നോക്കിയെങ്കിലും ഫലിച്ചില്ല. അപ്പോള്‍ മന്ത്രി ജ്യോതിഷിയോട് ''രാജാവിന്റെ മരണം പ്രവചിച്ചത് ശരി തന്നെയോ'' എന്ന് ഒന്നുകൂടി ചോദിച്ചു. വീണ്ടും ഗണിച്ചതിന് ശേഷം ജ്യോതിഷി തന്റെ പ്രവചനത്തിലുറച്ചുനിന്നു. ഉടനെ മന്ത്രി ജ്യോതിഷിയോട് ''നിങ്ങള്‍ എപ്പോഴാണ് മരിക്കുക''യെന്ന് ചോദിച്ചു. ''പന്ത്രണ്ടുകൊല്ലം കഴിഞ്ഞാല്‍'' എന്ന് മറുപടി. അതുകേട്ടയുടനെ മന്ത്രി വാളെടുത്തു വീശി ആ ജ്യോതിഷിയെ വെട്ടിക്കൊന്നു. എന്നിട്ട് രാജാവിനോടു പറഞ്ഞു: ''അങ്ങേക്കിപ്പോള്‍ ബോധ്യമായല്ലോ, അവന്‍ കള്ളനാണെന്ന്. ഈ നിമിഷം തന്നെ അവന്‍ ചത്തുവല്ലോ.''(വിവേകാനന്ദ സാഹിത്യസര്‍വസ്വം, വാ. 4, പുറം 87)
സ്വാമി വിവേകാനന്ദന്‍ തന്നെ എഴുതുന്നു: ''വിധിയെപ്പറ്റി പുലമ്പിക്കൊണ്ടിരിക്കുന്നത് പ്രായം കൂടിവരുന്നവരാണ്. യുവജനങ്ങള്‍ പ്രായേണ ജ്യോതിഷത്തെ ആശ്രയിക്കാറില്ല. ഗ്രഹങ്ങള്‍ നമ്മുടെ മേല്‍ പ്രാഭവം പ്രയോഗിക്കുന്നുണ്ടാവാം. എന്നാല്‍ നാം അതിനത്ര പ്രാധാന്യം കല്‍പിക്കാന്‍ പാടില്ല.... ജ്യോതിര്‍ഗണങ്ങള്‍ വന്നുകൊള്ളട്ടെ; അതുകൊണ്ടെന്തു ദോഷം? ഒരു നക്ഷത്രത്തിന് താറുമാറാക്കാവുന്നതാണ് എന്റെ ജീവിതമെങ്കില്‍ അതൊരു കാശിനു വിലപിടിപ്പുള്ളതല്ല. ജ്യോതിഷവും അതുപോലുള്ള ഗൂഢവിദ്യകളും പ്രായേണ ദുര്‍ബലമനസ്സിന്റെ ചിഹ്നങ്ങളാണെന്ന് നിങ്ങള്‍ക്കറിയാനാകും. അതിനാലവ നമ്മുടെ മനസ്സില്‍ സ്ഥാനം പിടിക്കാന്‍ തുടങ്ങിയാല്‍ ഉടന്‍ നാം ഒരു വൈദ്യനെ കാണുകയും നല്ല ആഹാരം കഴിക്കുകയും വിശ്രമിക്കുകയും ചെയ്യേണ്ടതാണ്''(അതേ പുസ്തകം, പുറം 86).
ഏകദൈവത്തിനല്ലാതെ മറ്റാര്‍ക്കും അഭൗതികകാര്യങ്ങളറിയുകയില്ലെന്നത് കലര്‍പ്പില്ലാത്ത ദൈവവിശ്വാസത്തിന്റെ അവിഭാജ്യഘടകമത്രെ. ഇസ്‌ലാം അക്കാര്യം ഊന്നിപ്പറയാനുള്ള കാരണവും അതുതന്നെ.
\chapter{ദൈവത്തെ സൃഷ്ടിച്ചതാര്? }
 \section{ 'പ്രപഞ്ചത്തിന് ഒരു സ്രഷ്ടാവ് വേണമെന്നും ദൈവമാണ് അതിനെ സൃഷ്ടിച്ചതെന്നും നിങ്ങള്‍ മതവിശ്വാസികള്‍ പറയുന്നു. എന്നാല്‍ നിങ്ങളുടെ ദൈവത്തെ സൃഷ്ടിച്ചതാരാണ്? മതവും ദൈവവും വിശ്വാസകാര്യമാണെന്നും അതില്‍ യുക്തിക്ക് പ്രസക്തിയില്ലെന്നുമുള്ള പതിവു മറുപടിയല്ലാതെ വല്ലതും പറയാനുണ്ടോ?''}
 പ്രപഞ്ചത്തെപ്പറ്റി പ്രധാനമായും രണ്ടു വീക്ഷണമാണ് നിലനില്‍ക്കുന്നത്. ഒന്ന് മതവിശ്വാസികളുടേത്. അതനുസരിച്ച് പ്രപഞ്ചം സൃഷ്ടിയാണ്. ദൈവമാണതിന്റെ സ്രഷ്ടാവ്. രണ്ടാമത്തേത് പദാര്‍ഥ വാദികളുടെ വീക്ഷണമാണ്. പ്രപഞ്ചം അനാദിയാണെന്ന് അവരവകാശപ്പെടുന്നു. അഥവാ അതുണ്ടായതല്ല, ആദിയിലേ ഉള്ളതാണ്. അതിനാലതിന് ആദ്യവും അന്ത്യവുമില്ല.
ആയിരത്തി അഞ്ഞൂറോ രണ്ടായിരമോ കോടി കൊല്ലം മുമ്പ് പ്രപഞ്ചം അതീവസാന്ദ്രതയുള്ള ഒരു കൊച്ചു പദാര്‍ഥമായിരുന്നുവെന്ന് ഭൗതികവാദികള്‍ അവകാശപ്പെടുന്നു. അത് സാന്ദ്രതയുടെയും താപത്തിന്റെയും പാരമ്യതയിലെത്തിയപ്പോള്‍ പൊട്ടിത്തെറിച്ചു. ആ സ്‌ഫോടനം അനന്തമായ അസംഖ്യം പൊട്ടിത്തെറികളുടെ ശൃംഖലയായി. സംഖ്യകള്‍ക്ക് അതീതമായ അവസ്ഥയില്‍ ആദിപദാര്‍ഥത്തിന്റെ താപവും സാന്ദ്രതയും എത്തിയപ്പോഴാണ് അത് പൊട്ടിത്തെറിച്ചത്. ആ പൊട്ടിത്തെറിയുടെ ശബ്ദവും അളക്കാനാവാത്തതത്രെ. വന്‍ വിസ്‌ഫോടനത്തിന്റെ അതേ നിമിഷത്തില്‍ മുവ്വായിരം കോടി ഡിഗ്രി താപമുള്ള പ്രപഞ്ചം ഉണ്ടായി. പൊട്ടിത്തെറിയുടെ ഫലമായി വികാസമുണ്ടായി. വികാസം കാരണമായി താപനില കുറയുവാന്‍ തുടങ്ങി. സ്‌ഫോടനം കഴിഞ്ഞ് നാല് നിമിഷം പിന്നിട്ടപ്പോള്‍ ന്യൂട്രോണുകളും പ്രോട്ടോണുകളും കൂടിച്ചേര്‍ന്നു. ആ സംഗമമാണ് നക്ഷത്രങ്ങള്‍ തൊട്ട് മനുഷ്യന്‍ വരെയുള്ള എല്ലാറ്റിന്റെയും ജന്മത്തിന് നാന്ദി കുറിച്ചത്.
പ്രപഞ്ചോല്‍പത്തിയെ സംബന്ധിച്ച ഭൗതികവാദികളുടെ ഈ സങ്കല്‍പം ഉത്തരംകിട്ടാത്ത നിരവധി ചോദ്യങ്ങളുയര്‍ത്തുന്നു. ആദിപദാര്‍ഥം എന്നാണ് താപത്തിലും സാന്ദ്രതയിലും പാരമ്യതയിലെത്തിയത്? എന്തുകൊണ്ട് അതിനു മുമ്പായില്ല, അല്ലെങ്കില്‍ ശേഷമായില്ല? അനാദിയില്‍ തന്നെ താപത്തിലും സാന്ദ്രതയിലും പാരമ്യതയിലായിരുന്നെങ്കില്‍ എന്തുകൊണ്ട് അനാദിയില്‍ തന്നെ സ്‌ഫോടനം സംഭവിച്ചില്ല? അനാദിയില്‍ താപവും സാന്ദ്രതയും പാരമ്യതയിലായിരുന്നില്ലെങ്കില്‍ പിന്നെ എങ്ങനെ അവ പാരമ്യതയിലെത്തി? പുറത്തുനിന്നുള്ള ഇടപെടലുകള്‍ കാരണമാണോ? എങ്കില്‍ എന്താണ് ആ ഇടപെടല്‍? അല്ലെങ്കില്‍ അനാദിയിലില്ലാത്ത സാന്ദ്രതയും താപവും ആദിപദാര്‍ഥത്തില്‍ പിന്നെ എങ്ങനെയുണ്ടായി? എന്നാണ് ആദ്യത്തെ പൊട്ടിത്തെറിയുണ്ടായത്? എന്തുകൊണ്ട് അതിനു മുമ്പായില്ല? എന്തുകൊണ്ട് ശേഷമായില്ല? ഒന്നായിരുന്ന സൂര്യനും ഭൂമിയും രണ്ടായത് എന്ന്? എന്തുകൊണ്ട് അതിനു മുമ്പോ ശേഷമോ ആയില്ല? ഇത്തരം നിരവധി ചോദ്യങ്ങള്‍ക്ക് ഇന്നോളം ഭൗതികവാദികള്‍ മറുപടി നല്‍കിയിട്ടില്ല. നല്‍കാനൊട്ടു സാധ്യവുമല്ല.
ഈ പ്രപഞ്ചം വളരെ വ്യവസ്ഥാപിതവും ക്രമാനുസൃതവുമാണെന്ന് ഏവരും അംഗീകരിക്കുന്നു. ലോകഘടനയിലെങ്ങും തികഞ്ഞ താളൈക്യവും കൃത്യതയും കണിശതയും പ്രകടമാണ്. വ്യക്തമായ യുക്തിയുടെയും ഉയര്‍ന്ന ആസൂത്രണത്തിന്റെയും നിത്യ നിദര്‍ശനമാണീ പ്രപഞ്ചം. മനുഷ്യന്റെ അവസ്ഥ പരിശോധിച്ചാല്‍തന്നെ ഇക്കാര്യം ആര്‍ക്കും ബോധ്യമാകും. നാം ജീവിക്കുന്ന ലോകത്ത് അറുനൂറു കോടിയോളം മനുഷ്യരുണ്ട്. എല്ലാവരും ശ്വസിക്കുന്ന വായു ഒന്നാണ്. കുടിക്കുന്ന വെള്ളവും ഒന്നുതന്നെ. കഴിക്കുന്ന ആഹാരത്തിലും പ്രകടമായ അന്തരമില്ല. എന്നിട്ടും അറുനൂറു കോടി മനുഷ്യര്‍ക്ക് അറുനൂറു കോടി മുഖം. നമ്മെപ്പോലുള്ള മറ്റൊരാളെ ലോകത്തെവിടെയും കാണുക സാധ്യമല്ല. നമ്മുടെ തള്ളവിരല്‍ എത്ര ചെറിയ അവയവമാണ്. എന്നിട്ടും അറുനൂറു കോടി മനുഷ്യരിലൊരാള്‍ക്കും നമ്മുടേതുപോലുള്ള ഒരു കൈവിരലില്ല. മരിച്ചുപോയ കോടാനുകോടിക്കുമില്ല. ജനിക്കാനിരിക്കുന്ന കോടാനുകോടികള്‍ക്ക് ഉണ്ടാവുകയുമില്ല. നമ്മുടെ ഓരോരുത്തരുടെയും തലയില്‍ പതിനായിരക്കണക്കിന് മുടിയുണ്ട്. എന്നാല്‍ ലോകത്തിലെ അറുനൂറു കോടി തലയിലെ അസംഖ്യം കോടി മുടികളിലൊന്നു പോലും നമ്മുടെ മുടിപോലെയില്ല. അവസാനത്തെ ഡി.എന്‍.എ. പരിശോധനയില്‍ നമ്മുടേത് തിരിച്ചറിയുക തന്നെ ചെയ്യും. നമ്മുടെയൊക്കെ സിരകളില്‍ ആയിരക്കണക്കിന് രക്തത്തുള്ളികള്‍ ഒഴുകിക്കൊണ്ടേയിരിക്കുന്നു. അറുനൂറു കോടിയുടെ സിരകളിലെ കോടാനുകോടി രക്തത്തുള്ളികളില്‍ നിന്നും നമ്മുടേത് വ്യത്യസ്തമത്രെ. നമ്മുടെ ശരീരത്തിന്റെ ഗന്ധം പോലും മറ്റുള്ളവരുടേതില്‍നിന്ന് ഭിന്നമത്രെ. ഇത്രയേറെ അദ്ഭുതകരവും ആസൂത്രിതവും വ്യവസ്ഥാപിതവും കണിശവുമായ അവസ്ഥയില്‍ നാമൊക്കെ ആയിത്തീര്‍ന്നത് കേവലം യാദൃഛികതയും പദാര്‍ഥത്തിന്റെ പരിണാമവും ചലനവും മൂലവുമാണെന്ന് പറയുന്നത് ഒട്ടും യുക്തിനിഷ്ഠമോ ബുദ്ധിപൂര്‍വമോ അല്ല. അത് പരമാബദ്ധമാണെന്ന് അഹന്തയാല്‍ അന്ധത ബാധിക്കാത്തവരെല്ലാം അംഗീകരിക്കും. അറുനൂറു കോടി മനുഷ്യര്‍ക്ക് അത്രയും മുഖഭാവവും കൈവിരലുകളും ഗന്ധവും തലമുടിയും രക്തത്തുള്ളികളും മറ്റും നല്‍കിയത് സര്‍വശക്തനും സര്‍വജ്ഞനും യുക്തിമാനുമായ ശക്തിയാണെന്ന് അംഗീകരിക്കലും വിശ്വസിക്കലുമാണ് ബുദ്ധിപൂര്‍വകം. ആ ശക്തിയത്രെ പ്രപഞ്ചസ്രഷ്ടാവായ ദൈവം.
പ്രപഞ്ചത്തില്‍ പുതുതായൊന്നുമുണ്ടാവില്ലെന്ന് പദാര്‍ഥവാദികള്‍ പറയുന്നു. പുതുതായി വല്ലതും ഉണ്ടാവുകയാണെങ്കില്‍ അതുണ്ടാക്കിയത് ആര് എന്ന ചോദ്യമുയരുമല്ലോ. എന്നാല്‍ ഇന്നുള്ള അറുനൂറോളം കോടി മനുഷ്യര്‍ക്ക് ബുദ്ധിയും ബോധവും അറിവും യുക്തിയുമുണ്ട്. ഈ ബുദ്ധിയും ബോധവും അറിവും യുക്തിയുമൊക്കെ എവിടെയായിരുന്നു? വിസ്‌ഫോടനത്തിനു മുമ്പുള്ള ആദിപദാര്‍ഥം ഇതൊക്കെയും ഉള്‍ക്കൊണ്ടിരുന്നോ? എങ്കില്‍ അനാദിയില്‍ ആ പദാര്‍ഥത്തിന്റെ ബുദ്ധിയും ബോധവും അറിവും യുക്തിയും അതിരുകളില്ലാത്തത്ര ആയിരിക്കില്ലേ? അപ്പോള്‍ അചേതന പദാര്‍ഥം ഇത്രയേറെ സര്‍വജ്ഞനും യുക്തിജ്ഞനുമാവുകയോ?
അതിനാല്‍ അറുനൂറു കോടി മനുഷ്യര്‍ക്ക് അറിവും ബോധവും ബുദ്ധിയും യുക്തിയും നല്‍കിയത് അതിരുകളില്ലാത്ത അറിവിന്റെയും ബോധത്തിന്റെയും യുക്തിയുടെയും ഉടമയായ സര്‍വശക്തനായ ദൈവമാണെന്ന് വിശ്വസിക്കലും അംഗീകരിക്കലുമാണ് ന്യായവും ശരിയും. സത്യസന്ധവും വിവേകപൂര്‍വകവുമായ സമീപനവും അതുതന്നെ.
അനാദിയായ ഒന്നുണ്ട്; ഉണ്ടായേ തീരൂവെന്ന് ഏവരും അംഗീകരിക്കുന്നു. അത് അചേതനമായ പദാര്‍ഥമാണെന്ന് ഭൗതികവാദികളും, സര്‍വശക്തനും സര്‍വജ്ഞനുമായ ദൈവമാണെന്ന് മതവിശ്വാസികളും പറയുന്നു. അനാദിയായ, അഥവാ തുടക്കമില്ലാത്ത ഒന്നിനെ സംബന്ധിച്ച്, അതിനെ ആരുണ്ടാക്കി; എങ്ങനെയുണ്ടായി തുടങ്ങിയ ചോദ്യങ്ങള്‍ ഒട്ടും പ്രസക്തമല്ലെന്നതും സുസമ്മതമാണ്. അനാദിയായ ആദിപദാര്‍ഥത്തെ ആരുണ്ടാക്കിയെന്ന ചോദ്യം അപ്രസക്തമാണെന്ന് പറയുന്നവര്‍ തന്നെ അനാദിയായ ദൈവത്തെ ആരുണ്ടാക്കിയെന്ന് ചോദിക്കുന്നത് അര്‍ഥശൂന്യവും അബദ്ധപൂര്‍ണവുമത്രെ.
പദാര്‍ഥ നിഷ്ഠമായ ഒന്നും ഒരു നിര്‍മാതാവില്ലാതെ ഉണ്ടാവുകയില്ല. അതിനാല്‍ പദാര്‍ഥനിര്‍മിതമായ പ്രപഞ്ചത്തിന് ഒരു സ്രഷ്ടാവ് അനിവാര്യമാണ്. എന്നാല്‍ പദാര്‍ഥപരമായതിന്റെ നിയമവും അവസ്ഥയും പദാര്‍ഥാതീതമായതിനു ബാധകമല്ല. ദൈവം പദാര്‍ഥാതീതനാണ്. അതിനാല്‍ പദാര്‍ഥനിഷ്ഠമായ പ്രപഞ്ചത്തിന്റെ നിയമവും മാനദണ്ഡവും അടിസ്ഥാനമാക്കി പദാര്‍ഥാതീതനായ ദൈവത്തെ ആരു സൃഷ്ടിച്ചുവെന്ന ചോദ്യം തീര്‍ത്തും അപ്രസക്തമത്രെ. അനാദിയായ, ആരംഭമില്ലാത്ത, എന്നെന്നും ഉള്ളതായ ഒന്നുണ്ടായേ തീരൂവെന്നത് അനിഷേധ്യവും സര്‍വസമ്മതവുമാണ്. അതാണ് സര്‍വശക്തനും പ്രപഞ്ചങ്ങളുടെയൊക്കെ സ്രഷ്ടാവും സംരക്ഷകനുമായ ദൈവം.
\chapter{ദൈവം നീതിമാനോ? }
  \section{ദൈവം നീതിമാനാണെന്നാണല്ലോ പറഞ്ഞുവരുന്നത്. എന്നാല്‍ അനുഭവം മറിച്ചാണ്. മനുഷ്യരില്‍ ചിലര്‍ വികലാംഗരും മറ്റു ചിലര്‍ മന്ദബുദ്ധികളുമാണ്. ഇത് അവരോടു ചെയ്ത കടുത്ത അനീതിയല്ലേ?}
 ഈ ചോദ്യം പ്രത്യക്ഷത്തില്‍ വളരെ പ്രസക്തവും ന്യായവും തന്നെ. എന്നാല്‍ അല്‍പം ആലോചിച്ചാല്‍ അബദ്ധം അനായാസം ബോധ്യമാകും. നമുക്ക് ഈ ചോദ്യം ഒന്നുകൂടി വികസിപ്പിക്കാം. അപ്പോള്‍ പ്രസക്തമെന്ന് തോന്നുന്ന നിരവധി ചോദ്യങ്ങള്‍ നമ്മുടെ മുമ്പില്‍ ഉയര്‍ന്നുവരും. എനിക്ക് എന്തുകൊണ്ട് ആറടി നീളം നല്‍കിയില്ലെന്ന് കുറിയവനു ചോദിക്കാവുന്നതാണ്. തന്നെ എന്തുകൊണ്ട് തൊലി വെളുത്തവനാക്കിയില്ലെന്ന കറുത്തവന്റെ ചോദ്യവും സുന്ദരനാക്കിയില്ലെന്ന വിരൂപന്റെ ചോദ്യവും പ്രതിഭാധനനാക്കിയില്ലെന്ന സാമാന്യബുദ്ധിയുടെ ചോദ്യവും സുഖകരമായ കാലാവസ്ഥയുള്ളേടത്ത് ജനിപ്പിച്ചില്ലെന്ന മരുഭൂവാസിയുടെ ചോദ്യവും ധനികകുടുംബത്തിലാക്കിയില്ലെന്ന ദരിദ്രന്റെ ചോദ്യവും ഭരണാധികാരിയാക്കിയില്ലെന്ന ഭരണീയന്റെ ചോദ്യവും ഇരുപത്തൊന്നാം നൂറ്റാണ്ടില്‍ ജീവിപ്പിച്ചില്ല എന്ന പത്താം നൂറ്റാണ്ടുകാരന്റെ ചോദ്യവുമൊക്കെ പ്രസക്തവും ന്യായവുമത്രെ. ഇതൊക്കെയും ഓരോരുത്തരുടെയും തലത്തില്‍നിന്ന് നോക്കുമ്പോള്‍ തികഞ്ഞ അനീതിയുമാണ്. നമ്മുടെ സഹധര്‍മിണിമാരുടെ ചോദ്യം ഇതിനെക്കാളെല്ലാം ന്യായവും പ്രസക്തവുമത്രെ. മാസത്തില്‍ നിശ്ചിത ദിവസം ചില പ്രയാസങ്ങളനുഭവിക്കുന്നവളും ഗര്‍ഭം ചുമക്കേണ്ടവളും പ്രസവിക്കേണ്ടവളും കുട്ടിക്ക് മുല കൊടുക്കേണ്ടവളുമായി തന്നെയെന്തിനു സൃഷ്ടിച്ചു; തന്റെ ഭര്‍ത്താവിന് ഇത്തരം പ്രയാസങ്ങളൊന്നുമില്ലല്ലോ; അതിനാല്‍ തന്നെ എന്തുകൊണ്ട് ആണായി സൃഷ്ടിച്ചില്ല എന്ന് ഏതൊരു സ്ത്രീക്കും ചോദിക്കാവുന്നതാണ്. തന്നോട് ചെയ്തത് കടുത്ത അനീതിയാണെന്ന് സമര്‍ഥിക്കുകയും ചെയ്യാം. മനുഷ്യര്‍ക്കിടയിലെ ഈ വ്യത്യാസങ്ങളെല്ലാം ഇല്ലാതാവലാണ് നീതിയെങ്കില്‍ എല്ലാവരും ഒരേ ഭൂപ്രദേശത്ത്, ഒരേ കാലാവസ്ഥയില്‍, ഒരേ കാലത്ത്, ഒരേ കുടുംബത്തില്‍, ഒരേ മാതാപിതാക്കളുടെ മക്കളായി, ബുദ്ധിപരമായി ഒരേ നിലവാരത്തിലുള്ളവരായി, ഒരേ ശരീരപ്രകൃതിയോടുകൂടി, ഒരേ ലിംഗക്കാരായി, ഒരേവിധം ആരോഗ്യാവസ്ഥയില്‍ ജനിക്കുകയും ജീവിക്കുകയും വേണ്ടിവരും. ഇത് തീര്‍ത്തും അസാധ്യവും അപ്രായോഗികവുമാണെന്ന് ഏവര്‍ക്കും അറിയാവുന്ന കാര്യമാണ്. അതിനാല്‍ എല്ലാ കാര്യങ്ങളിലെയും വൈവിധ്യം മനുഷ്യരാശിയുടെ നിലനില്‍പിന് അനിവാര്യമത്രെ.
മനുഷ്യജീവിതം മരണത്തോടെ അവസാനിക്കുമെങ്കില്‍ നീതിയെ സംബന്ധിച്ച ഈ ചോദ്യങ്ങളെല്ലാം തീര്‍ത്തും പ്രസക്തങ്ങളാണ്. എങ്കില്‍ ജനനം മുതല്‍ മരണം വരെയുള്ള ജീവിതം എല്ലാവര്‍ക്കും ഒരേപോലെ തുല്യമായി അനുഭവിക്കാനും ആസ്വദിക്കാനും അവസരം ലഭിക്കുകതന്നെ വേണം. എന്നാല്‍ നീതിമാനായ ദൈവം മരണത്തോടെ മനുഷ്യജീവിതം അവസാനിപ്പിക്കുന്നില്ല. ഐഹിക ജീവിതം കര്‍മകാലമാണ്. വിചാരണയും വിധിയും കര്‍മഫലവും മരണശേഷം മറുലോകത്താണ്. ഓരോരുത്തര്‍ക്കും ഭൂമിയില്‍ നിര്‍വഹിക്കാനുള്ള ബാധ്യത അവരവര്‍ക്ക് നല്‍കപ്പെട്ട കഴിവുകള്‍ക്കനുസരിച്ചാണ്. പണക്കാരന്റെയത്ര ബാധ്യത പാവപ്പെട്ടവന്നില്ല. പണ്ഡിതന്റെ ചുമതല പാമരന്നില്ല. പ്രതിഭാശാലികളുടെ ഉത്തരവാദിത്വം സാമാന്യബുദ്ധിക്കില്ല. മന്ദബുദ്ധിക്ക് അത്രയുമില്ല. വികലാംഗന് പൂര്‍ണ ആരോഗ്യവാനെ അപേക്ഷിച്ച് കുറഞ്ഞ ഉത്തരവാദിത്വമേയുള്ളൂ. ഈ ബാധ്യതകളുടെ നിര്‍വഹണമാണ് ജീവിതത്തിന്റെ ജയാപജയങ്ങളുടെ നിദാനം. ഓരോരുത്തര്‍ക്കും ലഭ്യമായ കഴിവുകള്‍ ഏതുവിധം വിനിയോഗിച്ചുവെന്നതാണ് വിലയിരുത്തപ്പെടുക. തദനുസൃതമായി ഓരോരുത്തരിലും അര്‍പ്പിതമായ ചുമതലകളുടെ പൂര്‍ത്തീകരണത്തിന്റെ അടിസ്ഥാനത്തിലാണ് മരണാനന്തരജീവിതത്തിലെ രക്ഷാശിക്ഷകള്‍ നിശ്ചയിക്കപ്പെടുക. അതിനാല്‍ മരണശേഷം മറുലോകമില്ലെങ്കില്‍ മാത്രമേ ഈ ഭൂമിയിലെ മനുഷ്യന്റെ അവസ്ഥാന്തരങ്ങള്‍ അനീതിപരമാവുകയുള്ളൂ. അനശ്വരമായ പരലോകം അനിഷേധ്യമാണെന്നതിനാല്‍ അംഗവൈകല്യവും ആരോഗ്യാവസ്ഥയിലെ അന്തരവും ധൈഷണിക നിലവാരത്തിലെ വ്യത്യാസവുമൊന്നും ദൈവനീതിക്കു തീരെ വിരുദ്ധമാവുന്നില്ല.
വികലാംഗന്‍ അംഗവൈകല്യമില്ലാത്തവന്റെയും പാവപ്പെട്ടവന്‍ പണക്കാരന്റെയും മുമ്പിലെ പരീക്ഷണം കൂടിയത്രെ. തങ്ങള്‍ക്ക് ദൈവം നല്‍കിയ അനുഗ്രഹങ്ങളനുസ്മരിച്ച് വികലാംഗരോടും ദരിദ്രരോടും സഹാനുഭൂതിയും അനുകമ്പയും പുലര്‍ത്തി, സ്‌നേഹം പകര്‍ന്നുകൊടുക്കുകയും സഹായസഹകരണങ്ങള്‍ നല്‍കുകയും ഉദാരമായി പെരുമാറുകയും ചെയ്യുന്നുണ്ടോയെന്നതിന്റെ കൂടി അടിസ്ഥാനത്തിലായിരിക്കും അവരുടെ മരണാനന്തരജീവിതത്തിലെ രക്ഷാശിക്ഷകള്‍.
എന്നാല്‍ ദൈവത്തെയും പരലോകത്തെയും നിഷേധിക്കുന്ന ഭൗതിക വാദികള്‍ക്ക് നീതിയെ സംബന്ധിച്ച ഇത്തരം ചോദ്യങ്ങള്‍ക്കൊന്നും ഉത്തരം നല്‍കുക സാധ്യമല്ല. പ്രപഞ്ചത്തിന് നിയതമായ താളവും വ്യക്തമായ ക്രമവും വ്യവസ്ഥയും നല്‍കിയ 'പ്രകൃതി' എന്തുകൊണ്ട് മനുഷ്യരോട് നീതി കാണിച്ചില്ല? പ്രകൃതി ചെയ്ത ഈ കടുത്ത അനീതിക്ക് എന്തു പരിഹാരമാണ് ഭൗതികവാദിക്ക് നിര്‍ദേശിക്കാനുള്ളത്? ഉത്തരം നല്‍കേണ്ടത് ദൈവനിഷേധികളും മതവിരുദ്ധരുമാണ്.


